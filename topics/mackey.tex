\section{Project: Mackey's theorem}

\begin{definition}
    Two representations of a finite group 
    are said to be \emph{disjoint} if they have no common 
    irreducible constituent. 
\end{definition}

\begin{exercise}
\label{xca:disjoint}
    Prove that 
    two representations are disjoint if and only if their characters are orthogonal. 
\end{exercise}

Let $H$ and $K$ be subgroups of a group $G$. 
The group $H\times K$ acts on $G$ via $(h,k)\cdot g=hgk^{-1}$. 
The orbit of $g$ under this action is 
the \emph{double coset} 
\[
HgK=\{(h,k)\cdot g:h\in H,k\in K\}
=\{hgk^{-1}:h\in H,k\in K\}
\]
with representative $g$. 

\begin{theorem}[Mackey]
\label{thm:Mackey}
\index{Mackey!theorem}
    Let $G$ be a finite group and $H$ and $K$ be 
    subgroups of $G$. Let $S$ be a complete set 
    of representatives of double $(H,K)$-cosets. If $\alpha\cf(K)$, then 
    \[
    \Res_H^G\Ind_K^G\alpha=\sum_{s\in S}\Ind_{H\cap sKs^{-1}}^H\Res_{H\cap sKs^{-1}}^{sKs^{-1}}(s\cdot f).
    \]
\end{theorem}

\begin{proof}
    For $s\in S$, let $X(s)$ be a left transversal 
    for $H\cap sKs^{-1}$ on $H$. Then 
    \[
    H=\bigcup_{x\in X(s)}x(H\cap sKs^{-1}),
    \]
    where the union is disjoint. 

    \begin{claim}
        $HsK=\bigcup_{x\in X(s)}xsK$, where the union is disjoint.
    \end{claim} 
    
    Let $z\in HsK$. Then $z=hsk$ for some $h\in H$ and $k\in K$. Since $h\in x(H\cap sKs^{-1})$ 
    for some $x\in X(s)$, 
    \[
    z=hsk\in x(H\cap sKs^{-1})sK\subseteq xsK. 
    \]
    Conversely, let $z\in xsK$ for some $x\in X(s)\subseteq H$. Then $z\in xsK\subseteq HsK$. To see that the union is disjoint, 
    suppose that $xsK=x_1sK$ for some $x,x_1\in X(s)$. Then 
    $x_1^{-1}x\in sKs^{-1}\cap H$. Thus $x(sKs^{-1}\cap H)=x_1(sKs^{-1}\cap H)$ and hence $x=x_1$, because $X(s)$ 
    is a left transversal for $sKs^{-1}\cap H$ in $H$. 

    \bigskip 
    Let $T(s)=\{xs:x\in X(s)\}$ and 
    \[
    T=\bigcup_{s\in S}T(s)
    \]
    To see that the union is disjoint, we proceed as follows. 
    Let $xs=x_1s_1$ for some $s,s_1\in S$, $x\in X(s)$ and $x_1\in X(s_1)$. 
    Since $x^{-1}x_1\in H$ and 
    $HsK=Hx^{-1}x_1s_1K=Hs_1K$, $s=s_1$ and hence $x=x_1$. 

    Then
    \[
    G=\bigcup_{s\in S}HsK
    =\bigcup_{s\in S}\bigcup_{x\in X(s)}xsK
    =\bigcup_{s\in S}\bigcup_{t\in T(s)}tK
    =\bigcup_{t\in T}tK.
    \]
    Since the unions are disjoint, 
    it follows that $T$ is a left transversal of $K$ in $G$. 

    For $h\in H$, 
    \begin{align*}
        (\Ind_K^G\alpha)(h) &= \sum_{t\in T}\alpha^0(t^{-1}ht)\\
        &=\sum_{s\in S}\sum_{t\in T(s)}\alpha^0(t^{-1}ht)\\
        &=\sum_{s\in S}\sum_{x\in X(s)}\alpha^0(s^{-1}x^{-1}hxs)\\
        &=\sum_{s\in S}\sum_{\substack{x\in X(s)\\x^{-1}hx\in sKs^{-1}}}(s\cdot \alpha)(x^{-1}hx)\\
        &=\sum_{s\in S}\sum_{\substack{x\in X(s)\\x^{-1}hx\in H\cap sKs^{-1}}}\Res_{H\cap sKs^{-1}}^{sKs^{-1}}(s\cdot \alpha)(\underbrace{x^{-1}hx)}_{\in H\cap sKs^{-1}}\\
        &=\sum_{s\in S}\Ind_{H\cap sKs^{-1}}^H\Res_{H\cap sKs^{-1}}^{sKs^{-1}}(s\cdot \alpha)(h).\qedhere 
    \end{align*}
\end{proof}

\begin{theorem}[Mackey's irreducibility criterion]
\label{thm:Mackey_irreducibility}
\index{Mackey!irreducibility criterion}
Let $H$ be a subgroup of a finite group $G$ and $\chi\in\Char(H)$.
Then $\Ind_H^G\chi\in\Irr(G)$ if and only if $\chi\in\Irr(H)$ and $\Res_{H\cap sHs^{-1}}^H\chi$ and 
$\Res_{H\cap sHs^{-1}}^{sHs^{-1}}(s\cdot\chi)$ are disjoint for all $s\not\in H$. 
\end{theorem}
\begin{proof}
    Let $S$ be a complete set of representatives of $(H,H)$-double cosets. Without loss of generality, 
    we may assume that $1\in S$. Note that if $s=1$, then 
    $H\cap sHs^{-1}=H$ and $s\cdot\chi=\chi$. By Mackey's theorem, 
    \begin{align*}
        \Res_H^G\Ind_H^G\chi 
        &=\sum_{s\in S}\Ind_{H\cap sHs^{-1}}^H\Res_{H\cap sHs^{-1}}^{sHs^{-1}}(s\cdot\chi)\\
        &=\chi+\sum_{1\ne s\in S}\Ind_{H\cap sHs^{-1}}^H\Res_{H\cap sHs^{-1}}^{sHs^{-1}}(s\cdot\chi).
    \end{align*}
    By Frobenius' reciprocity, 
    \begin{align*}
        \langle\Ind_H^G\chi,\Ind_H^G\chi\rangle
        &=\langle\Res_H^G\Ind_H^G\chi,\chi\rangle\\
        &=\underbrace{\langle\chi,\chi\rangle}_{\geq1}+\sum_{1\ne s\in S}\underbrace{\langle\Ind_{H\cap sHs^{-1}}^H\Res_{H\cap sHs^{-1}}^{sHs^{-1}}(s\cdot\chi),\chi\rangle}_{\geq0}.
    \end{align*}

    If $\chi\in\Irr(H)$ and $\Res_{H\cap sHs^{-1}}^H\chi$ and 
    $\Res_{H\cap sHs^{-1}}^{sHs^{-1}}(s\cdot\chi)$ are disjoint for all $s\not\in H$, then 
    $\langle\Ind_H^G\chi,\Ind_H^G\chi\rangle=1$ and hence $\Ind_H^G\chi\in\Irr(G)$. 

    Conversely, if $\Ind_H^G\chi\in\Irr(G)$, then $\langle\Ind_H^G\chi,\Ind_H^G\chi\rangle=1$. Thus 
    $\langle\chi,\chi\rangle=1$ and 
    \[
    \langle\Res_{H\cap sHs^{-1}}^{sHs^{-1}}(s\cdot\chi),\Res_{H\cap sHs^{-1}}^H\chi\rangle=
    \langle\Ind_{H\cap sHs^{-1}}^H\Res_{H\cap sHs^{-1}}^{sHs^{-1}}(s\cdot\chi),\chi\rangle=0
    \]
    for all $s\in S\setminus\{1\}$. As every element $s\notin H$ 
    could serve as a representative of an $(H,H)$-double coset, the claim follows.
\end{proof}

Theorem~\ref{thm:Mackey_irreducibility} takes a particularly elegant form when the subgroup is normal.

\begin{exercise}
\label{xca:Mackey}
    Let $H$ be a normal subgroup of a finite group $G$ and $\chi\in\Char(H)$.
    Then $\Ind_H^G\chi\in\Irr(G)$ if and only if $\chi\in\Irr(H)$ and $\chi\ne s\cdot\chi$ 
    for all $s\not\in H$. 
\end{exercise}

\begin{example}
\label{exa:p(p-1)}
    For a prime number $p\geq3$, let 
    \[
        G=\left\{\begin{pmatrix}a&b\\0&1\end{pmatrix}:0\ne a\in\Z/p,\,b\in\Z/p\right\}\text{ and }
        H=\left\{\begin{pmatrix}1&b\\0&1\end{pmatrix}:b\in\Z/p\right\}.
    \]
    Then $|G|=p(p-1)$, $H$ is a normal subgroup of $G$ and $|G/H|=p-1$. Let 
    \[
    \chi\colon H\to\C^{\times},\quad\begin{pmatrix}1&b\\0&1\end{pmatrix}\mapsto \exp(2\pi ib/p).
    \]
    Then $\chi$ is a group homomorphism. For each $a\in\Z/p\setminus\{0,1\}$, let $s(a)=\begin{pmatrix}a&0\\0&1\end{pmatrix}$. Then 
    \[
    (s(a)\cdot\chi)\begin{pmatrix}1&b\\0&1\end{pmatrix}=\chi\begin{pmatrix}1&a^{-1}b\\0&1\end{pmatrix}=\exp(2\pi ia^{-1}b/p)
    \ne \exp(2\pi ib/p).
    \]
    Hence $s(a)\cdot\chi\ne\chi$ for all $a\in\Z/p\setminus\{0,1\}$. By Exercise~\ref{xca:Mackey}, 
    $\Ind_H^G\chi\in\Irr(G)$ and 
    \[
    \deg\Ind_H^G\chi=(\Ind_H^G\chi)(1)=(G:H)\chi(1)=p-1.
    \]
    
    Since 
    $|G|-(p-1)^2=p-1$, 
    we still need additional irreducible characters to fully determine $\Irr(G)$. 
    The group $G/H$ is cyclic of order $p-1$, so it has $p-1$ irreducible characters, all of degree one. 
    These irreducible characters lift to irreducible characters of $G$ (see Theorem~\ref{thm:correspondence}). 
\end{example}

\begin{bonus}
\label{xca:p(p-1)}
    Find the character table of the group of Example~\ref{exa:p(p-1)}. 
\end{bonus}