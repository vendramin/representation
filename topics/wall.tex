\section{Project: Wall's theorem}

We now present a character-theoretic proof of a theorem of Wall \cite{MR125156}, 
which bounds the number of maximal subgroups of a finite solvable group. 

\begin{exercise}
    \label{xca:number_maximals}
    Let $G$ be a finite group and $M$ be a maximal subgroup of $G$.
    Prove that $M$ has exactly $(G:M)$ conjugates. 
\end{exercise}

\begin{lemma}
\label{lem:multiplicity-free}
    Let $G$ be a finite solvable group and $M$ be a maximal subgroup of $G$. If $M$ is core-free, then 
    every irreducible constituent of $\Ind_M^G\Tchar_M$ has multiplicity one. 
\end{lemma}

\begin{proof}
    Let $N$ be a minimal normal subgroup of $G$. Then 
    $G=NM$ and $N\cap M=\{1\}$ (see 
    By Lemma~\ref{lem:Ore_step2}). In particular, 
    $|N|=(G:M)$. 

    We claim that 
    $\Res_N^G\Ind_M^G\Tchar_M$ is the regular character of $N$. For 
    $n\in N$, 
    \[
    (\Ind_M^G\Tchar_M)(n)=\frac{1}{|M|}\sum_{x\in G}\Tchar_M^0(x^{-1}nx)
    =\begin{cases}
        (G:M) & \text{if $n=1$,}\\
        0 & \text{otherwise.}
    \end{cases}
    \]
    By Theorem~\ref{thm:regular}, 
    $\Res_N^G\Ind_M^G\Tchar_M$ is the regular character of $N$ and hence 
    \[
    \Res_N^G\Ind_M^G\Tchar_M=\sum_{\lambda\in\Irr(N)}\lambda(1)\lambda.
    \]
    Since $N$ is abelian, $\lambda(1)=1$ for all $\lambda\in\Irr(N)$. Now assume
    that some irreducible constituent $\psi\in\Irr(G)$ of $\Ind_M^G\Tchar_M$ has
    multiplicity $m\geq2$, say $\Ind_M^G\Tchar_M=m\psi+\xi$, where $\langle\psi,\xi\rangle=0$. 
    Then 
    \[
    \sum_{\lambda\in\Irr(N)}\lambda=\Res_N^G\Ind_M^G\Tchar_M=m\Res_N^G\psi+\Res_N^G\xi,
    \]
    a contradiction. 
\end{proof}

\begin{lemma}
\label{lem:kernel}
    Let $G$ be a finite solvable group and $M$ be a maximal subgroup of $G$. If $M$ is core-free, then 
    \[
    \ker\Ind_M^G\Tchar_M=\bigcap\{\ker\chi:\chi\in\Irr(G)\text{ such that }\langle\Ind_M^G\Tchar_M,\chi\rangle\ne0\}.
    \]
    Thus $\ker\Ind_M^G\Tchar_M$ is the intersection of the kernels of the irreducible constituents 
    of $\Ind_M^G\Tchar_M$.
\end{lemma}

\begin{proof}
    By Lemma~\ref{lem:multiplicity-free}, every irreducible constituent of 
    $\Ind_M^G\Tchar_M$ appears with multiplicity one, that is 
    \[
    \Ind_M^G\Tchar_M=\sum_{j=1}^k\chi_j
    \]
    for some subset $\{\chi_1,\dots,\chi_k\}\subseteq\Irr(G)$. If $g\in\ker\chi_1\cap\cdots\cap\ker\chi_k$, then
    $\chi_j(g)=\chi_j(1)$ for all $j\in\{1,\dots,k\}$. Thus 
    \[
    \sum_{j=1}^k\chi_j(g)=(\Ind_M^G\Tchar_M)(g)=(\Ind_M^G\Tchar_M)(1)=\sum_{j=1}^k\chi_j(1). 
    \]
    Conversely, let $g\in\ker\Ind_M^G\Tchar_M$. 
    Then 
    \[
    \sum_{j=1}^k\chi_j(g)=(\Ind_M^G\Tchar_M)(g)=(\Ind_M^G\Tchar_M)(1)=\sum_{j=1}^k\chi_j(1).
    \]
    Assume that there exists 
    $i\in\{1,\dots,k\}$ such that 
    $g\not\in\ker\chi_i$. Then 
    $|\chi_i(g)|<\chi_i(1)$ and hence 
    \[
    \sum_{j=1}^k\chi_j(1)=\left|\sum_{j=1}^k\chi(g)\right|\leq
    \sum_{j=1}^k|\chi_j(g)|<\sum_{j=1}^k\chi_j(1),
    \]
    a contradiction. 
\end{proof}

\begin{lemma}
    Let $G$ be a finite solvable group and $M$ be a  
    maximal subgroup of $G$. If $M$ is not normal in $G$, 
    then $\Tchar_G$ is the only degree-one constituent of $\Ind_M^G\Tchar_M$. 
\end{lemma}

\begin{proof}
    Let $\psi\in\Irr(G)$ be a degree-one constituent of $\Ind_M^G\Tchar_M$. 
    By Frobenius' reciprocity, $0\ne\langle\Ind_M^G\Tchar_M,\psi\rangle=\langle\Tchar_M,\Res_M^G\psi\rangle$.
    Since $\Res_M^G\psi$ is a degree-one character, it follows the irreducibility that 
    $\Tchar_M=\Res_M^G\psi$. In particular, 
    \[
    M=\ker\Tchar_M=\ker(\Res_M^G\psi)=M\cap\ker\psi\subseteq\ker\psi. 
    \]
    
    Assume now that $\psi\ne\Tchar_G$. Then 
    \[
    M\subseteq \ker\psi=\Core_GM\subsetneq M,
    \]
    a contradiction. Hence $\psi=\Tchar_G$. 
\end{proof}

\begin{theorem}[Wall]
\index{Wall theorem}
\label{thm:Wall}
Let $G$ be a finite solvable group. Then the number of maximal subgroups of $G$ is at most $|G|-1$. 
\end{theorem}

\begin{proof}
    For a maximal subgroup $M$ of $G$, let 
    $C(M)$ be the set of non-trivial constituents
    of $\Ind_M^G\Tchar_M$, that is
    \[
    C(M)=\{\chi\in\Irr(G):\chi\ne\Tchar_G\text{ and }\langle\eta,\Ind_M^G\Tchar_M\rangle\ne0\}.
    \]

    Let $M_1$ and $M_2$ be maximal subgroups of $G$. 
    We claim that $C(M_1)\cap C(M_2)=\emptyset$ if $M_1$ and $M_2$ are
    not conjugate. 
    
    Assume that $M_1$ and $M_2$ are not conjugate. By  
    Ore's theorem~\ref{thm:Ore}, $G=M_1M_2$. In particular, 
    there is only one double $(M_1,M_2)$-coset, with representative $1$. Thus
    \[
    \Res_{M_1}^G\Ind_{M_2}^G\Tchar_{M_2}=\Ind_{M_1\cap M_2}^{M_1}\Res_{M_1\cap M_2}^{M_2}\Tchar_{M_2}
    =\Ind_{M_1\cap M_2}^{M_1}\Tchar_{M_1\cap M_2}.
    \]
    Using Frobenius' reciprocity (Theorem~\ref{thm:reciprocity}) and 
    Mackey's theorem~\ref{thm:Mackey}, 
    \begin{align*}
    \langle\Ind_{M_1}^G\Tchar_{M_1},\Ind_{M_2}^G\Tchar_{M_2}\rangle
    &=\langle \Tchar_{M_1},\Res_{M_1}^G\Ind_{M_2}^G\Tchar_{M_2}\rangle\\
    &=\langle \Tchar_{M_1},\Ind_{M_1\cap M_2}^{M_1}\Tchar_{M_1\cap M_2}\rangle\\
    &=\langle \Res_{M_1\cap M_2}^{M_1}\Tchar_{M_1},\Tchar_{M_1\cap M_2}\rangle\\
    &=\langle \Tchar_{M_1\cap M_2},\Tchar_{M_1\cap M_2}\rangle\\
    &=1.
    \end{align*}
    For $i\in\{1,2\}$, by Frobenius' reciprocity, 
    \[
    \langle\Ind_{M_i}^G\Tchar_{M_i},\Tchar_G\rangle
    =\langle\Tchar_{M_i},\Res_{M_i}^G\Tchar_G\rangle=\langle \Tchar_{M_i},\Tchar_{M_i}\rangle=1.
    \]
    Thus $\Tchar_{G}$ is an irreducible constituent of both 
    $\Ind_{M_1}^G\Tchar_{M_1}$ and $\Ind_{M_2}^G\Tchar_{M_2}$. Since 
    \[
    \langle\Ind_{M_1}^G\Tchar_{M_1},\Ind_{M_2}^G\Tchar_{M_2}\rangle=1,
    \]
    the set of non-trivial 
    irreducible
    constituents of $\Ind_{M_1}^G\Tchar_{M_1}$ and $\Ind_{M_2}^G\Tchar_{M_2}$ are disjoint, that~is  
    \[
    \Ind_{M_1}^G\Tchar_{M_1}=\Tchar_G+\eta_1,
    \quad 
    \Ind_{M_2}^G\Tchar_{M_2}=\Tchar_G+\eta_2,
    \quad 
    \langle\eta_1,\eta_2\rangle=0.
    \]
    Hence $C(M_1)\cap C(M_2)=\emptyset$. 

    
   Let $X$ be the set of maximal subgroups of $G$ that are normal in $G$, and 
   let $Y$ be the set of representatives of non-normal 
   maximal subgroups of $G$. By Exercise~\ref{xca:number_maximals}, 
   the number of 
   maximal subgroups of $G$ is 
   \[
   m=|X|+\sum_{M\in Y}(G:M).
   \]

   For every maximal subgroup $M$ of $G$ such that $M$ is normal in $G$, 
   we know that $\Ind_M^G\Tchar_M$ decomposes 
   as $\Ind_M^G\Tchar_M=\Tchar_G+\eta_1+\cdots+\eta_k$ for some  
   $\eta_1,\dots,\eta_k\in\Irr(G)\setminus\{\Tchar_G\}$ such that 
   $\eta_j(1)=1$ for all $j\in\{1,\dots,k\}$. Then
   \[
   (G:M)-1=\eta_1(1)+\cdots+\eta_k(1)=\sum_{i=1}^k\eta_i(1)^2
   \]
   since $\eta_j(1)=1$ for all $j$. Since $p$ is the smallest prime divisor of $|G|$, 
   it follows that 
   \[
   \sum_{\eta\in C(M)}\eta(1)^2
   =(G:M)-1\geq p-1.
   \]

   For every maximal subgroup $M$ of $G$ such that $M$ is not normal in $G$, 
   $\Ind_M^G\Tchar_M$ decomposes 
   as $\Ind_M^G\Tchar_M=\Tchar_G+\xi_1+\cdots+\xi_l$ for some distinct characters 
   $\xi_1,\dots,\xi_k\in\Irr(G)$ 
   such that $\xi_j(1)\geq p$ for all $j\in\{1,\dots,k\}$. 
   Then 
   \[
   \sum_{\xi\in C(M)}\xi(1)^2
   \geq p\sum_{\xi\in C(M)}\xi(1)
   =p((G:M)-1)
   \geq (p-1)(G:M).
   \]

   Now 
   \begin{align*}
    |G|-1 &=\sum_{\Tchar_G\ne\chi\in\Irr(G)}\chi(1)^2
    \geq\sum_{M\in X}\sum_{\eta\in C(X)}\eta(1)^2
    +\sum_{M\in Y}\sum_{\xi\in C(X)}\xi(1)^2\\
    &\geq (p-1)|X|+(p-1)\sum_{M\in Y}(G:M)\\
    &=(p-1)\left(|X|+\sum_{M\in Y}(G:M)\right)=(p-1)m.\qedhere 
   \end{align*}
\end{proof}

\begin{exercise}
    Prove that a finite solvable group has exactly $|G|-1$ maximal subgroups
    if and only if it is an elementary abelian $2$-group. 
\end{exercise}

\begin{example}
    Let $G=\Sym_3$. 
    Recall from Table~\ref{tab:S3} that $\Irr(G)=\{\Tchar_G,\sgn,\chi\}$, where $\sgn$ is the sign representation
    and $\chi\colon G\to\C^{\times}$ is given by
    \[
    \chi(g)=\begin{cases}
        2 & \text{if $g=\id$},\\
        0 & \text{if $g\in\{(12),(13),(23)\}$},\\
        -1 & \text{if $g\in\{(123),(132)\}.$}
    \end{cases}
    \]

    The group $G$ has two conjugacy classes of maximal subgroups, namely 
    \[
    \{\langle (123)\rangle\}\text{ and }\{\langle(12)\rangle,\langle(23)\rangle,\langle(13)\rangle\}.
    \]

    As the group $G$ is rather small, this can be easily verified 
    by a direct calculation. In any case, here is the Magma code:
    \begin{lstlisting}
> S3 := Sym(3);
> max := MaximalSubgroups(S3);
> max;
Conjugacy classes of subgroups
------------------------------

[1]     Order 2            Length 3
        Permutation group M acting on a set of cardinality 3
        Order = 2
            (2, 3)
[2]     Order 3            Length 1
        Permutation group N acting on a set of cardinality 3
        Order = 3
            (1, 2, 3)
    \end{lstlisting}

    Let $M=\langle (23)\rangle$ and $N=\langle (123)\rangle$. Thus $X=\{N\}$ and 
    $Y=\{M\}$. 
    
    Let us compute $C(N)$. For that purpose, let $t_1=1$ 
    and $t_2=(12)$ be a transversal of $N$ in $G$. Then 
    \[
    (\Ind_{N}^G\Tchar_N)(g)=\Tchar_N^0(g)+\Tchar_N^0((12)g(12))=\begin{cases}
        2 & \text{if $g=\id$},\\
        0 & \text{if $g\in\{(12),(23),(13)\}$},\\
        2 & \text{otherwise.}
    \end{cases}
    \]
    Thus $\Ind_{N}^G\Tchar_N=\Tchar_G+\sgn$ and $C(N)=\{\sgn\}$. Here is the Magma code: 
\begin{lstlisting}
> N := max[2]`subgroup;
> g := Character(TrivialRepresentation(N));
> ind_N := Induction(g, S3);
> ind_N;
( 2, 0, 2 )
\end{lstlisting}    

To decompose our induced character, with Magma we proceed as follows:
\begin{lstlisting}
> T := CharacterTable(S3);
> Decomposition(T, ind_N);
[
    1,
    1,
    0
]
( 0, 0, 0 )
> InnerProduct(T[3], ind_N);
0
> InnerProduct(T[2], ind_N);
1
> InnerProduct(T[1], ind_N);
1
\end{lstlisting}


    Let us now compute $C(M)$. A direct calculation shows that 
    \[
    (\Ind_{M}^G\Tchar_{M})(g)=\begin{cases}
        3 & \text{if $g=\id$},\\
        1 & \text{if $g\in\{(12),(23),(13)\}$},\\
        0 & \text{otherwise.}
    \end{cases}
    \]
    Thus $\Ind_{M}\Tchar_{M}=\Tchar_G+\chi$ and $C(M)=\{\chi\}$. 
    We leave it as an exercise to verify these calculations, either by 
    hand, with Magma, or perhaps both.
\end{example}

\begin{example}
    Let $G=\Alt_4$. There are two conjugacy classes of maximal subgroups of $G$ with representatives
    are $M=\langle (234)\rangle$ and $N=\{\id,(12)(34),(13)(24),(14)(23)\}$. Then 
    $X=\{N\}$ and $Y=\{M\}$. 

    In Exercise~\ref{xca:A4}, we asked for the construction of the character table of \( \Alt_4 \). The completed table is shown in Table~\ref{tab:A4}.

    \index{Character table!of $\Alt_4$}
    \begin{table}[h]
        \centering\makegapedcells
        \caption{The Character table of $\Alt_4$.}
        \label{tab:A4}
        \begin{tabular}{|c|cccc|}
             \hline
             & $\id$ & $(12)(34)$ & $(123)$ & $(132)$\\
             \hline
             $\chi_1$ & $1$ & $1$ & $1$ & $1$\\
             $\chi_2$ & $1$ & $1$ & $\frac{-1+\sqrt{-3}}{2}$ & $\frac{-1-\sqrt{-3}}{2}$\\
             $\chi_3$ & $1$ & $1$ & $\frac{-1-\sqrt{-3}}{2}$ & $\frac{-1+\sqrt{-3}}{2}$\\
             $\chi_4$ & $3$ & $-1$ & $0$ & $0$\\
             \hline
        \end{tabular}
    \end{table}

    
    A direct calculation shows that
    $\Ind_N^G\Tchar_N=\Tchar_G+\chi_2+\chi_3$ and 
    $\Ind_M^G\Tchar_M=\Tchar_G+\chi_4$. 
\end{example}
% > A4 := Alt(4);
% > max := MaximalSubgroups(A4);
% > #max;
% 2
% > max;
% Conjugacy classes of subgroups
% ------------------------------

% [1]     Order 3            Length 4
%         Permutation group acting on a set of cardinality 4
%         Order = 3
%             (2, 3, 4)
% [2]     Order 4            Length 1
%         Permutation group acting on a set of cardinality 4
%         Order = 4 = 2^2
%             (1, 3)(2, 4)
%             (1, 2)(3, 4)
% > M := max[1]`subgroup;
% > f := Character(TrivialRepresentation(M));
% > f;
% ( 1, 1, 1 )
% > ind_M := Induction(f, A4);
% > ind_M;
% ( 4, 0, 1, 1 )
% > T := CharacterTable(A4);
% > Decomposition(T, ind_M);
% [
%     1,
%     0,
%     0,
%     1
% ]
% ( 0, 0, 0, 0 )
% > T[4];
% ( 3, -1, 0, 0 )
% > T[1];
% ( 1, 1, 1, 1 )
The ideas used here to prove Theorem~\ref{thm:Wall} can be applied 
to obtain the following variant of Wall’s theorem, established  
by Cook, Wiegold, and Williamson in \cite{MR896628}.

\begin{theorem}[Cook--Wiegold--Williamson]
    \label{thm:CookWiegold-Williamson}
    \index{Cook--Wiegold--Williamson theorem}
    Let $G$ be a finite solvable group and $p$ the smallest prime divisor of $|G|$. 
    Then the number of maximal subgroups of $G$ is at most $\frac{|G|-1}{p-1}$. Equality 
    holds if and only if $G$ is an elementary $p$-group.  
\end{theorem}

\begin{bonus}
    Prove Theorem~\ref{thm:CookWiegold-Williamson}.    
\end{bonus}

\begin{exercise}
    \label{xca:kernel_ind}
    Let $H$ be a subgroup of a finite group $G$ and 
    $\chi\in\Char(H)$. Prove that 
    $\ker\Ind_H^G\chi=\bigcap_{x\in G}x(\ker\chi)x^{-1}$. 
\end{exercise}

\begin{sol}{xca:kernel_ind}
    Let $g\in\ker\Ind_H^G\chi$. Then \[
    \sum_{x\in G}\chi^0(x^{-1}gx)=\sum_{x\in G}\chi(1).
    \]
    Then $\chi^0(x^{-1}gx)=\chi(1)$ for all $x\in G$. (Otherwise, $|\chi^0(y^{-1}gy)|<\chi(1)$ for some
    $y\in G$ and hence 
    \[
    \sum_{x\in G}\chi(1)=\left|\sum_{x\in G}|\chi^0(x^{-1}gx)\right|
    \leq\sum_{x\in G}|\chi^0(x^{-1}gx)|<\sum_{x\in G}\chi(1),
    \]
    a contradiction.) In particular, $x^{-1}gx\in\ker\chi\subseteq H$ for all $x\in G$, that 
    is $g\in x(\ker\chi)x^{-1}$ for all $x\in G$. 
    
    Conversely, let $g\in\bigcap_{x\in G}x(\ker\chi)x^{-1}$. Then $g\in x(\ker\chi)x^{-1}$ for all $x\in G$. 
    This implies that $x^{-1}gx\in\ker\chi\subseteq H$ for all $x\in G$. Thus 
    \[
    \chi^0(x^{-1}gx)=\chi(x^{-1}gx)=\chi(1)
    \]
    for all $x\in G$. Summing over all $x\in G$ and dividing by $|H|$, 
    \[
    (\Ind_H^G\chi)(g)=\frac{1}{|H|}\sum_{x\in G}\chi^0(x^{-1}gx)=(G:H)\chi(1)=(\Ind_H^G\chi)(1).
    \]
\end{sol}

\begin{exercise}
\label{xca:kernel_constituent}
    Let $M$ be a maximal subgroup of a finite group $G$, and $\Tchar_G\ne\chi\in\Irr(G)$ be
    a constituent
    of $\Ind_M^G\Tchar_M$. Prove that $\ker\chi=\Core_GM$. 
\end{exercise}

\begin{sol}{xca:kernel_constituent}
    We first prove that $\ker\chi\subseteq\Core_GM$. If not, $G=(\ker\chi)M$ because $M$ is a maximal subgroup of $G$.
    
    We claim that $\Res_M^G\chi\in\Irr(M)$. Let $\rho\colon G\to\GL(V)$ be a representation with character $\chi$.  
    Every $g\in G$ can be written as $g=xm$ for $x\in\ker\chi$ and $m\in M$. 
    Thus $\rho_g=\rho_x\rho_m=\rho_m$ and a subspace $W$ of $V$ such that $\rho|_M(W)\subseteq W$ 
    will also be such that $\rho(W)\subseteq W$. 

    By Frobenius' reciprocity, 
    \[
    0\ne\langle\chi,\Ind_M^G\Tchar_M\rangle=\langle\Res_M^G\chi,\Tchar_M\rangle.
    \]
    Hence $\Res_M^G\chi=\Tchar_M$. This implies that $M\subseteq\ker\chi$ and hence 
    $M=(\ker\chi)M=G$, a contradiction.  Therefore $\ker\chi\subseteq M$ and 
    hence $\ker\chi\subseteq\Core_GM$. 

    For the other inclusion, use Exercise~\ref{xca:kernel_ind}.
\end{sol}

