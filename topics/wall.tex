\section{Project: Wall's theorem}

\subsection{Ore's theorem}

\begin{lemma}
    \label{lem:Ore_step1}
    Let $G$ be a finite group and $M$ be a maximal subgroup with 
    $\Core_GM=\{1\}$. If $H$ is a non-trivial nilpotent 
    normal subgroup of $G$, then $G=HM$ and $H\cap M=\{1\}$. 
\end{lemma}

\begin{proof}
    Since $H$ is normal in $G$, then $HM$ is a subgroup of $G$. Since $M$ 
    is maximal and $M\subseteq HM$, either $HM=M$ or $HM=G$. In the first case, 
    $H\subseteq HM=M$ and for every $x\in G$, 
    \[
    H=xHx^{-1}\subseteq xMx^{-1}.
    \]
    Thus $\{1\}\ne H\subseteq\bigcap_{x\in G}xMx^{-1}=\Core_GM=\{1\}$, a contradiction. 
    Hence $HM=G$. 

    Since $H$ is normal in $G$, $M\subseteq N_G(M\cap H)$. The maximality of $M$ implies that  
    either $N_G(M\cap H)=M$ or $N_G(M\cap H)=G$. If $N_G(M\cap H)=M$, then 
    \[
    N_H(M\cap H)=H\cap N_G(M\cap H)=M\cap H,
    \]
    a contradiction to the nilpotency of $H$. Thus $N_G(M\cap H)=G$ and 
    $M\cap H$ is normal in $G$. For each $x\in G$, 
    \[
    M\cap H=x(M\cap H)x^{-1}\subseteq xMx^{-1}.
    \]
    Thus $M\cap H\subseteq\Core_GM=\{1\}$ and therefore $M\cap H=\{1\}$. 
\end{proof}

\begin{lemma}
\label{lem:Ore_step2}
    Let $G$ be a finite solvable group and $M$ be a maximal subgroup with 
    $\Core_GM=\{1\}$. There exists a unique 
    non-trivial nilpotent normal subgroup $N$ such that 
    $NM=G$ and $N\cap M=\{1\}$.   
\end{lemma}

\begin{proof}
    Let $N$ be a minimal normal subgroup of $G$. Since $G$ is solvable, 
    $N$ is an elementary abelian $p$-group for some prime number $p$. 
    In particular, $N$ is nilpotent. By Lemma~\ref{lem:Ore_step2}, 
    $G=NM$ and $N\cap M=\{1\}$. 

    To prove the uniqueness of $N$, let $N_1$ be a normal
    subgroup of $G$ such that $G=N_1M$ and $N_1\cap M=\{1\}$. Then
    \[
    N=N\cap G=N\cap (N_1M)=N_1(N\cap M)=N_1
    \]
    by Dedekind's lemma. 
\end{proof}

\begin{lemma}
    \label{lem:Ore_step3}
    Let $G$ be a solvable group and $M_1$ and $M_2$ be core-free 
    maximal subgroups of $G$. If $N$ is a minimal normal subgroup of $G$, then
    $M_1$ and $M_2$ are conjugate. 
\end{lemma}

\begin{proof}
    Let $i\in\{1,2\}$. 
    By Lemma~\ref{lem:Ore_step2}, $G=NM_i$ and $N\cap M_i=\{1\}$. If $G=N$, then 
    $M_1=M_2=\{1\}$ and the lemma is proved. Suppose then that $G\ne N$. Since 
    $G$ is solvable, $N$ is an elementary abelian $p$-group for some prime number $p$, 
    say $|N|=p^{\alpha}$. Let $\pi\colon G\to G/N$ be the canonical map. Let $A$ be a normal subgroup of $G$ 
    containing $N$ 
    such that $\pi(A)$ is minimal normal in $G/N$. Since $G/N$ is solvable, $\pi(A)$ is an 
    elementary abelian $q$-group for some prime number $q$, say $|\pi(A)|=q^\beta$, 
    By the correspondence theorem, 
    \begin{equation}
        \label{eq:Ore}
        \frac{|G|}{|N|q^\beta}=(\pi(G):\pi(A))=(G:A)=\frac{|G|}{|A|}=\frac{|G|}{p^\alpha}.
    \end{equation}
    We claim that $p\ne q$. In fact, if $p=q$, then~\eqref{eq:Ore} implies that $|A|=p^{\alpha+\beta}$. In particular, 
    $A$ is non-trivial nilpotent normal subgroup of $G$. 
    By Lemma~\ref{lem:Ore_step1}, $G=AM_i$ and $A\cap M_i=\{1\}$. Hence $A=N$ by Lemma~\ref{lem:Ore_step2}, a contradiction. 
    Therefore $p\ne q$ and $|A|=p^{\alpha}q^\beta$ by~\eqref{eq:Ore}. 

    Since $M_i$ is maximal and $M_i\subseteq M_iA$, either $M_iA=M_i$ or $M_iA=G$. As $M_i$ is core-free, 
    $G=AM_i$ (otherwise, $A\subseteq\Core_GM=\{1\}$). 
    Since $|G|=|NM_i|=|N||M_i|=q^\beta|M_i|$ and 
    \[
    p^\alpha|M_i|=|G|=|AM_i|=\frac{|A||M_i|}{|A\cap M_i|}=\frac{p^\alpha q^\beta|M_i|}{|A\cap M_i|}, 
    \]
    it follows that $|A\cap M_i|=q^\beta$. Thus $A\cap M_i\in\Syl_q(A)$. 
    By the second Sylow's theorem, there exists $a\in A$ such that 
    \[
    a(A\cap M_1)a^{-1}=A\cap M_2.
    \]

    We claim that $aM_1a^{-1}=M_2$. 
    If $aM_1a^{-1}\ne M_2$, then $G=\langle M_2,aM_1a^{-1}\rangle$ by the maximality of $M_2$. Note that  
    $A\cap M_i$ is normal in $M_i$ (because $A$ is normal in $G$). It follows that 
    $A\cap M_2$ is a nilpotent non-trivial 
    normal subgroup of $G$, since for example 
    \begin{align*}
    (am_1a^{-1})(A\cap M_2)(am_1a^{-1})^{-1}
    =am_1(A\cap M_1)m_1^{-1}a^{-1}
    =a(A\cap M_1)a^{-1}
    =A\cap M_2
    \end{align*}
    for all $a\in A$, $m_1\in M_1$ and $x\in A\cap M_2$. By Lemma~\ref{lem:Ore_step2}, 
    $N=A\cap M_2\subseteq M_2$, a contradiction to $\Core_GM_2=\{1\}$. Hence 
    $aM_1a^{-1}=M_2$. 
\end{proof}

Now we are ready to state and prove the theorem. 

\begin{theorem}[Ore]
\label{thm:Ore}
\index{Ore's theorem}
    Let $G$ be a finite solvable group. If $M_1$ and $M_2$ 
    are two maximal subgroups of $G$, then 
    $M_1M_2=G$ or $M_1$ and $M_2$ are conjugate. 
\end{theorem}

\begin{proof}
    We proceed by induction on $|G|$. We divide the proof in three cases. 
    
    Assume first that there exists a minimal normal
    subgroup $N$ of $G$ contained in $M_1\cap M_2$. Let 
    $\pi\colon G\to G/N$ be the canonical map. Since $M_1$ and $M_2$ are both maximal subgroups of $G$ containing $N$, 
    the subgroups $\pi(M_1)$ and $\pi(M_2)$ 
    are maximal in the solvable group $G/N$. Since 
    $|G/N|<|G|$, the inductive hypothesis 
    implies that either $G/N=\pi(M_1)\pi(M_2)$ or
    $\pi(M_1)$ and $\pi(M_2)$ are conjugate. If $G/N=\pi(M_1)\pi(M_2)$, then
    \[
    G=M_1M_2N\subseteq M_1M_2
    \]
    because $N\subseteq M_1\cap M_2$.  
    If $\pi(M_1)$ and 
    $\pi(M_2)$ are conjugate, then 
    \[
    \pi(xM_1x^{-1})=\pi(x)\pi(M_1)\pi(x)^{-1}=\pi(M_2)
    \]
    for some $x\in G$. Thus $xM_1x^{-1}\subseteq M_2N=M_2$ (again, because $N\subseteq M_2$). 
    By the maximality of $M_2$, $xM_1x^{-1}=M_2$.  

    Assume now that there exists a minimal normal subgroup $N$ such that 
    $N\subseteq M_1$ and $N\not\subseteq M_2$. 
    Since $G$ is solvable, $N$ is nilpotent. By Lemma~\ref{lem:Ore_step1}, 
    $G=NM_2$ and $N\cap M_2=\{1\}$. By Dedekind's lemma,
    \[
    M_1=M_1\cap G=M_1\cap (NM_2)=N(M_1\cap M_2).
    \]
    Thus
    \[
    M_1M_2=N(M_1\cap M_2)M_2=NM_2(M_1\cap M_2)=G. 
    \]

    Finally, if neither $M_1$ nor $M_2$ contain a normal subgroup of $G$, 
    then $M_1$ and $M_2$ are conjugate by Lemma~\ref{lem:Ore_step3}. 
\end{proof}

    
%     Since $N$ is normal in $G$, 
%     $LM_2$ is a subgroup of $G$ containing $M_2$. Since 
%     $M_2$ is maximal, either $M_2=LM_2$ or $NM_2=G$. In the first case, 
%     $N\subseteq NM_2=M_2$, a contradiction. Thus $NM_2=G$. Now 
%     $M_2\cap N$ is normal in $N$ and 
%     $M_2\cap N$ is normal in $M_2$ (because $NM_2=G$). Thus 
%     $N\cap L$ is a normal subgroup of $G$ contained in $L$. By the minimality of $L$, 
%     $N\cap L=\{1\}$ (because otherwise $L=N\cap L\subseteq N$, a contradiction). Hence 
%     $G=NL$ with $N\cap L=\{1\}$. By Dedekind's lemma, 
%     \[
%     M=M\cap G=M\cap (NL)=L(M\cap N).
%     \]
%     Thus
%     \[
%     MN=L(M\cap N)N=LN(N\cap M)=G, 
%     \]
%     by Exercise~\ref{xca:AcapB}.

% We state a useful lemma.

% \begin{lemma}
% \label{lem:Ore}
%     Let $G=AB$ for some subgroups $A$ and $B$ of a finite group $G$. Let 
%     $N$ be a normal subgroup of $B$ such that $N\subseteq A\cap B$. 
%     Then there exists a normal subgroup $K$ of $G$ such that 
%     $N\subseteq K\subseteq A$. 
% \end{lemma}

% \begin{proof}
%     Since $G=AB$, the conjugates of $A$ in $G$ are
%     $A=b_1Ab_1^{-1},b_2Ab_2^{-1},...,b_nAb_n^{-1}$ for some 
%     $b_1,\dots,b_n\in B$. Let $K=\bigcap_{i=1}^nb_iAb_i^{-1}$. 
%     Then $K\subseteq A$ and $K$ is normal in $G$. Moreover, 
%     since $N$ is normal in $B$, 
%     \[
%     N=b_iNb_i^{-1}\subseteq b_iAb_i
%     \]
%     for all $i\in\{1,\dots,n\}$. Thus $N\subseteq K$.     
% \end{proof}

% \begin{exercise}
%     \label{xca:Ore_abelian}
%     Let $G=AB$ for some subgroups $A$ and $B$ of a finite group $G$. Assume that $B$ is abelian. If $A\cap B\ne\{1\}$, then $A$ contains a normal subgroup of $G$. 
% \end{exercise}

% \begin{sol}{xca:Ore_abelian}
%     Let $N=A\cap B\ne\{1\}$. Since 
%     $B$ is abelian, $N$ is normal in $B$. 
%     By Lemma~\ref{lem:Ore}, 
%     there exists a normal subgroup $K$ of $G$
%     such that $A\cap B\subseteq K\subseteq A$. 
% \end{sol}

% \begin{lemma}
%     Let $G$ be a finite solvable group and $M$ be a maximal subgroup of $G$ 
%     that does not contain any normal subgroup of $G$. Then 
%     there exists a unique minimal normal subgroup $L$ of $G$
%     such that $G=...$
% \end{lemma}

% \begin{proof}
%     Let $L$ be a minimal normal subgroup of $G$. Since $G$ is solvable, $|L|=p^\alpha$ 
%     for some prime number $p$ and $L$ is elementary abelian. In particular, 
%     $ML$ is a subgroup of $G$ containing $M$. By the maximality of $M$, 
%     $G=ML$ since $L\subsetneq M$. We claim that $L\cap M$ is normal in $G$. 
%     Let $g\in G$ and $x\in L\cap M$. Since $G=ML$, $g=ml$ for some $m\in M$ and 
%     $l\in L$. Since $L$ is abelian, $lxl^{-1}=x$. Moreover,  
%     \[
%     gxg^{-1}=m(lxl^{-1})m^{-1}=mxm^{-1}\in L\cap M
%     \]
%     since $L$ is normal in $G$. 

%     Note that the subgroup $L$ is unique such that $G=ML$ and $M\cap L=\{1\}$. In fact, 
%     if $G=ML_1$ and $L_1\cap M=\{1\}$, then
%     \[
%     L=L\cap G=L\cap (ML_1)=L_1(L\cap M)=L_1
%     \]
%     by Dedenkind's lemma. 

%     Let $\pi\colon G\to G/L$ be the canonical map and $\pi(A)$ 
%     be a minimal normal subgroup of $G/L$. Then $A$ is a normal subgroup of $G$ 
%     such that $L\subsetneq A$. Since $G/L$ is solvable, $\pi(A)$ is an elementary 
%     abelian $q$-group for some prime number $q$, say $|\pi(A)|=q^\beta$. 

%     Note that 
%     \[
%     \frac{|G|}{q^{\beta}p^{\alpha}}=(\pi(G):\pi(A))=(G:A)=\frac{|G|}{|A|}.
%     \]

%     We claim that $p\ne q$. If not, the previous equality implies that $A$ is a $p$-group. Since 
%     $L$ is normal in $A$, $L\cap Z(A)\ne\{1\}$ (because $A$ is a $p$-group). Then 
%     $L\cap Z(A)$ is a non-trivial normal subgroup of $G$ contained in $L$. By the minimality
%     of $L$, $L\cap Z(A)=L$. Hence $L\subseteq Z(A)$. This implies that
%     $A\cap M$ is a normal subgroup of $G$. In fact, if $x\in A\cap M$ and $g\in G$, write
%     $g=ml$ for $m\in M$ and $l\in L$. Then 
%     \[
%     gxg^{-1}=mxm^{-1}\in L\cap M.
%     \]
%     Since $A\cap M$ is a normal subgroup of $G$ contained in $A$, $\pi(A\cap M)$ is a normal 
%     subgroup of $G/L$ contained in $\pi(A)$. The minimality of $\pi(A)$ implies that 
%     either $\pi(A\cap M)=\pi(A)$ or $\pi(A\cap M)=L$. Note that 
%     \[
%     A=A\cap G=A\cap (ML)=L(A\cap M)
%     \]
%     by Dedekind's lemma. 
%     Thus $\pi(A)=\pi(A\cap M)$...
    
%     Now that $p\ne q$, it follows that $|A|=p^{\alpha}q^\beta$. Let $Q\in\Syl_p(A)$. Then 
%     $A=QL$ and $Q\cap L=\{1\}$. By the Frattini argument, 
%     \[
%     G=N_G(Q)A\subseteq N_G(Q)(QL)=N_G(Q)L...
%     \]
%     We claim that $N_G(Q)$ is a maximal subgroup of $G$. 
%     Suppose that $N_G(Q)\subseteq M_1$ for some
%     maximal subgroup $M_1$ of $G$. Then $G=M_1L$ and $M_1\cap L=\{1\}$. Moreover, 
%     \[
%     M_1=M_1\cap G=M_1\cap (N_G(Q)L)=N_G(Q)(M_1\cap L)=N_G(Q)
%     \]
%     by Dedekind's lemma. \framebox{What?}
%     % Now $M_1\cap L$ is a normal subgroup of $G$ \framebox{why?}. Then 
%     % either $M_1\cap L=\{1\}$ or $M_1\cap L=L$. If $M_1\cap L=\{1\}$, then 
%     % $M_1=N_G(Q)$. If $M_1\cap L=L$, then $M_1=N_G(Q)L=G$.  

%     % https://math.stackexchange.com/questions/3411417/if-maximal-subgroups-of-solvable-group-have-equal-cores-then-they-are-conjugate
% \end{proof}

% \begin{theorem}[Ore]
% \label{thm:Ore}
% \index{Ore's theorem}
%     Let $G$ be a finite solvable group. If $M$ and $N$ 
%     are two maximal subgroups of $G$, then 
%     $MN=G$ or $M$ and $N$ are conjugate. 
% \end{theorem}

% \begin{proof}
%     We proceed by induction on $|G|$. We divide the proof in three cases. 
    
%     Assume first that there exists a minimal normal
%     subgroup $L$ of $G$ contained in $M\cap N$. Let 
%     $\pi\colon G\to G/L$ be the canonical map. Since $M$ and $N$ are both maximal subgroups of $G$ containing $L$, 
%     the subgroups $\pi(M)$ and $\pi(N)$ 
%     are maximal in the solvable group $G/L$. Since 
%     $|G/L|<|G|$, the inductive hypothesis 
%     implies that either $G/L=\pi(M)\pi(L)$ or
%     $\pi(M)$ and $\pi(N)$ are conjugate. If $G/L=\pi(M)\pi(N)$, then
%     \[
%     G=MNL\subseteq MN
%     \]
%     because $N\subseteq L=\ker\pi$. If $\pi(M)$ and 
%     $\pi(N)$ are conjugate, then $\pi(x)\pi(M)\pi(x)^{-1}=\pi(N)$ 
%     for some $x\in G$. Thus $xMx^{-1}\subseteq NL=N$. By the maximality of $N$, $xMx^{-1}=N$.  

%     Assume now that 
%     $L\subseteq M$ and $L\not\subseteq N$. Since $L$ is normal in $G$, 
%     $NL$ is a subgroup of $G$ containing $N$. Since 
%     $N$ is maximal, either $N=NL$ or $NL=G$. But in the first case, 
%     $L\subseteq NL=N$, a contradiction. Thus $NL=G$. Now 
%     $N\cap L$ is normal in $L$ and 
%     $N\cap L$ is normal in $N$ (because $NL=G$). Thus 
%     $N\cap L$ is a normal subgroup of $G$ contained in $L$. By the minimality of $L$, 
%     $N\cap L=\{1\}$ (because otherwise $L=N\cap L\subseteq N$, a contradiction). Hence 
%     $G=NL$ with $N\cap L=\{1\}$. By Dedekind's lemma, 
%     \[
%     M=M\cap G=M\cap (NL)=L(M\cap N).
%     \]
%     Thus
%     \[
%     MN=L(M\cap N)N=LN(N\cap M)=G, 
%     \]
%     by Exercise~\ref{xca:AcapB}.

%     Finally, we may assume that neither $M$ nor $N$ contain
%     a normal subgroup of $G$. Let $L$ be a minimal
%     normal subgroup of $G$. Then $L$ is an abelian $p$-group, 
%     say $|L|=p^\alpha$, for some prime number $p$, 
%     $ML=G$ and $L\cap M=\{1\}$ (see Exercise...). Let 
%     $\pi\colon G\to G/L$ be the canonical map 
%     and $\pi(A)$ be a minimal normal subgroup of $G/L$. 
%     Since $G/L$ is solvable, $|\pi(A)|=q^\beta$ for 
%     some prime number $q$. We claim that $p\ne q$. If $p=q$, then 
%     since $(G:A)=(\pi(G):\pi(A))$, it follows
%     that $A$ is a $p$-group. Thus $Z(A)\ne\{1\}$. Since 
%     $Z(A)$ is characteristic in $A$ and $A$ is normal
%     in $G$, $Z(A)$ is normal in $G$. Hence $Z(A)\cap L$ is a non-trivial normal subgroup of $G$ (is non-trivial because $L$ is a $A$ is a $p$-group). By the minimality 
%     of $L$, $Z(A)\cap L=L$. We claim that $A\cap M$ is normal in $G$. 
%     Let $g\in G$ 
%     Then $g=ml$ for some $m\in M$ and $l\in L$. If $a\in A\cap M$,
%     then 
%     \[
%     gag^{-1}=m(lal^{-1})m^{-1}=mam^{-1}\in A\cap M
%     \]
%     because $A$ is normal in $G$ and $L\subseteq Z(A)$. Thus $A\cap M$ is normal in $G$, a contradiction. 

%     Now we now that $p\ne q$. Let $Q\in\Syl_p(A)$. Then $Q\cap L=\{1\}$ since $p\ne q$. Note that since $L$ is a normal subgroup of $G$, $QL$ is a subgroup of $A$. Since 
%     \[
%     \frac{|G|}{|A|}=(G:A)=(\pi(G):\pi(A))=\frac{|G/L|}{q^\beta}=\frac{|G|}{p^{\alpha}q^{\beta}},
%     \]
%     it follows that $|A|=p^{\alpha}q^{\beta}$. Now 
%     $|Q||L|=|A|$ and hence $QL=A$.

% \end{proof}


We now present a character-theoretic proof of a theorem of Wall \cite{MR125156}, 
which bounds the number of maximal subgroups of a finite solvable group. 

\begin{exercise}
    \label{xca:number_maximals}
    Let $G$ be a finite group and $M$ be a maximal subgroup of $G$.
    Prove that $M$ has exactly $(G:M)$ conjugates. 
\end{exercise}

\begin{lemma}
\label{lem:multiplicity-free}
    Let $G$ be a finite solvable group and $M$ be a maximal subgroup of $G$. If $M$ is core-free, then 
    every irreducible constituent of $\Ind_M^G\Tchar_M$ has multiplicity one. 
\end{lemma}

\begin{proof}
    Let $N$ be a minimal normal subgroup of $G$. Then 
    $G=NM$ and $N\cap M=\{1\}$ (see 
    By Lemma~\ref{lem:Ore_step2}). In particular, 
    $|N|=(G:M)$. 

    We claim that 
    $\Res_N^G\Ind_M^G\Tchar_M$ is the regular character of $N$. For 
    $n\in N$, 
    \[
    (\Ind_M^G\Tchar_M)(n)=\frac{1}{|M|}\sum_{x\in G}\Tchar_M^0(x^{-1}nx)
    =\begin{cases}
        (G:M) & \text{if $n=1$,}\\
        0 & \text{otherwise.}
    \end{cases}
    \]
    By Theorem~\ref{thm:regular}, 
    $\Res_N^G\Ind_M^G\Tchar_M$ is the regular character of $N$ and hence 
    \[
    \Res_N^G\Ind_M^G\Tchar_M=\sum_{\lambda\in\Irr(N)}\lambda(1)\lambda.
    \]
    Since $N$ is abelian, $\lambda(1)=1$ for all $\lambda\in\Irr(N)$. Now assume
    that some irreducible constituent $\psi\in\Irr(G)$ of $\Ind_M^G\Tchar_M$ has
    multiplicity $m\geq2$, say $\Ind_M^G\Tchar_M=m\psi+\xi$, where $\langle\psi,\xi\rangle=0$. 
    Then 
    \[
    \sum_{\lambda\in\Irr(N)}\lambda=\Res_N^G\Ind_M^G\Tchar_M=m\Res_N^G\psi+\Res_N^G\xi,
    \]
    a contradiction. 
\end{proof}

\begin{lemma}
\label{lem:kernel}
    Let $G$ be a finite solvable group and $M$ be a maximal subgroup of $G$. If $M$ is core-free, then 
    \[
    \ker\Ind_M^G\Tchar_M=\bigcap\{\ker\chi:\chi\in\Irr(G)\text{ such that }\langle\Ind_M^G\Tchar_M,\chi\rangle\ne0\}.
    \]
    Thus $\ker\Ind_M^G\Tchar_M$ is the intersection of the kernels of the irreducible constituents 
    of $\Ind_M^G\Tchar_M$.
\end{lemma}

\begin{proof}
    By Lemma~\ref{lem:multiplicity-free}, every irreducible constituent of 
    $\Ind_M^G\Tchar_M$ appears with multiplicity one, that is 
    \[
    \Ind_M^G\Tchar_M=\sum_{j=1}^k\chi_j
    \]
    for some subset $\{\chi_1,\dots,\chi_k\}\subseteq\Irr(G)$. If $g\in\ker\chi_1\cap\cdots\cap\ker\chi_k$, then
    $\chi_j(g)=\chi_j(1)$ for all $j\in\{1,\dots,k\}$. Thus 
    \[
    \sum_{j=1}^k\chi_j(g)=(\Ind_M^G\Tchar_M)(g)=(\Ind_M^G\Tchar_M)(1)=\sum_{j=1}^k\chi_j(1). 
    \]
    Conversely, let $g\in\ker\Ind_M^G\Tchar_M$. 
    Then 
    \[
    \sum_{j=1}^k\chi_j(g)=(\Ind_M^G\Tchar_M)(g)=(\Ind_M^G\Tchar_M)(1)=\sum_{j=1}^k\chi_j(1).
    \]
    Assume that there exists 
    $i\in\{1,\dots,k\}$ such that 
    $g\not\in\ker\chi_i$. Then 
    $|\chi_i(g)|<\chi_i(1)$ and hence 
    \[
    \sum_{j=1}^k\chi_j(1)=\left|\sum_{j=1}^k\chi(g)\right|\leq
    \sum_{j=1}^k|\chi_j(g)|<\sum_{j=1}^k\chi_j(1),
    \]
    a contradiction. 
\end{proof}

\begin{lemma}
    Let $G$ be a finite solvable group and $M$ be a  
    maximal subgroup of $G$. If $M$ is not normal in $G$, 
    then $\Tchar_G$ is the only degree-one constituent of $\Ind_M^G\Tchar_M$. 
\end{lemma}

\begin{proof}
    Let $\psi\in\Irr(G)$ be a degree-one constituent of $\Ind_M^G\Tchar_M$. 
    By Frobenius' reciprocity, $0\ne\langle\Ind_M^G\Tchar_M,\psi\rangle=\langle\Tchar_M,\Res_M^G\psi\rangle$.
    Since $\Res_M^G\psi$ is a degree-one character, it follows the irreducibility that 
    $\Tchar_M=\Res_M^G\psi$. In particular, 
    \[
    M=\ker\Tchar_M=\ker(\Res_M^G\psi)=M\cap\ker\psi\subseteq\ker\psi. 
    \]
    
    Assume now that $\psi\ne\Tchar_G$. Then 
    \[
    M\subseteq \ker\psi=\Core_GM\subsetneq M,
    \]
    a contradiction. Hence $\psi=\Tchar_G$. 
\end{proof}

\begin{theorem}[Wall]
\index{Wall theorem}
\label{thm:Wall}
Let $G$ be a finite solvable group. Then the number of maximal subgroups of $G$ is at most $|G|-1$. 
\end{theorem}

\begin{proof}
    For a maximal subgroup $M$ of $G$, let 
    $C(M)$ be the set of non-trivial constituents
    of $\Ind_M^G\Tchar_M$, that is
    \[
    C(M)=\{\chi\in\Irr(G):\chi\ne\Tchar_G\text{ and }\langle\eta,\Ind_M^G\Tchar_M\rangle\ne0\}.
    \]

    Let $M_1$ and $M_2$ be maximal subgroups of $G$. 
    We claim that $C(M_1)\cap C(M_2)=\emptyset$ if $M_1$ and $M_2$ are
    not conjugate. 
    
    Assume that $M_1$ and $M_2$ are not conjugate. By  
    Ore's theorem~\ref{thm:Ore}, $G=M_1M_2$. In particular, 
    there is only one double $(M_1,M_2)$-coset, with representative $1$. Thus
    \[
    \Res_{M_1}^G\Ind_{M_2}^G\Tchar_{M_2}=\Ind_{M_1\cap M_2}^{M_1}\Res_{M_1\cap M_2}^{M_2}\Tchar_{M_2}
    =\Ind_{M_1\cap M_2}^{M_1}\Tchar_{M_1\cap M_2}.
    \]
    Using Frobenius' reciprocity (Theorem~\ref{thm:reciprocity}) and 
    Mackey's theorem~\ref{thm:Mackey}, 
    \begin{align*}
    \langle\Ind_{M_1}^G\Tchar_{M_1},\Ind_{M_2}^G\Tchar_{M_2}\rangle
    &=\langle \Tchar_{M_1},\Res_{M_1}^G\Ind_{M_2}^G\Tchar_{M_2}\rangle\\
    &=\langle \Tchar_{M_1},\Ind_{M_1\cap M_2}^{M_1}\Tchar_{M_1\cap M_2}\rangle\\
    &=\langle \Res_{M_1\cap M_2}^{M_1}\Tchar_{M_1},\Tchar_{M_1\cap M_2}\rangle\\
    &=\langle \Tchar_{M_1\cap M_2},\Tchar_{M_1\cap M_2}\rangle\\
    &=1.
    \end{align*}
    For $i\in\{1,2\}$, by Frobenius' reciprocity, 
    \[
    \langle\Ind_{M_i}^G\Tchar_{M_i},\Tchar_G\rangle
    =\langle\Tchar_{M_i},\Res_{M_i}^G\Tchar_G\rangle=\langle \Tchar_{M_i},\Tchar_{M_i}\rangle=1.
    \]
    Thus $\Tchar_{G}$ is an irreducible constituent of both 
    $\Ind_{M_1}^G\Tchar_{M_1}$ and $\Ind_{M_2}^G\Tchar_{M_2}$. Since 
    \[
    \langle\Ind_{M_1}^G\Tchar_{M_1},\Ind_{M_2}^G\Tchar_{M_2}\rangle=1,
    \]
    the set of non-trivial 
    irreducible
    constituents of $\Ind_{M_1}^G\Tchar_{M_1}$ and $\Ind_{M_2}^G\Tchar_{M_2}$ are disjoint, that~is  
    \[
    \Ind_{M_1}^G\Tchar_{M_1}=\Tchar_G+\eta_1,
    \quad 
    \Ind_{M_2}^G\Tchar_{M_2}=\Tchar_G+\eta_2,
    \quad 
    \langle\eta_1,\eta_2\rangle=0.
    \]
    Hence $C(M_1)\cap C(M_2)=\emptyset$. 

    
   Let $X$ be the set of maximal subgroups of $G$ that are normal in $G$, and 
   let $Y$ be the set of representatives of non-normal 
   maximal subgroups of $G$. By Exercise~\ref{xca:number_maximals}, 
   the number of 
   maximal subgroups of $G$ is 
   \[
   m=|X|+\sum_{M\in Y}(G:M).
   \]

   For every maximal subgroup $M$ of $G$ such that $M$ is normal in $G$, 
   we know that $\Ind_M^G\Tchar_M$ decomposes 
   as $\Ind_M^G\Tchar_M=\Tchar_G+\eta_1+\cdots+\eta_k$ for some  
   $\eta_1,\dots,\eta_k\in\Irr(G)\setminus\{\Tchar_G\}$ such that 
   $\eta_j(1)=1$ for all $j\in\{1,\dots,k\}$. Then
   \[
   (G:M)-1=\eta_1(1)+\cdots+\eta_k(1)=\sum_{i=1}^k\eta_i(1)^2
   \]
   since $\eta_j(1)=1$ for all $j$. Since $p$ is the smallest prime divisor of $|G|$, 
   it follows that 
   \[
   \sum_{\eta\in C(M)}\eta(1)^2
   =(G:M)-1\geq p-1.
   \]

   For every maximal subgroup $M$ of $G$ such that $M$ is not normal in $G$, 
   $\Ind_M^G\Tchar_M$ decomposes 
   as $\Ind_M^G\Tchar_M=\Tchar_G+\xi_1+\cdots+\xi_l$ for some distinct characters 
   $\xi_1,\dots,\xi_k\in\Irr(G)$ 
   such that $\xi_j(1)\geq p$ for all $j\in\{1,\dots,k\}$. 
   Then 
   \[
   \sum_{\xi\in C(M)}\xi(1)^2
   \geq p\sum_{\xi\in C(M)}\xi(1)
   =p((G:M)-1)
   \geq (p-1)(G:M).
   \]

   Now 
   \begin{align*}
    |G|-1 &=\sum_{\Tchar_G\ne\chi\in\Irr(G)}\chi(1)^2
    \geq\sum_{M\in X}\sum_{\eta\in C(X)}\eta(1)^2
    +\sum_{M\in Y}\sum_{\xi\in C(X)}\xi(1)^2\\
    &\geq (p-1)|X|+(p-1)\sum_{M\in Y}(G:M)\\
    &=(p-1)\left(|X|+\sum_{M\in Y}(G:M)\right)=(p-1)m.\qedhere 
   \end{align*}
\end{proof}

\begin{exercise}
    Prove that a finite solvable group has exactly $|G|-1$ maximal subgroups
    if and only if it is an elementary abelian $2$-group. 
\end{exercise}

\begin{example}
    Let $G=\Sym_3$. 
    Recall from Table~\ref{tab:S3} that $\Irr(G)=\{\Tchar_G,\sgn,\chi\}$, where $\sgn$ is the sign representation
    and $\chi\colon G\to\C^{\times}$ is given by
    \[
    \chi(g)=\begin{cases}
        2 & \text{if $g=\id$},\\
        0 & \text{if $g\in\{(12),(13),(23)\}$},\\
        -1 & \text{if $g\in\{(123),(132)\}.$}
    \end{cases}
    \]

    The group $G$ has two conjugacy classes of maximal subgroups, namely 
    \[
    \{\langle (123)\rangle\}\text{ and }\{\langle(12)\rangle,\langle(23)\rangle,\langle(13)\rangle\}.
    \]

    As the group $G$ is rather small, this can be easily verified 
    by a direct calculation. In any case, here is the Magma code:
    \begin{lstlisting}
> S3 := Sym(3);
> max := MaximalSubgroups(S3);
> max;
Conjugacy classes of subgroups
------------------------------

[1]     Order 2            Length 3
        Permutation group M acting on a set of cardinality 3
        Order = 2
            (2, 3)
[2]     Order 3            Length 1
        Permutation group N acting on a set of cardinality 3
        Order = 3
            (1, 2, 3)
    \end{lstlisting}

    Let $M=\langle (23)\rangle$ and $N=\langle (123)\rangle$. Thus $X=\{N\}$ and 
    $Y=\{M\}$. 
    
    Let us compute $C(N)$. For that purpose, let $t_1=1$ 
    and $t_2=(12)$ be a transversal of $N$ in $G$. Then 
    \[
    (\Ind_{N}^G\Tchar_N)(g)=\Tchar_N^0(g)+\Tchar_N^0((12)g(12))=\begin{cases}
        2 & \text{if $g=\id$},\\
        0 & \text{if $g\in\{(12),(23),(13)\}$},\\
        2 & \text{otherwise.}
    \end{cases}
    \]
    Thus $\Ind_{N}^G\Tchar_N=\Tchar_G+\sgn$ and $C(N)=\{\sgn\}$. Here is the Magma code: 
\begin{lstlisting}
> N := max[2]`subgroup;
> g := Character(TrivialRepresentation(N));
> ind_N := Induction(g, S3);
> ind_N;
( 2, 0, 2 )
\end{lstlisting}    

To decompose our induced character, with Magma we proceed as follows:
\begin{lstlisting}
> T := CharacterTable(S3);
> Decomposition(T, ind_N);
[
    1,
    1,
    0
]
( 0, 0, 0 )
> InnerProduct(T[3], ind_N);
0
> InnerProduct(T[2], ind_N);
1
> InnerProduct(T[1], ind_N);
1
\end{lstlisting}


    Let us now compute $C(M)$. A direct calculation shows that 
    \[
    (\Ind_{M}^G\Tchar_{M})(g)=\begin{cases}
        3 & \text{if $g=\id$},\\
        1 & \text{if $g\in\{(12),(23),(13)\}$},\\
        0 & \text{otherwise.}
    \end{cases}
    \]
    Thus $\Ind_{M}\Tchar_{M}=\Tchar_G+\chi$ and $C(M)=\{\chi\}$. 
    We leave it as an exercise to verify these calculations, either by 
    hand, with Magma, or perhaps both.
\end{example}

\begin{example}
    Let $G=\Alt_4$. There are two conjugacy classes of maximal subgroups of $G$ with representatives
    are $M=\langle (234)\rangle$ and $N=\{\id,(12)(34),(13)(24),(14)(23)\}$. Then 
    $X=\{N\}$ and $Y=\{M\}$. 

    In Exercise~\ref{xca:A4}, we asked for the construction of the character table of \( \Alt_4 \). The completed table is shown in Table~\ref{tab:A4}.

    \index{Character table!of $\Alt_4$}
    \begin{table}[h]
        \centering\makegapedcells
        \caption{The Character table of $\Alt_4$.}
        \label{tab:A4}
        \begin{tabular}{|c|cccc|}
             \hline
             & $\id$ & $(12)(34)$ & $(123)$ & $(132)$\\
             \hline
             $\chi_1$ & $1$ & $1$ & $1$ & $1$\\
             $\chi_2$ & $1$ & $1$ & $\frac{-1+\sqrt{-3}}{2}$ & $\frac{-1-\sqrt{-3}}{2}$\\
             $\chi_3$ & $1$ & $1$ & $\frac{-1-\sqrt{-3}}{2}$ & $\frac{-1+\sqrt{-3}}{2}$\\
             $\chi_4$ & $3$ & $-1$ & $0$ & $0$\\
             \hline
        \end{tabular}
    \end{table}

    
    A direct calculation shows that
    $\Ind_N^G\Tchar_N=\Tchar_G+\chi_2+\chi_3$ and 
    $\Ind_M^G\Tchar_M=\Tchar_G+\chi_4$. 
\end{example}
% > A4 := Alt(4);
% > max := MaximalSubgroups(A4);
% > #max;
% 2
% > max;
% Conjugacy classes of subgroups
% ------------------------------

% [1]     Order 3            Length 4
%         Permutation group acting on a set of cardinality 4
%         Order = 3
%             (2, 3, 4)
% [2]     Order 4            Length 1
%         Permutation group acting on a set of cardinality 4
%         Order = 4 = 2^2
%             (1, 3)(2, 4)
%             (1, 2)(3, 4)
% > M := max[1]`subgroup;
% > f := Character(TrivialRepresentation(M));
% > f;
% ( 1, 1, 1 )
% > ind_M := Induction(f, A4);
% > ind_M;
% ( 4, 0, 1, 1 )
% > T := CharacterTable(A4);
% > Decomposition(T, ind_M);
% [
%     1,
%     0,
%     0,
%     1
% ]
% ( 0, 0, 0, 0 )
% > T[4];
% ( 3, -1, 0, 0 )
% > T[1];
% ( 1, 1, 1, 1 )
The ideas used here to prove Theorem~\ref{thm:Wall} can be applied 
to obtain the following variant of Wall’s theorem, established  
by Cook, Wiegold, and Williamson in \cite{MR896628}.

\begin{theorem}[Cook--Wiegold--Williamson]
    \label{thm:CookWiegold-Williamson}
    \index{Cook--Wiegold--Williamson theorem}
    Let $G$ be a finite solvable group and $p$ the smallest prime divisor of $|G|$. 
    Then the number of maximal subgroups of $G$ is at most $\frac{|G|-1}{p-1}$. Equality 
    holds if and only if $G$ is an elementary $p$-group.  
\end{theorem}

\begin{bonus}
    Prove Theorem~\ref{thm:CookWiegold-Williamson}.    
\end{bonus}

\begin{exercise}
    \label{xca:kernel_ind}
    Let $H$ be a subgroup of a finite group $G$ and 
    $\chi\in\Char(H)$. Prove that 
    $\ker\Ind_H^G\chi=\bigcap_{x\in G}x(\ker\chi)x^{-1}$. 
\end{exercise}

\begin{sol}{xca:kernel_ind}
    Let $g\in\ker\Ind_H^G\chi$. Then \[
    \sum_{x\in G}\chi^0(x^{-1}gx)=\sum_{x\in G}\chi(1).
    \]
    Then $\chi^0(x^{-1}gx)=\chi(1)$ for all $x\in G$. (Otherwise, $|\chi^0(y^{-1}gy)|<\chi(1)$ for some
    $y\in G$ and hence 
    \[
    \sum_{x\in G}\chi(1)=\left|\sum_{x\in G}|\chi^0(x^{-1}gx)\right|
    \leq\sum_{x\in G}|\chi^0(x^{-1}gx)|<\sum_{x\in G}\chi(1),
    \]
    a contradiction.) In particular, $x^{-1}gx\in\ker\chi\subseteq H$ for all $x\in G$, that 
    is $g\in x(\ker\chi)x^{-1}$ for all $x\in G$. 
    
    Conversely, let $g\in\bigcap_{x\in G}x(\ker\chi)x^{-1}$. Then $g\in x(\ker\chi)x^{-1}$ for all $x\in G$. 
    This implies that $x^{-1}gx\in\ker\chi\subseteq H$ for all $x\in G$. Thus 
    \[
    \chi^0(x^{-1}gx)=\chi(x^{-1}gx)=\chi(1)
    \]
    for all $x\in G$. Summing over all $x\in G$ and dividing by $|H|$, 
    \[
    (\Ind_H^G\chi)(g)=\frac{1}{|H|}\sum_{x\in G}\chi^0(x^{-1}gx)=(G:H)\chi(1)=(\Ind_H^G\chi)(1).
    \]
\end{sol}

\begin{exercise}
\label{xca:kernel_constituent}
    Let $M$ be a maximal subgroup of a finite group $G$, and $\Tchar_G\ne\chi\in\Irr(G)$ be
    a constituent
    of $\Ind_M^G\Tchar_M$. Prove that $\ker\chi=\Core_GM$. 
\end{exercise}

\begin{sol}{xca:kernel_constituent}
    We first prove that $\ker\chi\subseteq\Core_GM$. If not, $G=(\ker\chi)M$ because $M$ is a maximal subgroup of $G$.
    
    We claim that $\Res_M^G\chi\in\Irr(M)$. Let $\rho\colon G\to\GL(V)$ be a representation with character $\chi$.  
    Every $g\in G$ can be written as $g=xm$ for $x\in\ker\chi$ and $m\in M$. 
    Thus $\rho_g=\rho_x\rho_m=\rho_m$ and a subspace $W$ of $V$ such that $\rho|_M(W)\subseteq W$ 
    will also be such that $\rho(W)\subseteq W$. 

    By Frobenius' reciprocity, 
    \[
    0\ne\langle\chi,\Ind_M^G\Tchar_M\rangle=\langle\Res_M^G\chi,\Tchar_M\rangle.
    \]
    Hence $\Res_M^G\chi=\Tchar_M$. This implies that $M\subseteq\ker\chi$ and hence 
    $M=(\ker\chi)M=G$, a contradiction.  Therefore $\ker\chi\subseteq M$ and 
    hence $\ker\chi\subseteq\Core_GM$. 

    For the other inclusion, use Exercise~\ref{xca:kernel_ind}.
\end{sol}

