\section{Project: Ore's theorem}

% \begin{exercise}
% \label{xca:AcapB}
%     Let $A$ and $B$ be permutable subgroups of $G$. 
%     Then \[
%     A(A\cap B)=(B\cap A)A.
%     \]
% \end{exercise}

% \begin{sol}{xca:AcapB}
%     If $x=ay$ for some $a\in A$ and $y\in A\cap B$, then 
%     $x=y(y^{-1}ay)\in (A\cap B)A$. 
%     Conversely, if $x=ya$ for some $y\in A\cap B$ and $a\in A$, then $x=(yay^{-1})y\in A(A\cap B)$. 
% \end{sol}

\begin{lemma}
    \label{lem:Ore_step1}
    Let $G$ be a finite group and $M$ be a maximal subgroup with 
    $\Core_GM=\{1\}$. If $H$ is a non-trivial nilpotent 
    normal subgroup of $G$, then $G=HM$ and $H\cap M=\{1\}$. 
\end{lemma}

\begin{proof}
    Since $H$ is normal in $G$, then $HM$ is a subgroup of $G$. Since $M$ 
    is maximal and $M\subseteq HM$, either $HM=M$ or $HM=G$. In the first case, 
    $H\subseteq HM=M$ and for every $x\in G$, 
    \[
    H=xHx^{-1}\subseteq xMx^{-1}.
    \]
    Thus $\{1\}\ne H\subseteq\bigcap_{x\in G}xMx^{-1}=\Core_GM=\{1\}$, a contradiction. 
    Hence $HM=G$. 

    Since $H$ is normal in $G$, $M\subseteq N_G(M\cap H)$. The maximality of $M$ implies that  
    either $N_G(M\cap H)=M$ or $N_G(M\cap H)=G$. If $N_G(M\cap H)=M$, then 
    \[
    N_H(M\cap H)=H\cap N_G(M\cap H)=M\cap H,
    \]
    a contradiction to the nilpotency of $H$. Thus $N_G(M\cap H)=G$ and 
    $M\cap H$ is normal in $G$. For each $x\in G$, 
    \[
    M\cap H=x(M\cap H)x^{-1}\subseteq xMx^{-1}.
    \]
    Thus $M\cap H\subseteq\Core_GM=\{1\}$ and therefore $M\cap H=\{1\}$. 
\end{proof}

\begin{lemma}
\label{lem:Ore_step2}
    Let $G$ be a finite solvable group and $M$ be a maximal subgroup with 
    $\Core_GM=\{1\}$. There exists a unique 
    non-trivial nilpotent normal subgroup $N$ such that 
    $NM=G$ and $N\cap M=\{1\}$.   
\end{lemma}

\begin{proof}
    Let $N$ be a minimal normal subgroup of $G$. Since $G$ is solvable, 
    $N$ is an elementary abelian $p$-group for some prime number $p$. 
    In particular, $N$ is nilpotent. By Lemma~\ref{lem:Ore_step2}, 
    $G=NM$ and $N\cap M=\{1\}$. 

    To prove the uniqueness of $N$, let $N_1$ be a normal
    subgroup of $G$ such that $G=N_1M$ and $N_1\cap M=\{1\}$. Then
    \[
    N=N\cap G=N\cap (N_1M)=N_1(N\cap M)=N_1
    \]
    by Dedekind's lemma. 
\end{proof}

\begin{lemma}
    \label{lem:Ore_step3}
    Let $G$ be a solvable group and $M_1$ and $M_2$ be core-free 
    maximal subgroups of $G$. If $N$ is a minimal normal subgroup of $G$, then
    $M_1$ and $M_2$ are conjugate. 
\end{lemma}

\begin{proof}
    Let $i\in\{1,2\}$. 
    By Lemma~\ref{lem:Ore_step2}, $G=NM_i$ and $N\cap M_i=\{1\}$. If $G=N$, then 
    $M_1=M_2=\{1\}$ and the lemma is proved. Suppose then that $G\ne N$. Since 
    $G$ is solvable, $N$ is an elementary abelian $p$-group for some prime number $p$, 
    say $|N|=p^{\alpha}$. Let $\pi\colon G\to G/N$ be the canonical map. Let $A$ be a normal subgroup of $G$ 
    containing $N$ 
    such that $\pi(A)$ is minimal normal in $G/N$. Since $G/N$ is solvable, $\pi(A)$ is an 
    elementary abelian $q$-group for some prime number $q$, say $|\pi(A)|=q^\beta$, 
    By the correspondence theorem, 
    \begin{equation}
        \label{eq:Ore}
        \frac{|G|}{|N|q^\beta}=(\pi(G):\pi(A))=(G:A)=\frac{|G|}{|A|}=\frac{|G|}{p^\alpha}.
    \end{equation}
    We claim that $p\ne q$. In fact, if $p=q$, then~\eqref{eq:Ore} implies that $|A|=p^{\alpha+\beta}$. In particular, 
    $A$ is non-trivial nilpotent normal subgroup of $G$. 
    By Lemma~\ref{lem:Ore_step1}, $G=AM_i$ and $A\cap M_i=\{1\}$. Hence $A=N$ by Lemma~\ref{lem:Ore_step2}, a contradiction. 
    Therefore $p\ne q$ and $|A|=p^{\alpha}q^\beta$ by~\eqref{eq:Ore}. 

    Since $M_i$ is maximal and $M_i\subseteq M_iA$, either $M_iA=M_i$ or $M_iA=G$. As $M_i$ is core-free, 
    $G=AM_i$ (otherwise, $A\subseteq\Core_GM=\{1\}$). 
    Since $|G|=|NM_i|=|N||M_i|=q^\beta|M_i|$ and 
    \[
    p^\alpha|M_i|=|G|=|AM_i|=\frac{|A||M_i|}{|A\cap M_i|}=\frac{p^\alpha q^\beta|M_i|}{|A\cap M_i|}, 
    \]
    it follows that $|A\cap M_i|=q^\beta$. Thus $A\cap M_i\in\Syl_q(A)$. 
    By the second Sylow's theorem, there exists $a\in A$ such that 
    \[
    a(A\cap M_1)a^{-1}=A\cap M_2.
    \]

    We claim that $aM_1a^{-1}=M_2$. 
    If $aM_1a^{-1}\ne M_2$, then $G=\langle M_2,aM_1a^{-1}\rangle$ by the maximality of $M_2$. Note that  
    $A\cap M_i$ is normal in $M_i$ (because $A$ is normal in $G$). It follows that 
    $A\cap M_2$ is a nilpotent non-trivial 
    normal subgroup of $G$, since for example 
    \begin{align*}
    (am_1a^{-1})(A\cap M_2)(am_1a^{-1})^{-1}
    =am_1(A\cap M_1)m_1^{-1}a^{-1}
    =a(A\cap M_1)a^{-1}
    =A\cap M_2
    \end{align*}
    for all $a\in A$, $m_1\in M_1$ and $x\in A\cap M_2$. By Lemma~\ref{lem:Ore_step2}, 
    $N=A\cap M_2\subseteq M_2$, a contradiction to $\Core_GM_2=\{1\}$. Hence 
    $aM_1a^{-1}=M_2$. 
\end{proof}

Now we are ready to state and prove the theorem. 

\begin{theorem}[Ore]
\label{thm:Ore}
\index{Ore's theorem}
    Let $G$ be a finite solvable group. If $M_1$ and $M_2$ 
    are two maximal subgroups of $G$, then 
    $M_1M_2=G$ or $M_1$ and $M_2$ are conjugate. 
\end{theorem}

\begin{proof}
    We proceed by induction on $|G|$. We divide the proof in three cases. 
    
    Assume first that there exists a minimal normal
    subgroup $N$ of $G$ contained in $M_1\cap M_2$. Let 
    $\pi\colon G\to G/N$ be the canonical map. Since $M_1$ and $M_2$ are both maximal subgroups of $G$ containing $N$, 
    the subgroups $\pi(M_1)$ and $\pi(M_2)$ 
    are maximal in the solvable group $G/N$. Since 
    $|G/N|<|G|$, the inductive hypothesis 
    implies that either $G/N=\pi(M_1)\pi(M_2)$ or
    $\pi(M_1)$ and $\pi(M_2)$ are conjugate. If $G/N=\pi(M_1)\pi(M_2)$, then
    \[
    G=M_1M_2N\subseteq M_1M_2
    \]
    because $N\subseteq M_1\cap M_2$.  
    If $\pi(M_1)$ and 
    $\pi(M_2)$ are conjugate, then 
    \[
    \pi(xM_1x^{-1})=\pi(x)\pi(M_1)\pi(x)^{-1}=\pi(M_2)
    \]
    for some $x\in G$. Thus $xM_1x^{-1}\subseteq M_2N=M_2$ (again, because $N\subseteq M_2$). 
    By the maximality of $M_2$, $xM_1x^{-1}=M_2$.  

    Assume now that there exists a minimal normal subgroup $N$ such that 
    $N\subseteq M_1$ and $N\not\subseteq M_2$. 
    Since $G$ is solvable, $N$ is nilpotent. By Lemma~\ref{lem:Ore_step1}, 
    $G=NM_2$ and $N\cap M_2=\{1\}$. By Dedekind's lemma,
    \[
    M_1=M_1\cap G=M_1\cap (NM_2)=N(M_1\cap M_2).
    \]
    Thus
    \[
    M_1M_2=N(M_1\cap M_2)M_2=NM_2(M_1\cap M_2)=G. 
    \]

    Finally, if neither $M_1$ nor $M_2$ contain a normal subgroup of $G$, 
    then $M_1$ and $M_2$ are conjugate by Lemma~\ref{lem:Ore_step3}. 
\end{proof}

    
%     Since $N$ is normal in $G$, 
%     $LM_2$ is a subgroup of $G$ containing $M_2$. Since 
%     $M_2$ is maximal, either $M_2=LM_2$ or $NM_2=G$. In the first case, 
%     $N\subseteq NM_2=M_2$, a contradiction. Thus $NM_2=G$. Now 
%     $M_2\cap N$ is normal in $N$ and 
%     $M_2\cap N$ is normal in $M_2$ (because $NM_2=G$). Thus 
%     $N\cap L$ is a normal subgroup of $G$ contained in $L$. By the minimality of $L$, 
%     $N\cap L=\{1\}$ (because otherwise $L=N\cap L\subseteq N$, a contradiction). Hence 
%     $G=NL$ with $N\cap L=\{1\}$. By Dedekind's lemma, 
%     \[
%     M=M\cap G=M\cap (NL)=L(M\cap N).
%     \]
%     Thus
%     \[
%     MN=L(M\cap N)N=LN(N\cap M)=G, 
%     \]
%     by Exercise~\ref{xca:AcapB}.

% We state a useful lemma.

% \begin{lemma}
% \label{lem:Ore}
%     Let $G=AB$ for some subgroups $A$ and $B$ of a finite group $G$. Let 
%     $N$ be a normal subgroup of $B$ such that $N\subseteq A\cap B$. 
%     Then there exists a normal subgroup $K$ of $G$ such that 
%     $N\subseteq K\subseteq A$. 
% \end{lemma}

% \begin{proof}
%     Since $G=AB$, the conjugates of $A$ in $G$ are
%     $A=b_1Ab_1^{-1},b_2Ab_2^{-1},...,b_nAb_n^{-1}$ for some 
%     $b_1,\dots,b_n\in B$. Let $K=\bigcap_{i=1}^nb_iAb_i^{-1}$. 
%     Then $K\subseteq A$ and $K$ is normal in $G$. Moreover, 
%     since $N$ is normal in $B$, 
%     \[
%     N=b_iNb_i^{-1}\subseteq b_iAb_i
%     \]
%     for all $i\in\{1,\dots,n\}$. Thus $N\subseteq K$.     
% \end{proof}

% \begin{exercise}
%     \label{xca:Ore_abelian}
%     Let $G=AB$ for some subgroups $A$ and $B$ of a finite group $G$. Assume that $B$ is abelian. If $A\cap B\ne\{1\}$, then $A$ contains a normal subgroup of $G$. 
% \end{exercise}

% \begin{sol}{xca:Ore_abelian}
%     Let $N=A\cap B\ne\{1\}$. Since 
%     $B$ is abelian, $N$ is normal in $B$. 
%     By Lemma~\ref{lem:Ore}, 
%     there exists a normal subgroup $K$ of $G$
%     such that $A\cap B\subseteq K\subseteq A$. 
% \end{sol}

% \begin{lemma}
%     Let $G$ be a finite solvable group and $M$ be a maximal subgroup of $G$ 
%     that does not contain any normal subgroup of $G$. Then 
%     there exists a unique minimal normal subgroup $L$ of $G$
%     such that $G=...$
% \end{lemma}

% \begin{proof}
%     Let $L$ be a minimal normal subgroup of $G$. Since $G$ is solvable, $|L|=p^\alpha$ 
%     for some prime number $p$ and $L$ is elementary abelian. In particular, 
%     $ML$ is a subgroup of $G$ containing $M$. By the maximality of $M$, 
%     $G=ML$ since $L\subsetneq M$. We claim that $L\cap M$ is normal in $G$. 
%     Let $g\in G$ and $x\in L\cap M$. Since $G=ML$, $g=ml$ for some $m\in M$ and 
%     $l\in L$. Since $L$ is abelian, $lxl^{-1}=x$. Moreover,  
%     \[
%     gxg^{-1}=m(lxl^{-1})m^{-1}=mxm^{-1}\in L\cap M
%     \]
%     since $L$ is normal in $G$. 

%     Note that the subgroup $L$ is unique such that $G=ML$ and $M\cap L=\{1\}$. In fact, 
%     if $G=ML_1$ and $L_1\cap M=\{1\}$, then
%     \[
%     L=L\cap G=L\cap (ML_1)=L_1(L\cap M)=L_1
%     \]
%     by Dedenkind's lemma. 

%     Let $\pi\colon G\to G/L$ be the canonical map and $\pi(A)$ 
%     be a minimal normal subgroup of $G/L$. Then $A$ is a normal subgroup of $G$ 
%     such that $L\subsetneq A$. Since $G/L$ is solvable, $\pi(A)$ is an elementary 
%     abelian $q$-group for some prime number $q$, say $|\pi(A)|=q^\beta$. 

%     Note that 
%     \[
%     \frac{|G|}{q^{\beta}p^{\alpha}}=(\pi(G):\pi(A))=(G:A)=\frac{|G|}{|A|}.
%     \]

%     We claim that $p\ne q$. If not, the previous equality implies that $A$ is a $p$-group. Since 
%     $L$ is normal in $A$, $L\cap Z(A)\ne\{1\}$ (because $A$ is a $p$-group). Then 
%     $L\cap Z(A)$ is a non-trivial normal subgroup of $G$ contained in $L$. By the minimality
%     of $L$, $L\cap Z(A)=L$. Hence $L\subseteq Z(A)$. This implies that
%     $A\cap M$ is a normal subgroup of $G$. In fact, if $x\in A\cap M$ and $g\in G$, write
%     $g=ml$ for $m\in M$ and $l\in L$. Then 
%     \[
%     gxg^{-1}=mxm^{-1}\in L\cap M.
%     \]
%     Since $A\cap M$ is a normal subgroup of $G$ contained in $A$, $\pi(A\cap M)$ is a normal 
%     subgroup of $G/L$ contained in $\pi(A)$. The minimality of $\pi(A)$ implies that 
%     either $\pi(A\cap M)=\pi(A)$ or $\pi(A\cap M)=L$. Note that 
%     \[
%     A=A\cap G=A\cap (ML)=L(A\cap M)
%     \]
%     by Dedekind's lemma. 
%     Thus $\pi(A)=\pi(A\cap M)$...
    
%     Now that $p\ne q$, it follows that $|A|=p^{\alpha}q^\beta$. Let $Q\in\Syl_p(A)$. Then 
%     $A=QL$ and $Q\cap L=\{1\}$. By the Frattini argument, 
%     \[
%     G=N_G(Q)A\subseteq N_G(Q)(QL)=N_G(Q)L...
%     \]
%     We claim that $N_G(Q)$ is a maximal subgroup of $G$. 
%     Suppose that $N_G(Q)\subseteq M_1$ for some
%     maximal subgroup $M_1$ of $G$. Then $G=M_1L$ and $M_1\cap L=\{1\}$. Moreover, 
%     \[
%     M_1=M_1\cap G=M_1\cap (N_G(Q)L)=N_G(Q)(M_1\cap L)=N_G(Q)
%     \]
%     by Dedekind's lemma. \framebox{What?}
%     % Now $M_1\cap L$ is a normal subgroup of $G$ \framebox{why?}. Then 
%     % either $M_1\cap L=\{1\}$ or $M_1\cap L=L$. If $M_1\cap L=\{1\}$, then 
%     % $M_1=N_G(Q)$. If $M_1\cap L=L$, then $M_1=N_G(Q)L=G$.  

%     % https://math.stackexchange.com/questions/3411417/if-maximal-subgroups-of-solvable-group-have-equal-cores-then-they-are-conjugate
% \end{proof}

% \begin{theorem}[Ore]
% \label{thm:Ore}
% \index{Ore's theorem}
%     Let $G$ be a finite solvable group. If $M$ and $N$ 
%     are two maximal subgroups of $G$, then 
%     $MN=G$ or $M$ and $N$ are conjugate. 
% \end{theorem}

% \begin{proof}
%     We proceed by induction on $|G|$. We divide the proof in three cases. 
    
%     Assume first that there exists a minimal normal
%     subgroup $L$ of $G$ contained in $M\cap N$. Let 
%     $\pi\colon G\to G/L$ be the canonical map. Since $M$ and $N$ are both maximal subgroups of $G$ containing $L$, 
%     the subgroups $\pi(M)$ and $\pi(N)$ 
%     are maximal in the solvable group $G/L$. Since 
%     $|G/L|<|G|$, the inductive hypothesis 
%     implies that either $G/L=\pi(M)\pi(L)$ or
%     $\pi(M)$ and $\pi(N)$ are conjugate. If $G/L=\pi(M)\pi(N)$, then
%     \[
%     G=MNL\subseteq MN
%     \]
%     because $N\subseteq L=\ker\pi$. If $\pi(M)$ and 
%     $\pi(N)$ are conjugate, then $\pi(x)\pi(M)\pi(x)^{-1}=\pi(N)$ 
%     for some $x\in G$. Thus $xMx^{-1}\subseteq NL=N$. By the maximality of $N$, $xMx^{-1}=N$.  

%     Assume now that 
%     $L\subseteq M$ and $L\not\subseteq N$. Since $L$ is normal in $G$, 
%     $NL$ is a subgroup of $G$ containing $N$. Since 
%     $N$ is maximal, either $N=NL$ or $NL=G$. But in the first case, 
%     $L\subseteq NL=N$, a contradiction. Thus $NL=G$. Now 
%     $N\cap L$ is normal in $L$ and 
%     $N\cap L$ is normal in $N$ (because $NL=G$). Thus 
%     $N\cap L$ is a normal subgroup of $G$ contained in $L$. By the minimality of $L$, 
%     $N\cap L=\{1\}$ (because otherwise $L=N\cap L\subseteq N$, a contradiction). Hence 
%     $G=NL$ with $N\cap L=\{1\}$. By Dedekind's lemma, 
%     \[
%     M=M\cap G=M\cap (NL)=L(M\cap N).
%     \]
%     Thus
%     \[
%     MN=L(M\cap N)N=LN(N\cap M)=G, 
%     \]
%     by Exercise~\ref{xca:AcapB}.

%     Finally, we may assume that neither $M$ nor $N$ contain
%     a normal subgroup of $G$. Let $L$ be a minimal
%     normal subgroup of $G$. Then $L$ is an abelian $p$-group, 
%     say $|L|=p^\alpha$, for some prime number $p$, 
%     $ML=G$ and $L\cap M=\{1\}$ (see Exercise...). Let 
%     $\pi\colon G\to G/L$ be the canonical map 
%     and $\pi(A)$ be a minimal normal subgroup of $G/L$. 
%     Since $G/L$ is solvable, $|\pi(A)|=q^\beta$ for 
%     some prime number $q$. We claim that $p\ne q$. If $p=q$, then 
%     since $(G:A)=(\pi(G):\pi(A))$, it follows
%     that $A$ is a $p$-group. Thus $Z(A)\ne\{1\}$. Since 
%     $Z(A)$ is characteristic in $A$ and $A$ is normal
%     in $G$, $Z(A)$ is normal in $G$. Hence $Z(A)\cap L$ is a non-trivial normal subgroup of $G$ (is non-trivial because $L$ is a $A$ is a $p$-group). By the minimality 
%     of $L$, $Z(A)\cap L=L$. We claim that $A\cap M$ is normal in $G$. 
%     Let $g\in G$ 
%     Then $g=ml$ for some $m\in M$ and $l\in L$. If $a\in A\cap M$,
%     then 
%     \[
%     gag^{-1}=m(lal^{-1})m^{-1}=mam^{-1}\in A\cap M
%     \]
%     because $A$ is normal in $G$ and $L\subseteq Z(A)$. Thus $A\cap M$ is normal in $G$, a contradiction. 

%     Now we now that $p\ne q$. Let $Q\in\Syl_p(A)$. Then $Q\cap L=\{1\}$ since $p\ne q$. Note that since $L$ is a normal subgroup of $G$, $QL$ is a subgroup of $A$. Since 
%     \[
%     \frac{|G|}{|A|}=(G:A)=(\pi(G):\pi(A))=\frac{|G/L|}{q^\beta}=\frac{|G|}{p^{\alpha}q^{\beta}},
%     \]
%     it follows that $|A|=p^{\alpha}q^{\beta}$. Now 
%     $|Q||L|=|A|$ and hence $QL=A$.

% \end{proof}
