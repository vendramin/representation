\section{Project: Irreducible characters of dihedral groups}
\index{Character table!of dihedral groups}
Let $n\geq3$. 
Recall that the \emph{dihedral group} of order $2n$ 
is the group
\[
\D_{n}=\langle r,s:r^n=s^2=1,srs=r^{-1}\rangle. 
\]
Every element of $\D_{n}$ is of the form
$s^ir^j$ for some $i\in\{0,1\}$ and $j\in\{0,\dots,n-1\}$. 

Our goal is the construct the character table of $\D_{n}$. 

\begin{proposition}
\label{pro:classes_dihedral}
    Let $n\geq3$. If $n$ is odd, then 
    \[
    \{1\},
    \{r,r^{-1}\},
    \{r^2,r^{-2}\},
    \cdots,\{r^{(n-1)/2},r^{(1-n)/2}\},
    \{s,sr,sr^2,\dots,sr^{n-1}\}
    \]
    are the conjugacy classes of $\D_n$. If $n$ is even, then
    \begin{align*}
        \{1\},
    \{r,r^{-1}\},
    \{r^2,r^{-2}\},
    \cdots,&\{r^{n/2-1},r^{1-n/2}\},\\
    &\{r^{n/2}\},
    \{s,sr^2,sr^4,\dots,sr^{n-2}\},
    \{sr,sr^3,\dots,sr^{n-1}\}
    \end{align*}
    are the conjugacy classes of $\D_n$.
\end{proposition}

\begin{proof}
    Recall that $sr^j=r^{-j}s$ for all $j$. Let $g=s^ir^j\in\D_n$ and 
    $x=s^kr^l\in\D_n$. Let us compute $xgx^{-1}$. We split the
    proof into several steps. 

    Assume first that $i=0$, that is $g=r^j$. Then
    \[
    xgx^{-1}=(s^kr^l)r^j(r^{-l}s^{-k})=s^kr^js^{-k}
    =\begin{cases}
        r^j & \text{if $k=0$,}\\
        r^{-j} & \text{if $k=1$.}
    \end{cases}
    \]
    Hence the conjugacy class of $g=r^j$ is $\{r^j,r^{-j}\}$. 

    Now assume that $i=1$, that is $g=sr^j$. Since $k\in\{0,1\}$, a direct
    calculation using the fact that $r^ls=sr^{-l}$ yields 
    \[
    xgx^{-1}=\begin{cases} 
        sr^{-2l+j} & \text{if $k=0$,}\\
        sr^{2l-j} & \text{if $k=1$.}
        \end{cases}
    \]
    Hence the conjugacy class of $g=sr^j$ is $\{sr^{2l-j},sr^{-2l+j}:0\leq l\leq n-1\}$. 

    Assume that $n$ is odd. We have determined the conjugacy classes 
    \[
    \{1\},\{b,b^{-1}\},\{b^2,b^{-2}\},\dots,\{b^{(n-1)/2},b^{(1-n)/2}\}
    \]
    which together cover all the elements of the subgroup 
    $\langle b\rangle=\{1,b,b^2,\dots,b^{n-1}\}$. Since $n$ is odd, for every 
    integer 
    $m$ there exists an integer $x$ such that $2x\equiv m\bmod n$. Thus 
    the conjugacy class of $s$ is $\{s,sr,sr^2,\dots,sr^{n-1}\}$. These classes together cover 
    all the elements of $\D_n$. 
    
    Now assume that $n$ is even. We have determined the conjugacy classes 
    \[
    \{1\},\{b,b^{-1}\},\{b^2,b^{-2}\},\dots,\{b^{n/2-1},b^{1-n/2}\},\{b^{n/2}\}
    \]
    which together cover all the elements of the subgroup 
    $\langle b\rangle=\{1,b,b^2,\dots,b^{n-1}\}$. The class of $s$ is $\{s,sr^2,sr^4,\dots,sr^{n-2}\}$ and 
    the class of $sr$ is $\{sr,sr^3,\dots,sr^{n-1}\}$. These classes together cover 
    all the elements of $\D_n$. 
\end{proof}

The previous proposition gives the number of conjugacy classes of the dihedral group $\D_n$, namely
\[
\frac{2n+9+(-1)^n3}{4}=\begin{cases}
    \frac{n+6}{2} & \text{if $n$ is even},\\
    \frac{n+3}{2} & \text{if $n$ is odd}.\\
\end{cases}
\]
This number is precisely the number of irreducible representations of $\D_n$. 

\begin{exercise}
\label{xca:center_dihedral}
    Compute $Z(\D_n)$. 
\end{exercise}

\begin{exercise}
\label{xca:cp_dihedral}
    Prove that $\lim_{n\to\infty}\cp(\D_{n})=1/4$.
\end{exercise}

To determine the number of degree-one representations of our group, 
we need the index of the commutator subgroup.

\begin{exercise}
\label{xca:commutator_dihedral}
    Prove that $[\D_n,\D_n]=\langle r^2\rangle$. Moreover, 
    \[
    (G:[G,G])=\begin{cases}
        2 & \text{if $n$ is odd.}\\
        4 & \text{if $n$ is even.}
    \end{cases}
    \]
\end{exercise}

\subsection{$n$ odd}

By Proposition~\ref{pro:classes_dihedral}, the representatives of the conjugacy classes of $\D_n$ are 
$1,r,r^2,\dots,r^{(n-1)/2},s$. 
By Exercise~\ref{xca:commutator_dihedral}, there are two degree-one characters, namely the trivial character 
and the character $\eta$ such that $r\mapsto 1$ and $s\mapsto -1$. 

\bigskip 
\begin{center}
    \begin{tabular}{|c|cccccc|}
         \hline 
         & $1$ & $r$ & $r^2$ & $\cdots$  & $r^{(n-1)/2}$ & $s$\\
         \hline 
         trivial & $1$ & $1$ & $1$ & $\cdots$ & $1$ & $1$\\
         $\eta$ & $1$ & $1$ & $1$ & $\cdots$ & $1$ & $-1$\\
         \hline 
    \end{tabular}
\end{center}
\bigskip 

Assume that $n=2k-1$. 
We need $\frac{n-1}{2}=k-1$ additional irreducible characters.
For $m\in\{1,\dots,k-1\}$, let $\omega_m=e^{2\pi im/k}$ and 
\[
\rho_m\colon\D_n\to\GL_2(\C),\quad 
r\mapsto\begin{pmatrix}\omega_m&0\\0&\omega_m^{-1}\end{pmatrix},\quad 
s\mapsto\begin{pmatrix}0&1\\1&0\end{pmatrix}.
\]

\begin{exercise}
\label{xca:rho_m}
    Prove that each $\rho_m$ is a group homomorphism.
\end{exercise}

\begin{exercise}
\label{xca:Ind_chi}
    Let $G=\D_n$ and 
    $H=\langle r\rangle$. For each $m\in\{1,\dots,k-1\}$, 
    let 
    \[
    \chi_m\colon H\to\C^{\times},\quad r\mapsto\omega_m.
    \]
    Prove that 
    $\Ind_H^G\chi_m=\rho_m$.    
\end{exercise}

A direct calculation produces the values of the 
character $\chi_m$ of $\rho_m$.

\bigskip 
\begin{center}
    \begin{tabular}{|c|cccccc|}
         \hline 
         & $1$ & $r$ & $r^2$ & $\cdots$  & $r^{(n-1)/2}$ & $s$\\
         \hline 
         $\chi_m$ & $1$ & $\omega_m+\omega_m^{-1}$ & $\omega_m^2+\omega_m^{-2}$ & $\cdots$ & $\omega_m^{(n-1)/2}+\omega_m^{(1-n)/2}$ & 0\\
         \hline 
    \end{tabular}
\end{center}
\bigskip 

\begin{exercise}
\label{xca:chim_alldifferent}
    Let $i,j\in\{1,\dots,k-1\}$. Prove that $\chi_i\ne\chi_j$ whenever $i\ne j$.
\end{exercise}

\begin{exercise}
\label{xca:chim_irreducible}
    Prove that each $\chi_m$ is irreducible. 
\end{exercise}

It remains only to note that we have constructed 
$\frac{n+3}{2}$ irreducible characters of $\D_n$, 
so the character table of $\D_n$ for odd $n$ 
is complete!

\subsection{$n$ even}

In this case, by Exercise~\ref{xca:commutator_dihedral}, there are four 
degree-one representations. These are the group homomorphisms defined as follows: For $j\in\{1,2,3,4\}$, let $\eta_j\colon\D_n\to\C$ be given by 
\begin{align*}
    \eta_1(r)&=1, & \eta_2(r)&=1, &  \eta_3(r)&=-1, & \eta_4(r)&=-1,\\
    \eta_1(s)&=1, & \eta_2(s)&=-1, & \eta_3(s)&=1, &  \eta_4(s)&=-1.\\
\end{align*}

Of course, $\eta_1$ is the trivial character of $\D_n$. 
By a direct calculation, we compute the values of the other characters: 
\bigskip 
\begin{center}
    \begin{tabular}{|c|ccccccc|}
         \hline 
         & $1$ & $r$ & $r^2$ & $\cdots$  & $r^{n/2}$ & $s$ & $sr$\\
         \hline 
         $\eta_1$ & $1$ & $1$ & $1$ & $\cdots$ & $1$ & $1$ & $1$\\
         $\eta_2$ & $1$ & $1$ & $1$ & $\cdots$ & $1$ & $-1$ & $-1$ \\
         $\eta_3$ & $1$ & $-1$ & $1$ & $\cdots$ & $(-1)^{n/2}$ & $1$ & $(-1)^{n/2}$ \\
         $\eta_4$ & $1$ & $-1$ & $1$ & $\cdots$ & $(-1)^{n/2}$ & $-1$ & $(-1)^{n/2+1}$\\
         \hline 
    \end{tabular}
\end{center}
\bigskip 

Assume now that $n=2k$. 
We need $\frac{n-1}{2}=k-1$ additional irreducible characters. For 
$m\in\{1,\dots,k-1\}$, let $\omega_m=e^{2\pi im/k}$ and 
\[
\rho_m\colon\D_n\to\GL_2(\C),\quad 
r\mapsto\begin{pmatrix}\omega_m&0\\0&\omega_m^{-1}\end{pmatrix},\quad 
s\mapsto\begin{pmatrix}0&1\\1&0\end{pmatrix}.
\]

Each $\rho_m$ is a group homomorphism (see Exercise~\ref{xca:rho_m}). 
A direct calculation produces the values of the 
character $\chi_m$ of $\rho_m$.

\bigskip 
\begin{center}
    \begin{tabular}{|c|ccccccc|}
         \hline 
         & $1$ & $r$ & $r^2$ & $\cdots$  & $r^{n/2}$ & $s$ & $sr$\\
         \hline 
         $\chi_m$ & $1$ & $\omega_m+\omega_m^{-1}$ & $\omega_m^2+\omega_m^{-2}$ & $\cdots$ & $\omega_m^{n/2}+\omega_m^{-n/2}$ & $0$ & $0$\\
         \hline 
    \end{tabular}
\end{center}
\bigskip 

In the same way that we constructed the character table when $n$ is odd, 
we now need to verify that we have constructed 
$\frac{n+6}{2}$ irreducible characters of $\D_n$. 

\begin{exercise}
\label{xca:n_even}
    Prove that we have constructed $\frac{n+6}{2}$ irreducible characters of $\D_n$. 
\end{exercise}