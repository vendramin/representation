\section{Project: Induced representations}


% \index{Restriction}
% Let $G$ be a finite group, $H$ a subgroup of $G$, and 
% $V$ be a $\C[G]$-module. By restricting the action of $G$ on $V$ to $H$, we obtain that $V$ is a $\C[H]$-module. This module is denoted $\Res_H^GV$ and 
% is called the \emph{restriction} of $V$ to $H$. 

\begin{definition}
    \index{Bimodule}
    Let $R$ and $S$ be rings. An abelian group $M$ is called 
    a \emph{$(R,S)$-bimodule} if $M$ is a left $R$-module, 
    $M$ is a right $S$-module, and 
    \[
    r\cdot (m\cdot s)=(r\cdot m)\cdot s
    \]
    holds for all $r\in R$, $s\in S$ and $m\in M$. 
\end{definition}

Note that every left $R$-module is an $(R,\Z)$-bimodule. Similarly, every right $S$-module is an $(\Z,S)$-bimodule. Every 
ring $R$ is an $(R,R)$-bimodule. 

\begin{example}
If $M$ is an $(R,S)$-bimodule and $N$ is a left 
$R$-module, then the set 
$\Hom_R(M,N)$ of left $R$-module homomorphisms $M\to N$ is a left 
$S$-module with 
\[
(s\cdot \varphi)(m)=\varphi(m\cdot s),\quad s\in S,\,\varphi\in\Hom_R(M,N),\,m\in M.
\]
\end{example}

\index{Map!balanced}
\index{Balanced Map}
Let $M$ be an $(R,S)$-bimodule, $N$ be an $S$-module and $U$ be a $R$-module. We say that a map  
$f\colon M\times N\to U$ 
is \emph{balanced} if 
\begin{align*}
    &f(m_1+m_2,n)=f(m_1,n)+f(m_2,n),\\
    &f(m,n_1+n_2)=f(m,n_1)+f(m,n_2),\\
    &f(m\cdot s,n)=f(m,s\cdot n),\\
    &f(r\cdot m,n)=r\cdot f(m,n)
\end{align*}
for all $m,m_1,m_2\in M$, $n,n_1,n_2\in N$, $r\in R$ and $s\in S$. 

\begin{example}
If $M$ is an $R$-module, the map $f\colon R\times M\to M$, $(r,m)\mapsto r\cdot m$, is balanced.  
\end{example}

\index{Tensor product!of bimodules}
Let $M$ be an $(R,S)$-bimodule, $N$ be an $S$-module and $U$ be an $R$-module. 
A \emph{tensor product} $M\otimes_S N$ is an $R$-module with a balanced map
$\eta\colon M\times N\to M\otimes_S N$ satisfying the following universal property:
\begin{quote}
If $f\colon M\times N\to U$ is a balanced map, then 
there exists a unique $R$-module homomorphism $\alpha\colon M\otimes_S N\to U$ such that $f=\alpha\circ\eta$. 
\end{quote}

Notation: $m\otimes n=\eta(m,n)$ for $m\in M$ and $n\in N$.
The tensor product of bimodules exists and one can show it is unique up to isomorphism. More precisely,  $M\otimes_S N$
is the $R$-module generated by 
the set $\{m\otimes n:m\in M,\,n\in N\}$, where the elements $m\otimes n$ satisfy the following properties: 
\begin{align}
    &(m+m_1)\otimes n=m\otimes n+m_1\otimes n &&\text{$m,m_1\in M$, $n\in N$},\\
    &m\otimes(n+n_1)=m\otimes n+m\otimes n_1 &&\text{$m\in M$, $n,n_1\in N$},\\
    &(ms)\otimes n=m\otimes (sn) &&\text{$m\in M$, $n\in N$, $s\in S$},\\
    &(rm)\otimes n=r(m\otimes n) &&\text{$m\in M$, $n\in N$, $r\in R$}.
\end{align}

An arbitrary element of $M\otimes_S N$ is a finite sum of the form 
$\sum_{i=1}^k m_i\otimes n_i$,
where $m_1,\dots,m_k\in M$ and $n_1,\dots,n_k\in N$, and not necessarily an element of the form 
$m\otimes n$. 

\begin{example}
$M\simeq R\otimes_R M$ as $R$-modules. Since the map $R\times M\to M$, $(r,m)\mapsto r\cdot m$, is balanced, it induces an isomomorphism $R\otimes_R M\to M$, $r\otimes m\mapsto r\cdot m$ with inverse $M\to R\otimes_R M$, $m\mapsto 1\otimes m$. 
\end{example}

\begin{example}
If $M_1,\dots,M_k$ are $(R,S)$-bimodules and $N$ is an $S$-module, then
\[
(M_1\oplus\cdots\oplus M_k)\otimes_S N\simeq (M_1\otimes_S N)\oplus\cdots\oplus (M_k\otimes_S N).
\]
\end{example}

Some exercises:

\begin{exercise}
    Prove that  $M\otimes_RN\simeq N\otimes_{R^{\op}}M$.
\end{exercise}

\begin{exercise}
    Prove that  $\Z/n\otimes_{\Z}\Q=\{0\}$.
\end{exercise}

\begin{exercise}
    Let $M$ be an $(R,S)$-bimodule and $N$ be an $(S,T)$-bimodule. 
    Prove that  $M\otimes_SN$ is an $(R,T)$-bimodule 
    with $r(m\otimes n)t=(rm)\otimes (nt)$, 
    where $m\in M$, $n\in N$, $r\in R$, $t\in T$.
\end{exercise}

\begin{exercise}
    Prove that  $(M\otimes_R N)\otimes_RT\simeq M\otimes_R (N\otimes_RT)$.
\end{exercise}

\begin{exercise}
    State and prove the associativity of tensor product of bimodules. 
\end{exercise}

% Atiyah-Mac Donald
% https://math.stackexchange.com/questions/2586211/associativity-of-tensor-products

If $G$ is a finite group, $H$ is a subgroup of $G$
and $V$ is a $\C[H]$-module, then  
$\C[G]$ is a $(\C[G],\C[H])$-bimodule.

\begin{definition}
\index{Module!induced}
Let $G$ be a finite group and  
$H$ be a subgroup of $G$. 
If $V$ is a $\C[H]$-module of $G$, 
we define the \emph{induced} $\C[G]$-module of $V$ 
as 
\[
\Ind_H^GV=\C[G]\otimes_{\C[H]}V.
\]
\end{definition}

% \index{Transversal}
% Si $H$ es un subgrupo de $G$, un \textbf{transversal} (a izquierda) 
% de $H$ en $G$ es un subconjunto $T$ de $G$ que contiene exactamente un elemento de cada coclase (a izquierda) 
% de $H$ en $G$. 

\begin{example}
Let $G=\Sym_3$ and $H=\{\id,(12)\}$. Then 
$T=\{\id,(123),(23)\}$ is a transversal of $H$ in $G$. We can decompose $G$ as 
\[
G=\{\id,(12)\}\cup \{(123),(13)\}\cup\{(132),(23)\}=\bigcup_{t\in T}tH.
\]
Each $g\in G$ can be written uniquely as $g=th$ for some $t\in T$ and $h\in H$. We can define define a linear transformation 
$\varphi\colon \C[G]\to \C[H]\oplus \C[H]\oplus \C[H]=|T|\C[H]$, such that for each $g=th$ returns $h$ in the position corresponding to $t\in T$, namely 
\begin{align*}
\id&\mapsto (\id,0,0), && (12)\mapsto ((12),0,0), && (123)\mapsto (0,\id,0),\\
(23)&\mapsto (0,0,\id), && (13)\mapsto (0,(12),0), && (132)\mapsto (0,0,(12)).
\end{align*}
For example, 
\[
\varphi( 5(12)-3(123)+7\id )=(7\id+5(12),-3\id,0).
\]
Note that $\varphi$ is an isomorphism of right $\C[H]$-modules. 
\end{example}

The previous example is crucial to understand the following 
result:

\begin{proposition}
Let $G$ be a finite group and 
$H$ be a subgroup of $G$. If $V$ is a $\C[H]$-module, then  
\[
    \Ind_H^G(V)=\bigoplus_{t\in T}t\otimes V,
\]
where $T$ is a transversal of $H$ in $G$ and $t\otimes V=\{t\otimes v:v\in V\}$. In particular, 
\[
\dim\Ind_H^GV=(G:H)\dim V.
\]
\end{proposition}

\begin{proof}
Decompose $G$ into $H$-cosets with the transversal 
$T$, that is 
\[
G=\bigcup_{t\in T}tH.
\]
Each $g\in G$ can be written uniquely as $g=th$ for some $t\in T$ and $h\in H$. As we did in the previous example, this 
produces an isomorphism 
$\varphi\colon \C[G]\to |T|\C[H]$ of right $\C[H]$-modules, where $\varphi(g)$ is $h$ in the summand corresponding to $t\in T$
and is zero in the rest of the summands. Hence 
\[
\Ind_H^GV=\C[G]\otimes_{\C[H]}V\simeq (|T|\C[H])\otimes_{\C[H]}V\simeq |T|(\C[H]\otimes_{\C[H]}V)\simeq |T|V
\]
as $\C[H]$-modules. In particular, $\dim\Ind_H^GV=|T|\dim V$. 

 Write $g=th$ with $t\in T$ and $h\in H$. Then $g\otimes v=(th)\otimes v=t\otimes h\cdot v\in t\otimes V$. 
Hence $\C[G]\otimes_{\C[H]}V\subseteq \oplus_{t\in T}t\otimes V$. The other inclusion is trivial. By definition, 
the sum over $t\in T$ of the $t\otimes V$'s is direct.
\end{proof}

\begin{theorem}[Frobenius' reciprocity]
\index{Frobenius' reciprocity}
Sea $G$ un grupo finito y $H$ un subgrupo de $G$. 
Si $U$ es un $\C[G]$-module y $V$ es un $\C[H]$-module, entonces
\[
\Hom_{\C[H]}(V,\Res_H^GU)\simeq \Hom_{\C[G]}(\Ind_H^GV,U)
\]
como espacios vectoriales.
\end{theorem}

\begin{proof}
Si $\varphi\in\Hom_{\C[H]}(V,\Res_H^GU)$, sea 
\[
f_{\varphi}\colon \C[G]\times V\to U,
\quad
(g,v)\mapsto g\cdot\varphi(v).
\]
Veamos que $f_{\varphi}$ es balanceada. Un cálculo directo muestra que
\begin{align*}
    &f_{\varphi}(g+g_1,v)=f_{\varphi}(g,v)+f_{\varphi}(g_1,v),&&
    f_{\varphi}(g,v+w)=f_{\varphi}(g,v)+f_{\varphi}(g,w).
\end{align*}
Como $\varphi$ es morfismo de $\C[H]$-modules,
\begin{align*}
    &f_{\varphi}(gh,v)=(gh)\cdot\varphi(v)
    =g\cdot (h\cdot \varphi(v))
    =g\cdot (h\cdot\varphi(v))
    =g\cdot \varphi(h\cdot v)=f_{\varphi}(g,h\cdot v)
\end{align*}
para todo $g\in G$, $h\in H$ y $v\in V$. Por último,
\begin{align*}
    &f_{\varphi}(gg_1,v)=(gg_1)\cdot\varphi(v)=g\cdot(g_1\cdot\varphi(v))=g\cdot f_{\varphi}(g_1,v)
\end{align*}
para todo $g,g_1\in G$ y $v\in V$. Para cada $\varphi\in\Hom_{\C[H]}(V,\Res_H^GU)$ tenemos 
entonces un $\Gamma(\varphi)\in\Hom_{\C[G]}(\Ind_H^GV,U)$ tal que
$\Gamma(\varphi)(g\otimes v)=g\cdot\varphi(v)$. 
Tenemos así definida una función 
\[
\Gamma\colon \Hom_{\C[H]}(V,\Res_H^GU)\to\Hom_{\C[G]}(\Ind_H^GV,U),
\quad
\varphi\mapsto\Gamma(\varphi).
\]

La función $\Gamma$ es lineal e inyectiva, ambas afirmaciones fáciles de verificar. 

Es también sobreyectiva, pues si $\theta\in\Hom_{\C[H]}(\Ind_H^GV,U)$, entonces
la función $\varphi(v)=\theta(1\otimes v)$ es tal que $\varphi\in\Hom_{\C[H]}(V,\Res_H^GU)$ y 
cumple 
\[
\Gamma(\varphi)(g\otimes v)=g\cdot\varphi(v)=g\cdot\theta(1\otimes v)=\theta(g\otimes v).\qedhere
\]
\end{proof}

Supongamos ahora que $K=\C$. 

Sea $H$ un subgrupo de $G$. Si $U$ es un $\C[G]$-module con caracter $\chi$, el caracter de $\Res_H^GU$ se denota por $\chi|_H$ y vale que 
que $\chi|_H(1)=\chi(1)$. Si $V$ es un $\C[H]$-module con 
caracter $\phi$, el module $\Ind_H^GV$ tiene caracter $\phi^G$ y vale que $\phi^G(1)=(G:H)\phi(1)$. 
\begin{align*}
\langle \phi,\chi|_H\rangle_H 
&=\dim\Hom_{\C[H]}(V,\Res_H^GU)
=\dim\Hom_{\C[G]}(\Ind_H^GV,U)
=\langle\phi^G,\chi\rangle_G,
\end{align*}
donde $\langle \alpha,\beta\rangle_X=\sum_{x\in X}\alpha(x)\overline{\beta(x)}$ denota el producto 
interno del espacio de funciones $X\to\C$. 

\begin{definition}
Si $\Irr(G)=\{\chi_1,\dots,\chi_k\}$ e $\Irr(H)=\{\phi_1,\dots,\phi_l\}$, se define
la \textbf{matriz de inducción--restricción} como la matriz $(c_{ij})\in\C^{l\times k}$, donde
\[
c_{ij}=\langle \phi_i^G,\chi_j\rangle_G=\langle\phi_i,\chi_j|_H\rangle_H.
\]
\end{definition}

La fila $i$-ésima de la matriz de inducción--restricción da la multiplicidad con que el caracter $\chi_j$ aparece
en la descomposición de $\phi_i^G$. La columna $j$-ésima da la multiplicidad con que el caracter $\phi_i$ aparece 
en la descomposición de $\chi_j|H$.

\begin{example}
Sea $G=\Sym_3$. 
La tabla de caracteres de $G$ es 
	\begin{center}
		\begin{tabular}{|c|rrr|}
			\hline
			& $1$ & $3$ & $2$\tabularnewline
			& $1$ & $(12)$ & $(123)$ \tabularnewline
			\hline 
			$\chi_{1}$ & $1$ & $1$ & $1$\tabularnewline
			$\chi_{2}$ & $1$ & $-1$ & $1$ \tabularnewline
			$\chi_{3}$ & $2$ & $0$ & $-1$ \tabularnewline
			\hline
		\end{tabular}
	\end{center}
La tabla de caracteres del subgrupo 
$H=\{\id,(12)\}$ es 
\begin{center}
\begin{tabular}{|c|rr|}
\hline 
& $1$ & $1$ \tabularnewline
& $\id$ & $(12)$ \tabularnewline
\hline 
$\phi_{1}$ & $1$ & $1$ \tabularnewline
$\phi_{2}$ & $1$ & $-1$\tabularnewline
\hline
\end{tabular}
\end{center}
A simple vista vemos que $\chi_1|_H=\phi_1$, $\chi_2|_H=\phi_2$ y que $\chi_3|_H=\phi_1+\phi_2$. 
La matriz de inducción--restricción es entonces
\[
\begin{pmatrix}
1 & 0 & 1\\
0 & 1 & 1
\end{pmatrix}.
\]
Observemos que además $\phi_1^G=\chi_1+\chi_3$ y que $\phi_2^G=\chi_2+\chi_3$. 
\end{example}

Veamos cómo calcular explícitamente caracteres inducidos. 

\begin{proposition}
Sea $H$ un subgrupo de $G$ y sea $V$ es un $\C[H]$-module con caracter $\chi$. Si 
$T$ es un trasversal de $H$ en $G$, entonces
\[
\chi^G(g)=\sum_{\substack{t\in T\\t^{-1}gt\in H}}\chi(t^{-1}gt)
\]
para todo $g\in G$. 
\end{proposition}

\begin{proof}
    Sabemos que $\Ind_H^GV=\oplus_{t\in T}t\otimes V$. 
    Supongamos que $T=\{t_1,\dots,t_m\}$ 
    y sea $\{v_1,\dots,v_n\}$ una base de $V$. 
    Entonces $\{t_i\otimes v_k:1\leq i\leq m,\,1\leq k\leq n\}$ es 
    una base de $\Ind_H^GV$ y la acción
    de $g$ en $\Ind_H^GV$ está dada por
    \[
    \rho^G(g)=\begin{cases}
    \rho(t_j^{-1}gt_i) & \text{si $t_j^{-1}gt_i\in H$},\\
    0 & \text{en otro caso}.
    \end{cases}
    \]
    En efecto, si $gt_i=t_jh$ para $h\in H$ y ciertos $i,j$, entonces 
    \[
    g\cdot (t_i\otimes v_k)=gt_i\otimes v_k=t_jh\otimes v_k=t_j\otimes h\cdot v_k
    \]
    y además $gt_i=t_jh$ si y sólo si $t_j^{-1}gt_i=h\in H$. Se concluye entones
    que $g$ actúa como $t^{-1}gt$ en $V$ en caso en que $t^{-1}gt\in H$ y 
    como la transformación nula en otro caso. 
\end{proof}

\begin{corollary}
\label{cor:induccion}
    Sea $H$ un subgrupo de $G$ 
    y sea $V$ es un $\C[H]$-module con caracter $\chi$.
    Si $g\in G$, entonces
    \[
    \chi^G(g)=\frac{1}{|H|}\sum_{\substack{x\in G\\x^{-1}gx\in H}}\chi(x^{-1}gx).
    \]
\end{corollary}

\begin{proof}
    Sea $T$ un transversal de $H$ en $G$. Si $x\in G$, escribimos $x=th$ para $t\in T$ y $h\in H$. 
    Como $x^{-1}gx=h^{-1}(t^{-1}gt)h$, entonces $x^{-1}gx\in H\Longleftrightarrow t^{-1}gt\in H$ y además, en ese caso, 
    $\chi(x^{-1}gx)=\chi(t^{-1}gt)$ pues $\chi$ es una función de clases. Eso implica que existen $|H|$ elementos $x\in G$ 
    tales que $x^{-1}gx\in H$. Para esos $x$, se tiene $\chi(x^{-1}gx)=\chi(t^{-1}gt)$, lo que implica 
    el corolario. 
\end{proof}