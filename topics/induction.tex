\section{Induced representations}


% \index{Restriction}
% Let $G$ be a finite group, $H$ a subgroup of $G$, and 
% $V$ be a $\C[G]$-module. By restricting the action of $G$ on $V$ to $H$, we obtain that $V$ is a $\C[H]$-module. This module is denoted $\Res_H^GV$ and 
% is called the \emph{restriction} of $V$ to $H$. 




\begin{definition}
    \index{Bimodule}
    Let $R$ and $S$ be rings. An abelian group $M$ is called 
    a \emph{$(R,S)$-bimodule} if $M$ is a left $R$-module, 
    $M$ is a right $S$-module, and 
    \[
    r\cdot (m\cdot s)=(r\cdot m)\cdot s
    \]
    holds for all $r\in R$, $s\in S$ and $m\in M$. 
\end{definition}

Note that every left $R$-module is an $(R,\Z)$-bimodule. Similarly, every right $S$-module is an $(\Z,S)$-bimodule. Every 
ring $R$ is an $(R,R)$-bimodule. 

\begin{example}
Si $M$ es un $(R,S)$-bimódulo y $N$ es un $R$-módulo, entonces el conjunto 
$\Hom_R(M,N)$ de morfismos de $R$-módulos $M\to N$ es un 
$S$-módulo con 
\[
(s\cdot \varphi)(m)=\varphi(m\cdot s),\quad s\in S,\,\varphi\in\Hom_R(M,N),\,m\in M.
\]
\end{example}

Sean $M$ un $(R,S)$-bimódulo, $N$ un $S$-módulo y $U$ un $R$-módulo. 
Diremos que una función $f\colon M\times N\to U$ 
es \textbf{balanceada} si 
\begin{align*}
    &f(m_1+m_2,n)=f(m_1,n)+f(m_2,n),\\
    &f(m,n_1+n_2)=f(m,n_1)+f(m,n_2),\\
    &f(m\cdot s,n)=f(m,s\cdot n),\\
    &f(r\cdot m,n)=r\cdot f(m,n)
\end{align*}
para todo $m,m_1,m_2\in M$, $n,n_1,n_2\in N$, $r\in R$ y $s\in S$. 

\begin{example}
Si $M$ es un $R$-módulo, la función $f\colon R\times M\to M$, $(r,m)\mapsto r\cdot m$, es balanceada. 
\end{example}

\index{Producto tensorial!de bimódulos}
Sean $M$ un $(R,S)$-bimódulo, $N$ un $S$-módulo y $U$ un $R$-módulo. 
Se define el \textbf{producto tensorial} $M\otimes_S N$ es un $R$-módulo provisto con una función balanceada 
$\eta\colon M\times N\to M\otimes_S N$ que cumple con la siguiente propiedad universal: 
\begin{quote}
Si $f\colon M\times N\to U$ es una función balanceada, entonces
existe un único morfismo de $R$-módulos $\alpha\colon M\otimes_S N\to U$ tal que $f=\alpha\circ\eta$. 
\end{quote}
Notación: $m\otimes n=\eta(m,n)$ para $m\in M$ y $n\in N$.
El producto tensorial existe y puede demostrarse que es único salvo isomorfismos. Más precisamente, $M\otimes_S N$
se define como el $R$-módulo generado por
el conjunto $\{m\otimes n:m\in M,\,n\in N\}$, donde los $m\otimes n$ satisfacen 
las siguientes identidades:
\begin{align}
    &(m+m_1)\otimes n=m\otimes n+m_1\otimes n &&\text{$m,m_1\in M$, $n\in N$},\\
    &m\otimes(n+n_1)=m\otimes n+m\otimes n_1 &&\text{$m\in M$, $n,n_1\in N$},\\
    &(ms)\otimes n=m\otimes (sn) &&\text{$m\in M$, $n\in N$, $s\in S$},\\
    &(rm)\otimes n=r(m\otimes n) &&\text{$m\in M$, $n\in N$, $r\in R$}.
\end{align}
Un elemento arbitrario de $M\otimes_S N$ es una suma finita
de la forma 
$\sum_{i=1}^k m_i\otimes n_i$,
donde $m_1,\dots,m_k\in M$ y $n_1,\dots,n_k\in N$, y no necesariamente un tensor elemental $m\otimes n$. 

\begin{example}
$M\simeq R\otimes_R M$ como $R$-módulos. Como la función $R\times M\to M$, $(r,m)\mapsto r\cdot m$, es balanceada, 
induce un morfismo $R\otimes_R M\to M$, $r\otimes m\mapsto r\cdot m$ con inversa $M\to R\otimes_R M$, $m\mapsto 1\otimes m$. 
\end{example}

\begin{example}
Si $M_1,\dots,M_k$ son $(R,S)$-bimódulos y $N$ es un $S$-módulo, entonces
\[
(M_1\oplus\cdots\oplus M_k)\otimes_S N\simeq (M_1\otimes_S N)\oplus\cdots\oplus (M_k\otimes_S N).
\]
\end{example}

Algunos ejercicios:

\begin{exercise}
    Demuestre que $M\otimes_RN\simeq N\otimes_{R^{\op}}M$.
\end{exercise}

\begin{exercise}
    Demuestre que $\Z/n\otimes_{\Z}\Q=\{0\}$.
\end{exercise}

\begin{exercise}
    Sean $M$ un $(R,S)$-bimódulo y $N$ un $(S,T)$-bimódulo. 
    Demuestre que $M\otimes_SN$ es un $(R,T)$-bimódulo 
    con $r(m\otimes n)t=(rm)\otimes (nt)$, 
    donde $m\in M$, $n\in N$, $r\in R$, $t\in T$.
\end{exercise}

\begin{exercise}
    Demuestre que $(M\otimes_R N)\otimes_RT\simeq M\otimes_R (N\otimes_RT)$.
\end{exercise}

\begin{exercise}
    Enuncie y demuestre la asociatividad del producto tensorial de bimódulos. 
\end{exercise}

% Atiyah-Mac Donald
% https://math.stackexchange.com/questions/2586211/associativity-of-tensor-products

Si $G$ es un grupo finito, $H$ es un subgrupo de $G$
y $V$ es un $K[H]$-módulo, entonces 
$K[G]$ es un $(K[G],K[H])$-bimódulo.

\begin{definition}
\index{Módulo!inducido}
Sea $G$ un grupo finito y sea 
$H$ un subgrupo de $G$. 
Si $V$ es un $K[H]$-módulo de $G$, 
se define el $K[G]$-módulo \textbf{inducido} de $V$ 
como
\[
\Ind_H^GV=K[G]\otimes_{K[H]}V.
\]
\end{definition}

\index{Transversal}
Si $H$ es un subgrupo de $G$, un \textbf{transversal} (a izquierda) 
de $H$ en $G$ es un subconjunto $T$ de $G$ que contiene exactamente un elemento de cada coclase (a izquierda) 
de $H$ en $G$. 

\begin{example}
Si $G=\Sym_3$ y $H=\{\id,(12)\}$, entonces
$T=\{\id,(123),(23)\}$ es un transversal de $H$ en $G$. Podemos descomponer 
a $G$ como
\[
G=\{\id,(12)\}\cup \{(123),(13)\}\cup\{(132),(23)\}=\bigcup_{t\in T}tH.
\]
Como cada $g\in G$ se escribe en forma única como $g=th$ para $t\in T$ y $h\in H$, podemos 
definir una transformación lineal 
$\varphi\colon K[G]\to K[H]\oplus K[H]\oplus K[H]=|T|K[H]$, que para $g=th$ devuelve $h$ en el lugar que corresponde a $t\in T$, es decir
\begin{align*}
\id&\mapsto (\id,0,0), && (12)\mapsto ((12),0,0), && (123)\mapsto (0,\id,0),\\
(23)&\mapsto (0,0,\id), && (13)\mapsto (0,(12),0), && (132)\mapsto (0,0,(12)).
\end{align*}
Por ejemplo, 
\[
\varphi( 5(12)-3(123)+7\id )=(7\id+5(12),-3\id,0).
\]
Es importante observar que $\varphi$ es un isomorfismo de $K[H]$-módulos (a derecha). 
\end{example}

La observación hecha en el ejemplo anterior es la clave del siguiente resultado.

\begin{proposition}
Sea $G$ un grupo finito y sea 
$H$ un subgrupo de $G$. Si $V$ es un $K[H]$-módulo de $G$, entonces 
\[
    \Ind_H^G(V)=\bigoplus_{t\in T}t\otimes V,
\]
donde $T$ es un transversal de $H$ en $G$ y $t\otimes V=\{t\otimes v:v\in V\}$. En particular, 
$\dim\Ind_H^GV=(G:H)\dim V$.
\end{proposition}

\begin{proof}
Descomponemos a $G$ como unión disjunta de coclases de $H$ con el transversal $T$, es decir
\[
G=\bigcup_{t\in T}tH.
\]
Cada $g\in G$ se escribe entonces unívocamente como $g=th$ con $t\in T$ y $h\in H$. Tal como 
hicimos en el ejemplo anterior, esto nos permite obtener un isomorfismo 
$\varphi\colon K[G]\to |T|K[H]$ de $K[H]$-módulos (a derecha), donde $\varphi(g)$ es $h$ en el sumando que corresponde a $t\in T$
y es cero en el resto de los sumandos. Luego
\[
\Ind_H^GV=K[G]\otimes_{K[H]}V\simeq (|T|K[H])\otimes_{K[H]}V\simeq |T|(K[H]\otimes_{K[H]}V)\simeq |T|V
\]
como $K[H]$-módulos. En particular, $\dim\Ind_H^GV=|T|\dim V$. 

Si escribimos $g=th$ con $t\in T$ y $h\in H$, entonces $g\otimes v=(th)\otimes v=t\otimes h\cdot v\in t\otimes V$. 
Luego $K[G]\otimes_{K[H]}V\subseteq \oplus_{t\in T}t\otimes V$. La otra inclusión es trivial. Por definición, 
la suma sobre $t\in T$ de los $t\otimes V$ es directa. 
\end{proof}

\begin{theorem}[Reciprocidad de Frobenius]
\index{Teorema!de reciprocidad de Frobenius}
Sea $G$ un grupo finito y $H$ un subgrupo de $G$. 
Si $U$ es un $K[G]$-módulo y $V$ es un $K[H]$-módulo, entonces
\[
\Hom_{K[H]}(V,\Res_H^GU)\simeq \Hom_{K[G]}(\Ind_H^GV,U)
\]
como espacios vectoriales.
\end{theorem}

\begin{proof}
Si $\varphi\in\Hom_{K[H]}(V,\Res_H^GU)$, sea 
\[
f_{\varphi}\colon K[G]\times V\to U,
\quad
(g,v)\mapsto g\cdot\varphi(v).
\]
Veamos que $f_{\varphi}$ es balanceada. Un cálculo directo muestra que
\begin{align*}
    &f_{\varphi}(g+g_1,v)=f_{\varphi}(g,v)+f_{\varphi}(g_1,v),&&
    f_{\varphi}(g,v+w)=f_{\varphi}(g,v)+f_{\varphi}(g,w).
\end{align*}
Como $\varphi$ es morfismo de $K[H]$-módulos,
\begin{align*}
    &f_{\varphi}(gh,v)=(gh)\cdot\varphi(v)
    =g\cdot (h\cdot \varphi(v))
    =g\cdot (h\cdot\varphi(v))
    =g\cdot \varphi(h\cdot v)=f_{\varphi}(g,h\cdot v)
\end{align*}
para todo $g\in G$, $h\in H$ y $v\in V$. Por último,
\begin{align*}
    &f_{\varphi}(gg_1,v)=(gg_1)\cdot\varphi(v)=g\cdot(g_1\cdot\varphi(v))=g\cdot f_{\varphi}(g_1,v)
\end{align*}
para todo $g,g_1\in G$ y $v\in V$. Para cada $\varphi\in\Hom_{K[H]}(V,\Res_H^GU)$ tenemos 
entonces un $\Gamma(\varphi)\in\Hom_{K[G]}(\Ind_H^GV,U)$ tal que
$\Gamma(\varphi)(g\otimes v)=g\cdot\varphi(v)$. 
Tenemos así definida una función 
\[
\Gamma\colon \Hom_{K[H]}(V,\Res_H^GU)\to\Hom_{K[G]}(\Ind_H^GV,U),
\quad
\varphi\mapsto\Gamma(\varphi).
\]

La función $\Gamma$ es lineal e inyectiva, ambas afirmaciones fáciles de verificar. 

Es también sobreyectiva, pues si $\theta\in\Hom_{K[H]}(\Ind_H^GV,U)$, entonces
la función $\varphi(v)=\theta(1\otimes v)$ es tal que $\varphi\in\Hom_{K[H]}(V,\Res_H^GU)$ y 
cumple 
\[
\Gamma(\varphi)(g\otimes v)=g\cdot\varphi(v)=g\cdot\theta(1\otimes v)=\theta(g\otimes v).\qedhere
\]
\end{proof}

Supongamos ahora que $K=\C$. 

Sea $H$ un subgrupo de $G$. Si $U$ es un $\C[G]$-módulo con caracter $\chi$, el caracter de $\Res_H^GU$ se denota por $\chi|_H$ y vale que 
que $\chi|_H(1)=\chi(1)$. Si $V$ es un $\C[H]$-módulo con 
caracter $\phi$, el módulo $\Ind_H^GV$ tiene caracter $\phi^G$ y vale que $\phi^G(1)=(G:H)\phi(1)$. 
\begin{align*}
\langle \phi,\chi|_H\rangle_H 
&=\dim\Hom_{\C[H]}(V,\Res_H^GU)
=\dim\Hom_{\C[G]}(\Ind_H^GV,U)
=\langle\phi^G,\chi\rangle_G,
\end{align*}
donde $\langle \alpha,\beta\rangle_X=\sum_{x\in X}\alpha(x)\overline{\beta(x)}$ denota el producto 
interno del espacio de funciones $X\to\C$. 

\begin{definition}
Si $\Irr(G)=\{\chi_1,\dots,\chi_k\}$ e $\Irr(H)=\{\phi_1,\dots,\phi_l\}$, se define
la \textbf{matriz de inducción--restricción} como la matriz $(c_{ij})\in\C^{l\times k}$, donde
\[
c_{ij}=\langle \phi_i^G,\chi_j\rangle_G=\langle\phi_i,\chi_j|_H\rangle_H.
\]
\end{definition}

La fila $i$-ésima de la matriz de inducción--restricción da la multiplicidad con que el caracter $\chi_j$ aparece
en la descomposición de $\phi_i^G$. La columna $j$-ésima da la multiplicidad con que el caracter $\phi_i$ aparece 
en la descomposición de $\chi_j|H$.

\begin{example}
Sea $G=\Sym_3$. 
La tabla de caracteres de $G$ es 
	\begin{center}
		\begin{tabular}{|c|rrr|}
			\hline
			& $1$ & $3$ & $2$\tabularnewline
			& $1$ & $(12)$ & $(123)$ \tabularnewline
			\hline 
			$\chi_{1}$ & $1$ & $1$ & $1$\tabularnewline
			$\chi_{2}$ & $1$ & $-1$ & $1$ \tabularnewline
			$\chi_{3}$ & $2$ & $0$ & $-1$ \tabularnewline
			\hline
		\end{tabular}
	\end{center}
La tabla de caracteres del subgrupo 
$H=\{\id,(12)\}$ es 
\begin{center}
\begin{tabular}{|c|rr|}
\hline 
& $1$ & $1$ \tabularnewline
& $\id$ & $(12)$ \tabularnewline
\hline 
$\phi_{1}$ & $1$ & $1$ \tabularnewline
$\phi_{2}$ & $1$ & $-1$\tabularnewline
\hline
\end{tabular}
\end{center}
A simple vista vemos que $\chi_1|_H=\phi_1$, $\chi_2|_H=\phi_2$ y que $\chi_3|_H=\phi_1+\phi_2$. 
La matriz de inducción--restricción es entonces
\[
\begin{pmatrix}
1 & 0 & 1\\
0 & 1 & 1
\end{pmatrix}.
\]
Observemos que además $\phi_1^G=\chi_1+\chi_3$ y que $\phi_2^G=\chi_2+\chi_3$. 
\end{example}

Veamos cómo calcular explícitamente caracteres inducidos. 

\begin{proposition}
Sea $H$ un subgrupo de $G$ y sea $V$ es un $\C[H]$-módulo con caracter $\chi$. Si 
$T$ es un trasversal de $H$ en $G$, entonces
\[
\chi^G(g)=\sum_{\substack{t\in T\\t^{-1}gt\in H}}\chi(t^{-1}gt)
\]
para todo $g\in G$. 
\end{proposition}

\begin{proof}
    Sabemos que $\Ind_H^GV=\oplus_{t\in T}t\otimes V$. 
    Supongamos que $T=\{t_1,\dots,t_m\}$ 
    y sea $\{v_1,\dots,v_n\}$ una base de $V$. 
    Entonces $\{t_i\otimes v_k:1\leq i\leq m,\,1\leq k\leq n\}$ es 
    una base de $\Ind_H^GV$ y la acción
    de $g$ en $\Ind_H^GV$ está dada por
    \[
    \rho^G(g)=\begin{cases}
    \rho(t_j^{-1}gt_i) & \text{si $t_j^{-1}gt_i\in H$},\\
    0 & \text{en otro caso}.
    \end{cases}
    \]
    En efecto, si $gt_i=t_jh$ para $h\in H$ y ciertos $i,j$, entonces 
    \[
    g\cdot (t_i\otimes v_k)=gt_i\otimes v_k=t_jh\otimes v_k=t_j\otimes h\cdot v_k
    \]
    y además $gt_i=t_jh$ si y sólo si $t_j^{-1}gt_i=h\in H$. Se concluye entones
    que $g$ actúa como $t^{-1}gt$ en $V$ en caso en que $t^{-1}gt\in H$ y 
    como la transformación nula en otro caso. 
\end{proof}

\begin{corollary}
\label{cor:induccion}
    Sea $H$ un subgrupo de $G$ 
    y sea $V$ es un $\C[H]$-módulo con caracter $\chi$.
    Si $g\in G$, entonces
    \[
    \chi^G(g)=\frac{1}{|H|}\sum_{\substack{x\in G\\x^{-1}gx\in H}}\chi(x^{-1}gx).
    \]
\end{corollary}

\begin{proof}
    Sea $T$ un transversal de $H$ en $G$. Si $x\in G$, escribimos $x=th$ para $t\in T$ y $h\in H$. 
    Como $x^{-1}gx=h^{-1}(t^{-1}gt)h$, entonces $x^{-1}gx\in H\Longleftrightarrow t^{-1}gt\in H$ y además, en ese caso, 
    $\chi(x^{-1}gx)=\chi(t^{-1}gt)$ pues $\chi$ es una función de clases. Eso implica que existen $|H|$ elementos $x\in G$ 
    tales que $x^{-1}gx\in H$. Para esos $x$, se tiene $\chi(x^{-1}gx)=\chi(t^{-1}gt)$, lo que implica 
    el corolario. 
\end{proof}