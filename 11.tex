\section{Lecture: Week 11}

\subsection{Clifford theory}

We begin with a routine exercise. 

%\begin{exercise}
%$    Let $G$ be a finite group and $\chi\in\Char(G)$. 
%$    Let $f\colon G\to H$ be an isomorphism of 
%$    groups. Then 
%$    \[
%$    (f\cdot )(f(g))
%$    
%\end{exercise}

\begin{exercise}
\label{xca:conjugate_chars1}
Let $G$ be a finite group and $N$ be a normal subgroup
of $G$. Prove that $G$ acts on $\Irr(N)$ via 
\[
(g\cdot\theta)(n)=\theta(g^{-1}ng),\quad 
g\in G,\theta\in\Irr(N),n\in N.
\]
\end{exercise}

\begin{exercise}
\label{xca:conjugate_chars2}
Let $G$ be a finite group and $N$ be a normal subgroup of $G$. 
Let $\chi\in\cf(G)$, $\theta\in\cf(N)$ and $g\in G$. Prove that
\[
\langle\Res_N^G\chi,g\cdot\theta\rangle=\langle\Res_N^G\chi,\theta\rangle.
\]
% for all $\chi\in\cf(G)$.
%     \begin{enumerate}
%         \item $\langle g\cdot\alpha,g\cdot\beta\rangle=\langle\alpha,\beta\rangle$.
%         \item $\langle\Res_N^G\chi,g\cdot\alpha\rangle=\langle\Res_N^G\chi,\alpha\rangle$ for all $\chi\in\cf(G)$. 
%     \end{enumerate}
\end{exercise}

\index{Irreducible constituent}
Recall that every character $\chi$ of a finite group is uniquely 
a sum of irreducible characters. These are called
the \emph{irreducible constituents} of $\chi$. The set 
of irreducible constituents of $\chi$ is the set  
\[
\{\eta\in\Irr(G):\langle\chi,\eta\rangle>0\}.
\]

\begin{theorem}[Clifford]
\label{thm:Clifford}
\index{Clifford theorem}
    Let $G$ be a finite group and $N$ be a normal
    subgroup of $G$. Let $\chi\in\Irr(G)$ and $\theta\in\Irr(N)$ be 
    an irreducible constituent of $\Res_N^G\chi$. 
    Then 
    \[
    \Res_N^G\chi = e(\theta_1+\cdots+\theta_t),
    \]
    where $\theta=\theta_1,\dots,\theta_t$ are the conjugates 
    of $\theta$ in $G$, 
    and $e$ is a positive integer. In particular, all the constituents of $\Res_N^G\chi$ have the same degree. 
\end{theorem}

\begin{proof}
    Let $G\cdot\theta=\{\theta_1,\dots,\theta_t\}$ be the 
    orbit of $\theta$. For each $i\in\{1,\dots,t\}$, 
    \[
    \langle\Res_N^G\chi,\theta_i\rangle 
    =\langle g_i\cdot \Res_H^G\chi,g_i\cdot\theta\rangle
    =\langle\Res_H^G\chi,\theta\rangle>0
    \]
    since by assumption $\theta$ is an irreducible constituent of $\Res_N^G\chi$. 
    Let $e=\langle\Res_N^G\chi,\theta\rangle$. Then 
    \[
    \Res_N^G\chi=e(\theta_1+\cdots+\theta_t)+\eta
    \]
    for some $\eta=0$ or $\eta\in\Char(N)$. Since 
    \[
    e=\langle\Res_H^G\chi,\theta\rangle
    =\langle e(\theta_1+\cdots+\theta_t)+\eta,\theta\rangle
    =e\sum_{i=1}^t\langle\theta_i,\theta\rangle+\langle\eta,\theta\rangle
    =e+\langle\eta,\theta\rangle,
    \]
    it follows that $\langle\eta,\theta\rangle=0$. By Frobenius' reciprocity, 
    $\langle\chi,\Ind_N^G\theta\rangle
    =\langle\Res_N^G\chi,\theta\rangle=e$.
    Thus 
    \[
    \Ind_N^G\theta=e\chi+\lambda 
    \]
    for some $\lambda=0$ or $\lambda\in\Char(G)$. Since 
    \[
    e=\langle\chi,\Ind_N^G\theta\rangle 
    =\langle\chi, e\chi+\lambda\rangle
    =e\langle\chi,\chi\rangle+\langle\chi,\lambda\rangle 
    =e+\langle\chi,\lambda\rangle,
    \]
    it follows that $\langle\chi,\lambda\rangle=0$.

    \begin{claim}
        $\Res_N^G\Ind_N^G\theta=t\frac{1}{|N|}\sum_{i=1}^t\theta_i$.
        %=\frac{1}{|N|}\sum_{x\in G}(x\cdot\theta)$.
    \end{claim}

    Let $n\in N$. For $i\in\{1,\dots,t\}$ let 
    $x_i\in G$ be such that $x_i\cdot\theta=\theta_i$. Then 
    \begin{align*}
        (\Ind_N^G\theta)(n) &= \frac{1}{|N|}\sum_{x\in G}\theta^0(x^{-1}nx)\\
        &=\frac{1}{|N|}\sum_{x\in G}(x\cdot\theta)(n)\\
        &=\frac{1}{|N|}\sum_{1=1}^t t(x_i\cdot\theta)(n)\\
        &=\frac{t}{|N|}\sum_{1=1}^t \theta_i(n), 
    \end{align*}
    where we have used that $n\in N$ and $N$ is normal in $G$ (because $x^{-1}nx\in N$ if and only if
    $n\in xNx^{-1}=N$). 

    \bigskip 
    Therefore 
    \begin{align*}
        \frac{t}{|N|}(\theta_1+\cdots+\theta_t)
        &=\Res_N^G\Ind_N^G\theta\\
        &=\Res_N^G(e\chi+\lambda)\\
        &=e\Res_N^G\chi+\Res_N^G\lambda\\
        &=e^2(\theta_1+\cdots+\theta_t)+e\eta+\Res_N^G\lambda.
    \end{align*}
    Taking inner product against $\eta$,  
    \[
    \frac{t}{|N|}\sum_{i=1}^t\langle\theta_i,\eta\rangle 
    =e^2\sum_{i=1}^t\langle\theta_i,\eta\rangle+e\langle\eta,\eta\rangle+\langle\Res_N^G\lambda,\eta\rangle. 
    \]
    Since $\langle\theta_i,\eta\rangle=0$ for all $i\in\{1,\dots,t\}$, 
    \begin{equation}
    \label{eq:eta}
    0=e\langle\eta,\eta\rangle+\langle\Res_N^G\lambda,\eta\rangle.
    \end{equation}
    We know that $e>0$. Moreover, since $\eta\in\Char(N)$ and 
    $\Res_N^G\lambda\in\Char(N)$, each term of the right hand side of~\eqref{eq:eta} is non-negative, that is 
    $\langle\eta,\eta\rangle\geq0$ and 
    $\langle\Res_N^G\lambda,\eta\rangle\geq0$. Therefore 
    $\langle\eta,\eta\rangle=0$ and hence $\eta=0$. 
\end{proof}

% \begin{proof}
%     We claim that 
%     \[
%     \Res_N^G\Ind_N^G\theta=\frac{1}{|N|}\sum_{g\in G}(g\cdot\theta).
%     \]
%     To prove this formula, let $n\in N$. Then 
%     \[
%     \left(\Ind_N^G\theta\right)(n)
%     =\frac{1}{|N|}\sum_{g\in G}\theta^0(g^{-1}ng)
%     =\frac{1}{|N|}\sum_{g\in G}\theta(g^{-1}ng)
%     =\frac{1}{|N|}\sum_{g\in G}(g\cdot\theta)(n).
%     \]
%     By Frobenius' reciprocity, 
%     $\langle\Ind_N^G\theta,\chi\rangle=\langle\theta,\Res_N^G\chi\rangle>0$. 
%     Thus $\chi$ is an irreducible constituent of $\Ind_N^G\theta$. 
    
%     % Moreover, $g\cdot\chi=\chi$ 
%     % for all $g\in G$, since $\chi\in\Irr(G)\subseteq \cf(G)$. 
%     Now 
%     let $\varphi\in\Irr(N)\setminus\{\theta_1,\dots,\theta_t\}$. Then
%     \[
%     \langle \Ind_N^G\theta, \Ind_N^G\varphi\rangle
%     =\langle \Res_N^G\Ind_N^G\theta,\varphi\rangle
%     =\sum_{g\in G}\langle g\cdot\theta,\varphi\rangle=0,
%     \]
%     since each $g\cdot\theta=\theta_j$ for some $j\in\{1,\dots,t\}$. 
%     It follows that $\langle\Res_N^G\chi,\varphi\rangle=0$. 
%     Thus all irreducible constituents of $\Res_N^G\chi$ belong
%     to $\{\theta_1,\dots,\theta_t\}$, that is
%     \[
% \Res_N^G\chi=\sum_{i=1}^t\langle\Res_N^G\chi,\theta_i\rangle\theta_i.
%     \]
%     Moreover, for each $i\in\{1,\dots,t\}$, 
%     there exists $g_i\in G$ such that
%     $\theta_i=g_i\cdot\theta$. Thus, using Exercise~\ref{xca:conjugate_chars2}, 
%     \[
%     \langle\Res_N^G\chi,\theta_i\rangle=\langle\Res_N^G\chi,g_i\cdot\theta\rangle=\langle\Res_N^G\chi,\theta\rangle=e.\]
%     From this the theorem follows. 
% \end{proof}

\index{Ramification index}
The integer $e$ in Theorem~\ref{thm:Clifford} is known as the \emph{ramification index} of $\chi$ on $N$. In general, the number $e$ is not easy to control. 

\begin{exercise}
\label{xca:Clifford_divisibility}
    Let $G$ be a finite group and $N$ be a normal subgroup of $G$. Let $\chi\in\Irr(G)$ and $\theta$ 
    be an irreducible constituent of $\Res_N^G\chi$. 
    Prove that $\theta(1)$ divides $\chi(1)$. 
\end{exercise}

\index{Inertia subgroup}
Let $G$ be a group and $\theta\in\Irr(G)$. The set 
\[
I_G(\theta)=\{g\in G:g\cdot\theta=\theta\}
\]
is a subgroup of $G$ and is called \emph{inertia subgroup} of $\theta$ in $G$. Note that the inertia
subgroup is the stabilizer of the action of $G$ 
on characters by conjugation (see 
of Exercise~\ref{xca:conjugate_chars1}). In particular, 
$\theta$ has 
$(G:I_G(\theta))$ conjugates. 

\begin{theorem}[Clifford correspondence]
\label{thm:Clifford_correspondence}
\index{Clifford correspondence}
    Let $G$ be a finite group and $N$ be a normal subgroup of $G$. Let $\theta\in\Irr(N)$ and $I=I_G(\theta)$.  Then 
    the map 
    \[
    \{\psi\in\Irr(I):\langle\Res_N^I\psi,\theta\rangle>0\}\to 
    \{\chi\in\Irr(G):\langle\Res_N^G\chi,\theta\rangle>0\},\quad 
    \psi\mapsto\Ind_I^G\psi,
    \]
    is bijective. Moreover, if $\psi$ is a constituent of $\Res_N^I\theta$, then 
    $\langle\Res_N^I\psi,\theta\rangle=\langle\Res_N^G\chi,\theta\rangle$. 
\end{theorem}

\begin{proof}
    There are several things to prove. 

    \begin{claim}
        The map is $\psi\mapsto\Ind_I^G\psi$ well-defined. 
    \end{claim}

    Let $\psi\in\Irr(I)$ be such that $e=\langle\Res_N^I\psi,\theta\rangle>0$ and 
    let $\chi\in\Irr(G)$ be a constituent of $\Ind_I^G\psi$. By Frobenius' reciprocity, 
    \[
    \langle\psi,\Res_I^G\chi\rangle=\langle\Ind_I^G\psi,\chi\rangle>0.
    \]
    Thus $\psi$ is a constituent of $\Res_I^G\chi$, that is 
    \[
    \Res_I^G\chi=\psi+\lambda 
    \]
    for some $\lambda=0$ or $\lambda\in\Char(I)$. Thus
    \[
    \Res_N^G\chi=\Res_N^I\Res_I^G\chi=\Res_N^I(\psi+\lambda)
    =\Res_N^I\psi+\Res_N^I\lambda, 
    \]
    that is 
    $\Res_N^I\psi$ is a constituent of $\Res_N^I\Res_I^G\chi$. 
    Moreover, 
    \[
    \chi(1)\leq(\Ind_I^G\psi)(1)=(G:I)\psi(1).
    \]
    Let $f=\langle\Res_N^G\chi,\theta\rangle$. Then  
    \[
    f=\langle\Res_N^G\chi,\theta\rangle=\langle\Res_N^I\Res_I^G\chi,\theta\rangle
    \geq\langle\Res_N^I\psi,\theta\rangle=e>0.
    \]
    Since $\Res_N^G\chi=f(\theta_1+\cdots+\theta_t)$, where $G\cdot\theta=\{\theta_1,\dots,\theta_t\}$ is 
    the orbit of $\theta$ under the action of $G$ and $t=(G:I)$,  
    \[
    ft\theta(1)=\chi(1)=(\Res_N^G\chi)(1)\leq(\Ind_I^G\psi)(1)=t\psi(1)=et\theta(1)\leq ft\theta(1),
    \]
    where the last equality follows since 
    $\Res_N^I\psi=e\theta$ by Clifford's theorem. Therefore $e=f$ and 
    $\Ind_I^G\psi=\chi$. 

    \begin{claim}
        The map is $\psi\mapsto\Ind_I^G\psi$ is injective. 
    \end{claim}

    Let $\psi_1\in\Irr(I)$ and $\psi_2\in\Irr(I)$ 
    be such that 
    $\langle\Res_N^I\psi_i,\theta\rangle>0$ for all $i\in\{1,2\}$ and 
    $\chi=\Ind_I^G\psi_1=\Ind_I^G\psi_2$. In the first claim, we proved that $\chi\in\Irr(G)$. 
    We want to prove that $\psi_1=\psi_2$. 
    
    Suppose $\psi_1\ne\psi_2$. 
    We know that $\psi_1$ and $\psi_2$ from the first claim that
    are constituents
    of $\Res_I^G\chi$, that is 
    \[
    \Res_I^G\chi=\psi_1+\psi_2+\xi 
    \]
    for some map $\xi\colon I\to\C$. (The map $\xi$ is either zero
    or a character of $I$.) 
    Then both $\Res_N^I\psi_1$ and 
    $\Res_N^I\psi_2$ are constituents 
    of 
    $\Res_N^G\chi$, as  
    \begin{align*}
        \Res_N^G\chi&=\Res_N^I\Res_I^G\chi\\
        &=\Res_N^I(\psi_1+\psi_2+\xi)\\
        &=\Res_N^I\psi_1+\Res_N^I\psi_2+\Res_N^I\xi.       
    \end{align*}

    Moreover, 
    \begin{align*}
        \langle\Res_N^G\chi,\theta\rangle &= \langle\Res_N^I\psi_1+\Res_N^I\psi_2+\Res_N^I\xi,\theta\rangle\\
        &=\langle\Res_N^I\psi_1,\theta\rangle+\langle\Res_N^I\psi_2,\theta\rangle+\langle\Res_N^I\xi,\theta\rangle\\
&\geq\langle\Res_N^I\psi_1,\theta\rangle+\langle\Res_N^I\psi_2,\theta\rangle\\
        &=\langle\Res_N^G\chi,\theta\rangle+\langle\Res_N^G\chi,\theta\rangle,
    \end{align*}
    where the last equality holds because we proved in the previous claim that 
    \[
        \langle\Res_N^I\psi_i,\theta\rangle=\langle\Res_N^G\chi,\theta\rangle
    \]
    for all $i\in\{1,2\}$. 
    This implies that $\langle\Res_N^G\chi,\theta\rangle=0$, a contradiction. 

    \begin{claim}
        The map is $\psi\mapsto\Ind_I^G\psi$ is surjective. 
    \end{claim}

    Let $\chi\in\Irr(G)$ be such that 
    $e=\langle\Res_N^G\chi,\theta\rangle>0$. Since 
    \[
    \Res_I^G\chi=\sum_{\psi\in\Irr(I)}\langle\Res_I^G,\psi\rangle\psi,
    \]
    it follows that 
    \[
    \Res_N^G\chi=\Res_N^I\Res_I^G\chi
    =\sum_{\psi\in\Irr(I)}\langle\Res_I^G\chi,\psi\rangle\Res_N^I\psi.
    \]
    Since 
    \[
    e=\langle\Res_N^G\chi,\theta\rangle=
    \sum_{\psi\in\Irr(I)}\langle\Res_I^G\chi,\psi\rangle\langle\Res_N^I\psi,\theta\rangle
    \]
    is a positive number, there exists some $\psi\in\Irr(I)$ 
    such that $\langle\Res_I^G\chi,\psi\rangle\langle\Res_N^I\psi,\theta\rangle>0$. In particular, $\langle\Res_N^I\psi,\theta\rangle$ and 
    \[
    \langle\chi,\Ind_I^G\psi\rangle=\langle\Res_I^G\chi,\psi\rangle>0
    \]
    Hence $\chi=\Ind_I^G\psi$. 
\end{proof}

\subsection{It\^o's theorem}

We now present a result that is stronger than Schur’s Theorem~\ref{thm:Schur_chi(1)}.
To that end, we introduce some exercises on basic properties of the center of characters.

\begin{definition}
    \index{Center of a character}
    Let $G$ be a finite group and $\chi\in\Char(G)$. 
    The \emph{center} of $\chi$ is
    \[
    Z(\chi)=\{g\in G:|\chi(g)|=\chi(1)\}.
    \]
\end{definition}

\begin{exercise}
\label{xca:center}
    Let $G$ be a finite group and $\rho\colon G\to\GL_n(\C)$ be a representation with character 
    $\chi$. Prove the following statements: 
    \begin{enumerate}
        \item $Z(\chi)=\{g\in G:\rho_g\text{ is a scalar matrix}\}$. 
       % $=\lambda I\text{ for some $\lambda\in\C\}$}\}$.
        \item $Z(\chi)$ is a normal subgroup of $G$. 
        \item $Z(\chi)/\ker\chi$ is cyclic.
    \end{enumerate}
\end{exercise}

% Recall that an $n\times n$ matrix $A$ is a \emph{scalar matrix} if 
% $A=\lambda I$ for some $\lambda\in\C$, where $I$ is the $n\times n$ identity matrix. 

\begin{exercise}
\label{xca:center_quotient}
    Let $G$ be a finite group and $\chi\in\Irr(G)$. 
    Prove that 
    \[
    Z(\chi)/\ker\chi=Z(G/\ker\chi).
    \]
\end{exercise}

\begin{exercise}
\label{xca:center_ofG}
    Let $G$ be a finite group. Prove that
    \[
    Z(G)=\bigcap\{Z(\chi):\chi\in\Irr(G)\}.
    \]
\end{exercise}

The previous exercise shows that the center of a finite group can be determined
from its character table. It follows that the character table detects
nilpotency. To do this, one computes $Z(G)$ from the character table of $G$, then
the character table of $G/Z(G)$, and by iterating this process, one obtains the
upper central series of the group $G$. 

\begin{lemma}
\label{lem:Ito}
    Let $G$ be a finite group and 
    $\chi\in\Irr(G)$. Then $\chi(1)$ divides $(G:Z(\chi))$. 
\end{lemma}

\begin{proof}
    Let $Q=G/\ker\chi$. 
    By Theorem~\ref{thm:correspondence}, $\chi$ corresponds to 
    $\eta\in\Char(Q)$. 
    By Schur's theorem~\ref{thm:Schur_chi(1)}, 
    $\chi(1)=\eta(1)$ divides $(Q:Z(Q))$. By Exercise~\ref{xca:center_quotient}, 
    $(Q:Z(Q))=(G:Z(\chi))$.    
\end{proof}

\begin{theorem}[It\^o]
\index{It\^o theorem}
\label{thm:Ito}
Let $G$ be a finite group and $\chi\in\Irr(G)$. Then 
$\chi(1)$ divides $(G:A)$ for all normal abelian subgroup $A$ of $G$.  
\end{theorem}

\begin{proof}
    Let $A$ be a normal abelian subgroup of $G$ and 
    $\theta\in\Irr(A)$ be an irreducible constituent of $\Res_A^G\chi$, that 
    is $\langle\Res_A^G\chi,\theta\rangle>0$. Let $I=I_G(\theta)$. 
    By Clifford correspondence (Theorem~\ref{thm:Clifford_correspondence}), 
    $\chi=\Ind_I^G(\psi)$ for some $\psi\in\Irr(I)$ such that 
    $\langle\Res_A^I\psi,\theta\rangle>0$. By Clifford's theorem, since 
    $I$ acts trivially on $\theta$, 
    $\Res_A^I\psi=e\theta$, where $e=\langle\Res_A^I\psi,\theta\rangle>0$. Since $A$ is abelian and
    $\theta\in\Irr(A)$, $\theta(1)=1$. 
    
    We claim that $A\subseteq Z(\psi)$. In fact, if $a\in A$, then 
    \[
    |\psi(a)|=|\Res_A^I\psi(a)|=|e\theta(a)|=e|\theta(a)|=e1=e=\psi(1).
    \]
    By Lagrange's theorem, $|A|$ divides $|Z(\psi)|$. Thus $(I:Z(\psi))$ divides $(I:A)$. 

    By Lemma~\ref{lem:Ito}, 
    $e\theta(1)=\psi(1)$ divides $(I:Z(\psi))$. Then 
    $\psi(1)$ divides $(I:A)$. Now 
    \[
    \chi(1)=(\Ind_I^G\psi)(1)=(G:I)\psi(1)
    \]
    divides $(G:I)(I:A)=(G:A)$.
\end{proof}

\begin{bonus}
    \label{xca:Reynolds}
	Prove that Itô’s theorem remains valid under the assumption that $A$ is subnormal in $G$.
\end{bonus}

%The proof of Theorem~\ref{thm:Ito} is no more difficult than that of Schur's Theorem~\ref{thm:Schur_chi(1)}. For a proof, %see \cite[\S8.1]{MR0450380}.

