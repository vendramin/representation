\chapter{}

\topic{Brauer–Fowler theorem}

\index{Symmetric}
\index{Antisymmetric}
Let $\rho\colon G\to\GL(V)$ 
be a representation with character $\chi$. The $\C[G]$-module $V\otimes V$ 
has character $\chi^2$. Let 
$\{v_1,\dots,v_n\}$ be a basis of $V$ and 
\[
T\colon V\to V,\quad
v_i\otimes v_j\mapsto v_j\otimes v_i.
\]
It is an exercise to check that $T(v\otimes w)=w\otimes v$ for all 
$v,w\in V$. It follows that  
$T$ does not depend on the chosen basis. Note that
$T$ is a homomorphism of $\C[G]$-modules, as
\[
T(g\cdot (v\otimes w))=T((g\cdot v)\otimes (g\cdot w))=(g\cdot w)\otimes (g\cdot v)=g\cdot T(w\otimes v)
\]
for all $g\in G$ y $v,w\in V$. 
In particular, the \textbf{symmetric part} 
\begin{gather*}
S(V\otimes V)=\{x\in V\otimes V:T(x)=x\}
\shortintertext{and the \textbf{antisymmetric part}}
A(V\otimes V)=\{x\in V\otimes V:T(x)=-x\}
\end{gather*}
of $V\otimes V$ are both  
$\C[G]$-submodules of $V\otimes V$. 
The terminology is motivated by the following fact:
\[
V\otimes V=S(V\otimes V)\oplus A(V\otimes V).
\]
In fact, 
$S(V\otimes V)\cap A(V\otimes V)=\{0\}$, as   
$x\in S(V\otimes V)\cap A(V\otimes V)$ implies
$x=T(x)$ and $x=-T(x)$. Hence $x=0$. Moreover, 
$V\otimes V=S(V\otimes V)+ A(V\otimes V)$, as every $x\in V\otimes V$ can be written 
as 
\[
x=\frac12(x+T(x))+\frac12(x-T(x))
\]
with $\frac12(x+T(x))\in S(V\otimes V)$ and $\frac12(x-T(x))\in A(V\otimes V)$. 

We claim that $\{v_i\otimes v_j+v_j\otimes v_i:1\leq i,j\leq n\}$ is
a basis of $S(V\otimes V)$ 
and that  
\[
\{v_i\otimes v_j-v_j\otimes v_i:1\leq i<j\leq n\}
\]
is a basis of $A(V\otimes V)$. Since both sets are linearly independent, 
\[
\dim S(V\otimes V)\geq n(n+1)/2\text{ and }
\dim A(V\otimes V)\geq n(n-1)/2.
\]
Moreover, 
\[
n^2=\dim (V\otimes V)=\dim S(V\otimes V)+\dim A(V\otimes V),
\]
so it follows that
$\dim S(V\otimes V)=n(n+1)/2$ and $\dim A(V\otimes V)=n(n-1)/2$. 

\begin{proposition}
    Sea $G$ un grupo finito y  
    sea $V$ un $\C[G]$-módulo de dimensión finita con caracter $\chi$. Si el módulo $S(V\otimes V)$ 
    tiene caracter $\chi_S$ y el módulo $A(V\otimes V)$ tiene caracter $\chi_A$, entonces 
    \begin{align*}
        &\chi_S(g)=\frac12(\chi^2(g)+\chi(g^2)),\\
        &\chi_A(g)=\frac12(\chi^2(g)-\chi(g^2)).
    \end{align*}
\end{proposition}

\begin{proof}
    Sea $g\in G$. Sea $\rho\colon G\to\GL(V)$ la representación asociada al módulo $V$, es decir $\rho(g)(v)=\rho_g(v)=g\cdot v$. 
    Sabemos que $\rho_g$ es diagonalizable. Sea $\{e_1,\dots,e_n\}$ una base de autovectores de $\rho_g$, digamos
    $g\cdot e_i=\lambda_ie_i$ con $\lambda_i\in\C$ para $i\in\{1,\dots,n\}$. En particular, $\chi(g)=\sum_{i=1}^n\lambda_i$. 
    
    Como $\{e_i\otimes e_j-e_j\otimes e_i:1\leq i<j\leq n\}$ es base de $A(V\otimes V)$ y además 
    \[
    g\cdot (e_i\otimes e_j-e_j\otimes e_i)=\lambda_i\lambda_j(e_i\otimes e_j-e_j\otimes e_i),
    \]
    tenemos $\chi_A(g)=\sum_{1\leq i<j\leq n}\lambda_i\lambda_j$. Por otro lado, como $g^2\cdot e_i=\lambda_i^2e_i$ para todo $i$,
    $\chi(g^2)=\sum_{i=1}^n\lambda_i^2$. Luego
    \[
    \chi^2(g)=\chi(g)^2=\sum_{i=1}^n\sum_{j=1}^n\lambda_i\lambda_j=2\sum_{1\leq i<j\leq n}\lambda_i\lambda_j+\sum_{i=1}^n\lambda_i^2=2\chi_A(g)+\chi(g^2).
    \]
    Como además $V\otimes V=S(V\otimes V)\oplus A(V\otimes V)$, se tiene 
    $\chi^2(g)=\chi_S(g)+\chi_A(g)$, es decir 
    $\chi_S(g)=\frac12(\chi^2(g)+\chi(g^2))$.
\end{proof}

\index{Involución}
Una \textbf{involución} en un grupo es un elemento $x\ne 1$ tal que $x^2=1$. 
Es posible la cantidad de involuciones 
con la tabla de caracteres:

\begin{proposition}
Si $G$ es un grupo finito con $t$ involuciones, entonces 
\[
1+t=\sum_{\chi\in\Irr(G)}\langle\chi_S-\chi_A,\chi_1\rangle\chi(1).
\]
\end{proposition}

\begin{proof}
Supongamos que $\Irr(G)=\{\chi_1,\dots,\chi_k\}$, donde $\chi_1$ es el caracter trivial de $G$. 
Para $x\in G$ sea 
\[
\theta(x)=|\{y\in G:y^2=x\}|.
\]
Como $\theta$ es una función de clases
$\theta$ puede escribirse como combinación lineal de los $\chi_j$, digamos
\[
\theta=\sum_{\chi\in\Irr(G)}\langle\theta,\chi\rangle\chi.
\]
Calculamos
\begin{align*}
    \langle\chi_S-\chi_A,\chi_1\rangle 
    &=\frac{1}{|G|}\sum_{g\in G}\chi(g^2)\\
    &=\frac{1}{|G|}\sum_{x\in G}\sum_{\substack{g\in G\\g^2=x}}\chi(g^2)
    =\frac{1}{|G|}\sum_{x\in G}\theta(x)\chi(x)=\langle\theta,\chi\rangle.
\end{align*}
Luego $\theta(x)=\sum_{\chi\in\Irr(G)}\langle\chi_S-\chi_A,\chi_1\rangle\chi$ y el resultado se obtiene
al evaluar esta expresión en $x=1$. 
\end{proof}

\index{Desigualdad!de Cauchy--Schwartz}
Necesitamos un lema:
% Recordemos la desigualdad de Cauchy--Schwartz. Si $x_1,\dots,x_n\in\R$, entonces
% $\sum x_i^2\geq\frac{1}{n}(\sum x_i)^2$. 

\begin{lemma}
Sea $G$ un grupo finito con $k$ clases de conjugación. 
Si $t$ es la cantidad de involuciones de $G$, entonces 
$t^2\leq (k-1)(|G|-1)$. 
\end{lemma}

\begin{proof}
Supongamos que $\Irr(G)=\{\chi_1,\dots,\chi_k\}$, donde $\chi_1$ es el 
carácter trivial de $G$. 
Si $\chi\in\Irr(G)$, entonces
\[
\langle\chi^2,\chi_1\rangle=\frac{1}{|G|}\sum_{g\in G}\chi(g)\chi(g)=\langle\chi,\overline{\chi}\rangle=\begin{cases}
1 & \text{si $\chi=\overline{\chi}$},\\
0 & \text{en otro caso}.
\end{cases}
\]
Como $\chi^2=\chi_S+\chi_A$, si $\langle\chi^2,\chi_1\rangle=1$, entonces el caracter trivial 
o bien es $\chi_1$ es parte de $\chi_S$ o bien es parte de $\chi_A$, pero no de ambos. Esto implica que
\[
\langle\chi_S-\chi_A,\chi_1\rangle\in\{-1,1,0\}.
\]
Vamos a demostrar ahora que 
$t\leq\sum_{i=2}^k\chi_i(1)$. En efecto, 
como $|\langle\chi_S-\chi_A,\chi_1\rangle|\leq 1$, 
entonces 
\begin{align*}
1+t=\theta(1)
&=\left|\sum_{\chi\in\Irr(G)}\langle\chi_S-\chi_A,\chi_1\rangle\chi(1)\right|\\
&\leq\sum_{\chi\in\Irr(G)}|\langle\chi_S-\chi_A,\chi_1\rangle|\chi(1)
\leq\sum_{\chi\in\Irr(G)}\chi(1),
\end{align*}
de donde se obtiene inmediatamente que $t\leq\sum_{i=2}^k\chi_i(1)$. 
Si utilizamos ahora 
la desigualdad de Cauchy--Schwartz, 
\[
t^2\leq\left(\sum_{i=2}^k\chi_i(1)\right)^2
\leq(k-1)\sum_{i=2}^k\chi(1)^2=(k-1)(|G|-1).\qedhere
\]
\end{proof}

Ahora sí estamos en condiciones de dar la primera demostración del teorema de
Brauer--Fowler. 

\begin{theorem}[Brauer--Fowler]
\index{Teorema!de Brauer--Fowler}
Sea $G$ un grupo finito y simple y sea $x$ una involución. Si $|C_G(x)|=n$, entonces $|G|\leq (n^2)!$	
\end{theorem}

\begin{proof}
Supongamos primero que existe un subgrupo propio $H$ de $G$ tal que
$(G:H)\leq n^2$. En ese caso, hacemos actuar a $G$ en $G/H$ por multiplicación a izquierda 
y tenemos un morfismo de grupos $\rho\colon G\to\Sym_{n^2}$. Como $G$ es un grupo simple, 
$\ker\rho=\{1\}$ o bien $\ker\rho=G$. Si $\ker\rho=G$, entonces $\rho(g)(yH)=yH$ para todo
$g\in G$ e $y\in G$, lo que implica que $g\in H$, una contradicción. Luego $\rho$ es inyectiva
y entonces $G$ es isomorfo a un subgrupo de $\Sym_{n^2}$. En particular, $|G|$ divide a $(n^2)!$

Sea $m=(|G|-1)/t$. 
Como $|C_G(x)|=n$, el grupo $G$ tiene al menos $|G|/n$ involuciones (pues la clase de conjugación
de $x$ tiene tamaño $|G|/n$ y todos sus elementos son involuciones), es decir $t\geq |G|/n$. Luego
$m=(|G|-1)/t<n$. Basta demostrar entonces que $G$ contiene un subgrupo de índice $\leq m^2$. 

Sean $C_1,\dots,C_k$ las clases de conjugación de $G$, donde $C_1=\{1\}$. 
Como $G$ es simple, $|C_i|>1$ 
para todo $i\in\{2,\dots,k\}$. Notar que 
\[
|G|-1\leq\frac{(k-1)(|G|-1)^2}{t^2}\Longleftrightarrow t^2\leq(k-1)(|G|-1),
\]
que vale gracias al lema anterior. 
Si $|C_i|>m$ para todo $i\in\{2,\dots,k\}$, entonces, como
\[
|G|-1\leq\frac{(k-1)(|G|-1)^2}{t^2}=(k-1)m^2,
\]
tendríamos 
\[
|G|-1=\sum_{i=2}^k|C_i|>(k-1)m^2,
\]
una contradicción. Luego existe una clase de conjugación $C$ de $G$ tal que $|C|\leq m^2$. Si $g\in C$, entonces
$C_G(g)$ es un subgrupo de $G$ de índice $|C|\leq m^2$.
\end{proof}

The bound of Brauer--Fowler's is not important.

\begin{corollary}
    Let $n\geq 1$ be an integer. There are at most finitely many 
    finite simple groups with an involution with a centralizer of order $n$.
\end{corollary}

As an exercise, a simple applications: 

\begin{exercise}
    If $G$ is a finite simple group and $x$ is an involution with
    centralizer of order two, then  
    $G\simeq\Z/2$. 
\end{exercise}

\topic{Induction and restriction}

Let $N$ be a normal subgroup of $G$ 
and $\pi\colon G\to G/N$, $g\mapsto gN$, be the canonical map. 
If $\widetilde{\rho}\colon G/N\to\GL(V)$ 
is a representation of $G/N$ with 
character
$\widetilde{\chi}$, the composition 
$\rho=\widetilde{\rho}\circ\pi\colon G\to \GL(V)$, $\rho(g)=\widetilde{\rho}(gN)$, 
is a representation of $G$. 
Thus
\[
\chi(g)=\trace{\rho_g}=\trace(\widetilde{\chi}(gN))=\widetilde{\chi}(gN).
\]
In particular, $\chi(1)=\widetilde{\chi}(1)$. The character $\chi$ 
is the \textbf{lifting} to $G$ of the character 
$\widetilde{\chi}$ of $G/N$. 

\begin{proposition}
If $\chi\in\Irr(G)$, then 
\[
\ker\chi=\{g\in G:\chi(g)=\chi(1)\}
\]
is a normal subgroup of $G$. 
\end{proposition}

\begin{proof}
Let $\rho\colon G\to\GL_n(\C)$ be a representation with character $\chi$. Then 
$\ker\rho\subseteq\ker\chi$, as $\rho_g=\id$ implies 
$\chi(g)=\trace(\rho_g)=n=\chi(1)$. We claim that  
$\ker\chi\subseteq\ker\rho$. If $g\in G$ is such that $\chi(g)=\chi(1)$, since 
$\rho_g$ is diagonalizable, there exist eigenvalues $\lambda_1,\dots,\lambda_n\in\C$ such that
\[
n=\chi(1)=\chi(g)=\sum_{i=1}^n\lambda_i.
\]
Since each $\lambda_i$ is a root of one,  
$\lambda_1=\cdots=\lambda_n=1$. Hence $\rho_g=\id$. 
\end{proof}

\index{Kernel!of a character}
If $\chi$ is an irreducible character, the subgroup $\ker\chi$ 
is the \textbf{kernel} of $\chi$. 

\begin{theorem}[Correspondence theorem]
\index{Correspondence theorem!for characters}
Let $N$ be a normal subgroup of a finite group $G$. There exists
a bijective correspondence 
\[
\Char(G/N) \longleftrightarrow \{\chi\in\Char(G): 
N\subseteq\ker\chi\}
\]
that maps irreducible characters to irreducible characters.
\end{theorem}

\begin{proof}
If $\widetilde{\chi}\in\Char(G/N)$, let $\chi$ be the lifting of $\widetilde{\chi}$ to $G$. If $n\in N$, 
then
\[
\chi(n)=\widetilde{\chi}(nN)=\widetilde{\chi}(N)=\chi(1)
\]
and thus $N\subseteq\ker\chi$. 

If $\chi\in\Char(G)$ is such that $N\subseteq\ker\chi$, let $\rho\colon G\to\GL(V)$ be a representationn
with character $\chi$. 
Let $\widetilde{\rho}\colon G/N\to\GL(V)$, $gN\mapsto \rho(g)$. We claim that $\widetilde{\rho}$
is well-defined: 
\[
gN=hN\Longleftrightarrow h^{-1}g\in N\Longleftrightarrow\rho(h^{-1}g)=\id\Longleftrightarrow \rho(h)=\rho(g).
\]
Moreover, $\widetilde{\rho}$ is a representation, as 
\[
\widetilde{\rho}((gN)(hN))=\widetilde{\rho}(ghN)=\rho(gh)=\rho(g)\rho(h)=\widetilde{\rho}(gN)\widetilde{\rho}(hN).
\]
If $\widetilde{\chi}$ is the character of $\widetilde{\rho}$, then 
$\widetilde{\chi}(gN)=\chi(g)$.

We now prove that $\chi$ is irreducible if and only if 
$\widetilde{\chi}$ is irreducible. If $U$ is a subspace of $V$, then 
\begin{align*}
\text{$U$ is invariant}
%&\Longleftrightarrow g\cdot U\subseteq U\text{ for all $g\in G$}\\
&\Longleftrightarrow \rho(g)(U)\subseteq U\text{ for all $g\in U$}\\
&\Longleftrightarrow \widetilde{\rho}(gN)(U)\subseteq U\text{ for all $g\in U$}.
\shortintertext{Thus}
\chi\text{ is irreducible }&\Longleftrightarrow
\rho\text{ is irreducible }\\
&\Longleftrightarrow\widetilde{\rho}\text{ is irreducible }\Longleftrightarrow
\widetilde{\chi}\text{ is irreducible }\qedhere.
\end{align*}
\end{proof}

\begin{example}
    Let $G=\Sym_4$ and $N=\{\id,(12)(34),(13)(24),(14)(23)\}$. We know that $N$ is normal in $G$ 
    and that $G/N=\langle a,b\rangle\simeq\Sym_3$, where 
    $a=(123)N$ and $b=(12)N$. 
    The character table of $G/N$ is 
    \begin{center}
		\begin{tabular}{|c|rrr|}
			\hline
			%& $1$ & $3$ & $2$\tabularnewline
			& $1$ & $(12)N$ & $(123)N$ \tabularnewline
			\hline 
			$\widetilde{\chi}_{1}$ & $1$ & $1$ & $1$\tabularnewline
			$\widetilde{\chi}_{2}$ & $1$ & $-1$ & $1$ \tabularnewline
			$\widetilde{\chi}_{3}$ & $2$ & $0$ & $-1$ \tabularnewline
			\hline
		\end{tabular}
	\end{center}
    For each $i\in\{1,2,3\}$ we compute the lifting $\chi_i$ to $G$ of the character  
    $\widetilde{\chi}_i$ of $G/N$. 
    Since $(12)(34)\in N$ and $(13)(1234)=(12)(34)\in N$, 
    \begin{align*}
        \chi( (12)(34) )=\widetilde{\chi}(N),\quad
        \chi( (1234) )=\widetilde{\chi}((13)N)=\widetilde{\chi}((12)N).
    \end{align*}
    Since the characters $\widetilde{\chi_i}$ are irreducibles, 
    the liftings $\chi_i$ are also irreducibles. With this process
    we obtain the following irreducible characters of $G$:
    	\begin{center}
		\begin{tabular}{|c|rrrrr|}
			\hline
			& $1$ & $(12)$ & $(123)$ & $(12)(34)$ & $(1234)$ \tabularnewline
			\hline 
			$\chi_{1}$ & $1$ & $1$ & $1$ & 1 & 1\tabularnewline
			$\chi_{2}$ & $1$ & $-1$ & $1$ & 1 & -1 \tabularnewline
			$\chi_{3}$ & $2$ & $0$ & $-1$ & 2 & 0\tabularnewline
			\hline
		\end{tabular}
	\end{center}
\end{example}

The character table of a group can be used to find the lattice 
of normal subgroups. In particular, the character table detect simple groups. 

\begin{lemma}
    Let $G$ be a finite group and 
    let $g,h\in G$. Then $g$ and $h$ 
    are conjugate if and only if 
    $\chi(g)=\chi(h)$ for all
    $\chi\in\Char(G)$. 
\end{lemma}

\begin{proof}
    If $g$ and $h$ are conjugate, then $\chi(g)=\chi(h)$, as characters are class functions
    of $G$.
    Conversely, if $\chi(g)=\chi(h)$ for all $\chi\in\Char(G)$, then 
    $f(g)=f(h)$ for all class function $f$ of $G$, 
    as characters $G$ generate the space of class functions of $G$. In particular, 
    $\delta(g)=\delta(h)$, where
    \[
    \delta(x)=\begin{cases}
    1 & \text{if $x$ and $g$ are conjugate},\\
    0 & \text{otherwise}.
    \end{cases}
    \]
    This implies that $g$ and $h$ are conjugate.
\end{proof}

As a consequence, we get that 
\begin{equation}
\label{eq:kernels}
\bigcap_{\chi\in\Irr(G)}\ker\chi=\{1\}.
\end{equation}
Indeed, if $g\in\ker\chi$ for all $\chi\in\Irr(G)$, then $g=1$ since 
the lemma implies that $g$ and $1$ are conjugate
because 
$\chi(g)=\chi(1)$ for all $\chi\in\Irr(G)$.

\begin{proposition}
\label{pro:normal}
    Let $G$ be a finite group. 
    If $N$ is a normal subgroup of $G$, 
    then there exist characters
    $\chi_1,\dots,\chi_k\in\Irr(G)$ 
    such that
    \[
    N=\bigcap_{i=1}^k\ker\chi_i.
    \]
\end{proposition}

\begin{proof}
    Apply the previous remark to the group $G/N$ to obtain that 
    \[
    \bigcap_{\widetilde{\chi}\in\Irr(G/N)}\ker\widetilde{\chi}=\{N\}.
    \]
    Assume that $\Irr(G/N)=\{\widetilde{\chi}_1,\dots,\widetilde{\chi}_k\}$. 
    We lift the irreducible characters of $G/N$ to $G$ 
    to obtain (some) irreducible characters $\chi_1,\dots,\chi_k$ 
    of $G$ such that 
    $N\subseteq\ker\chi_1\cap\cdots\cap\ker\chi_k$. 
    If $g\in\ker\chi_i$ for all $i\in\{1,\dots,k\}$, then 
    \[
    \widetilde{\chi}_i(N)=\chi_i(1)=\chi_i(g)=\widetilde{\chi}_i(gN)
    \]
    for all $i\in\{1,\dots,k\}$. This implies that
    \[
    gN\in\bigcap_{i=1}^k\ker\widetilde{\chi}_i=\{N\},
    \]
    that is $g\in N$. 
\end{proof}

\index{Group!simple}
Recall that a non-trivial group is \textbf{simple} if it contains no non-trivial normal 
proper subgroups. Examples of simple groups are cyclic groups of prime order
and the alternating groups $\Alt_n$ for $n\geq5$. 
As a corollary of Proposition \ref{pro:normal}, 
we can use the character table to detect simple groups.

\begin{proposition}
    Let $G$ be a finite group. Then $G$ is not simple if and only if 
    there exists a non-trivial irreducible character $\chi$ such that
    $\chi(g)=\chi(1)$ 
    for some $g\in G\setminus\{1\}$. 
\end{proposition}

\begin{proof}
    If $G$ is not simple, there exists a normal subgroup $N$ of $G$ such that
    $N\ne G$ and $N\ne\{1\}$. 
    By Proposition \ref{pro:normal}, there exist characters 
    $\chi_1,\dots,\chi_k\in\Irr(G)$
    such that 
    $N=\ker\chi_1\cap\cdots\cap\ker\chi_k$.
    In particular, there exists a non-trivial character
    $\chi_i$ such that $\ker\chi_i\ne\{1\}$. Thus 
    there exists $g\in G\setminus\{1\}$ such that
    $\chi_i(g)=\chi_i(1)$. 
    
    Assume now that there exists a non-trivial character $\chi$ 
    such that $\chi(g)=\chi(1)$ for some $g\in G\setminus\{1\}$. In particular, $g\in\ker\chi$ 
    and hence $\ker\chi\ne\{1\}$. Since $\chi$ is non-trivial, $\ker\chi\ne G$. 
    Thus $\ker\chi$ is a proper non-trivial normal subgroup of $G$.
\end{proof}

\begin{example}
\index{Mathieu's group $M_9$}
    If there exists a group $G$ with
    a character table 
    of the form
    \begin{center}
		\begin{tabular}{|c|rrrrrr|}
			\hline
			$\chi_{1}$ & 1 & 1 & 1 & 1 & 1 & 1\tabularnewline
			$\chi_{2}$ & 1 & 1 & 1 & -1 & 1 & -1 \tabularnewline
			$\chi_{3}$ & 1 & 1 & 1 & 1 & -1 & -1\tabularnewline
		    $\chi_{4}$ & 1 & 1 & 1 & -1 & -1 & 1\tabularnewline
			$\chi_{5}$ & 2 & -2 & 2 & 0 & 0 & 0\tabularnewline
			$\chi_{6}$ & 8 & 0 & -1 & 0 & 0 & 0\tabularnewline
			\hline
		\end{tabular}
	\end{center}
	then $G$ cannot be simple. Note that such a group $G$ would have order $\sum_{i=1}^6\chi_i(1)^2=72$. 
	Mathieu's group $M_{9}$ has precisely this character table! 
\end{example}

\begin{example}
    Let $\alpha=\frac{1}{2}(-1+\sqrt{7}i)$. 
    If there exists a group $G$ with a character table
    of the form
    \begin{center}
		\begin{tabular}{|c|rrrrrr|}
			\hline
			$\chi_{1}$ & 1 & 1 & 1 & 1 & 1 & 1\tabularnewline
			$\chi_{2}$ & 7 & -1 & -1 & 1 & 0 & 0 \tabularnewline
			$\chi_{3}$ & 8 & 0 & 0 & -1 & 1 & 1\tabularnewline
		    $\chi_{4}$ & 3 & -1 & 1 & 0 & $\alpha$ & $\overline{\alpha}$ \tabularnewline
			$\chi_{5}$ & 3 & -1 & 1 & 0 & $\overline{\alpha}$ & $\alpha$\tabularnewline
			$\chi_{6}$ & 6 & 2 & 0 & 0 & 0 & 0\tabularnewline
			\hline
		\end{tabular}
	\end{center}    
	then $G$ is simple. Note that such a group $G$ would have order 
	$\sum_{i=1}^6\chi_i(1)^2=168$. 
	The group  
	\[
	\PSL_2(7)=\SL_2(7)/Z(\SL_2(7))
	\]
	is a simple group that has precisely this character table!  
\end{example}



\begin{definition}
\index{Restricción}
Si $U$ es un $K[G]$-módulo y $H$ es un subgrupo de $G$, podemos pensar a $U$ como $K[H]$-módulo al restringir la acción
al subgrupo $H$. Este módulo será denotado por $\Res_H^GU$ y se conoce como la \textbf{restricción} de $U$ a $H$.
\end{definition}

La restricción de un módulo irreducible puede no ser irreducible. 

\begin{example}
    Sea $G=\D_4=\langle r,s:r^4=s^2=1,\,srs=r^{-1}\rangle$ el grupo diedral de ocho elementos. Sea
    $V$ un espacio vectorial con base $\{v_1,v_2\}$. Entonces $V$ es un $\C[\D_4]$-módulo con 
    \[
    r\cdot v_1=v_2,\quad
    r\cdot v_2=-v_1,\quad
    s\cdot v_1=v_1,\quad
    s\cdot v_2=-v_2.
    \]
    El caracter de $V$ es 
    \[
    \chi(g)=\begin{cases}
    2 & \text{si $g=1$},\\
    -2 & \text{si $g=r^2$},\\
    0 & \text{en otro caso}.
    \end{cases}
    \]
    Observemos que $\chi$ es irreducible, pues $\langle\chi,\rangle\chi=1$.
    Sea 
    $H=\langle r^2,s\rangle=\{1,r^2,s,r^2s\}$. Entonces $\Res_H^GV$ es $V$ como $\C[H]$-módulo con
    \[
    r^2\cdot v_1=-v_1,\quad
    r^2\cdot v_2=-v_1,\quad
    s\cdot v_1=-v_1,\quad
    s\cdot v_2=-v_2.
    \]
    El caracter de $\Res_H^GV$ es
    \[
    \chi_H(h)=\chi|_H(h)
    =\begin{cases}
    2 & \text{si $h=1$},\\
    -2 & \text{si $h=r^2$},\\
    0 & \text{en otro caso}.
    \end{cases}
    \]
    El carater $\chi_H$ no es irreducible ya que $\langle\chi_H,\chi_H\rangle=0$. 
\end{example}

\index{Parte irreducible!de un caracter}
Sea $H$ un subgrupo de $G$ y supogamos que $\Irr(H)=\{\phi_1,\dots,\phi_l\}$.
Si $\chi\in\Char(G)$, entonces
\[
\chi|_H=\sum_{i=1}^ld_i\phi_i
\]
para ciertos enteros $d_1,\dots,d_l\geq 0$. 
Cada $\phi_i$ tal que $d_i=\langle\chi|_H,\phi_i\rangle\ne 0$ 
es una \textbf{parte irreducible} del caracter $\chi|_H$ y esos
$\phi_i$ son las \textbf{partes irreducibles que constituyen} al caracter $\chi|_H$. 

\begin{proposition}
    Si $H$ es un subgrupo de $G$ y $\phi\in\Char(H)$, 
    entonces $\chi\in\Irr(G)$ tal que $\langle\chi|_H,\phi\rangle_H\ne 0$.
\end{proposition}

\begin{proof}
    Supongamos que $\Irr(G)=\{\chi_1,\dots,\chi_k\}$. 
    Sabemos que si $L$ es la representación regular de $G$, entonces
    \[
    \chi_L(g)=\begin{cases}
    |G| & \text{si $g=1$},\\
    0 & \text{en otro caso}.
    \end{cases}
    \]
    Si escribimos $\chi_L=\sum_{i=1}^k\chi_i(1)\chi_i$, entonces, como
    \[
    0\ne \frac{|G|}{|H|}\phi(1)=\langle \chi_L|_H,\phi\rangle_H=\sum_{i=1}^k\chi_i(1)\langle\chi_i|_H,\phi\rangle_H,
    \]
    existe algún $i\in\{1,\dots,k\}$ 
    tal que $\langle\chi_i|_H,\phi\rangle_H\ne 0$. 
\end{proof}

\begin{proposition}
    Sean $H$ un subgrupo de $G$ y $\chi\in\Irr(G)$. Si $\Irr(H)=\{\phi_1,\dots,\phi_l\}$, entonces
    \[
    \chi|_H=\sum_{i=1}^ld_i\phi_i,
    \]
    donde $\sum_{i=1}^l d_i^2\leq (G:H)$. Más aún, $\sum_{i=1}^l d_i^2=(G:H)$ 
    si y sólo si $\chi(g)=0$ para todo $g\in G\setminus H$. 
\end{proposition}

\begin{proof}
Como 
\[
\sum_{i=1}^ld_i^2=\langle\chi|_H,\chi|_H\rangle_H=\frac{1}{|H|}\sum_{h\in H}\chi(h)\overline{\chi(h)}.
\]
Además, como $\chi$ es irreducible, 
\begin{align*}
1=\langle\chi,\chi\rangle_G&=\frac{1}{|G|}\sum_{g\in G}\chi(g)\overline{\chi(g)}\\
&=\frac{1}{|G|}\sum_{h\in H}\chi(h)\overline{\chi(h)}
+\frac{1}{|G|}\sum_{g\in G\setminus H}\chi(g)\overline{\chi(g)}\\
&=\frac{|H|}{|G|}\sum_{i=1}^l d_i^2+\frac{1}{|G|}\sum_{g\in G\setminus H}\chi(g)\overline{\chi(g)}.
\end{align*}
Como $\sum_{g\in G\setminus H}\chi(g)\overline{\chi(g)}\geq0$, se concluye que $\sum_{i=1}^ld_i^2\leq(G:H)$. Además
vale la igualdad si y sólo si $\sum_{g\in G\setminus H}\chi(g)\overline{\chi(g)}=0$, 
es decir si sólo si $\chi(g)=0$ para todo $g\in G\setminus H$.
\end{proof}

\index{Bimódulo}
Discutiremos ahora la inducción de módulos. Para eso, repasaremos algunas nociones básicas sobre
\textbf{bimódulos} y \textbf{producto tensorial de bimódulos}. 
Si $R$ y $S$ son anillos, un grupo abeliano $M$ se dirá un $(R,S)$-bimódulo si 
$M$ es un $R$-módulo a izquierda, $M$ es un $S$-módulo a derecha y además
\[
r\cdot (m\cdot s)=(r\cdot m)\cdot s
\]
para todo $r\in R$, $s\in S$ y $m\in M$. 

\begin{examples}\
\begin{enumerate}
    \item Un $R$-módulo a izquierda es un $(R,\Z)$-bimódulo.
    \item Un $S$-módulo a derecha es un $(\Z,S)$-bimódulo.
    \item Todo anillo $R$ es un $(R,R)$-bimódulo.
%    \item Si $R$ es un anillo conmutativo...
\end{enumerate}
\end{examples}

\begin{example}
Si $M$ es un $(R,S)$-bimódulo y $N$ es un $R$-módulo, entonces el conjunto 
$\Hom_R(M,N)$ de morfismos de $R$-módulos $M\to N$ es un 
$S$-módulo con 
\[
(s\cdot \varphi)(m)=\varphi(m\cdot s),\quad s\in S,\,\varphi\in\Hom_R(M,N),\,m\in M.
\]
\end{example}

Sean $M$ un $(R,S)$-bimódulo, $N$ un $S$-módulo y $U$ un $R$-módulo. 
Diremos que una función $f\colon M\times N\to U$ 
es \textbf{balanceada} si 
\begin{align*}
    &f(m_1+m_2,n)=f(m_1,n)+f(m_2,n),\\
    &f(m,n_1+n_2)=f(m,n_1)+f(m,n_2),\\
    &f(m\cdot s,n)=f(m,s\cdot n),\\
    &f(r\cdot m,n)=r\cdot f(m,n)
\end{align*}
para todo $m,m_1,m_2\in M$, $n,n_1,n_2\in N$, $r\in R$ y $s\in S$. 

\begin{example}
Si $M$ es un $R$-módulo, la función $f\colon R\times M\to M$, $(r,m)\mapsto r\cdot m$, es balanceada. 
\end{example}

\index{Producto tensorial!de bimódulos}
Sean $M$ un $(R,S)$-bimódulo, $N$ un $S$-módulo y $U$ un $R$-módulo. 
Se define el \textbf{producto tensorial} $M\otimes_S N$ es un $R$-módulo provisto con una función balanceada 
$\eta\colon M\times N\to M\otimes_S N$ que cumple con la siguiente propiedad universal: 
\begin{quote}
Si $f\colon M\times N\to U$ es una función balanceada, entonces
existe un único morfismo de $R$-módulos $\alpha\colon M\otimes_S N\to U$ tal que $f=\alpha\circ\eta$. 
\end{quote}
Notación: $m\otimes n=\eta(m,n)$ para $m\in M$ y $n\in N$.
El producto tensorial existe y puede demostrarse que es único salvo isomorfismos. Más precisamente, $M\otimes_S N$
se define como el $R$-módulo generado por
el conjunto $\{m\otimes n:m\in M,\,n\in N\}$, donde los $m\otimes n$ satisfacen 
las siguientes identidades:
\begin{align}
    &(m+m_1)\otimes n=m\otimes n+m_1\otimes n &&\text{$m,m_1\in M$, $n\in N$},\\
    &m\otimes(n+n_1)=m\otimes n+m\otimes n_1 &&\text{$m\in M$, $n,n_1\in N$},\\
    &(ms)\otimes n=m\otimes (sn) &&\text{$m\in M$, $n\in N$, $s\in S$},\\
    &(rm)\otimes n=r(m\otimes n) &&\text{$m\in M$, $n\in N$, $r\in R$}.
\end{align}
Un elemento arbitrario de $M\otimes_S N$ es una suma finita
de la forma 
$\sum_{i=1}^k m_i\otimes n_i$,
donde $m_1,\dots,m_k\in M$ y $n_1,\dots,n_k\in N$, y no necesariamente un tensor elemental $m\otimes n$. 

\begin{example}
$M\simeq R\otimes_R M$ como $R$-módulos. Como la función $R\times M\to M$, $(r,m)\mapsto r\cdot m$, es balanceada, 
induce un morfismo $R\otimes_R M\to M$, $r\otimes m\mapsto r\cdot m$ con inversa $M\to R\otimes_R M$, $m\mapsto 1\otimes m$. 
\end{example}

\begin{example}
Si $M_1,\dots,M_k$ son $(R,S)$-bimódulos y $N$ es un $S$-módulo, entonces
\[
(M_1\oplus\cdots\oplus M_k)\otimes_S N\simeq (M_1\otimes_S N)\oplus\cdots\oplus (M_k\otimes_S N).
\]
\end{example}

Algunos ejercicios:

\begin{exercise}
    Demuestre que $M\otimes_RN\simeq N\otimes_{R^{\op}}M$.
\end{exercise}

\begin{exercise}
    Demuestre que $\Z/n\otimes_{\Z}\Q=\{0\}$.
\end{exercise}

\begin{exercise}
    Sean $M$ un $(R,S)$-bimódulo y $N$ un $(S,T)$-bimódulo. 
    Demuestre que $M\otimes_SN$ es un $(R,T)$-bimódulo 
    con $r(m\otimes n)t=(rm)\otimes (nt)$, 
    donde $m\in M$, $n\in N$, $r\in R$, $t\in T$.
\end{exercise}

\begin{exercise}
    Demuestre que $(M\otimes_R N)\otimes_RT\simeq M\otimes_R (N\otimes_RT)$.
\end{exercise}

\begin{exercise}
    Enuncie y demuestre la asociatividad del producto tensorial de bimódulos. 
\end{exercise}

% Atiyah-Mac Donald
% https://math.stackexchange.com/questions/2586211/associativity-of-tensor-products

Si $G$ es un grupo finito, $H$ es un subgrupo de $G$
y $V$ es un $K[H]$-módulo, entonces 
$K[G]$ es un $(K[G],K[H])$-bimódulo.

\begin{definition}
\index{Módulo!inducido}
Sea $G$ un grupo finito y sea 
$H$ un subgrupo de $G$. 
Si $V$ es un $K[H]$-módulo de $G$, 
se define el $K[G]$-módulo \textbf{inducido} de $V$ 
como
\[
\Ind_H^GV=K[G]\otimes_{K[H]}V.
\]
\end{definition}

\index{Transversal}
Si $H$ es un subgrupo de $G$, un \textbf{transversal} (a izquierda) 
de $H$ en $G$ es un subconjunto $T$ de $G$ que contiene exactamente un elemento de cada coclase (a izquierda) 
de $H$ en $G$. 

\begin{example}
Si $G=\Sym_3$ y $H=\{\id,(12)\}$, entonces
$T=\{\id,(123),(23)\}$ es un transversal de $H$ en $G$. Podemos descomponer 
a $G$ como
\[
G=\{\id,(12)\}\cup \{(123),(13)\}\cup\{(132),(23)\}=\bigcup_{t\in T}tH.
\]
Como cada $g\in G$ se escribe en forma única como $g=th$ para $t\in T$ y $h\in H$, podemos 
definir una transformación lineal 
$\varphi\colon K[G]\to K[H]\oplus K[H]\oplus K[H]=|T|K[H]$, que para $g=th$ devuelve $h$ en el lugar que corresponde a $t\in T$, es decir
\begin{align*}
\id&\mapsto (\id,0,0), && (12)\mapsto ((12),0,0), && (123)\mapsto (0,\id,0),\\
(23)&\mapsto (0,0,\id), && (13)\mapsto (0,(12),0), && (132)\mapsto (0,0,(12)).
\end{align*}
Por ejemplo, 
\[
\varphi( 5(12)-3(123)+7\id )=(7\id+5(12),-3\id,0).
\]
Es importante observar que $\varphi$ es un isomorfismo de $K[H]$-módulos (a derecha). 
\end{example}

La observación hecha en el ejemplo anterior es la clave del siguiente resultado.

\begin{proposition}
Sea $G$ un grupo finito y sea 
$H$ un subgrupo de $G$. Si $V$ es un $K[H]$-módulo de $G$, entonces 
\[
    \Ind_H^G(V)=\bigoplus_{t\in T}t\otimes V,
\]
donde $T$ es un transversal de $H$ en $G$ y $t\otimes V=\{t\otimes v:v\in V\}$. En particular, 
$\dim\Ind_H^GV=(G:H)\dim V$.
\end{proposition}

\begin{proof}
Descomponemos a $G$ como unión disjunta de coclases de $H$ con el transversal $T$, es decir
\[
G=\bigcup_{t\in T}tH.
\]
Cada $g\in G$ se escribe entonces unívocamente como $g=th$ con $t\in T$ y $h\in H$. Tal como 
hicimos en el ejemplo anterior, esto nos permite obtener un isomorfismo 
$\varphi\colon K[G]\to |T|K[H]$ de $K[H]$-módulos (a derecha), donde $\varphi(g)$ es $h$ en el sumando que corresponde a $t\in T$
y es cero en el resto de los sumandos. Luego
\[
\Ind_H^GV=K[G]\otimes_{K[H]}V\simeq (|T|K[H])\otimes_{K[H]}V\simeq |T|(K[H]\otimes_{K[H]}V)\simeq |T|V
\]
como $K[H]$-módulos. En particular, $\dim\Ind_H^GV=|T|\dim V$. 

Si escribimos $g=th$ con $t\in T$ y $h\in H$, entonces $g\otimes v=(th)\otimes v=t\otimes h\cdot v\in t\otimes V$. 
Luego $K[G]\otimes_{K[H]}V\subseteq \oplus_{t\in T}t\otimes V$. La otra inclusión es trivial. Por definición, 
la suma sobre $t\in T$ de los $t\otimes V$ es directa. 
\end{proof}

\topic{Frobenius's reciprocity}

\begin{theorem}[Reciprocidad de Frobenius]
\index{Teorema!de reciprocidad de Frobenius}
Sea $G$ un grupo finito y $H$ un subgrupo de $G$. 
Si $U$ es un $K[G]$-módulo y $V$ es un $K[H]$-módulo, entonces
\[
\Hom_{K[H]}(V,\Res_H^GU)\simeq \Hom_{K[G]}(\Ind_H^GV,U)
\]
como espacios vectoriales.
\end{theorem}

\begin{proof}
Si $\varphi\in\Hom_{K[H]}(V,\Res_H^GU)$, sea 
\[
f_{\varphi}\colon K[G]\times V\to U,
\quad
(g,v)\mapsto g\cdot\varphi(v).
\]
Veamos que $f_{\varphi}$ es balanceada. Un cálculo directo muestra que
\begin{align*}
    &f_{\varphi}(g+g_1,v)=f_{\varphi}(g,v)+f_{\varphi}(g_1,v),&&
    f_{\varphi}(g,v+w)=f_{\varphi}(g,v)+f_{\varphi}(g,w).
\end{align*}
Como $\varphi$ es morfismo de $K[H]$-módulos,
\begin{align*}
    &f_{\varphi}(gh,v)=(gh)\cdot\varphi(v)
    =g\cdot (h\cdot \varphi(v))
    =g\cdot (h\cdot\varphi(v))
    =g\cdot \varphi(h\cdot v)=f_{\varphi}(g,h\cdot v)
\end{align*}
para todo $g\in G$, $h\in H$ y $v\in V$. Por último,
\begin{align*}
    &f_{\varphi}(gg_1,v)=(gg_1)\cdot\varphi(v)=g\cdot(g_1\cdot\varphi(v))=g\cdot f_{\varphi}(g_1,v)
\end{align*}
para todo $g,g_1\in G$ y $v\in V$. Para cada $\varphi\in\Hom_{K[H]}(V,\Res_H^GU)$ tenemos 
entonces un $\Gamma(\varphi)\in\Hom_{K[G]}(\Ind_H^GV,U)$ tal que
$\Gamma(\varphi)(g\otimes v)=g\cdot\varphi(v)$. 
Tenemos así definida una función 
\[
\Gamma\colon \Hom_{K[H]}(V,\Res_H^GU)\to\Hom_{K[G]}(\Ind_H^GV,U),
\quad
\varphi\mapsto\Gamma(\varphi).
\]

La función $\Gamma$ es lineal e inyectiva, ambas afirmaciones fáciles de verificar. 

Es también sobreyectiva, pues si $\theta\in\Hom_{K[H]}(\Ind_H^GV,U)$, entonces
la función $\varphi(v)=\theta(1\otimes v)$ es tal que $\varphi\in\Hom_{K[H]}(V,\Res_H^GU)$ y 
cumple 
\[
\Gamma(\varphi)(g\otimes v)=g\cdot\varphi(v)=g\cdot\theta(1\otimes v)=\theta(g\otimes v).\qedhere
\]
\end{proof}

Supongamos ahora que $K=\C$. 

Sea $H$ un subgrupo de $G$. Si $U$ es un $\C[G]$-módulo con caracter $\chi$, el caracter de $\Res_H^GU$ se denota por $\chi|_H$ y vale que 
que $\chi|_H(1)=\chi(1)$. Si $V$ es un $\C[H]$-módulo con 
caracter $\phi$, el módulo $\Ind_H^GV$ tiene caracter $\phi^G$ y vale que $\phi^G(1)=(G:H)\phi(1)$. 
\begin{align*}
\langle \phi,\chi|_H\rangle_H 
&=\dim\Hom_{\C[H]}(V,\Res_H^GU)
=\dim\Hom_{\C[G]}(\Ind_H^GV,U)
=\langle\phi^G,\chi\rangle_G,
\end{align*}
donde $\langle \alpha,\beta\rangle_X=\sum_{x\in X}\alpha(x)\overline{\beta(x)}$ denota el producto 
interno del espacio de funciones $X\to\C$. 

\begin{definition}
Si $\Irr(G)=\{\chi_1,\dots,\chi_k\}$ e $\Irr(H)=\{\phi_1,\dots,\phi_l\}$, se define
la \textbf{matriz de inducción--restricción} como la matriz $(c_{ij})\in\C^{l\times k}$, donde
\[
c_{ij}=\langle \phi_i^G,\chi_j\rangle_G=\langle\phi_i,\chi_j|_H\rangle_H.
\]
\end{definition}

La fila $i$-ésima de la matriz de inducción--restricción da la multiplicidad con que el caracter $\chi_j$ aparece
en la descomposición de $\phi_i^G$. La columna $j$-ésima da la multiplicidad con que el caracter $\phi_i$ aparece 
en la descomposición de $\chi_j|H$.

\begin{example}
Sea $G=\Sym_3$. 
La tabla de caracteres de $G$ es 
	\begin{center}
		\begin{tabular}{|c|rrr|}
			\hline
			& $1$ & $3$ & $2$\tabularnewline
			& $1$ & $(12)$ & $(123)$ \tabularnewline
			\hline 
			$\chi_{1}$ & $1$ & $1$ & $1$\tabularnewline
			$\chi_{2}$ & $1$ & $-1$ & $1$ \tabularnewline
			$\chi_{3}$ & $2$ & $0$ & $-1$ \tabularnewline
			\hline
		\end{tabular}
	\end{center}
La tabla de caracteres del subgrupo 
$H=\{\id,(12)\}$ es 
\begin{center}
\begin{tabular}{|c|rr|}
\hline 
& $1$ & $1$ \tabularnewline
& $\id$ & $(12)$ \tabularnewline
\hline 
$\phi_{1}$ & $1$ & $1$ \tabularnewline
$\phi_{2}$ & $1$ & $-1$\tabularnewline
\hline
\end{tabular}
\end{center}
A simple vista vemos que $\chi_1|_H=\phi_1$, $\chi_2|_H=\phi_2$ y que $\chi_3|_H=\phi_1+\phi_2$. 
La matriz de inducción--restricción es entonces
\[
\begin{pmatrix}
1 & 0 & 1\\
0 & 1 & 1
\end{pmatrix}.
\]
Observemos que además $\phi_1^G=\chi_1+\chi_3$ y que $\phi_2^G=\chi_2+\chi_3$. 
\end{example}

Veamos cómo calcular explícitamente caracteres inducidos. 

\begin{proposition}
Sea $H$ un subgrupo de $G$ y sea $V$ es un $\C[H]$-módulo con caracter $\chi$. Si 
$T$ es un trasversal de $H$ en $G$, entonces
\[
\chi^G(g)=\sum_{\substack{t\in T\\t^{-1}gt\in H}}\chi(t^{-1}gt)
\]
para todo $g\in G$. 
\end{proposition}

\begin{proof}
    Sabemos que $\Ind_H^GV=\oplus_{t\in T}t\otimes V$. 
    Supongamos que $T=\{t_1,\dots,t_m\}$ 
    y sea $\{v_1,\dots,v_n\}$ una base de $V$. 
    Entonces $\{t_i\otimes v_k:1\leq i\leq m,\,1\leq k\leq n\}$ es 
    una base de $\Ind_H^GV$ y la acción
    de $g$ en $\Ind_H^GV$ está dada por
    \[
    \rho^G(g)=\begin{cases}
    \rho(t_j^{-1}gt_i) & \text{si $t_j^{-1}gt_i\in H$},\\
    0 & \text{en otro caso}.
    \end{cases}
    \]
    En efecto, si $gt_i=t_jh$ para $h\in H$ y ciertos $i,j$, entonces 
    \[
    g\cdot (t_i\otimes v_k)=gt_i\otimes v_k=t_jh\otimes v_k=t_j\otimes h\cdot v_k
    \]
    y además $gt_i=t_jh$ si y sólo si $t_j^{-1}gt_i=h\in H$. Se concluye entones
    que $g$ actúa como $t^{-1}gt$ en $V$ en caso en que $t^{-1}gt\in H$ y 
    como la transformación nula en otro caso. 
\end{proof}

\begin{corollary}
\label{cor:induccion}
    Sea $H$ un subgrupo de $G$ 
    y sea $V$ es un $\C[H]$-módulo con caracter $\chi$.
    Si $g\in G$, entonces
    \[
    \chi^G(g)=\frac{1}{|H|}\sum_{\substack{x\in G\\x^{-1}gx\in H}}\chi(x^{-1}gx).
    \]
\end{corollary}

\begin{proof}
    Sea $T$ un transversal de $H$ en $G$. Si $x\in G$, escribimos $x=th$ para $t\in T$ y $h\in H$. 
    Como $x^{-1}gx=h^{-1}(t^{-1}gt)h$, entonces $x^{-1}gx\in H\Longleftrightarrow t^{-1}gt\in H$ y además, en ese caso, 
    $\chi(x^{-1}gx)=\chi(t^{-1}gt)$ pues $\chi$ es una función de clases. Eso implica que existen $|H|$ elementos $x\in G$ 
    tales que $x^{-1}gx\in H$. Para esos $x$, se tiene $\chi(x^{-1}gx)=\chi(t^{-1}gt)$, lo que implica 
    el corolario. 
\end{proof}


\topic{Frobenius' groups}
\label{Frobenius}

Recordemos que si $p$ es un número primo, entonces
las unidades $(\Z/p)^{\times}$ 
de $\Z/p$ forman un grupo con la multiplicación. Más aún, 
$(\Z/p)^{\times}$ 
es un grupo ciclico de orden $p-1$. 

Sean $p$ y $q$ números primos tales que $q$ divide a $p-1$ y sea
\[
G=\left\{\begin{pmatrix}
x & y\\
0 & 1
\end{pmatrix}
:x\in(\Z/p)^\times,\,y\in\Z/p\right\}.
\]
Es sencillo verificar que $G$ es un grupo con la multiplicación usual de matrices y que
$|G|=p(p-1)$. Sea $z\in\Z$ un elemento de orden $q$ módulo $p$ y sean  
\[
a=\begin{pmatrix}
1&1\\
0&1
\end{pmatrix},
\quad
b=\begin{pmatrix}
z&1\\
0&1
\end{pmatrix},
\quad
H=\langle a,b\rangle.
\]
Un cálculo directo muestra
que 
\begin{equation}
\label{eq:pq}
a^p=b^q=\begin{pmatrix}
1&0\\
0&1
\end{pmatrix},
\quad
bab^{-1}=\begin{pmatrix}
1&z\\
0&1
\end{pmatrix}
=a^z.
\end{equation}
Todo elemento de $H$ es de la forma $a^ib^j$ para $i\in\{0,\dots,p-1\}$ y $j\in\{0,\dots,q-1\}$. 
Luego $|H|=pq$ y además las relaciones~\eqref{eq:pq} nos permiten 
calcular completamente la tabla de multiplicación de $G$. 

\begin{exercise}
Sean $p$ y $q$ dos primos tales que $q\mid p-1$. Sean $u,v\in\Z$ de orden $q$ módulo $p$. 
Demuestre que
\[
\langle a,b:a^p=b^q=1,bab=a^u\rangle
\simeq \langle a,b:a^p=b^q=1,bab=a^v\rangle.
\]
\end{exercise}

El grupo  
\[
F_{p,q}=\langle a,b:a^p=b^q=1,bab=a^u\rangle,
\]
donde $u\in\Z$ tiene orden $q$ módulo $p$, 
es un caso particular de 
\emph{grupo de Frobenius}. 

\begin{proposition}
    Sean $p$ y $q$ números primos tales que $p>q$ y 
    sea $G$ un grupo de orden $pq$. Entonces $G$ es abeliano o bien 
    $q\mid p-1$ y 
    $G\simeq F_{p,q}$.
\end{proposition}

\begin{proof}
    Supongamos que $G$ es no abeliano. Los teoremas de Sylow implican que 
    $q$ divide a $p-1$ y que además 
    existe un único $p$-subgrupo de Sylow $P$ de $G$. Sean $a,b\in G$ tales que
    $P=\langle a\rangle\simeq\Z/p$ y $G/P=\langle bP\rangle\simeq\Z/q$. Por el teorema
    de Lagrange, $G=\langle a,b\rangle$. Calculemos el orden de $b^q$. Como 
    $G$ no es cíclico (pues es no abeliano) y $b^q\in P$, se concluye que $|b^q|=q$. 
    Como $P$ es normal en $G$, 
    $bab^{-1}\in P$ y entonces $bab^{-1}=a^z$ para algún $z\in\Z$. Luego $b^qab^{-q}=a^{z^q}$, lo que
    implica que $z^q\equiv1\bmod p$. El orden de $u$ en $(\Z/p)^{\times}$ divide entonces al primo $q$ y 
    luego es igual a $q$, pues de lo contrario, $u=1$ y entonces $bab^{-1}=a$, lo que implicaría que $G$ es abeliano.
    En conclusion, $G\simeq F_{p,q}$. 
\end{proof}

La proposición anterior nos permite demostrar, por ejemplo, 
que todo grupo de orden 15 es abeliano
y que, salvo isomorfismos, $\Z/20$ y 
$F_{5,4}$ son los 
únicos grupos de orden 20.

\begin{definition}
  \index{Frobenius!complemento de}
  \index{Frobenius!núcleo de}
  \index{Frobenius!grupo de}
  Diremos que un grupo $G$ es un 
  \textbf{grupo de Frobenius} si $G$ 
  tiene un subgrupo propio no trivial $H$ tal que $H\cap
  xHx^{-1}=\{1\}$ para todo $x\in G\setminus H$. En este caso, el
  subgrupo $H$ se llama \textbf{complemento de Frobenius}.
\end{definition}

\begin{theorem}[Frobenius]
  \label{theorem:Frobenius}
  \index{Frobenius!Teorema de}
  \index{Teorema!de Frobenius}
  Sea $G$ un grupo de Frobenius con complemento $H$. Entonces
  \[
	N=\left( G\setminus\bigcup_{x\in G}xHx^{-1}\right)\cup\{1\}
  \]
  es un subgrupo normal de $G$.
\end{theorem}

\begin{proof}
  Para cada $\chi\in\Irr(H)$, $\chi\ne1_H$ definimos
  $\alpha=\chi-\chi(1)1_H\in\cf(H)$, donde $1_H$ denota el caracter trivial de $H$. 

  Demostremos que $(\alpha^G)_H=\alpha$.
  Primero, $\alpha^G(1)=\alpha(1)=0$. Si $h\in H\setminus\{1\}$, entonces, gracias al corolario~\ref{cor:induccion}, 
  \[
    \alpha^G(h)=\frac{1}{|H|}\sum_{\substack{x\in G\\x^{-1}hx\in H}}\alpha(x^{-1}hx)
    =\frac{1}{|H|}\sum_{x\in H}\alpha(h)=\alpha(h),
  \]
  pues si $x\not\in H$, entonces, como $x^{-1}hx\in H$, se tiene que $h\in H\cap xHx^{-1}=\{1\}$.

  Por la reciprocidad de Frobenius, 
  \begin{equation}
    \label{eq:<a,a>=1+chi2}
    \langle\alpha^G,\alpha^G\rangle
    =\langle\alpha,(\alpha^G)_H\rangle=\langle\alpha,\alpha\rangle
    =1+\chi(1)^2.
  \end{equation}
  Nuevamente por la reciprocidad de Frobenius, 
  \[
  \langle\alpha^G,1_G\rangle
  =\langle\alpha,(1_G)_H\rangle
  =\langle\alpha,1_H\rangle
  =\langle\chi-\chi(1)1_H,1_H\rangle
  =-\chi(1),
  \]
  donde $1_G$ denota al caracter trivial de $G$. Si escribimos
  \[
  \alpha^G=\sum_{\eta\in\Irr(G)}\langle\alpha^G,\eta\rangle\eta
  =\langle\alpha^G,1_G\rangle1_G+\underbrace{\sum_{\substack{1_G\ne\eta\\\eta\in\Irr(G)}}\langle\alpha^G,\eta\rangle\eta}_{\phi}
  \]
  entonces $\alpha^G=-\chi(1)1_G+\phi$, donde $\phi$ es una 
  combinación lineal entera de caracteres irreducibles no triviales de $G$. 
  Calculamos además
  \[
  1+\chi(1)^2=\langle\alpha^G,\alpha^G\rangle
  =\langle\phi-\chi(1)1_G,\phi-\chi(1)1_G\rangle
  =\langle\phi,\phi\rangle+\chi(1)^2
  \]
  y luego $\langle\phi,\phi\rangle=1$. 
  
  \begin{claim}
  Si $\eta\in\Irr(G)$ es tal que $\eta\ne 1_G$, entonces $\langle\alpha^G,\eta\rangle\in\Z$. 
  \end{claim}
  
  En efecto, por la reciprocidad de Frobenius, $\langle\alpha^G,\eta\rangle=\langle\alpha,\eta_H\rangle$. 
  Si descomponemos a $\eta_H$ en irreducibles de $H$, digamos
  \[
  \eta_H=m_11_H+m_2\chi+m_3\theta_3+\cdots+m_t\theta_t
  \]
  para ciertos $m_1,m_2,\dots,m_t\geq0$, 
  entonces, como
  \begin{align*}
  \langle\alpha,1_H\rangle=\langle\chi-\chi(1)1_H,1_H\rangle=-\chi(1),
  &&\langle\alpha,\chi\rangle=\langle\chi-\chi(1)1_H,\chi\rangle=1,
  \end{align*}
  y además 
  \[
  \langle\alpha,\theta_j\rangle=\langle\chi-\chi(1)1_H,\theta_j\rangle=0
  \]
  para todo $j\in\{3,\dots,t\}$, se concluye que
  \[
  \langle\alpha^G,\eta\rangle=-m_1\chi(1)+m_2\in\Z.
  \]
  
  \begin{claim}
  $\phi\in\Irr(G)$.
  \end{claim}
  
  Como $\langle\alpha^G,\eta\rangle\in\Z$ para todo $\eta\in\Irr(G)$ tal que $\eta\ne 1_G$ y además 
  \[
  1=\langle\phi,\phi\rangle
  =\sum_{\substack{\eta,\theta\in\Irr(G)\\\eta,\theta\ne1_G}}\langle\alpha^G,\eta\rangle\langle\alpha^G,\theta\rangle\langle\eta,\theta\rangle
  =\sum_{\substack{\eta\ne 1_G\\\eta\in\Irr(G)}}\langle\alpha^G,\eta\rangle^2,
  \]
  entonces existe un único $\eta\in\Irr(G)$ tal que 
  $\langle\alpha^G,\eta\rangle^2=1$ y el resto de los productos es cero, es decir 
  $\alpha^G=\pm\eta$ para un cierto $\eta\in\Irr(G)$. Como además 
  \[
  \chi-\chi(1)1_H=\alpha=(\alpha^G)_H=(\phi-\chi(1)1_G)_H=\phi_H-\chi(1)1_H,
  \]
  se tiene que $\phi(1)=\phi_H(1)=\chi(1)\in\N$. Luego $\phi\in\Irr(G)$. 

  \medskip
  Observemos que hemos demostrado que si $\chi\in\Irr(H)$ es tal que $\chi\ne 1_H$, entonces
  existe $\phi_\chi\in\Irr(G)$ tal que $(\phi_\chi)_H=\chi$. 
  
  \medskip
  Vamos a demostrar que $N$ es igual a
  \[
	M=\bigcap_{\substack{\chi\in\Irr(H)\\\chi\ne1_H}}\ker\phi_{\chi}.
  \]

  Demostremos primero que $N\subseteq M$. 
  Sea $n\in N\setminus\{1\}$ y sea $\chi\in\Irr(H)\setminus\{1_H\}$. Como $n$ no pertenece
  a ningún conjugado de $H$, 
  \[
	\alpha^G(n)=\frac{1}{|H|}\sum_{\substack{x\in G\\x^{-1}nx\in H}}\chi(x^{-1}nx)=0
  \]
  pues como $n\in N$ el conjunto $\{x\in G:x^{-1}nx\in H\}$ es vacío. Como entonces 
  \[
  0=\alpha^G(n)
  =\phi_{\chi}(n)-\chi(1)=\phi_{\chi}(n)-\phi_{\chi}(1),
  \]
  se concluye que $n\in\ker\phi_{\chi}$. 
  
  Demostremos ahora que $M\subseteq N$. 
  Sea $h\in M\cap H$ y sea $\chi\in\Irr(H)\setminus\{1_H\}$. Entonces
  \[
    \phi_{\chi}(h)-\chi(1)=\alpha^G(h)=\alpha(h)=\chi(h)-\chi(1),
  \]
  y luego $h\in\ker\chi$ pues 
  \[
    \chi(h)=\phi_{\chi}(h)=\phi_{\chi}(1)=\chi(1).
  \]
  Por lo tanto $h\in\cap_{\chi}\ker\chi=\{1\}$, que que vimos en la fórmula~\eqref{eq:kernels} que
  la intersección de los núcleos de los irreducibles es trivial. Demostremos ahora que $M\cap
  xHx^{-1}=\{1\}$ para todo $x\in G$. Sean $x\in G$ y $m\in M\cap xHx^{-1}$. Como
  $m=xhx^{-1}$ para algún $h\in H$, $x^{-1}mx\in H\cap M=\{1\}$.  Esto implica que
  $m=1$.
\end{proof}

No se conoce una demostración del teorema de Frobenius que no use teoría de caracteres. 

\begin{definition}
  \index{Frobenius!núcleo de}
  Sea $G$ un grupo de Frobenius. El subgrupo normal
  $N$ construido en el teorema de Frobenius se llama \textbf{núcleo de
  Frobenius}.
\end{definition}

\begin{corollary}
  Sea $G$ un grupo de Frobenius con complemento $H$. 
  Entonces existe un subgrupo normal $N$ de $G$ tal que
  $G=HN$, $H\cap N=\{1\}$.
\end{corollary}

\begin{proof}
  La existencia del subgrupo normal $N$ está garantizada por el
  teorema de Frobenius. Demostremos que $H\subseteq N_H(H)$. Si $h\in
  H\setminus\{1\}$ y $g\in G$ son tales que $ghg^{-1}\in H$, entonces $h\in
  g^{-1}Hg\cap H$ y luego $g\in H$. Como entonces $H=N_G(H)$, el subgrupo $H$
  tiene $(G:H)$ conjugados y luego $|G|=|H||N|$ pues 
  \[
    |N|=|G|-(G:H)(|H|-1)=(G:H).
  \]
  Como $N\cap H=\{1\}$, entonces 
  \[
  |HN|=|N||H|/|H\cap N|=|N||H|=|G|
  \]
  y luego $G=NH$.
\end{proof}

\begin{corollary}[Teorema de Frobenius, versión combinatoria]
  \label{corollary:Frobenius_combinatorio}
  \index{Frobenius!Teorema de}
  \index{Teorema!de Frobenius}
  Sea $X$ un conjunto finito y sea $G$ un grupo que actúa transitivamente en
  $X$. Supongamos que todo $g\in G\setminus\{1\}$ fija a lo sumo un punto de
  $X$. El conjunto $N$ formado por la identidad y las permutaciones que mueven
  todos los puntos de $X$ es un subgrupo de $G$.
\end{corollary}

\begin{proof}
  Sea $x\in X$ y sea $H=G_x$. Veamos que si $g\in G\setminus H$ entonces $H\cap
  gHg^{-1}=1$. Si $h\in H\cap gHg^{-1}$ entonces $h\cdot x=x$ y $g^{-1}hg\cdot
  x=x$. Como $g\cdot x\ne x$, entonces $h$ fija dos puntos de $X$. Esto implica
  que $h=1$ (pues todo elemento no trivial fija a lo sumo un punto de $X$). 

  Por el teorema~\ref{theorem:Frobenius}, el conjunto
  \[
    N=\left(G\setminus\bigcup_{g\in G}gHg^{-1}\right)\cup\{1\}
  \]
  es un subgrupo de $G$. Veamos cómo son los elementos de $N$: Si
  $h\in\cup_{g\in G}gHg^{-1}$ entonces existe $g\in G$ tal que $g^{-1}hg\in H$,
  es decir $(g^{-1}hg)\cdot x=x$ o quivalentemente $h\in G_{g\cdot x}$. Luego,
  a excepción de la identidad, los elementos de $N$ son los elementos de $G$
  que mueven algún punto de $X$.
\end{proof}

\begin{example}
  Sea $F$ un cuerpo finito y sea $G$ el grupo de funciones $f\colon G\to G$ de
  la forma $f(x)=ax+b$, $a,b\in F$ con $a\ne0$. El grupo $G$ actúa en $F$ y toda
  $f\ne\id$ fija a lo sumo un punto de $F$ pues 
  \[
	x=f(x)=ax+b\implies x=1-(b/a).
  \]
  En este caso, $N=\{f:f(x)=x+b\,,b\in F\}$ que es
  un subgrupo de $G$.
\end{example}

\begin{exercise}
  Demuestre que el teorema~\ref{theorem:Frobenius} puede deducirse del
  corolario~\ref{corollary:Frobenius_combinatorio}.
\end{exercise}


% Wielandt 8.5.4
% 8.5.6 para ver algo de grupos de permutaciones
% 7.1 para ejemplo H(q)
% 10.5.6 (Thompson) N es nilpotente, se usa 10.5.4 

En su tesis doctoral Thompson demostró el siguiente resultado, que había sido conjeturado por Frobenius:

\begin{theorem}[Thompson]
Sea $G$ un grupo de Frobenius. Si $N$ es el núcleo de Frobenius, entonces $N$ es nilpotente. 
\end{theorem}

La demostración puede consultarse en el capítulo 6 
de~\cite{MR2426855}, más precisamente en el teorema 6.24. 


