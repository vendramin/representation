\chapter{}

\topic{Brauer–Fowler theorem}

\index{Parte!simétrica}
\index{Parte!antisimétrica}
Sea $\rho\colon G\to\GL(V)$ 
una representación
con caracter $\chi$. Vimos que el $\C[G]$-módulo $V\otimes V$ tiene caracter $\chi^2$. Sea 
$v_1,\dots,v_n$ una base de $V$ y sea 
\[
T\colon V\to V,\quad
v_i\otimes v_j\mapsto v_j\otimes v_i.
\]
Dejamos como ejercicio verificar que $T(v\otimes w)=w\otimes v$ para todo 
$v,w\in V$. Luego la transformación lineal 
$T$ no depende de la base elegida. Observemos que
además $T$ es morfismo de $\C[G]$-módulos, pues
\[
T(g\cdot (v\otimes w))=T((g\cdot v)\otimes (g\cdot w))=(g\cdot w)\otimes (g\cdot v)=g\cdot T(w\otimes v)
\]
para todo $g\in G$ y $v,w\in V$. 
En particular, la \textbf{parte simétrica} 
\begin{gather*}
S(V\otimes V)=\{x\in V\otimes V:T(x)=x\}
\shortintertext{la \textbf{parte antisimétrica}}
A(V\otimes V)=\{x\in V\otimes V:T(x)=-x\}
\end{gather*}
de $V\otimes V$ son ambas 
$\C[G]$-submódulos de $V\otimes V$. Estos nombres están motivados por la siguiente observación 
\[
V\otimes V=S(V\otimes V)\oplus A(V\otimes V).
\]
En efecto, 
$S(V\otimes V)\cap A(V\otimes V)=\{0\}$ pues 
si $x\in S(V\otimes V)\cap A(V\otimes V)$, entonces $x=T(x)$ y $x=-T(x)$ y luego $x=0$. Además 
$V\otimes V=S(V\otimes V)+ A(V\otimes V)$ pues todo $x\in V\otimes V$ puede escribirse como 
\[
x=\frac12(x+T(x))+\frac12(x-T(x))
\]
con $\frac12(x+T(x))\in S(V\otimes V)$ y $\frac12(x-T(x))\in A(V\otimes V)$. 

Veamos que el conjunto $\{v_i\otimes v_j+v_j\otimes v_i:1\leq i,j\leq n\}$ es
base de $S(V\otimes V)$ 
y que el conjunto 
\[
\{v_i\otimes v_j-v_j\otimes v_i:1\leq i<j\leq n\}
\]
es base de $A(V\otimes V)$. Como ambos conjuntos son linealmente independientes, 
entonces 
$\dim S(V\otimes V)\geq n(n+1)/2$ y también 
$\dim A(V\otimes V)\geq n(n-1)/2$. Como además 
\[
n^2=\dim (V\otimes V)=\dim S(V\otimes V)+\dim A(V\otimes V),
\]
se concluye que $\dim S(V\otimes V)=n(n+1)/2$ y que $\dim A(V\otimes V)=n(n-1)/2$. 

\begin{proposition}
    Sea $G$ un grupo finito y  
    sea $V$ un $\C[G]$-módulo de dimensión finita con caracter $\chi$. Si el módulo $S(V\otimes V)$ 
    tiene caracter $\chi_S$ y el módulo $A(V\otimes V)$ tiene caracter $\chi_A$, entonces 
    \begin{align*}
        &\chi_S(g)=\frac12(\chi^2(g)+\chi(g^2)),\\
        &\chi_A(g)=\frac12(\chi^2(g)-\chi(g^2)).
    \end{align*}
\end{proposition}

\begin{proof}
    Sea $g\in G$. Sea $\rho\colon G\to\GL(V)$ la representación asociada al módulo $V$, es decir $\rho(g)(v)=\rho_g(v)=g\cdot v$. 
    Sabemos que $\rho_g$ es diagonalizable. Sea $\{e_1,\dots,e_n\}$ una base de autovectores de $\rho_g$, digamos
    $g\cdot e_i=\lambda_ie_i$ con $\lambda_i\in\C$ para $i\in\{1,\dots,n\}$. En particular, $\chi(g)=\sum_{i=1}^n\lambda_i$. 
    
    Como $\{e_i\otimes e_j-e_j\otimes e_i:1\leq i<j\leq n\}$ es base de $A(V\otimes V)$ y además 
    \[
    g\cdot (e_i\otimes e_j-e_j\otimes e_i)=\lambda_i\lambda_j(e_i\otimes e_j-e_j\otimes e_i),
    \]
    tenemos $\chi_A(g)=\sum_{1\leq i<j\leq n}\lambda_i\lambda_j$. Por otro lado, como $g^2\cdot e_i=\lambda_i^2e_i$ para todo $i$,
    $\chi(g^2)=\sum_{i=1}^n\lambda_i^2$. Luego
    \[
    \chi^2(g)=\chi(g)^2=\sum_{i=1}^n\sum_{j=1}^n\lambda_i\lambda_j=2\sum_{1\leq i<j\leq n}\lambda_i\lambda_j+\sum_{i=1}^n\lambda_i^2=2\chi_A(g)+\chi(g^2).
    \]
    Como además $V\otimes V=S(V\otimes V)\oplus A(V\otimes V)$, se tiene 
    $\chi^2(g)=\chi_S(g)+\chi_A(g)$, es decir 
    $\chi_S(g)=\frac12(\chi^2(g)+\chi(g^2))$.
\end{proof}

\index{Involución}
Una \textbf{involución} en un grupo es un elemento $x\ne 1$ tal que $x^2=1$. 
Es posible la cantidad de involuciones 
con la tabla de caracteres:

\begin{proposition}
Si $G$ es un grupo finito con $t$ involuciones, entonces 
\[
1+t=\sum_{\chi\in\Irr(G)}\langle\chi_S-\chi_A,\chi_1\rangle\chi(1).
\]
\end{proposition}

\begin{proof}
Supongamos que $\Irr(G)=\{\chi_1,\dots,\chi_k\}$, donde $\chi_1$ es el caracter trivial de $G$. 
Para $x\in G$ sea 
\[
\theta(x)=|\{y\in G:y^2=x\}|.
\]
Como $\theta$ es una función de clases
$\theta$ puede escribirse como combinación lineal de los $\chi_j$, digamos
\[
\theta=\sum_{\chi\in\Irr(G)}\langle\theta,\chi\rangle\chi.
\]
Calculamos
\begin{align*}
    \langle\chi_S-\chi_A,\chi_1\rangle 
    &=\frac{1}{|G|}\sum_{g\in G}\chi(g^2)\\
    &=\frac{1}{|G|}\sum_{x\in G}\sum_{\substack{g\in G\\g^2=x}}\chi(g^2)
    =\frac{1}{|G|}\sum_{x\in G}\theta(x)\chi(x)=\langle\theta,\chi\rangle.
\end{align*}
Luego $\theta(x)=\sum_{\chi\in\Irr(G)}\langle\chi_S-\chi_A,\chi_1\rangle\chi$ y el resultado se obtiene
al evaluar esta expresión en $x=1$. 
\end{proof}

\index{Desigualdad!de Cauchy--Schwartz}
Necesitamos un lema:
% Recordemos la desigualdad de Cauchy--Schwartz. Si $x_1,\dots,x_n\in\R$, entonces
% $\sum x_i^2\geq\frac{1}{n}(\sum x_i)^2$. 

\begin{lemma}
Sea $G$ un grupo finito con $k$ clases de conjugación. 
Si $t$ es la cantidad de involuciones de $G$, entonces 
$t^2\leq (k-1)(|G|-1)$. 
\end{lemma}

\begin{proof}
Supongamos que $\Irr(G)=\{\chi_1,\dots,\chi_k\}$, donde $\chi_1$ es el 
carácter trivial de $G$. 
Si $\chi\in\Irr(G)$, entonces
\[
\langle\chi^2,\chi_1\rangle=\frac{1}{|G|}\sum_{g\in G}\chi(g)\chi(g)=\langle\chi,\overline{\chi}\rangle=\begin{cases}
1 & \text{si $\chi=\overline{\chi}$},\\
0 & \text{en otro caso}.
\end{cases}
\]
Como $\chi^2=\chi_S+\chi_A$, si $\langle\chi^2,\chi_1\rangle=1$, entonces el caracter trivial 
o bien es $\chi_1$ es parte de $\chi_S$ o bien es parte de $\chi_A$, pero no de ambos. Esto implica que
\[
\langle\chi_S-\chi_A,\chi_1\rangle\in\{-1,1,0\}.
\]
Vamos a demostrar ahora que 
$t\leq\sum_{i=2}^k\chi_i(1)$. En efecto, 
como $|\langle\chi_S-\chi_A,\chi_1\rangle|\leq 1$, 
entonces 
\begin{align*}
1+t=\theta(1)
&=\left|\sum_{\chi\in\Irr(G)}\langle\chi_S-\chi_A,\chi_1\rangle\chi(1)\right|\\
&\leq\sum_{\chi\in\Irr(G)}|\langle\chi_S-\chi_A,\chi_1\rangle|\chi(1)
\leq\sum_{\chi\in\Irr(G)}\chi(1),
\end{align*}
de donde se obtiene inmediatamente que $t\leq\sum_{i=2}^k\chi_i(1)$. 
Si utilizamos ahora 
la desigualdad de Cauchy--Schwartz, 
\[
t^2\leq\left(\sum_{i=2}^k\chi_i(1)\right)^2
\leq(k-1)\sum_{i=2}^k\chi(1)^2=(k-1)(|G|-1).\qedhere
\]
\end{proof}

Ahora sí estamos en condiciones de dar la primera demostración del teorema de
Brauer--Fowler. 

\begin{theorem}[Brauer--Fowler]
\index{Teorema!de Brauer--Fowler}
Sea $G$ un grupo finito y simple y sea $x$ una involución. Si $|C_G(x)|=n$, entonces $|G|\leq (n^2)!$	
\end{theorem}

\begin{proof}
Supongamos primero que existe un subgrupo propio $H$ de $G$ tal que
$(G:H)\leq n^2$. En ese caso, hacemos actuar a $G$ en $G/H$ por multiplicación a izquierda 
y tenemos un morfismo de grupos $\rho\colon G\to\Sym_{n^2}$. Como $G$ es un grupo simple, 
$\ker\rho=\{1\}$ o bien $\ker\rho=G$. Si $\ker\rho=G$, entonces $\rho(g)(yH)=yH$ para todo
$g\in G$ e $y\in G$, lo que implica que $g\in H$, una contradicción. Luego $\rho$ es inyectiva
y entonces $G$ es isomorfo a un subgrupo de $\Sym_{n^2}$. En particular, $|G|$ divide a $(n^2)!$

Sea $m=(|G|-1)/t$. 
Como $|C_G(x)|=n$, el grupo $G$ tiene al menos $|G|/n$ involuciones (pues la clase de conjugación
de $x$ tiene tamaño $|G|/n$ y todos sus elementos son involuciones), es decir $t\geq |G|/n$. Luego
$m=(|G|-1)/t<n$. Basta demostrar entonces que $G$ contiene un subgrupo de índice $\leq m^2$. 

Sean $C_1,\dots,C_k$ las clases de conjugación de $G$, donde $C_1=\{1\}$. 
Como $G$ es simple, $|C_i|>1$ 
para todo $i\in\{2,\dots,k\}$. Notar que 
\[
|G|-1\leq\frac{(k-1)(|G|-1)^2}{t^2}\Longleftrightarrow t^2\leq(k-1)(|G|-1),
\]
que vale gracias al lema anterior. 
Si $|C_i|>m$ para todo $i\in\{2,\dots,k\}$, entonces, como
\[
|G|-1\leq\frac{(k-1)(|G|-1)^2}{t^2}=(k-1)m^2,
\]
tendríamos 
\[
|G|-1=\sum_{i=2}^k|C_i|>(k-1)m^2,
\]
una contradicción. Luego existe una clase de conjugación $C$ de $G$ tal que $|C|\leq m^2$. Si $g\in C$, entonces
$C_G(g)$ es un subgrupo de $G$ de índice $|C|\leq m^2$.
\end{proof}

La cota del teorema de Brauer--Fowler no es importante, ya que
para considerar una forma posible de atacar la clasificación de grupos simples 
solamente es necesario saber que existen  
finitos grupos simples finitos con un cierto centralizador de involusiones.

\begin{corollary}
    Sea $n\in\N$. Existe (a lo sumo) una cantidad finita de grupos simples 
    finitos con una involución con centralizador de orden $n$. 
\end{corollary}

Veamos un ejemplo sencillo que da una idea de cómo es que pueden clasificarse grupos simples
una vez que se tiene fija la estructura del centralizador de una involución. 

\begin{exercise}
Si $G$ es un grupo simple finito y $x$ es una involución con centralizador de orden dos, entonces 
$G\simeq\Z/2$. 
\end{exercise}

\topic{Induction and restriction}

\topic{Frobenius' theorem}

\topic{Some theorems of Burnside}

For $n\geq1$ let $\{e_1,\dots,e_n\}$ be the standard basis of $\C^n$.  
The \textbf{natural representation} of $\Sym_n$ is 
$\rho\colon\Sym_n\to\GL(n,\C)$, $\sigma\mapsto\rho_{\sigma}$, 
where $\rho_\sigma(e_j)=e_{\sigma(j)}$ for all $j\in\{1,\dots,n\}$. 
The matrix of $\rho_\sigma$ in the standard basis is  
\begin{equation}
    \label{eq:Sn_natural}
    (\rho_\sigma)_{ij}=\begin{cases}
      1 & \text{if $i=\sigma(j)$},\\
      0 & \text{otherwise}.
    \end{cases}
\end{equation}

\begin{lemma}
	\label{lem:permutaciones}
	For $n\geq1$ let $\rho\colon\Sym_n\to\GL(n,\C)$ be the natural 
	representation of the symmetric group. 
	If $A\in\C^{n\times n}$ and $\sigma\in\Sym_n$, then
	\[
		A_{ij}=(\rho_{\sigma}A)_{\sigma(i)j}=(A\rho_{\sigma})_{i\sigma^{-1}(j)}
	\]
    for all $i,j\in\{1,\dots,n\}$.
\end{lemma}

\begin{proof}
	With~\eqref{eq:Sn_natural} we compute:
	\[
		(A\rho_{\sigma})_{ij}=\sum_{k=1}^n A_{ik}(\rho_{\sigma})_{kj}=A_{i\sigma(j)},
		\quad
		(\rho_\sigma A)_{ij}=\sum_{k=1}^n (\rho_\sigma)_{ik}A_{kj}=A_{\sigma^{-1}(i)j}.\qedhere
	\]
\end{proof}

\begin{definition}
  \index{Real!character}
  Let $G$ be a finite group. A character $\chi$ of $G$ is said to be
  \textbf{real} if
  $\chi=\overline{\chi}$, that is $\chi(g)\in\R$ for all $g\in G$. 
\end{definition}

\begin{exercise}
	\label{xca:chi_irreducible}
	Let $G$ be a finite group. If $\chi\in\Irr(G)$, then 
	$\overline{\chi}$ is irreducible.
\end{exercise}

\begin{definition}
  \index{Real!conjugacy class}
  Let $G$ be a group. A conjugacy class $C$ of $G$ is said to be
  \textbf{real} if for every $g\in C$ one has $g^{-1}\in C$. 
\end{definition}

We use the following notation: if $G$ is a group and $C=\{xgx^{-1}:x\in G\}$ is a conjugacy class of  
$G$, then $C^{-1}=\{xg^{-1}x^{-1}:x\in G\}$.  

\begin{theorem}[Burnside]
    \index{Burnside's!theorem}
    Let $G$ be a finite group. The number of real conjugacy classes is equal 
    to the number of real irreducible characters. 
\end{theorem}

\begin{proof}
  Let $C_1,\dots,C_r$ be the conjugacy classes of $G$ and  
  let $\chi_1,\dots,\chi_r$ be the irreducible characters of $G$. 
  Let $\alpha,\beta\in\Sym_r$ be such that $\overline{\chi_i}=\chi_{\alpha(i)}$ and 
  $C_i^{-1}=C_{\beta(i)}$ for all $i\in\{1,\dots,r\}$. Note that $\chi_i$
  is real if and only if $\alpha(i)=i$ and that $C_i$ is real if and only if 
  $\beta(i)=i$. The number $n$ of fixed points of $\alpha$ is equal to the number
  of irreducible characters of $G$ and the number $m$ of fixed points of $\beta$ is equal
  to the number of real classes. 
  Let $\rho\colon\Sym_r\to\GL(r,\C)$ be the natural representation of $\Sym_r$. Then
  $\chi_\rho(\alpha)=n$ and $\chi_\rho(\beta)=m$. We claim that 
  $\trace\rho_\alpha=\trace\rho_\beta$. 
  Let $X\in\GL(r,\C)$ be the character matrix of $G$. 
  By Lemma~\ref{lem:permutaciones}, 
  \[
	\rho_\alpha X=\overline{X}=X\rho_\beta.
  \]
  Since $X$ is invertible, $\rho_{\alpha}=X\rho_{\beta}X^{-1}$. Thus 
  \[
    n=\chi_{\rho}(\alpha)=\trace\rho_{\alpha}=\trace\rho_{\beta}=\chi_{\rho}(\beta)=m.\qedhere
  \]
\end{proof}

\begin{corollary}
  \label{corollary:|G|impar}
  Let $G$ be a finite group. Then $|G|$ is odd if and only if
  the only real $\chi\in\Irr(G)$ is the trivial character. 
\end{corollary}

\begin{proof}
    We first prove $\impliedby$. If $|G|$ is even, there exists 
    $g\in G$ of order two (Cauchy's theorem). The conjugacy class of $g$ 
    is real. 

    We now prove $\implies$. Assume that $G$ has a non-trivial 
    real conjugacy class $C$. Let $g\in C$. We claim that 
    $G$ has an element of even order. Let $h\in G$ be such that
    $hgh^{-1}=g^{-1}$. Then $h^2\in C_G(g)$, as $h^2gh^{-2}=g$. 
    If $h\in\langle h^2\rangle\subseteq C_G(g)$, then $g$ has 
    even order, as $g^{-1}=g$. If $h\not\in\langle h^2\rangle$, then 
    $h^2$ does not generate $\langle h\rangle$. Hence $h$ has odd order, 
    as $|h|\ne|h^2|=|h|/(|h|:2)$.  
\end{proof}

\begin{theorem}[Burnside]
  \index{Burnside's!theorem}
  \label{thm:Burnside_mod16}
  Let $G$ be a finite group of odd order 
  with $r$ conjugacy classes. Then
  $r\equiv|G|\bmod{16}$.
\end{theorem}

\begin{proof}
  Since $|G|$ is odd, every non-trivial $\chi\in\Irr(G)$ is not real by
  the previous corollary. The irreducible characters 
  of $G$ are then 
  \[
    \chi_1,\chi_2,\overline{\chi_2},\dots,\chi_k,\overline{\chi_k},
    \quad
    r=1+2k,
  \]
  where $\chi_1$ denotes the trivial character. 
  For every $j\in\{2,\dots,k\}$ let $d_j=\chi_j(1)$. 
  Since each $d_j$ divides 
  $|G|$ by Frobenius' theorem and  $|G|$ is odd, 
  every $d_j$ is an odd number, 
  say $d_j=1+2m_j$. Thus  
  \begin{align*}
    |G|&=1+\sum_{j=2}^k 2d_j^2=1+\sum_{j=2}^k2(2m_j+1)^2\\
    &=1+\sum_{j=2}^k2(4m_j^2+4m_j+1)
    =1+2k+8\sum_{j=2}^km_j(m_j+1).
  \end{align*}
  Hence $|G|\equiv r\bmod{16}$, 
  as $r=1+2k$ and every $m_j(m_j+1)$ is even. 
\end{proof}

\begin{exercise}
    Prove that every group of order 15 is abelian. 
\end{exercise}

\topic{Solvable groups and Burnside's theorem}

\index{Derived series}
For a group $G$ let 
$G^{(0)}=G$ and 
$G^{(i+1)}=[G^{(i)},G^{(i)}]$ for $i\geq0$.
The \textbf{derived series} of $G$ is the sequence
\[
G=G^{(0)}\supseteq G^{(1)}\supseteq G^{(2)}\supseteq\cdots
\]
Each $G^{(i)}$ is a characteristic subgroup of $G$. We say that 
$G$ is \textbf{solvable} if $G^{(n)}=\{1\}$ for some $n$.  

\begin{example}
	Abelian groups are solvable. 
\end{example}

\begin{example}
	The group $\SL_2(3)$ is solvable, as the derived series is 
	\[
	\SL_2(3)\supseteq Q_8\supseteq C_4\supseteq C_2\supseteq \{1\}.
	\]
	Here is the what the computer says:
\begin{lstlisting}
gap> IsSolvable(SL(2,3));
true
gap> List(DerivedSeries(SL(2,3)),StructureDescription);
[ "SL(2,3)", "Q8", "C2", "1" ]
\end{lstlisting}
\end{example}

\begin{example}
	Non-abelian simple groups cannot be solvable. 
\end{example}

\begin{exercise}
	\label{xca:solvable}
	Let $G$ be a group. Prove the following statements:
	\begin{enumerate}
		\item A subgroup $H$ of $G$ is solvable.
		\item Let $K$ be a normal subgroup of $G$. 
		    Then $G$ is solvable if and only if $K$ and $G/K$ are solvable.
	\end{enumerate}
\end{exercise}

\begin{example}
	For $n\geq5$ the group $\Alt_n$ is not solvable. It follows that 
	$\Sym_n$ is not solvable for $n\geq5$. 
\end{example}

\begin{exercise}
\label{xca:pgroups_solvable}
	Let $p$ be a prime number. Prove that 
	finite $p$-groups are solvable.
\end{exercise}

\begin{theorem}[Burnside]
	\index{Burnside's theorem}
	\label{thm:Burnside_auxiliar}
	Let $G$ be a finite group. If $\phi\colon G\to\GL(n,\C)$ is a representation
	with character $\chi$ and $C$ is a conjugacy class of $G$ such that 
	$\gcd(|C|,n)=1$, then for every $g\in C$ either 
	$\chi(g)=0$ or $\phi_g$ is a scalar matrix. 
\end{theorem}

We need a lemma.

\begin{lemma}
	\label{lem:4Burnside}
	Let $\epsilon_1,\dots,\epsilon_n$ be roots of one such that 
	$(\epsilon_1+\cdots+\epsilon_n)/n\in\A$. Then either 
	$\epsilon_1=\cdots=\epsilon_n$ or 
	$\epsilon_1+\cdots+\epsilon_n=0$.
\end{lemma}

\begin{proof}
	Let $\alpha=(\epsilon_1+\cdots+\epsilon_n)/n$.
	Si los $\epsilon_j$ no son todos iguales, entonces $N(\alpha)<1$. Además 
	$N(\beta)<1$ para todo conjugado algebraico $\beta$ de $\alpha$. Como el
	producto de los conjugados algebraicos de $\alpha$ es un entero de módulo
	$<1$, se conluye que es cero.
\end{proof}

Now we prove the theorem.

\begin{proof}[Proof of Theorem \ref{thm:Burnside_auxiliar}]
	Sean $\epsilon_1,\dots,\epsilon_n$ los autovalores de $\phi_g$. Como
	$(|C|:n)=1$, existen $a,b\in\Z$ tales que $a|C|+bn=1$.  Como
	$|C|\chi(g)/n\in\A$, al multiplicar por $\chi(g)/n$ obtenemos 
	\[
		a|C|\frac{\chi(g)}{n}+b\chi(g)=\frac{\chi(g)}{n}=\frac{1}{n}(\epsilon_1+\cdots+\epsilon_n)\in\A.
	\]
	El lema anterior nos dice que entonces hay dos posbilidades:
	$\epsilon_1=\cdots=\epsilon_n$ o bien $\epsilon_1+\cdots+\epsilon_n=0$. En
	el primer caso, como $\phi_g$ es diagonalizable, $\phi_g$ es una matriz
	escalar. El segundo caso dice exactamente que $\chi(g)=0$.
\end{proof}

\begin{theorem}[Burnside]
	\index{Teorema!de Burnside}
  Sea $p$ un número primo. Si $G$ es un grupo finito y $C$ es una clase de
  conjugación de $G$ con $p^k>1$ elementos, entonces $G$ no es simple.
\end{theorem}

\begin{proof}
	Sea $g\in C\setminus\{1\}$. Por la ortogonalidad de las columnas, 
	\begin{equation}
	\label{eq:Burnside}
	\begin{aligned}
		0&=\sum_{\chi\in\Irr(G)}\chi(1)\chi(g)\\
		&=\sum_{p\mid\chi(1)}\chi(1)\chi(g)+\sum_{p\nmid\chi(1)}\chi(1)\chi(g)+1,
	\end{aligned}
	\end{equation}
	donde el uno corresponde a la representación trivial de $G$. 
	
	Vamos a mirar esta ecuación módulo $p$. Más precisamente, si $\chi(g)=0$ para todo $\chi\in\Irr(G)$
	tal que $\chi\ne\chi_1$ y $p\nmid\chi(1)$, entonces
	podemos escribir
	\[
	-\frac{1}{p}=\sum\frac{\chi(1)}{p}\chi(g)\in\A\cap\Q=\Z,
	\]
	donde la suma se toma sobre todos los irreducibles no triviales de $G$ de grado divisible por $p$, 
	una contradicción. Luego existe una representación no
	trivial irreducible $\phi$ con carácter $\chi$ tal que $p$ no divide a
	$\chi(1)$ y además $\chi(g)\ne0$. Por el teorema anterior, $\phi_g$ es una
	matriz escalar. Si $\phi$ es fiel, entonces $g$ es un elemento central no
	trivial, una contradicción pues $|C|>1$. En caso contrario, $G$ no es simple pues
	$\ker\phi$ es un subgrupo propio no trivial de $G$.
%	Si $p$ divide a $\deg\phi$, entonces
%	$\frac1p(\deg\phi)\chi_\phi(g)\in\A$ y luego
%	\[
%		\alpha=\sum_{p\mid\deg\phi}\frac1p(\deg\phi)\chi_\phi(g)\in\A.
%	\]
%	La fórmula~\eqref{eq:Burnside} queda entonces 
%	\[
%		0=1+p\alpha+\sum_{p\nmid\deg\phi}(\deg\phi)\chi_{\phi}(g).
%	\]
\end{proof}

\begin{theorem}[Burnside]
  \index{Teorema!de Burnside}
  Sean $p,q$ primos. Si $G$ tiene orden $p^aq^b$ entonces $G$ es resoluble.
\end{theorem}

\begin{proof}
	Supongamos que el teorema no es cierto y sea $G$ un grupo de orden $p^aq^b$
	minimal con la propiedad de no ser resoluble. La minimalidad de $|G|$
	implica que $G$ es simple. Por el teorema anterior, $G$ no tiene clases de
	conjugación de tamaño $p^k$ ni clases de tamaño $q^l$ con $k,l\geq1$. El
	tamaño de toda clase de conjugación de $G$ es entonces igual a uno o es
	divisible por $pq$. Pero entonces la ecuación de clases, 
	\[
		|G|=1+\sum_{C:|C|>1}|C|,
	\]
	donde la suma se hace sobre todas las clases de conjugación que tienen más
	de un elemento, da una contradicción.
\end{proof}

Concluimos el capítulo con los enunciados de algunas 
generalizaciones del teorema de Burnside. 

\begin{theorem}[Kegel--Wielandt]
\index{Teorema!de Kegel--Wielandt}
Si $G$ es un grupo finito y existen subgrupos nilpotentes $A$ y $B$ de $G$ tales
que $G=AB$, entonces $G$ es resoluble. 
\end{theorem}

La demostración del teorema de Kegel--Wielandt puede consultarse en el segundo 
capítulo del libro~\cite{MR1211633}, más precisamente en el teorema 2.4.3. 

\begin{theorem}[Feit--Thompson]
\index{Teorema!de Feit--Thompson}
Todo grupo finito de orden impar es resoluble. 
\end{theorem}

La demostración del teorema de Feit--Thompson es extremadamente difícil y ocupa un volumen completo del 
\emph{Pacific Journal of Mathematics}~\cite{MR166261}. 
En~\cite{MR3111271} se anunció haber verificado formalmente 
demostración del teorema de Feit--Thompson con el 
sistema de ayuda para la demostración de teoremas Coq. 

\medskip
\index{Conjetura!de Feit--Thompson}
En los sesenta se sabía que la demostración del teorema de Feit--Thomson iba a poder simplificarse 
si la conjetura de Feit--Thompson es verdadera:

\begin{quote}
No existen primos distintos $p$ y $q$ tales que
$\frac {p^{q}-1}{p-1}$ divide a $\frac{q^{p} - 1}{q - 1}$. 
\end{quote}

Ya no es necesaria esa conjetura para simplificar la demostración, 
y la conjetura de Feit--Thompson permanece abierta. 
En~\cite{MR297686} 
Stephens demostró que la versión fuerte de la conjetura no es cierta, ya que 
los enteros $\frac {p^{q}-1}{p-1}$ y $\frac{q^{p} - 1}{q - 1}$ 
podrían tener factores en común. De hecho, si $p=17$ y $q=3313$, 
entonces 
\[
\gcd\left(\frac {p^{q}-1}{p-1},\frac{q^{p} - 1}{q - 1}\right)=112643.
\]
Hoy podemos reproducir los cálculos de 
Stephens con casi cualquier computadora de escritorio:
\begin{lstlisting}
gap> Gcd((17^3313-1)/16,(3313^17-1)/3312);
112643
\end{lstlisting}

Otra dirección en la que puede generalizarse el teorema de Burnside 
es con el uso de las funciones de palabra. 
Una \emph{función de palabra} de un grupo $G$ es una función 
\[
G^k\to G,\quad 
(x_1,\dots,x_k)\mapsto w(x_1,\dots,x_k)
\]
para alguna 
palabra $w(x_1,\dots,x_k)$ en el grupo libre $F_k$ de rango $k$. 
Algunas palabras son sobreyecticas en 
todo grupo o en cierta familia de grupos. Por ejemplo, 
la conjetura de Ore es la sobreyectividad de la función 
$(x,y)\mapsto [x,y]=xyx^{-1}y^{-1}$ en todo grupo finito simple no abeliano.

\begin{theorem}[Guralnick--Liebeck--O'Brien--Shalev--Tiep]
Sean $p$ y $q$ dos primos, $a,b\geq0$ y $N=p^aq^b$. La función $(x,y)\mapsto x^Ny^N$ es 
sobreyectiva en todo grupo simple.
\end{theorem}

El teorema fue demostrado en~\cite{MR3827208}. 

Veamos por qué implica el teorema de Burnside. Supongamos que $G$ es un grupo de orden $N=p^aq^b$ y que $G$ no es resoluble. 
Si fijamos una serie de composición de $G$, tenemos un factor $S$ no abeliano de orden que divide a $N$. Como entonces
$S$ es simple y no abeliano y $s^N=1$, se concluye que la función $(x,y)\mapsto x^Ny^N$ tiene imagen trivial en $S$, una contradicción al teorema. 

