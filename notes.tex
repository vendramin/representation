\documentclass[12pt]{amsproc}

\newcommand{\course}{Representation theory of algebras}
\input{mystuff}


\renewcommand{\sl}{\mathfrak{sl}}
\newcommand{\cC}{\mathcal{C}}
\renewcommand{\cf}{\operatorname{ClassFun}}
\newcommand{\cD}{\mathcal{D}}
\newcommand{\Sets}{\mathfrak{Sets}}
\newcommand{\Groups}{\mathfrak{Groups}}
\newcommand{\AbelianGroups}{\mathfrak{AbelianGroups}}
\newcommand{\Rings}{\mathfrak{Rings}}
\newcommand{\Vect}{\mathfrak{Vec}}



%\newcommand{\Q}{\mathbb{Q}}
%\newcommand{\Z}{\mathbb{Z}}
%\newcommand{\F}{\mathbb{F}}
%\newcommand{\R}{\mathbb{R}}
%\newcommand{\C}{\mathbb{C}}
%\renewcommand{\H}{\mathbb{H}}

%\usepackage{mathptmx}
%\usepackage{newtxtext}

\begin{document}

% \begin{abstract}
% The notes correspond to the master
% course \textbf{Representation Theory of Algebras} of the
% Vrije Universiteit Brussel,
% Faculty of Sciences,
% Department of Mathematics and Data Sciences. \end{abstract}

\maketitle

\setcounter{tocdepth}{1}
\tableofcontents

\thispagestyle{plain}
\section*{Introduction}

The notes correspond to the master  
course \emph{Representation theory of algebras} of the 
Vrije Universiteit Brussel, 
Faculty of Sciences, 
Department of Mathematics and Data Sciences. The course
is divided into twelve two-hour lectures. 

Most of the material is based on standard 
results of the representation theory of finite groups. 
Basic texts on representation theory are \cite{MR1369573} 
and \cite{MR2270898}. 

The notes include many exercises, some with full detailed solutions. Mandatory exercises have a \colorbox{green!5!white}{green background}, while optional ones
(bonus exercises) have a \colorbox{yellow!15!white}{yellow background}.

The notes also include some additional comments. While these are entirely optional, I hope they offer further insight. They are highlighted with a \colorbox{red!5!white}{pink background}.

The notes include Magma code, which we use to verify examples and offer alternative solutions to certain exercises. Magma \cite{zbMATH01077111} is a powerful software tool designed for working with algebraic structures. There is a free \href{https://magma.maths.usyd.edu.au/calc/}{online} version of Magma available.


Thanks go to Luca Descheemaeker, Wannes Malfait, Silvia Properzi, Lukas Simons.  



This version 
was compiled on \today~at~\currenttime.


 \begin{figure}[b]
     \includegraphics[scale=0.2]{VUB.jpg}
 \end{figure}
\include{subsections}


\chapter{}

\topic{Artin--Wedderburn theorem}

We first review the basic definitions concerning
finite-dimensional semisimple algebras. 
Proofs can be found in the
notes to the course \emph{Associative Algebras}, see
lectures 1, 2 and 3. 

Our base field will be the field $\C$ of complex numbers. 

\index{Algebra}
\index{Algebra!unitary}
A (complex) \textbf{algebra} $A$ is a is a (complex) vector space  
with an associative multiplication $A\times A\to A$ such that
\[
a(\lambda b+\mu c)=\lambda(ab)+\mu(ac),
\quad
(\lambda a+\mu b)c=\lambda(ac)+\mu (bc)
\]
for all $a,b,c\in A$. If $A$ contains 
an element $1_A\in A$ such that $1_Aa=a1_A=a$ for all $a\in A$, then $A$ is 
an unitary algebra. 

Our algebras will be finite-dimensional. 
Clearly, $\C$ is an algebra. Other 
examples of algebras are $\C[X]$ and $M_n(\C)$. 

\index{Module}
\index{Submodule}
A (left) \textbf{module} $M$ (over a unitary 
algebra $A$) is an abelian group $M$
together with a map $A\times M\to M$, $(a,m)\mapsto am$, such that
$1_Am=m$ for all $m\in M$ and 
$a(bm)=(ab)m$ and $a(m+m_1)=am+am_1$ for all $a,b\in A$ and $m,m_1\in M$. 
A \textbf{submodule} $N$ of $M$ is a subgroup 
$N$ such that $an\in N$ for all $a\in A$ and $n\in N$. 

\begin{exercise}
Let $A$ be a finite-dimensional algebra. If $M$ 
is a module, then $M$ is a vector space with 
$\lambda m=(\lambda 1_A)m$ for $\lambda\in\C$ and $m\in M$. Moreover, 
$M$ is finitely generated if and only if $M$ is finite-dimensional. 
\end{exercise}

\index{Module!simple}
\index{Module!semisimple}
A module $M$ is said to be \textbf{simple} if $M\ne\{0\}$ and $\{0\}$ and $M$ 
are the only submodules of $M$.	
A finite-dimensional module $M$  
is said to be \textbf{semisimple} if $M$ is a direct sum of 
finitely many simple submodules. 
Clearly, simple modules are semisimple. Moreover, any finite direct sum of semisimples is semisimple. 


\index{Algebra!semisimple}
A finite-dimensional algebra $A$ is said to be \textbf{semisimple} if
if every finitely-generated $A$-module is semisimple. 

\begin{theorem}[Artin--Wedderburn]
Let $A$ be a complex finite-dimensional semisimple algebra, say with  
$k$ isomorphism classes of simple modules. Then 
\[
A\simeq M_{n_1}(\C)\times\cdots\times M_{n_k}(\C)
\]
for some $n_1,\dots,n_k\in\Z_{>0}$.
\end{theorem}

We also basic some basic facts on the Jacobson radical
of finite-dimensional algebras. If $A$ is a finite-dimensional algebra, the \textbf{Jacobson radical} is defined as 
\[
J(A)=\bigcap\{M:M\text{ is a maximal left ideal of $A$}\}. 
\]
It turns out that $J(A)$ is an ideal of $A$. If $A$ is
unitary, then Zorn's lemma implies that there a 
maximal left ideal of $A$ and hence $J(A)\ne A$. 

An ideal $I$ of $A$ is said to be \textbf{nilpotent}
if $I^m=\{0\}$ for some $m$, that is 
$x_1\cdots x_m=0$ for all $x_1,\dots,x_m\in I$. 
One proves that the Jacboson radical of $A$ 
contains every nilpotent ideal of $A$. An important
fact is that 
\begin{align*}
A\text{ is semisimple }
&\Longleftrightarrow 
J(A)=\{0\}\\
&\Longleftrightarrow 
A\text{ has no non-zero nilpotent ideals}.
\end{align*}

\topic{Kolchin's theorem}
\label{Kolchin}

In this section it will be useful to consider 
non-unitary algebras. 

\begin{definition}
\index{Nil!element}
\index{Nilpotent!element}
\index{Nil!algebra}
    Let $A$ be an algebra (possibly without one). An element $a\in A$
    is said to be \textbf{nilpotent} if 
    $a^n=0$ for some $n\geq1$. The algebra $A$ is said to be
    \textbf{nil} if every element $a\in A$ is nilpotent. 
\end{definition}

Nilpotent elements are also called nil elements.  

\begin{example}
    Let $A=M_2(\R)$. Then $a=\begin{pmatrix}0&1\\0&0\end{pmatrix}$ is nilpotent. 
\end{example}

\begin{definition}
    \index{Nilpotent!algebra}
    An algebra $A$ is said to be \textbf{nilpotent} if there exists
    $n\geq1$ such that every product 
    $a_1a_2\cdots a_n$
    of $n$ elements of $A$ is zero. 
\end{definition}

Nilpotent algebras are trivially nil, whereas nil algebras may not be nilpotent, as each element being nilpotent does not force products of distinct elements to vanish.

\begin{exercise}
    Give an example of a nil algebra that is not nilpotent. 
\end{exercise}

Note that nil algebras cannot be unitary. 

\begin{proposition}
\label{pro:unit}
    Let $A$ be an algebra. There exists an algebra $B$ 
    with one $1_B$ and an ideal $I$ of $B$ 
    such that $B/I\simeq\C$ and $I\simeq A$. 
\end{proposition}

\begin{proof}[Sketch of the proof]
    Let $B=\C\times A$. The multiplication  
    \[
    (\lambda,u)(\mu,v)=(\lambda\mu,\lambda v+\mu u+uv)
    \]
    turns $B$ into an algebra with identity $(1,0)$. The subset
    $I=\{(0,a):a\in A\}$ is an ideal of $B$. Then $I\simeq A$ 
    and $B/I\simeq\C$. 
\end{proof}

\begin{exercise}
Let $A_1,\dots,A_k$ be algebras. 
Prove that the ideals of $A_1\times\cdots\times A_k$ 
are of the form $I_1\times\cdots\times I_k$, where
each $I_j$ is an ideal of $A_j$.  
\end{exercise}

\begin{exercise}
\label{xca:unit}
    Prove that the non-zero ideals of 
    $\prod_{i=1}^k M_{n_i}(\C)$ are unitary algebras.  
\end{exercise}

\begin{proposition}
    Let $A$ be non-zero algebra (possibly without one). If $A$ 
    does not have non-zero nilpotent 
    ideals, then $A$ is a unitary algebra. 
\end{proposition}

\begin{proof}
    Let $B$ be a unitary algebra such that there exists
    an ideal $I$ of $B$ with $B/I\simeq\C$ and $I\simeq A$ 
    (see Proposition \ref{pro:unit}). Let $J$ be 
    a nilpotent ideal of $B$. Since $J\cap I\subseteq I$ is a nilpotent
    ideal of $A$, 
    $J\cap I=\{0\}$. Thus 
    \[
    J\simeq J/(J\cap I)\simeq (I+J)/I
    \]
    is a nilpotent ideal of $B/I\simeq\C$. Thus $J=\{0\}$ 
    and hence $B$ is semisimple. By Artin--Wedderburn, 
    $B\simeq\prod_{i=1}^k M_{n_i}(\C)$. Since $A$ is isomorphic to an ideal of 
    $B$, Exercise \ref{xca:unit} shows
    that $A$ is a unitary algebra. 
\end{proof}

Now we prove another nice result of Wedderburn:

\begin{theorem}[Wedderburn]
\label{thm:Wedderburn}
\index{Wedderburn's theorem}
    Let $A$ be a complex finite-dimensional 
    algebra. If $A$ is generated (as a vector space) 
    by nilpotent elements, then $A$ is nilpotent. 
\end{theorem}

We shall need a lemma.

\begin{lemma}
    The vector space $M_n(\C)$ does not have a basis of nilpotent matrices. 
\end{lemma}

\begin{proof}
    If $\{A_1,\dots,A_{n^2}\}$ is a basis of 
    $M_n(\C)$ consisting of nilpotent matrices, 
    then there exist $\lambda_1,\dots,\lambda_{n^2}\in\C$ such that 
    \begin{equation}
        \label{eq:nilpotent}
        E_{11}=\begin{pmatrix}
        1&0&\cdots&0\\
        0&0&\cdots&0\\
        \vdots&\vdots&\ddots&\vdots\\
        0&0&\cdots&0
        \end{pmatrix}
        =\sum_{i=1}^{n^2}\lambda_iA_i.
    \end{equation}
    Note $\trace(A_i)=0$ for all $i\in\{1,\dots,n\}$, as 
    every $A_i$ is nilpotent. 
    Apply trace to \eqref{eq:nilpotent} to 
    obtain that $1=\trace(E_{11})=\sum\lambda_i\trace(A_i)=0$, a contradiction. 
\end{proof}

Now we prove Wedderburn's theorem. We note that
the theorem can be extended to any algebraically closed field. We 
state and proof Wedderburn's theorem in the case of complex numbers
to simplify a little bit the presentation. 

\begin{proof}[Proof of Theorem \ref{thm:Wedderburn}]
    We proceed by induction on $\dim A$. If $\dim A=1$ and 
    there exists a nilpotent element $a\in A$ such that 
    $\{a\}$ is a basis of $A$, then $A$ is nilpotent, 
    as every element of $A$ is nilpotent, as it is 
    of the form 
    $\lambda a$ for some $\lambda\in\C$. 
    
    Assume now that $\dim A>1$. Since $J(A)$ is nilpotent, $J(A)^n=\{0\}$ 
    for some $n$. 
    
    If $J(A)=A$, the result trivially holds. 
    
    If $J(A)\ne\{0\}$, 
    $\dim A/J(A)<\dim A$ and hence 
    $A/J(A)$ is nilpotent by 
    the inductive hypothesis, 
    say $(A/J(A))^m=\{0\}$. Let $\pi\colon A\to A/J(A)$ be the canonical map and 
    $N=nm$. We claim that $A^N=\{0\}$. Let $a_1,\dots,a_N\in A$. Write
    $a_1\cdots a_N=x_1\cdots x_n$ for some $x_1\cdots x_n\in A$. For example,
    \begin{align*}
    x_1&=a_1a_2\cdots a_m,\\
    x_2&=a_{m+1}a_{m+2}\cdots a_{2m},\\
    &\vdots
    \end{align*}
    Since 
    \[
    \pi(x_1)=\pi(a_1a_2\cdots a_m)=\pi(a_1)\pi(a_2)\cdots\pi(a_m)=0,
    \]
    it follows that $x_1\in J(A)$. Similarly, 
    $\pi(x_j)\in J(A)$
    for every $j\in\{1,\dots,n\}$. Thus 
    \[
    a_1a_2\cdots a_N=x_1x_2\cdots x_n\in J(A)^n=\{0\}. 
    \]
    Thus $A$ is nilpotent. 
    
    If $J(A)=\{0\}$, then 
    $A$ is semisimple. By Artin--Wedderburn, 
    $A\simeq\prod_{i=1}^k M_{n_i}(\C)$, a contradiction to 
    the previous lemma. 
\end{proof}

\begin{definition}
\index{Flag!complete}
    Let $V=\C^{n}$ (column vectors). A \textbf{complete flag} in $V$ 
    is a sequence $(V_1,V_2,\dots,V_n)$ of vector spaces
    such that 
    \[
    \{0\}\subsetneq V_1\subsetneq V_2\subsetneq\cdots\subsetneq V_n=V.
    \]
\end{definition}

\index{Flag!standard}
If $(V_1,\dots,V_n)$ is a complete flag, then $\dim V_i=i$ for all 
$i\in\{1,\dots,n\}$. 
Let $\{e_1,\dots,e_n\}$ be the standard basis of $\C^n$. 
The \textbf{standard flag} is the sequence $(E_1,\dots,E_n)$, where
$E_i=\langle e_1,\dots,e_i\rangle$ for all $i\in\{1,\dots,n\}$.  

Note that $\GL_n(\C)$ acts on the set of complete flags of $V$ 
by 
\[
g\cdot (V_1,\dots,V_n)=(T_g(V_1),\dots,T_g(V_n)),
\]
where $T_g\colon V\to V$, $x\mapsto gx$. 

The action is \emph{transitive}, 
which means that if $(V_1,\dots,V_n)$ 
is a complete flag, then there exists 
$g\in\GL_n(\C)$ such that $g\cdot (E_1,\dots,E_n)=(V_1,\dots,V_n)$. 
In fact, 
the matrix $g=(v_1|v_2|\cdots|v_n)$, where
$\{v_1,\dots,v_n\}$ is a basis of $V$, 
satisfies $ge_i=v_i$ for all $i\in\{1,\dots,n\}$. 

\label{Borel subgroup}
Let $B_n(\C)$ be the stabilizer    
\begin{align*}
G_{(E_1,\dots,E_n)}
&=\{g\in\GL_n(\C):T_g(E_i)=E_i\text{ for all $i$}\}
=\{(b_{ij}):b_{ij}=0\text{ if $i>j$}\}
\end{align*}
of the standard flag. Then $B_n(\C)$ is 
known as the \textbf{Borel subgroup}. 

Let $U_n(\C)$ be the subgroup of $\GL_n(\C)$ 
of matrices $(u_{ij})$ such that 
\[
u_{ij}=\begin{cases}
1&\text{if $i=j$},\\
0&\text{if $i>j$}.\end{cases}
\]
Let $T_n(\C)$ be the subgroup of $\GL_n(\C)$ diagonal matrices. 

\begin{proposition}
    $B_n(\C)=U_n(\C)\rtimes T_n(\C)$. 
\end{proposition}

\begin{proof}
    It is trivial that $U_n(\C)\cap T_n(\C)=\{I\}$, where $I$ is the 
    $n\times n$ identity matrix. Clearly, $U_n(\C)$ is a subgroup of $B_n(\C)$.
    To prove that 
    $U_n(\C)$ is normal in $B_n(\C)$ note that $U_n(\C)$ is the kernel
    of the group homomorphism
    \[
    f\colon B_n(\C)\to T_n(\C),\quad
    (b_{ij})\mapsto\begin{pmatrix}
        b_{11}\\
        &b_{22}\\
        &&\ddots\\
        &&&b_{nn}
    \end{pmatrix}.
    \]
    It remains to show that $B_n(\C)=U_n(\C)T_n(\C)$.
    Let us prove that  $B_n(\C)\subseteq U_n(\C)T_n(\C)$, as the other inclusion is trivial. 
    Let $b\in B_n(\C)$. Then
    $bf(b)^{-1}\in \ker f=U_n(\C)$ and therefore  
    $b=(bf(b)^{-1})f(b)\in U_n(\C)T_n(\C)$. 
\end{proof}

\begin{definition}
\index{Unipotent element}
    A matrix $a\in\GL_n(\C)$ is said to be \textbf{unipotent} 
    if its characteristic polynomial is of the form 
    $(X-1)^n$. 
\end{definition}

The matrix $\begin{pmatrix}1&1\\0&1\end{pmatrix}$ is unipotent, 
as its characteristic polynomial is $(X-1)^2$. 

\begin{definition}
\index{Unipotent group}
    A subgroup $G$ of $\GL_n(\C)$ is said to be \textbf{unipotent} if
    each $g\in G$ is unipotent. 
\end{definition}

Now an application of Wedderburn's theorem:

\begin{proposition}
\label{pro:unipotent}
    Let $G$ be an unipotent subgroup of $\GL_n(\C)$. 
    Then there exists a non-zero 
    $v\in C^{n}$ such that $gv=v$ for all $g\in G$. 
\end{proposition}

\begin{proof}
    Without loss of generality, we may assume that $G$ is non-trivial. 
    Let $V$ be the subspace of $\C^{n\times n}$ 
    generated by $\{g-I:g\in G\}$. If $g\in G$, then 
    $(g-I)^n=0$, as $g$ is unipotent. Thus
    every element of $V$ is nilpotent. If $g,h\in G$, 
    then 
    \[
    (g-I)(h-I)=(gh-I)-(g-I)-(h-I)\in V.
    \]
    This means that $V$ is closed under multiplication and
    hence $V$ is an algebra generated (as a vector space)
    by nilpotent elements. By Wedderburn's theorem, 
    $V$ is nilpotent. Let $m$ be minimal 
    such that 
    $(g_1-I)\cdots (g_m-I)=0$ 
    for all $g_1,\dots,g_m\in G$. The minimality of $m$ implies that  
    there exist $h_1,\dots,h_{m-1}\in G$ such that 
    \[
    (h_1-I)\cdots (h_{m-1}-I)\ne 0.
    \]
    In particular, there exists a non-zero 
    $w\in C^{n}$ such that 
    $v=(h_1-I)\cdots (h_{m-1}-I)w\ne 0$. For every 
    $g\in G$, 
    \[
    (g-I)v=(g-I)(h_1-I)\cdots (h_{m-1}-I)w=0
    \]
    and hence $gv=v$. 
\end{proof}

\begin{theorem}[Kolchin]
\label{thm:Kolchin}
\index{Kolchin's theorem}
Every unipotent subgroup of $\GL_n(\C)$ is conjugate
to some subgroup of $U_n(\C)$. 
\end{theorem}

\begin{proof}
    Let $G$ be an unipotent subgroup of $\GL_n(\C)$. 
    Assume first that there exists
    a complete flag $(V_1,\dots,V_n)$ of $\C^n$
    such that $G\subseteq G_{(V_1,\dots,V_n)}$. Let $g\in\GL_n(\C)$ be such that 
    $g\cdot (E_1,\dots,E_n)=(V_1,\dots,V_n)$. Then 
    \begin{gather*}
        G\subseteq G_{g\cdot (E_1,\dots,E_n)}
        =g G_{(E_1,\dots,E_n)}g^{-1}=gB_n(\C)g^{-1}.
    \shortintertext{Since $G$ is unipotent,}
        G=G\cap (gB_n(\C)g^{-1})\subseteq gU_n(\C)g^{-1}.
    \end{gather*}
    
    We claim that $G\subseteq G_{(V_1,\dots,V_n)}$ for
    some complete flag $(V_1,\dots,V_n)$. We proceed by induction on $n$. If $n=1$, the result is trivial. Assume the result holds for 
    $n-1$. By the previous proposition, there exists a non-zero $v\in\C^n$ 
    such that $gv=v$ for all $g\in G$. Let $Q=\C^n/\langle v\rangle$ and $\pi\colon \C^n\to Q$ be the canonical map. Then $\dim Q=n-1$. The group $G$ 
    acts on $Q$ by
    \[
    g\cdot (w+\langle v\rangle)=gw+\langle v\rangle.
    \]
    The action is well-defined: if $w+\langle v\rangle=w_1+\langle v\rangle$, then 
    $w-w_1=\lambda v$ for some $\lambda\in\C$. This implies
    that 
    \[
    gw-gw_1=g(w-w_1)=\lambda(gv)=\lambda v\in \langle v\rangle
    \]
    and hence $gw+\langle v\rangle=gw_1+\langle v\rangle$. 
    
    By the inductive hypothesis, $G$ stabilizes
    a complete flag $(Q_1,\dots,Q_{n-1})$, where
    \[
    Q_1=\langle\pi(v_1)\rangle,
    \quad
    Q_2=\langle\pi(v_1),\pi(v_2)\rangle,
    \quad
    \dots
    \quad
    Q_{n-1}=\langle\pi(v_1),\dots,\pi(v_{n-1})\rangle.
    \]
    Let 
    \[
    W_0=\langle v\rangle,
    \quad
    W_1=\langle v,v_1\rangle,
    \quad
    W_2=\langle v,v_1,v_2\rangle,
    \quad\dots\quad 
    W_{n-1}=\langle v,v_1,\dots,v_{n-1}\rangle.
    \]
    Since $(Q_1,\dots,Q_{n-1})$ is a complete flag, 
    the set $\{\pi(v_j):1\leq j\leq n-1\}$ is linearly
    independent. We claim that 
    $\{v,v_1,\dots,v_{n-1}\}$ is linearly independent. In fact, since $v\ne 0$, one obtains that 
    \[
    \sum_{i=1}^{n-1}\lambda_iv_i+\lambda v=0
    \implies
    \sum_{i=1}^{n-1}\lambda_i\pi(v_i)=0
    \implies 
    \lambda_1=\cdots=\lambda_{n-1}=0
    \implies
    \lambda=0.
    \]
    Thus $\dim W_i=i+1$ for all $i$. 
    
    Let $g\in G$. 
    Clearly, 
    $gW_0\subseteq W_0$, as $gv=v$. Let $j\in\{1,\dots,n-1\}$.
    There exist $\lambda_1,\dots,\lambda_j\in\C$ 
    such that 
    $\pi(gv_j)=\sum_{i\leq j}\lambda_i\pi(v_i)$. This means
    that 
    \[
    gv_j-\sum_{i\leq j}\lambda_iv_i=\lambda v\in\langle v\rangle
    \]
    for some $\lambda\in\C$. In particular, 
    \[
    gv_j=\sum_{i\leq j}\lambda_iv_i+\lambda v\in\langle v,v_1,\dots,v_{j}\rangle=W_j.
    \]
    Therefore $G\subseteq G_{(W_0,\dots,W_{n-1})}$. 
\end{proof}

\subsection{Some comments}

The ideas behind the theorem are somewhat connected to Sylow's theory. The key is to consider an explicit version of Sylow's theorem for the group $\GL_n(p)$ of invertible matrices 
with coefficients in the field $\F_p$ with $p$ elements. 

A group $G$ acts linearly on a vector space $V$ 
if $g\cdot (v+w)=g\cdot v+g\cdot w$
for all $g\in G$ and $v,w\in V$.
Proposition \ref{pro:unipotent} has the following
version:

\begin{proposition}
Let $P$ be a finite $p$-group acting on a finite-dimensional $\F_p$-vector space 
$V$ linearly. Then there exists a non-zero $v\in V$ 
such that $x\cdot v=v$ for all $x\in P$. 
\end{proposition}

\begin{proof}
    Let $n=\dim V$. There are $p^n-1$ non-zero vectors 
    in $V$. Since the action is linear, 
    $P$ acts  
    on $X=V\setminus\{0\}$. We decompose $V$ into orbits
    and collect those orbits with only one element, say 
    \[
    X=X_0\cup O(v_1)\cup \cdots\cup O(v_m),
    \]
    where $|O(v_j)|\geq 2$ for all $j\in\{1,\dots,m\}$. 
    Since 
    $p$ divides the order of each $O(v_j)$ and 
    $|X|=p^n-1$ is not divisible by $p$, 
    it follows that $X_0\ne\emptyset$. In particular, 
    there exists $v\in V$ such that $x\cdot v=v$ for
    all $x\in G$. 
\end{proof}

The analog of Kolchin's theorem is the following result:

\begin{proposition}
\label{pro:Kolchin}
    Every $p$-subgroup of $\GL_n(p)$ is conjugate to a subgroup
    of the unipotent subgroup $U_n(p)$. 
\end{proposition}

\begin{proof}[Sketch of the proof]
    Let $P$ be a $p$-subgroup of $\GL_n(p)$. 
    Then $P$ acts linearly on an $n$-dimensional 
    $\F_p$-vector space $V$ by left multiplication. 
    The previous
    proposition implies that there exists a non-zero
    $v_1\in V$
    such that $xv_1=v_1$ for all $x\in P$. Let 
    $V_1=\langle v_1\rangle$. The group $P$ 
    acts on the $(n-1)$-dimensional vector space 
    $V/V_1$ by 
    \[
    x\cdot (v+V_1)=xv+V_1.
    \]
    This action is well-defined. 
    As before, there exists a non-zero 
    vector of $V/V_1$ fixed by $P$. Thus 
    there exists $v_2\in V\setminus V_1$ such that 
    $xv_2+V_1=v_2+V_1$. Note that $\{v_1,v_2\}$ is linearly
    independent, as applying the canonical
    map $V\to V/V_1$ to 
    $\alpha v_1+\beta v_2=0$ one obtains
    that $\beta=0$ and therefore $\alpha=0$. This process
    produces a basis $\{v_1,\dots,v_n\}$ 
    of $V$ and a sequence $\{0\}\subsetneq V_1\subsetneq V_2\subsetneq\cdots\subsetneq V_n=V$, where 
    $V_j=\langle v_1,\dots,v_j\rangle$ for all $j\in\{1,\dots,n\}$. Moreover,  
    $PV_j\subseteq V_j$  and 
    $Pv_j=v_j+V_{j-1}$ for all $j$. This
    means precisely that in the basis 
    $\{v_1,\dots,v_n\}$ 
    every element of $P$ is an upper triangular
    matrix with ones in the main diagonal. 
\end{proof}

\index{Sylow's theorems}
Proposition \ref{pro:Kolchin} is deeply 
connected to Sylow's theorems. 

\begin{exercise}
    Prove that the normalizer of $U_n(p)$ in $\GL_n(p)$ is the
    Borel subgroup $B_n(p)$ of upper triangular matrices. 
\end{exercise}

Now we have the following explicit Sylow theory for
$\GL_n(p)$. The first two Sylow theorems 
appear in the following result. 

\begin{exercise}
    Prove that  $U_n(p)$ is a Sylow $p$-subgroup of $\GL_n(p)$. 
\end{exercise}

What about the third Sylow's theorem? 
First note that the number $n_p$
of conjugates of $U_n(p)$ in $\GL_n(p)$ 
is the number of complete flags 
in $\F_p^n$.

\begin{exercise}
    Prove that $n_p\equiv1\bmod p$. 
\end{exercise}


\chapter{}

\topic{Group algebras}

\index{Group algebra}
Let $G$ be a finite group. The (complex) \textbf{group
algebra} $\C[G]$ is the $\C$-vector space with
basis $\{g:g\in G\}$ and multiplication
\[
\left(\sum_{g\in G}\lambda_gg\right)\left(\sum_{h\in G}\mu_hh\right)
=\sum_{g,h\in G}\lambda_g\mu_h(gh).
\]

Clearly, $\dim \C[G]=|G|$. Moreover, 
$\C[G]$ is commutative if and only if $G$ is abelian. 

\index{Augmentation ideal}
If $G$ is non-trivial, 
then $\C[G]$ contains proper non-trivial ideals. For example, 
the \textbf{augmentation ideal} 
\[
I(G)=\left\{\sum_{g\in G}\lambda_gg\in \C[G]:\sum_{g\in G}\lambda_g=0\right\}
\]
is a non-zero proper ideal of $\C[G]$. 

\begin{exercise}
Let $C_n$ be the cyclic group of order $n$ (written multiplicatively).
Prove that $\C[G]\simeq \C[X]/(X^n-1)$. 
\end{exercise}

\begin{exercise}
    Let $G$ be a finite non-trivial group. Prove that
    $\C[G]$ has zero divisors. 
\end{exercise}

\begin{exercise}
    Let $G$ be a finite group. The set
    \[
    \Fun(G,\C)=\{\alpha\colon G\to\C\}
    \]
    is a complex vector space with 
    the operations 
    \[
    (\alpha+\beta)(x)=\alpha(x)+\beta(x),
    \quad
    (\lambda\alpha)(x)=\lambda\alpha(x),
    \]
    for all $\alpha,\beta\in\Fun(G,\C)$, $x\in G$ 
    and $\lambda\in\C$. It is an algebra
    with the \textbf{convolution product} 
    \[
    (\alpha*\beta)(x)=\sum_{y\in G}\alpha(xy^{-1})\beta(y).
    \]
    Let 
    \[
    \delta_x(y)=\begin{cases}
            1 & \text{if $x=y$},\\
            0 & \text{otherwise}.
        \end{cases}
    \]
    Prove the following statements:
    \begin{enumerate}
        \item 
        The set $\{\delta_x:x\in G\}$ is a basis
        of $\Fun(G,\C)$. 
        \item The map $G\to\Fun(G,\C)$, $g\mapsto\delta_g$, 
            extends linearly to an algebra isomorphism. 
    \end{enumerate}
\end{exercise}


\index{Module!semisimple}
Recall that a finite-dimensional module $M$ is semisimple 
if and only if for every submodule $S$ of $M$ there 
is a submodule $T$ of $M$ such that $M=S\oplus T$.    

\begin{theorem}[Maschke]
\index{Maschke's theorem}
    Let $G$ be a finite
    group and $M$ be a finite-dimensional $\C[G]$-module.
    Then $M$ is semisimple. 
\end{theorem}

\begin{proof}
We need to show that every submodule $S$ of $M$ admits a complement. 
Since $S$ is a subspace of $M$, there exists a subspace $T_0$ of $M$ 
such that $M=S\oplus T_0$ (as vector spaces). We use 
$T_0$ to construct a submodule $T$ of $M$ that complements $S$. Since $M=S\oplus T_0$, 
every $m\in M$ can be written uniquely as $m=s+t_0$ for some $s\in S$ and $t_0\in T$. 
Let 
\[
p_0\colon M\to S,\quad
p_0(m)=s,
\]
where $m=s+t_0$ with $s\in S$ and $t_0\in T$. 
If $s\in S$, then $p_0(s)=s$. In particular, $p_0^2=p_0$, as 
$p_0(m)\in S$. 

Note that, in general, $p_0$ is not a $K[G]$-modules homomorphism. 
Let 
\[
p\colon M\to S,\quad
p(m)=\frac{1}{|G|}\sum_{g\in G}g^{-1}\cdot p_0(g\cdot m).
\]

We claim that $p$ is a homomorphism of $K[G]$-modules. For that purpose, we need to show that 
$p(g\cdot m)=g\cdot p(m)$ for all $g\in G$ and $m\in M$. In fact, 
\[
p(g\cdot m)=\frac{1}{|G|}\sum_{h\in G}h^{-1}\cdot p_0(h\cdot (g\cdot m))
=\frac{1}{|G|}\sum_{h\in G}(gh^{-1})\cdot p_0(h\cdot m)=g\cdot p(m).
\]

We now claim that $p(M)=S$. The inclusion $\subseteq$ is trivial to prove, as $S$ is a submodule of $M$ 
and $p_0(M)\subseteq S$. Conversely, if $s\in S$, then $g\cdot s\in S$, as 
$S$ is a submodule. Thus 
$s=g^{-1}\cdot (g\cdot s)=g^{-1}\cdot p_0(g\cdot s)$ and hence 
\[
s=\frac{1}{|G|}\sum_{g\in G}g^{-1}\cdot (g\cdot s)=\frac{1}{|G|}\sum_{g\in G}g^{-1}\cdot (p_0(g\cdot s))=p(s).
\]
Since $p(m)\in S$ for all $m\in M$, it follows that $p^2(m)=p(m)$, so $p$ is a projector onto $S$. 
Hence $S$ admits a complement in $M$, that is $M=S\oplus\ker(p)$.
\end{proof}

\begin{exercise}
Let $G=\langle g\rangle$ be the cyclic group 
of order four and $\rho_g=\begin{pmatrix}
0&-1\\
1&0\end{pmatrix}$. 
Let $M=\R^{2\times 1}$ as an $\C[G]$-module with 
\[
g\cdot\begin{pmatrix}u\\v\end{pmatrix}
=\begin{pmatrix}-v\\u\end{pmatrix}.
\]
Prove that $M$ is a semisimple non-simple $\C[G]$-module.
\end{exercise}

\begin{exercise}
Let $G=\langle g\rangle$ be the cyclic group 
of order four and $\rho_g=\begin{pmatrix}
0&-1\\
1&0\end{pmatrix}$. 
Let $M=\R^{2\times 1}$ as an $\R[G]$-module with 
\[
g\cdot\begin{pmatrix}u\\v\end{pmatrix}
=\begin{pmatrix}-v\\u\end{pmatrix}.
\]
Prove that $M$ is a simple $\R[G]$-module. 
\end{exercise}

If $G$ is a finite group, 
then $\C[G]$ is semisimple. By Artin--Wedderburn theorem, 
\[
\C[G]\simeq\prod_{i=1}^r M_{n_i}(\C),
\]
where $r$ is the number of isomorphism classes of simple modules of $\C[G]$. Moreover, 
$|G|=\dim\C[G]=\sum_{i=1}^r n_i^2$. By convention, 
we always assume that $n_1=1$. 
This corresponds, of course, to the \textbf{trivial module}. 

\begin{theorem}
    Let $G$ be a finite group. The number of simple 
    modules of $\C[G]$ coincides with the number of conjugacy classes of $G$. 
\end{theorem}

\begin{proof}
    By Artin--Wedderburn theorem $\C[G]\simeq\prod_{i=1}^rM_{n_i}(\C)$. Thus 
    \[
		Z(\C[G])\simeq\prod_{i=1}^rZ(M_{n_i}(\C))\simeq\C^r.
	\]
	In particular, $\dim Z(\C[G])=r$. If $\alpha=\sum_{g\in
	G}\lambda_gg\in Z(\C[G])$, then $h^{-1}\alpha h=\alpha$ for all $h\in
	G$. Thus 
	\[
		\sum_{g\in G}\lambda_{hgh^{-1}}g=
		\sum_{g\in g}\lambda_g h^{-1}gh=\sum_{g\in G}\lambda_gg
	\]
	and hence $\lambda_{g}=\lambda_{hgh^{-1}}$ for all $g,h\in G$. A basis for 
	$Z(\C[G])$ is given by elements of the form 
	\[
		\sum_{g\in K}g,
	\]
	where $K$ is a conjugacy class of $G$. Therefore $\dim Z(\C[G])$ equals 
	the number of conjugacy classes of $G$.
\end{proof}

\begin{exercise}
    Let $G$ be a finite group of order $n$ with $k$ conjugacy classes.
    Let $m=(G:[G,G])$. Prove that $n+3m\geq4k$. 
\end{exercise}

For $n\in\Z_{\geq2}$, we write $C_n$ to denote the (multiplicative) cyclic group of order $n$. 

\begin{exercise}
    Prove that $\C[C_4]\simeq\C^4$. 
\end{exercise}

For $n\geq1$, let $\Sym_n$ denote the symmetric group in $n$ letters. 

\begin{example}
    The group $\Sym_3$ has three conjugacy classes:
    $\{\id\}$, $\{(12),(13),(23)\}$ and $\{(123),(132)\}$. 
    Since $6=1^2+a^2+b^2$, it follows that 
    $\C[G]\simeq\C\times\C\times M_2(\C)$. 
\end{example}    

\subsection{Some comments}


There is a multiplicative version of Maschke's theorem. A group $G$ acts 
by automorphisms on $A$ if there is a group homomorphism 
$\lambda\colon G\to\Aut(A)$. In this case, a subgroup $B$ of $A$ is said to be 
$G$-invariant if $\lambda(B)\subseteq B$. 

\begin{theorem}
\index{Maschke's theorem!multiplicative version}
    Let $K$ be a finite group of order $m$. Assume that 
    $K$ acts by automorphisms on $V=U\times W$, where
    $U$ and $W$ are subgroups of $V$ and $U$ is abelian and $K$-invariant. 
    If the map $U\to U$, $u\mapsto u^m$, is bijective, 
    then there exists a normal $K$-invariant subgroup $N$ of $V$ 
    such that $V=U\times N$. 
\end{theorem}

\begin{proof}
Let $\theta\colon U\times W\to U$, $(u,w)\mapsto u$. Then $\theta$ is a group homomorphism such that 
$\theta(u)=u$ for all $u\in U$. Since $U$ is $K$-invariant, 
\[
k^{-1}\cdot \theta(k\cdot v)\in U
\]
for all $k\in K$ and $v\in V$. 
Since $K$ is finite and $U$ is abelian, 
the map 
\[
\varphi\colon V\to U,\quad 
v\mapsto \prod_{k\in K}k^{-1}\cdot \theta(k\cdot v), 
\]
is well-defined. 
We claim that $\varphi$ is a group homomorphism. If $x,y\in V$, then 
\begin{align*}
    \varphi(xy) &= \prod_{k\in K}k^{-1}\cdot \theta(k\cdot (xy))\\
    &= \prod_{k\in K}k^{-1}\cdot (\theta(k\cdot x)\theta(k\cdot y))\\
    &= \prod_{k\in K}k^{-1}\cdot \theta(k\cdot x) \prod_{k\in K}k^{-1}\cdot \theta(k\cdot y)=\varphi(x)\varphi(y),
\end{align*}
since $U$ is abelian and $K$ acts by automorphisms on $V$. 

We claim that $N=\ker\varphi$ is $K$-invariant. 
We need to show that $\varphi(l\cdot x)=l\cdot\varphi(x)$ for all $l\in K$ and $x\in V$. 
If $l\in K$ and $x\in V$, then 
\begin{align*}
l^{-1}\cdot\varphi(l\cdot x)&=l^{-1}\cdot\left(\prod_{k\in K}k^{-1}\cdot \theta(k\cdot (l\cdot x))\right)=\prod_{k\in K}(kl)^{-1}\cdot\theta( (kl)\cdot x)=\varphi(x),
\end{align*}
since $kl$ runs over all the elements of $K$ whenever $k$ runs over all the elements of $K$.
In conclusion, $\ker\varphi$ is $K$-invariant. 

It remains to show that $V$ is the direct product of $U$ and $N$. By assumption, $U$ is normal in $V$. 
We first prove that $U\cap N=\{1\}$. If $u\in U$, then $k\cdot u\in U$ for all $k\in K$. This implies that 
$k^{-1}\cdot\theta(k\cdot u)=k^{-1}\cdot (k\cdot u)=u$. Hence $\varphi(u)=u^m$. Since this map is bijective by assumption,  
\[
U\cap N=U\cap\ker\varphi=\{1\}.
\]
We now show that $V\subseteq UN$, as the other inclusion is trivial. Since $N=\ker\varphi$,  
\[
\varphi(V)\subseteq U=\varphi(U)=\varphi(U)\varphi(N)=\varphi(UN) 
\]
and hence $V\subseteq (UN)N=UN$. 
Therefore $V$ is the direct product of $U$ and $N$, as $N$ is normal in $V$.
\end{proof}

\begin{corollary}
    Let $p$ be a prime number and $K$ be a finite
    group with order not divisible by $p$. Let $V$ be
    a $p$-elementary abelian group. Assume that $K$ acts
    by automorphism on $V$. If $U$ be a $K$-invariant subgroup of $V$, 
    then there exists a $K$-invariant subgroup $N$ of $V$ 
    such that $V=U\times N$. 
\end{corollary}

\begin{proof}
    Let $m=|K|$. Since $m$ and $|U|$ are coprime, the map 
    $u\mapsto u^m$ is bijective in $U$. Since $V$ is a vector space over the field 
    $\Z/p$, it follows that $V=U\times W$ for some subgroup $W$ of $V$. Now the claim follows
    from the previous theorem. 
\end{proof}
\chapter{} 

\topic{Representations}

Unless we state differently, we will always work
with finite groups. All our vector spaces will
be complex vector spaces. 

\begin{definition}
\index{Representation}
    Let $G$ be a finite group. A \textbf{representation}
    of $G$ is a group homomorphism $\rho\colon G\to\GL(V)$, where
    $V$ is a finite-dimensional vector space. The \textbf{degree} (or dimension) 
    of the representation is the integer $\deg\rho=\dim V$. 
\end{definition}

\index{Matrix representation}
Let $G\to\GL(V)$ be a representation. 
If we fix a basis of $V$, then we obtain
a \textbf{matrix representation} of $G$, that is a 
group homomorphism 
\[
\rho\colon G\to\GL(V)\simeq\GL_n(\C),
\quad 
g\mapsto\rho_g,
\]
where
$n=\dim V$. 

\begin{example}
Since $\Sym_3=\langle (12),(123)\rangle$, the map $\rho\colon \Sym_3\to\GL_3(\C)$,
\[
(12)\mapsto\begin{pmatrix}
0 & 1 & 0\\
1 & 0 & 0\\
0 & 0 & 1
\end{pmatrix},\quad
(123)\mapsto\begin{pmatrix}
0 & 0 & 1\\
1 & 0 & 0\\
0 & 1 & 0
\end{pmatrix}
\] 
is a representation of $\Sym_3$. 
\end{example}

\begin{example}
Let $G=\langle g\rangle$ be cyclic of order six. 
The map $\rho\colon G\to\GL_2(\C)$, 
\[
g\mapsto
\begin{pmatrix}
1&1\\
-1&0
\end{pmatrix}
\] 
is a representation of $G$. 
\end{example}

\begin{example}
Let $G=\langle g\rangle$ be cyclic of order four. 
The map $\rho\colon G\to\GL_2(\C)$, 
\[
g\mapsto
\begin{pmatrix}
0&-1\\
1&0
\end{pmatrix}
\] 
is a representation of $G$. 
\end{example}

\begin{example}
  Let $G=\langle a,b:a^2=b^3=(ab)^3=1\rangle$. The map 
  \[
    a\mapsto\begin{pmatrix}
    0 & 1 & -1\\
    1 & 0 & -1\\
    0 & 0 & -1
    \end{pmatrix},
    \quad
    b\mapsto\begin{pmatrix}
      0 & 0 & 1\\
      1 & 0 & 0\\
      0 & 1 & 0
    \end{pmatrix},
  \]
  defines a representation $G\to\GL_3(\C)$. 
\end{example}

\begin{example}
  Let $G$ be a finite group that acts on a finite set $X$. 
  Let $V=\C X$ the complex vector space with basis $\{x:x\in
  X\}$. The map 
  \[
	\rho\colon G\to\GL(V),\quad
	\rho_g\left(\sum_{x\in X}\lambda_x x\right)
	=\sum_{x\in X}\lambda_x\rho_g(x)
	=\sum_{x\in X}\lambda_{g^{-1}\cdot x}x, 
  \]
  is a representation of degree $|X|$.
\end{example}

\begin{example}
    The sign $\sgn\colon\Sym_n\to\GL_1(\C)=\C^{\times}$ is a representation of $\Sym_n$.
\end{example}

An important fact is that there exists a bijective
correspondence 
between 
representations of a finite group $G$ 
and 
finite-dimensional modules over $\C[G]$. The correspondence
is given as follows. If $\rho\colon G\to\GL(V)$ is a representation, 
then $V$ is a $\C[G]$-module with
\[
\left(\sum_{g\in G}\lambda_gg\right)\cdot v=\sum_{g\in G}\lambda_g\rho_g(v).
\]
Conversely, if $V$ is a $\C[G]$-module, then
$\rho\colon G\to\GL(V)$, $\rho_g\colon V\to V$, $v\mapsto g\cdot v$, 
is a representation. 

\begin{exercise}
    Let $G$ be a finite group and 
    $\rho\colon G\to\GL(V)$ be a representation. Prove that 
    each $\rho_g$ is diagonalizable. 
\end{exercise}

The previous exercise uses properties of the minimal polynomial. We will 
see a different proof later. 

\begin{example}
\end{example}

\begin{definition}

\end{definition}

\begin{definition}

\end{definition}

Let $\rho\colon G\to\GL(V)$ be a representation. 
If $W$ is a $G$-invariant subspace of $V$, 
then the restriction $\rho|_W\colon G\to\GL(W)$
is a representation. 

\begin{definition}
\index{Representation!irreducible}
\index{Module!simple}
    A representation $\rho\colon G\to\GL(V)$ is 
    said to be \textbf{irreducible} if 
    $\{0\}$ and $V$ are the only 
    $G$-invariant subspaces of $V$. 
\end{definition}

Note that a representation $\rho\colon G\to\GL(V)$ is irreducible
if and only if $V$ is simple. 

\begin{example}
    Degree-one representations are irreducible. 
\end{example}

\begin{example}

\end{example}

\begin{proposition}
\end{proposition}

\begin{proof}
\end{proof}

\begin{example}

\end{example}
\section{Lecture: Week 4}

We will use the following notation: if $\chi$ is a character
of a group $G$ 
and $C$ is a conjugacy class of $G$, then 
$\chi(g)=\chi(xgx^{-1})$ for all $x\in G$. We write 
$\chi(C)$ to denote the value $\chi(g)$ for any $g\in C$. 

\begin{theorem}
\label{thm:B}
    Let $G$ be a finite group, $\chi\in\Irr(G)$ 
    and $K$ be a conjugacy class of $G$. Then 
    \[
    \frac{\chi(K)}{\chi(1)}|K|\in\A. 
    \]
\end{theorem}

We need a lemma. 

\begin{lemma}
    Let $x\in\C$. Then $x\in\A$ if and only if 
    there exist $z_1,\dots,z_k\in\C$ not all zero such that 
    $xz_i=\sum_{j=1}^ka_{ij}z_j$ for some $a_{ij}\in\Z$ and 
    all $i\in\{1,\dots,k\}$. 
\end{lemma}

\begin{proof}
    Let us first prove $\implies$. Let $f=X^k+a_{k-1}X^{k-1}+\cdots+a_1X+a_0\in\Z[X]$
    be such that $f(x)=0$. For $i\in\{1,\dots,k\}$ let 
    $z_i=x^{i-1}$. Then 
    $xz_i=x^i=z_{i+1}$ for all $i\in\{1,\dots,k-1\}$. Moreover, 
    $xz_k=x^k=-a_0-a_1x-\cdots-a_{k-1}x^{k-1}$.
    
    We now prove $\impliedby$. Let $A=(a_{ij})\in\Z^{k\times k}$ and 
    $Z$ be the column vector 
    $Z=\begin{pmatrix}z_1\\\vdots\\z_k\end{pmatrix}$. Note that $Z$ is non-zero. 
    Moreover, $AZ=xZ$, as 
    \[
    (AZ)_i=\sum_{j=1}^ka_{ij}z_j=xz_i=(xZ)_i
    \]
    for all $i$. Thus $x$ is an eigenvalue of $A\in\Z^{k\times k}$ and
    hence $x\in\A$. 
\end{proof}

The previous lemma could be used to give an alternative proof of the fact 
that the algebraic integers form a ring. 

\begin{proof}[Proof of Theorem \ref{thm:B}]
    Let $\varphi$ be a representation of $G$ and 
    $\chi$ be its character. Note that $\varphi$ is irreducible. 
    Let $C_1,\dots,C_r$ be the conjugacy classes of $G$ 
    and for every $i\in\{1,\dots,r\}$ let 
    \[
    T_i=\sum_{x\in C_i}\varphi_x. 
    \]
    
    \begin{claim}
        $T_i=\left(\frac{|C_i|}{\chi(1)}\chi(C_i)\right)\id$. 
    \end{claim}
    
    We proceed in several steps. First, we prove that 
    $T_i=\lambda\id$ for some $\lambda\in\C$. 
    We prove that $T_i$ is a morphism of representations:
    \[
    \varphi_gT_i\varphi_g^{-1}=\sum_{x\in C_i}\varphi_g\varphi_x\varphi_g^{-1}
    =\sum_{x\in C_i}\varphi_{gxg^{-1}}=\sum_{y\in C_i}\varphi_y=T_i.
    \]
    Now Schur's lemma implies that $T_i=\lambda\id$ for some
    $\lambda\in\C$. 
    
    We now prove that 
    \[
    \lambda=\frac{|C_i|\chi(C_i)}{\chi(1)}.
    \]
    To prove
    this we compute $\lambda$:
    \[
    \lambda\chi(1)=\trace(\lambda\id)
    =\trace T_i
    =\sum_{x\in C_i}\trace\varphi_x
    =\sum_{x\in C_i}\chi(x)
    =|C_i|\chi(C_i).
    \]
    Then the claim follows. 
    
    Now we claim that 
    \[
    T_iT_j=\sum_{k=1}^r a_{ijk}T_k
    \]
    for some $a_{ijk}\in\Z_{\geq0}$. In fact, 
    \begin{align*}
        T_iT_j &= \sum_{x\in C_i}\sum_{y\in C_j}\varphi_x\varphi_y
        =\sum_{x\in C_i}\sum_{y\in C_j}\varphi_{xy}
        =\sum_{g\in G}a_{ijg}\varphi_g,
    \end{align*}
    where $a_{ijg}$ is the number of elements $(x,y)\in C_i\times C_j$ 
    such that $g=xy$. 
    
    \begin{claim}
        Once $i$ and $j$ are fixed, $a_{ijg}$ depends only on the conjugacy class of $g$. 
    \end{claim}
    
    Let $X_g=\{(x,y)\in C_i\times C_j:g=xy\}$. If $h=kgk^{-1}$, the map
    \[
    X_g\to X_h,\quad (x,y)\mapsto (kxk^{-1},kyk^{-1}),
    \]
    is well-defined. It is bijective with inverse
    \[
    X_h\to X_g,\quad
    (a,b)\mapsto (k^{-1}ak,k^{-1}bk).
    \]
    Hence $|X_g|=|X_h|$. 

    Let $a_{ijk}$ be the number of elements 
    $(x,y)\in C_i\times C_j$ such that $xy=g$ for some $g\in C_k$. 
    Then 
    \begin{align*}
        T_iT_j & 
        =\sum_{g\in G}a_{ijg}\varphi_g
        =\sum_{k=1}^r\sum_{g\in C_k}a_{ijg}\varphi_g
        =\sum_{k=1}^ra_{ijk}\sum_{g\in C_k}\varphi_g
        =\sum_{k=1}^ra_{ijk}T_k.
    \end{align*}
    Therefore 
    \begin{equation}
        \label{eq:omega}
    \left(\frac{|C_i|}{\chi(1)}\chi(C_i)\right)
    \left(\frac{|C_j|}{\chi(1)}\chi(C_j)\right)
    =\sum_{k=1}^r a_{ijk}\left(\frac{|C_k|}{\chi(1)}\chi(C_k)\right).
    \end{equation}
    By the previous lemma, $x=\frac{|C_j|}{\chi(1)}\chi(C_j)\in\A$.
\end{proof}

\subsection{Frobenius' theorem}
\label{degree}

\begin{theorem}[Frobenius]
\index{Frobenius' theorem}
\label{thm:Frobenius_chi(1)}
    Let $G$ be a finite group and $\chi\in\Irr(G)$. 
    Then $\chi(1)$ divides~$|G|$. 
\end{theorem}

\begin{proof}
    Let $\varphi$ be an irreducible representation with character $\chi$. 
    Since $\langle\chi,\chi\rangle=1$, 
    \[
    \frac{|G|}{\chi(1)}=\frac{|G|}{\chi(1)}\langle\chi,\chi\rangle
    =\sum_{g\in G}\frac{\chi(g)}{\chi(1)}\overline{\chi(g)}.
    \]
    Note that this is a rational number. 
    Let $C_1,\dots,C_r$ be the conjugacy classes of $G$. 
    Then 
    \[
        \frac{|G|}{\chi(1)}
        =\sum_{i=1}^r\sum_{g\in C_i}\frac{\chi(g)}{\chi(1)}\overline{\chi(g)}
        =\sum_{i=1}^r\left(\frac{|C_i|}{\chi(1)}\chi(C_i)\right)\overline{\chi(C_i)}\in\A\cap\Q=\Z,
    \]
    as $\overline{\chi(C_i)}\in\A$. This implies that $\chi(1)$ divides $|G|$. 
\end{proof}

The character table gives information on the structure of the group. For example,
with the previous result, one can easily prove that
groups of order $p^2$ (where $p$ is a prime number) are abelian. 

\begin{exercise}
    Let $p$ and $q$ be prime numbers such that $p<q$.
    If $q\not\equiv1\bmod p$, then a group of order $pq$ is abelian. 
\end{exercise}

Another application:

\begin{theorem}
\label{thm:simple}
    Let $G$ be a finite simple group. 
    Then $\chi(1)\ne2$ for all $\chi\in\Irr(G)$. 
\end{theorem}

\begin{proof}
    Let $\chi\in\Irr(G)$ be such that $\chi(1)=2$. Let $\rho\colon G\to\GL_2(\C)$
    be an irreducible representation of $G$ with character $\chi$. Since 
    $G$ is simple, $\ker\rho=\{1\}$. Since $\chi(1)=2$, 
    $G$ is non-abelian and hence $[G,G]=G$. Since 
    $G$ has $(G:[G,G])=1$ degree-one characters, it follows that
    $G$ has only one degree-one character, the trivial one. The composition
    \[
    \begin{tikzcd}
    	G & {\GL_2(\C)} & {\C^{\times}}
    	\arrow["{\rho }", hook, from=1-1, to=1-2]
    	\arrow["{\det }", from=1-2, to=1-3]
    \end{tikzcd}
    \]
    is a degree-one representation, which means that $\det\rho_g=1$ for all $g\in G$. 
    By Frobenius' theorem, $|G|$ is even (because 
    $2=\chi(1)$ divides $|G|$). Let $x\in G$ be such that $|x|=2$ (Cauchy's theorem). 
    Then $|\rho_x|=2$, as $\rho$ is injective. Since $\rho_x$ is diagonalizable, 
    there exists $C\in\GL_2(\C)$ such that
    \[
    C\rho_xC^{-1}=\begin{pmatrix}
    \lambda&0\\
    0&\mu
    \end{pmatrix}
    \]
    for some $\lambda,\mu\in\{-1,1\}$. Since $1=\det\rho_x=\lambda\mu$ and
    $\rho_x$ is not the identity matrix, $\lambda=\mu=-1$. In particular, $C\rho_xC^{-1}$ is central
    and hence $\rho_x$ is central. Since $\rho$ is injective, $x$ is central 
    and thus $Z(G)\ne\{1\}$, a contradiction. 
\end{proof}


\begin{theorem}[Schur]
\index{Schur's theorem}
\label{thm:Schur_chi(1)}
    Let $G$ be a finite group and $\chi\in\Irr(G)$. 
    Then $\chi(1)$ divides $(G:Z(G))$. 
\end{theorem}

Let $G$ and $G_1$ be groups. If $V$ is a $\C[G]$-module and 
$V_1$ is a $\C[G_1]$-module, then 
$V\otimes V_1$ is a $\C[G\times G_1]$-module 
with 
\[
(g,g_1)\cdot v\otimes v_1=(g\cdot v)\otimes (g_1\cdot v_1)
\]
for $(g,g_1)\in G\times G_1$, $v\in V$ and $v_1\in V_1$. 

\begin{lemma}
    Let $G$ and $G_1$ be finite groups. If $\rho$ is an irreducible
    representation of $G$ and $\rho_1$ is an irreducible representation
    of $G_1$, then 
    $\rho\otimes\rho_1$ is an irreducible representation of $G\times G_1$. 
\end{lemma}

\begin{proof}
    Write $\chi=\chi_{\rho}$ and $\chi_1=\chi_{\rho_1}$. Since
    $\chi$ is irreducible, $\langle\chi,\chi\rangle=1$. Similarly, 
    $\langle\chi_1,\chi_1\rangle=1$. Now
    $\rho\otimes\rho_1$ is irreducible, as 
    \begin{align*}
    \langle\chi\chi_1,\chi\chi_1\rangle
    &=\frac{1}{|G\times G_1|}\sum_{(g,g_1)\in G\times G_1}(\chi\chi_1)(g,g_1)\overline{(\chi\chi_1)(g,g_1)}\\
    &=\frac{1}{|G||G_1|}\sum_{g\in G}\sum_{g_1\in G}\chi(g)\chi_1(g_1)\overline{\chi(g)}\overline{\chi_1(g_1)}\\
    &=\frac{1}{|G||G_1|}\sum_{g\in G}\chi(g)\overline{\chi(g)}\sum_{g_1\in G}\chi_1(g_1)\overline{\chi_1(g_1)}\\
    &=\langle\chi,\chi\rangle\langle\chi_1,\chi_1\rangle=1.\qedhere 
    \end{align*}
\end{proof}

\begin{exercise}
    Let $G$ and $G_1$ be finite groups. 
    Prove that irreducible characters of $G\times G_1$ 
    are of the form $\chi\chi_1$ for  
    $\chi\in\Irr(G)$ and $\chi_1\in\Irr(G_1)$. 
\end{exercise}

\index{Tensor power trick}
We now prove Schur's theorem. The proof goes back to Tate; it uses the 
\emph{tensor power trick}. See
Tao's blog  
\url{https://terrytao.wordpress.com} for other applications of this powerful
trick. 

\begin{proof}[Proof of Theorem \ref{thm:Schur_chi(1)}]
    Let $\rho\colon G\to\GL(V)$ be an irreducible representation 
    with character $\chi$. Let $z\in Z(G)$. Then $\rho_z$ commutes
    with $\rho_g$ for all $g\in G$. By Schur's lemma, 
    $\rho_z(v)=\lambda(z)v$ for all $v\in V$. Note that
    $\lambda\colon Z(G)\to\C^{\times}$, $z\mapsto\lambda(z)$, 
    is a well-defined group homomorphism, as 
    \[
    \lambda(z_1z_2)v=\rho_{z_1z_2}(v)=\rho_{z_1}\rho_{z_2}(v)
    =\lambda(z_2)\rho_{z_1}(v)=\lambda(z_1)\lambda(z_2)v
    \]
    for all $v\in V$ and $z_1,z_2\in Z(G)$. 
    
    Let $n\in\Z_{\geq1}$. Write $G^n=G\times\cdots\times G$ ($n$-times). Let
    \[
    \sigma\colon G^n\to\GL(V^{\otimes n}),\quad
    (g_1,\dots,g_n)\mapsto \rho_{g_1}\otimes\cdots\otimes\rho_{g_n}.
    \]
    Then $\sigma$ is a representation. 
    The character of $\sigma$ is $\chi^n$. By the previous lemma, 
    $\sigma$ is
    irreducible. For $z_1,\dots,z_n\in Z(G)$, we compute
    \begin{align*}   
    \sigma(z_1,\dots,z_n)(v_1\otimes\cdots\otimes v_n)&=\rho_{z_1}v_1\otimes\cdots\otimes \rho_{z_n}v_n\\
    &=\lambda(z_1)\cdots\lambda(z_n)v_1\otimes\cdots\otimes v_n\\
    &=\lambda(z_1\cdots z_n)v_1\otimes\cdots\otimes v_n.
    \end{align*}
    Let 
    \[
    H=\{(z_1,\dots,z_n)\in Z(G)^n:z_1\cdots z_n=1\}.
    \]  
    Then $H$ is a central subgroup of $G^n$. Moreover, 
    $H$ acts trivially on $V^{\otimes n}$, so there exists
    a group homomorphism $\sigma$ that makes the diagram 
    \[\begin{tikzcd}
	{G^n} && {\GL(V^{\otimes n})} \\
	{G^n/H}
	\arrow["\sigma", from=1-1, to=1-3]
	\arrow[from=1-1, to=2-1]
	\arrow["\tau"', dashed, from=2-1, to=1-3]
    \end{tikzcd}\]
    commutative. Thus  
    \[
    \tau\colon G^n/H\to\GL(V^{\otimes n}),
    \]
    is a representation 
    of degree $\chi(1)^n$:
    Since $\sigma$ is irreducible, so is $\tau$. 
    By Frobenius' theorem, $\chi(1)$ divides $|G|$ 
    and $\chi(1)^n$ divides $|G^n/H|=\frac{|G|^n}{|Z(G)|^{n-1}}$. 
    Write 
    \[
    |G|=\chi(1)s\quad\text{ and }\quad 
    |G|(G:Z(G))^{n-1}=\chi(1)^nr
    \]
    for some $r,s\in\Z$. Let $a$ and $b$ be such that 
    $\gcd(a,b)=1$ and 
    $\frac{a}{b}=\frac{(G:Z(G))}{\chi(1)}$. Then
    \[
    s\left(\frac{a}{b}\right)^{n-1}=s\frac{(G:Z(G))^{n-1}}{\chi(1)^{n-1}}
    =\frac{|G|}{\chi(1)}\frac{(G:Z(G))^{n-1}}{\chi(1)^{n-1}}=r\in\Z.
    \]
    Thus $b^{n-1}$ divides $s$ and hence $b=1$ (because $n$ is arbitrary).  
\end{proof}

\begin{theorem}[It\^o]
\index{It\^o's theorem}
\label{thm:Ito}
Let $G$ be a finite group and $\chi\in\Irr(G)$. Then 
$\chi(1)$ divides $(G:A)$ for all normal abelian subgroup $A$ of $G$.  
\end{theorem}

The proof of Theorem~\ref{thm:Ito} is no more difficult than that of Schur's Theorem~\ref{thm:Schur_chi(1)}. For a proof, see \cite[\S8.1]{MR0450380}.


\subsection{Examples of character tables}

Let $G$ be a finite group and $\chi_1,\dots,\chi_r$ be the irreducible characters of $G$. Without loss of generality, 
we may assume that $\chi_1$ is the trivial character, i.e.,  $\chi_1(g)=1$ for all $g\in G$. 
Recall that $r$ is the number of conjugacy classes of $G$. Each $\chi_j$ is constant on conjugacy classes. 
The \emph{character table} of $G$ is presented as follows, arranging group elements and character values in a tabular format:
\bigskip 
\begin{center}
\begin{tabular}{|c|cccc|}
\hline 
 & $1$ & $k_{2}$ & $\cdots$ & $k_{r}$\tabularnewline
 & $1$ & $g_{2}$ & $\cdots$ & $g_{r}$\tabularnewline
\hline 
$\chi_{1}$ & $1$ & $1$ & $\cdots$ & $1$\tabularnewline
$\chi_{2}$ & $n_{2}$ & $\chi_{2}(g_{2})$ & $\cdots$ & $\chi_{2}(g_{r})$\tabularnewline
$\vdots$ & $\vdots$ & $\vdots$ & $\ddots$ & $\vdots$\tabularnewline
$\chi_{r}$ & $n_{r}$ & $\chi_{r}(g_{2})$ & $\cdots$ & $\chi_{r}(g_{r})$\tabularnewline
\hline
\end{tabular}
\end{center}
\bigskip 
Here, the numbers $n_j$ represent the degrees of the irreducible representations of $G$, and each $k_j$ denotes  
the size of the conjugacy class of the element $g_j$. By convention, the character table
contains only the values of the irreducible characters of the group. 

\begin{example}
    For $n\geq2$, let  
	$G=\langle g\rangle$ be the cyclic group of order $n$. Let $\lambda$ be a primitive $n$-th root of one. For each $i$, 
    let $V_i$ be a (complex) one-dimensional vector space with basis 
	$\{v\}$. Each $V_i$ is a $\C[G]$-module with 
    \[
		g\cdot v=\lambda^{i-1}v.
	\]
	Moreover, each $V_i$ is simple, as $\dim V_i=1$. The character $\chi_i$ associated with 
	$V_i$ is given by $\chi_i(g^m)=\lambda^{m(i-1)}$ for all 
	$m\in\{1,\dots,n\}$. Since the $\chi_1,\dots,\chi_n$ are all different and $G$ admits $n$ irreducible representations,
    it follows that $\Irr(G)=\{\chi_1,\dots,\chi_n\}$. The character
    table of $G$ is shown in Table~\ref{tab:C_n}.
    
    \begin{table}[h]
    \caption{The character table of the cyclic group $C_n$ of order $n$.} 
    \label{tab:C_n}
		\begin{tabular}{|c|ccccc|}
			\hline 
			& 1 & 1 & 1 & $\cdots$ & 1\tabularnewline
			& $1$ & $g$ & $g^2$ & $\cdots$ & $g^{n-1}$\tabularnewline
			\hline 
			$\chi_{1}$ & $1$ & $1$ & $1$ & $\cdots$ & $1$\tabularnewline
			$\chi_{2}$ & $1$ & $\lambda$ & $\lambda^2$ & $\cdots$ & $\lambda^{n-1}$\tabularnewline
			$\chi_{3}$ & $1$ & $\lambda^2$ & $\lambda^4$ & $\cdots$ & $\lambda^{n-2}$\tabularnewline
			$\vdots$ & $\vdots$ & $\vdots$ & $\vdots$ & $\ddots$ & $\vdots$\tabularnewline
			$\chi_{n}$ & $1$ & $\lambda^{n-1}$ & $\lambda^{n-2}$ & $\cdots$ & $\lambda$\tabularnewline
			\hline
		\end{tabular}
	\end{table}
\end{example}

\begin{example}
	Let $G=\langle g:g^4=1\rangle$ 
	be the cyclic group of order four. The character table of $G$ is given by Table~\ref{tab:C4}.  Let us see how to see this calculation on the computer:
% \begin{lstlisting}
% gap> C4 := CyclicGroup(4);;                       
% gap> T := CharacterTable(C4);;
% gap> Display(T);
% CT1

%      2  2  2  2  2

%       1a 4a 2a 4b

% X.1     1  1  1  1
% X.2     1 -1  1 -1
% X.3     1  A -1 -A
% X.4     1 -A -1  A

% A = E(4)
%   = Sqrt(-1) = i
% \end{lstlisting}
\begin{lstlisting}
> C4 := CyclicGroup(4);
> T := CharacterTable(C4);
> T;


Character Table of Group C4
---------------------------


--------------------
Class |   1  2  3  4
Size  |   1  1  1  1
Order |   1  2  4  4
--------------------
p  =  2   1  1  2  2
--------------------
X.1   +   1  1  1  1
X.2   +   1  1 -1 -1
X.3   0   1 -1  I -I
X.4   0   1 -1 -I  I


Explanation of Character Value Symbols
--------------------------------------

I = RootOfUnity(4)    
\end{lstlisting}
    
	\begin{table}[h]
    \caption{The character table of the cyclic group $C_4$ of order four.}
    \label{tab:C4}
		\begin{tabular}{|c|cccc|}
			\hline 
			& 1 & 1 & 1 & 1\tabularnewline
			& $1$ & $g$ & $g^2$ & $g^{3}$\tabularnewline
			\hline 
			$\chi_{1}$ & $1$ & $1$ & $1$ & $1$\tabularnewline
			$\chi_{2}$ & $1$ & $1$ & $-1$ & $-1$ \tabularnewline
			$\chi_{3}$ & $1$ & $-1$ & $i$ & $-i$\tabularnewline
			$\chi_{4}$ & $1$ & $-1$ & $-i$ & $i$\tabularnewline
			\hline
		\end{tabular}
	\end{table}

%\begin{lstlisting}
%julia> G = cyclic_group(4);
%
%julia> T = character_table(G)
%<pc group of size 4 with 2 generators>
%
%  2  2    2  2    2
%                   
%    1a   4a 2a   4b
%                   
%X_1  1    1  1    1
%X_2  1  z_4 -1 -z_4
%X_3  1   -1  1   -1
%X_4  1 -z_4 -1  z_4    
%\end{lstlisting}
%\begin{lstlisting}
%julia> G = cyclic_group(4);
%
%julia> Oscar.with_unicode() do
%       show(character_table(G))
%       end;
%<pc group of size 4 with 2 generators>
%
% 2  2   2  2   2
%                
%   1a  4a 2a  4b
%                
%χ₁  1   1  1   1
%χ₂  1  ζ₄ -1 -ζ₄
%χ₃  1  -1  1  -1
%χ₄  1 -ζ₄ -1  ζ₄  
%\end{lstlisting}
Some remarks: 
 \begin{enumerate}
     \item The symbol \lstinline{I} denotes a primitive fourth root of 1.
     \item 	The function \lstinline{CharacterTable} computes more than just the character table of the group; it also provides additional information.
 \end{enumerate}
%\begin{lstlisting}
%julia> orders_class_representatives(T)
%4-element Vector{Int64}:
% 1
% 4
% 2
% 4
%
%julia> class_lengths(T)
%4-element Vector{fmpz}:
% 1
% 1
% 1
% 1
%
%julia> orders_centralizers(T)
%4-element Vector{fmpz}:
% 4
% 4
% 4
% 4
%\end{lstlisting}
\begin{lstlisting}
> T[1];
( 1, 1, 1, 1 )
> Degree(T[1]);
1
> Degree(T[2]);
1
> Degree(T[3]);
1
> Degree(T[4]);
1    
\end{lstlisting}
% \begin{lstlisting}
% gap> OrdersClassRepresentatives(T);
% [ 1, 4, 2, 4 ]
% gap> SizesCentralizers(T);
% [ 4, 4, 4, 4 ]
% gap> SizesConjugacyClasses(T);
% [ 1, 1, 1, 1 ]
% \end{lstlisting}
\end{example}




\begin{example}
	The character table of the group $C_2\times C_2=\{1,a,b,ab\}$ is
    shown in Table~\ref{tab:C2xC2}. 
    
	\begin{table}[h]
    \caption{The character table of $C_2\times C_2$.}
    \label{tab:C2xC2}
		\begin{tabular}{|c|rrrr|}
			\hline 
			& 1 & 1 & 1 & 1\tabularnewline
			& $1$ & $a$ & $b$ & $ab$\tabularnewline
			\hline 
			$\chi_{1}$ & $1$ & $1$ & $1$ & $1$\tabularnewline
			$\chi_{2}$ & $1$ & $1$ & $-1$ & $-1$\tabularnewline
			$\chi_{3}$ & $1$ & $-1$ & $1$ & $-1$\tabularnewline
			$\chi_{4}$ & $1$ & $-1$ & $-1$ & $1$\tabularnewline
			\hline
		\end{tabular}
	\end{table}
    
	Let us do this by computer:
\begin{lstlisting}
> C2xC2 := AbelianGroup([2,2]);
> T := CharacterTable(C2xC2);
> T;


Character Table of Group C2xC2
------------------------------


--------------------
Class |   1  2  3  4
Size  |   1  1  1  1
Order |   1  2  2  2
--------------------
p  =  2   1  1  1  1
--------------------
X.1   +   1  1  1  1
X.2   +   1 -1  1 -1
X.3   +   1  1 -1 -1
X.4   +   1 -1 -1  1    
\end{lstlisting}
% \begin{lstlisting}
% gap> Display(CharacterTable(AbelianGroup([2,2])));
% CT2

%      2  2  2  2  2

%       1a 2a 2b 2c

% X.1     1  1  1  1
% X.2     1 -1  1 -1
% X.3     1  1 -1 -1
% X.4     1 -1 -1  1
% \end{lstlisting}
%\begin{lstlisting}
%julia> A = abelian_group(PcGroup, [2,2]);
%
%julia> character_table(A)
%<pc group of size 4 with 2 generators>
%
%  2  2  2  2  2
%               
%    1a 2a 2b 2c
%               
%X_1  1  1  1  1
%X_2  1 -1  1 -1
%X_3  1  1 -1 -1
%X_4  1 -1 -1  1
%\end{lstlisting}
\end{example}

% \begin{exercise}
%     Let $A$ and $B$ be abelian groups. 
%     We write $\Irr(A)=\{\rho_1,\dots,\rho_r\}$ and 
%     $\Irr(B)=\{\phi_1,\dots,\phi_s\}$. Prove
%     that the maps 
%     \[
%     \varphi_{ij}\colon A\times B\to\C^\times,\quad
%     (a,b)\mapsto\rho_i(a)\phi_j(b),
%     \]
%     where $i\in\{1,\dots,r\}$ and $j\in\{1,\dots,s\}$, are the irreducible representations of $A\times B$. 
% \end{exercise}

\begin{example}
	The character table of $\Sym_3$ was computed on Table~\ref{tab:S3}; see page~\pageref{tab:S3}. 
	Let us recall briefly one possible way to compute this table. 
	Degree-one characters were easy to compute. 
	To compute the third row of the table, one possible approach is to use
	the irreducible representation  
	\[
	(12)\mapsto \begin{pmatrix}-1&1\\0&1\end{pmatrix},
	\quad
	(123)\mapsto \begin{pmatrix}0&-1\\1&-1\end{pmatrix}.
	\]
    Then	
    \begin{align*}
		&\chi_3\left( (12) \right)=\trace\begin{pmatrix}-1&1\\0&1\end{pmatrix}=0,\\
		&\chi_3\left( (123) \right)=\chi_3\left( (12)(23)\right)=\trace\begin{pmatrix}0&-1\\1&-1\end{pmatrix}=-1.
	\end{align*}

	We should remark that the irreducible representation 
	mentioned is not needed to
	compute the third row of the character table. 
\begin{lstlisting}
> S3 := Sym(3);
> T := CharacterTable(S3);
> T;


Character Table of Group S3
---------------------------


-----------------
Class |   1  2  3
Size  |   1  3  2
Order |   1  2  3
-----------------
p  =  2   1  1  3
p  =  3   1  2  1
-----------------
X.1   +   1  1  1
X.2   +   1 -1  1
X.3   +   2  0 -1    
\end{lstlisting}
% \begin{lstlisting}
% gap> S3 := SymmetricGroup(3);;
% gap> T := CharacterTable(S3);;
% gap> Display(T);
% CT3

%      2  1  1  .
%      3  1  .  1

%       1a 2a 3a
%     2P 1a 1a 3a
%     3P 1a 2a 1a

% X.1     1 -1  1
% X.2     2  . -1
% X.3     1  1  1
% \end{lstlisting}
%\begin{lstlisting}
%julia> S3 = symmetric_group(3);
%
%julia> T = character_table(S3)
%Sym( [ 1 .. 3 ] )
%
%  2  1  1  .
%  3  1  .  1
%            
%    1a 2a 3a
% 2P 1a 1a 3a
% 3P 1a 2a 1a
%            
%X_1  1 -1  1
%X_2  2  . -1
%X_3  1  1  1
%\end{lstlisting}
%As we did before, some extra information was computed:
%\begin{lstlisting}
%julia> orders_class_representatives(T)
%3-element Vector{Int64}:
% 1
% 2
% 3
%
%julia> class_lengths(T)
%3-element Vector{fmpz}:
% 1
% 3
% 2
%
%julia> orders_centralizers(T)
%3-element Vector{fmpz}:
% 6
% 2
% 3
%\end{lstlisting}
%julia> GAP.Globals.SizesConjugacyClasses(T.GAPTable)
%GAP: [ 1, 3, 2 ]
%julia> GAP.Globals.SizesCentralizers(T.GAPTable)
%GAP: [ 6, 2, 3 ]    
% \begin{lstlisting}
% gap> SizesConjugacyClasses(T);
% [ 1, 3, 2 ]
% gap> SizesCentralizers(T);
% [ 6, 2, 3 ]
% gap> OrdersClassRepresentatives(T);
% [ 1, 2, 3 ]
% \end{lstlisting}
\end{example}

% \begin{exercise}
% \label{xca:S4}
%     Compute the character table of $\Sym_4$. 
% \end{exercise}

\begin{example} 
Let us compute the character table of $\Sym_4$. 
We know that $|\Sym_4|=24$ and that 
$\Sym_4$ has five conjugacy classes:
	\begin{center}
		\begin{tabular}{c|ccccc}
			Representative & $\id$ & $(12)$ & $(12)(34)$ & $(123)$ & $(1234)$\tabularnewline
			\hline
			Size & $1$ & $6$ & $3$ & $8$ & $6$
		\end{tabular}
	\end{center}
Thus $\Irr(\Sym_4)=\{\chi_1,\chi_2,\dots,\chi_5\}$. We may 
assume that $\chi_1$ is the trivial character and
that $\chi_2$ is the sign. Since 
$[\Sym_4,\Sym_4]\simeq\Alt_4$, the quotient 
$\Sym_4/[\Sym_4,\Sym_4]$ has order two and hence $\Sym_4$ 
admits exactly two degree-one irreducible representations. Hence
we know two rows of the character table of $\Sym_4$: 
\bigskip 
	\begin{center}
		\begin{tabular}{|c|rrrrr|}
			\hline
			& $\id$ & $(12)$ & $(12)(34)$ & $(123)$ & $(1234)$\tabularnewline
%			& $1$ & $6$ & $3$ & $8$ & $6$\tabularnewline
			\hline
			$\chi_1$ & $1$ & $1$ & $1$ & $1$ & $1$\tabularnewline
			$\chi_2$ & $1$ & $-1$ & $1$ & $1$ & $-1$\tabularnewline
			\hline
		\end{tabular}
	\end{center}
\bigskip 
    
	There exist $n_3,n_4,n_5\in\{2,3,4\}$ such that 
	$24=1+1+n_3^2+n_4^2+n_5^2$. A direct calculation shows that $(n_3,n_4,n_5)=(2,3,3)$ is the only solution with
    $n_3\leq n_4\leq n_5$.

To find the other characters, it is useful to use the action of $\Sym_4$ on the vector space 
	\[
		V=\{(x_1,x_2,x_3,x_4)\in\R^4:x_1+x_2+x_3+x_4=0\},
	\]
given by 
\[
g\cdot (x_1,x_2,x_3,x_4)=(x_{g^{-1}(1)},x_{g^{-1}(2)},x_{g^{-1}(3)},x_{g^{-1}(4)}).
\]
Let \[
		v_1=(1,0,0,-1),
		\quad
		v_2=(0,1,0,-1),
		\quad
		v_3=(0,0,1,-1).
	\]
	Then $\{v_1,v_2,v_3\}$ is a basis of $V$ and 
	\begin{align*}
		&(12)\cdot v_1=v_2,&&
		(12)\cdot v_2=v_1,&&
		(12)\cdot v_3=v_3,\\
		&(1432)\cdot v_1=-v_3,&&
		(1432)\cdot v_2=v_1-v_3,&&
		(1432)\cdot v_3=v_2-v_3.
	\end{align*}
	Since $\Sym_4=\langle (12),(1432)\rangle$, this 
    is enough to know how any element 
    $g\in\Sym_4$ acts on any $v\in V$. 
    This action yields a representation 
    $\rho\colon\Sym_4\to\GL(V)$: 
    \[
		\rho_{(12)}=\begin{pmatrix}
			0 & 1 & 0\\
			1 & 0 & 0\\
			0 & 0 & 1
		\end{pmatrix},\quad
		\rho_{(1432)}=\begin{pmatrix}
			0 & 1 & 0\\
			0 & 0 & 1\\
			-1 & -1 & -1
		\end{pmatrix}.
	\]
    Let $\chi$ be the character of $\rho$.  
	Then 
	$\chi(\id)=3$, $\chi\left( (12) \right)=1$, $\chi\left( (1234) \right)=-1$. How to compute the value of $\chi$ on 3-cycles? Here is the trick: 
    \begin{align*}
		&\chi\left( (234) \right)=\chi\left( (12)(1234) \right)=\trace(\rho_{(12)}\rho_{(1234)})=\trace\begin{pmatrix}
			0 & 0 & 1\\
			0 & 1 & 0\\
			1 & -1 & -1
		\end{pmatrix}
		=0.
	\end{align*}
	Similarly, to compute $\chi$ on products of two transpositions, 
    we note that 
    \begin{align*}
		&\chi\left( (13)(24) \right)=\chi\left( (1234)(1234) \right)=\trace(\rho_{(1234)}^2)=\trace\begin{pmatrix}
			0 & 0 & 1\\
			1 & -1 & -1\\
			-1 & 2 & 0
		\end{pmatrix}
		=-1.
	\end{align*}
     Now is an easy exercise to check that this $\chi$ is irreducible:
      \[
	 	\langle \chi,\chi\rangle=\frac{1}{24}(3^2+6+0+6+3)=1.
	 \]
     Moreover, $\sgn\otimes\chi$ is also an irreducible representation:
     	\[
		\langle \sgn\otimes\chi,\sgn\otimes\chi\rangle=\frac{1}{24}(3^2+(-1)^26+(-1)^23+6)=1.
	\]
     With the trivial representation $\chi_1$, the sign representation $\chi_2$ and these two new characters, namely $\chi_3=\chi$ and $\chi_4=\sgn\otimes\chi$, we are almost done. Only one irreducible character is missing. Let us call this
     character $\chi_5$. This character can be determined  
     using the left regular representation $L$: 
     \begin{align*}
		0 &= \chi_L\left( (12) \right)=1+(-1)+3+3(-1)+2\chi_5\left( (12) \right),\\
		0 &= \chi_L\left( (12)(34) \right)=1+1+3(-1)+3(-1)+2\chi_5\left( (12)(34) \right),\\
		0 &= \chi_L\left( (123) \right)=1+1+0+0+2\chi_5\left( (123) \right),\\
		0 &= \chi_L\left( (1234) \right)=1+(-1)+3(-1)+3+2\chi_5\left( (1234) \right)=0,
	\end{align*}
	
    Now we are ready to compute the character table of $\Sym_4$: 
	% Tenemos así cuatro de los cinco caracteres irreducibles de $G$. Nos falta
	% uno, digamos $\chi_5$.  Para calcular $\chi_5$ usamos el carácter de la
	% representación regular $L$:
	% \begin{align*}
	% 	0 &= \chi_L\left( (12) \right)=1+(-1)+3+3(-1)+2\chi_5\left( (12) \right),\\
	% 	0 &= \chi_L\left( (12)(34) \right)=1+1+3(-1)+3(-1)+2\chi_5\left( (12)(34) \right),\\
	% 	0 &= \chi_L\left( (123) \right)=1+1+0+0+2\chi_5\left( (123) \right),\\
	% 	0 &= \chi_L\left( (1234) \right)=1+(-1)+3(-1)+3+2\chi_5\left( (1234) \right)=0,
	% \end{align*}
	% de donde obtenemos los valores de $\chi_5$. Nos queda así la siguiente tabla:
    \begin{table}[h]
    \caption{The character table of $\Sym_4$.}      
		\begin{tabular}{|c|rrrrr|}
			\hline
            & $1$ & $6$ & $3$ & $8$ & $6$\tabularnewline
			& $\id$ & $(12)$ & $(12)(34)$ & $(123)$ & $(1234)$\tabularnewline
			\hline
			$\chi_1$ & $1$ & $1$ & $1$ & $1$ & $1$\tabularnewline
			$\sgn$ & $1$ & $-1$ & $1$ & $1$ & $-1$\tabularnewline
			$\chi$ & $3$ & $1$ & $-1$ & $0$ & $-1$\tabularnewline
			$\sgn\otimes\chi$ & $3$ & $-1$ & $-1$ & $0$ & $1$\tabularnewline
			$\chi_5$ & $2$ & $0$ & $2$ & $-1$ & $0$\tabularnewline
			\hline
		\end{tabular}
	\end{table}
\end{example}


\begin{exercise}
    Compute the character table of $\Alt_4$. 
\end{exercise}

\begin{exercise}
\label{xca:order5}
    Compute the character table of a non-abelian group of order eight.
\end{exercise}

There are two non-isomorphic non-abelian groups of order eight: the dihedral group $\D_4$ and the quaternion group $Q_8$. One does not need
to use this information to solve Exercise~\ref{xca:order5}.

% \begin{example}
% Let us compute the character table of a 
% non-abelian group of order eight. (There are two non-isomorphic non-abelian groups of order eight: the dihedral group $\D_4$ and the quaternion group $Q_8$. We will not use this information.) 

% Let $G$ be a non-abelian group of order eight. Since $G$
% is a $2$-group, $Z(G)$ is non-trivial. Moreover, 
% since $G$ is non-abelian, $G/Z(G)$ is non-cyclic. Thus
% $|Z(G)|=2$. Since $G/Z(G)$ has four elements, it is
% abelian. Thus $[G,G]\subseteq Z(G)$ and hence
% $[G,G]=Z(G)$. This means that 
% $|G/[G,G]|=4$, so there are 
% exactly four degree-one characters. Since 
% \[
% 8=1+1+1+1+n_5^2+\cdots+n_r^2,
% \]
% we conclude that 
% $r=5$ and $n_5=2$. We now know that 
% $G$ has give conjugacy classes, say 
% with representatives 
% $1,x,a,b,c$, where $[G,G]=Z(G)=\langle x\rangle$.  
% The class equation implies that 
% the conjugacy classes of $a$, $b$ and $c$ have
% two elements. 

% Since $G/[G,G]\simeq C_2\times C_2$, . Por la
% 	proposición~\ref{proposition:Lin(G)}, toda representación de grado uno de
% 	$G$ es de la forma $\chi_j\circ\pi$, donde $\chi_j$ es una representación
% 	de grado uno de $C_2\times C_2$ y $\pi\colon G\to G/[G,G]$ es el morfismo
% 	canónico. Esto nos permite calcular gran parte de los valores de los
% 	caracteres de grado uno:
% 	\begin{center}
% 		\begin{tabular}{|c|rrrrr|}
% 			\hline
% 			& $1$ & $x$ & $a$ & $b$ & $c$\tabularnewline
% 			\hline
% 			$\chi_1$ & $1$ & $1$ & $1$ & $1$ & $1$\tabularnewline
% 			$\chi_2$ & $1$ & $?$ & $-1$ & $1$ & $-1$\tabularnewline
% 			$\chi_3$ & $1$ & $?$ & $1$ & $-1$ & $-1$\tabularnewline
% 			$\chi_4$ & $1$ & $?$ & $-1$ & $-1$ & $1$\tabularnewline
% 			\hline
% 		\end{tabular}
% 	\end{center}
% 	Como $0=\langle \chi_1,\chi_2\rangle=\frac18(1+x+2+2(-1)+2(-1))$, se
% 	concluye que $\chi_2(x)=1$. De la misma forma probamos que $\chi_j(x)=1$
% 	para todo $j\in\{3,4\}$. 
	
% 	Nos falta calcular el valor del caracter de grado dos. Para eso usamos la
% 	representación regular $L$. Al resolver el sistema 
% 	\begin{align*}
% 		0&=\chi_L(x)=1+1+1+1+2\chi_5(x),\\
% 		0&=\chi_L(a)=1+1+-1-1+2\chi_5(a),\\
% 		0&=\chi_L(b)=1-1+1-1+2\chi_5(b),\\
% 		0&=\chi_L(c)=1-1-1+1+2\chi_5(c),
% 	\end{align*}
% 	obtenemos $\chi_5(x)=-2$ y $\chi_5(a)=\chi_5(b)=\chi_5(c)=0$. Luego la
% 	tabla de caracteres de $G$ es 
% 	\begin{center}
% 		\begin{tabular}{|c|rrrrr|}
% 			\hline
% 			& $1$ & $x$ & $a$ & $b$ & $c$\tabularnewline
% 			\hline
% 			$\chi_1$ & $1$ & $1$ & $1$ & $1$ & $1$\tabularnewline
% 			$\chi_2$ & $1$ & $1$ & $-1$ & $1$ & $-1$\tabularnewline
% 			$\chi_3$ & $1$ & $1$ & $1$ & $-1$ & $-1$\tabularnewline
% 			$\chi_4$ & $1$ & $1$ & $-1$ & $-1$ & $1$\tabularnewline
% 			$\chi_5$ & $2$ & $-2$ & $0$ & $0$ & $0$\tabularnewline
% 			\hline
% 		\end{tabular}
% 	\end{center}
%\end{example}


\chapter{}

We now prove Schur's second orthogonality relation. 

\begin{theorem}[Schur]
\index{Schur's second orthogonality relation}
    Let $G$ be a finite group and $g,h\in G$. 
    Then
    \[
    \sum_{\chi\in\Irr(G)}\chi(g)\overline{\chi(h)}
    =\begin{cases}
    |G_G(g)| & \text{if $g$ and $h$ are conjugate},\\
    0 & \text{otherwise}.
    \end{cases}
    \]
\end{theorem}

\begin{proof}
    Let $g_1,\dots,g_r$ be the representative of conjugacy classes of $G$. 
    Assume that $\Irr(G)=\{\chi_1,\dots,\chi_r\}$. For each $k\in\{1,\dots,r\}$ 
    let $c_k=(G:G_C(g_k))$ denote the size of the conjugacy class of $g_k$. Then
    \[
    \langle\chi_i,\chi_j\rangle
    =\frac{1}{|G|}\sum_{g\in G}\chi_i(g)\overline{\chi_j(g)}
    =\frac{1}{|G|}\sum_{k=1}^rc_k\chi_i(g_k)\overline{\chi_j(g_k)}.
    \]
    We write this as $I=\frac{1}{|G|}XDX^*$, where $I$ denotes the identity matrix, 
    $X_{ij}=\chi_i(g_j)$, 
    $X^*=\overline{X}^T$ and 
    \[
    D=\begin{pmatrix}
    c_1\\
    &c_2\\
    &&\ddots\\
    &&&c_k
    \end{pmatrix}.
    \]
    Since, in matrices, $AB=I$ implies $BA=I$, it follows that
    $I=\frac{1}{|G|}X^*XD$. Thus, using that $|G|=c_k|C_G(g_k)|$ 
    holds for all $k$, 
    \[
    |G|D^{-1}=X^*X=\sum_{k=1}^r\overline{\chi_k(g_i)}\chi_k(g_j)
    =\begin{cases}
    |C_G(g_j)| & \text{if $i=j$},\\
    0 & \text{otherwise}.
    \end{cases}\qedhere
    \]
\end{proof}

\begin{theorem}[Solomon]
\index{Solomon's theorem}
    Let $G$ be a finite group and $\Irr(G)=\{\chi_1,\dots,\chi_r\}$. 
    If $g_1,\dots,g_r$ are the representatives of conjugacy classes
    of $G$ and $i\in\{1,\dots,r\}$, then 
    \[
    \sum_{j=1}^r\chi_i(g_j)\in\Z_{\geq0}.
    \]
\end{theorem}

\begin{proof}
    Let $V=\C[G]$ be the vector space with basis $\{e_g:g\in G\}$. 
    The action of $G$ on $G$ by conjugation induces a group homomorphism 
    $\rho\colon G\to\GL(V)$, $g\mapsto\rho_g$, where
    $\rho_g(e_h)=e_{ghg^{-1}}$. The matrix of $\rho_g$ 
    in the basis $\{e_g:g\in G\}$ is
    \[
    (\rho_g)_{ij}=\begin{cases}
        1 & \text{if $g_ig=gg_j$},\\
        0 & \text{otherwise}.
        \end{cases}
    \]
    Then
    \[
    \chi_{\rho}(g)=\trace\rho_g=\sum_{k=1}^{|G|}(\rho_g)_{kk}
    =|\{k:g_kg=gg_k\}|=|C_G(g)|.
    \]
    Write $\chi=\sum_{i=1}^rm_i\chi_i$ for $m_1,\dots,m_r\geq0$. 
    For each $j$ let $c_j=(G:C_G(g_j))$. Then
    \begin{align*}
    m_i=\langle\chi_{\rho},\chi_i\rangle
    &=\frac{1}{|G|}\sum_{g\in G}\chi_{\rho}(g)\overline{\chi_i(g)}\\
    &=\frac{1}{|G|}\sum_{j=1}^r c_j|C_G(g_j)|\overline{\chi_i(g_j)}
    =\sum_{j=1}^r\overline{\chi_i(g_j)}.\qedhere
    \end{align*}
\end{proof}

\topic{Algebraic numbers and characters}

\begin{definition}
    Let $\alpha\in\C$. We say that $\alpha$ is \textbf{algebraic}
    if $f(\alpha)=0$ for some monic polynomial $f\in\Z[X]$. 
\end{definition}

Let $\A$ be the set of algebraic numbers.

\begin{proposition}
    $\Q\cap\A=\Z$. 
\end{proposition}

\begin{proof}
    Let $m/n\in\Q$ with $\gcd(m,n)=1$ and $n>0$. If 
    $f(m/n)=0$ for some 
    \[
    f=X^k+a_{k-1}X^{k-1}+\cdots+a_1X+a_0\in\Z[X]
    \]
    of degree $k\geq1$, then
    \[
    0=n^kf(m/n)=m^k+a_{k-1}m^{k-1}n+\cdots+a_1mn^{k-1}+a_0n^k.
    \]
    This implies that 
    \[
        m^k=-n\left(a_{k-1}m^{k-1}+\cdots+a_1mn^{k-2}+a_0n^{k-1}\right)
    \]
    and hence $n$ divides $m^k$. Thus $n\in\{-1,1\}$ and 
    therefore $m/n\in\Z$.
\end{proof}

\begin{proposition}
    Let $x\in\C$. Then $x\in\A$ if and only if $x$ is an eigenvalue of
    an integer matrix.
\end{proposition}

\begin{proof}
    Let us prove the non-trivial implication. Let 
    \[
    f=X^n+a_{n-1}X^{n-1}+\cdots+a_0\in\Z[X]
    \]
    be such that $f(x)=0$. Then $x$ is an eigenvalue
    of the companion matrix of $f$, that is the matrix
    \[
    C(f)=
    \begin{pmatrix}
    0&0&\cdots &0&-a_{0}\\
    1&0&\cdots &0&-a_{1}\\
    0&1&\cdots &0&-a_{2}\\
    \vdots &\vdots &\ddots &\vdots &\vdots \\
    0&0&\cdots &1&-a_{{n-1}}
    \end{pmatrix}
    \in\Z^{n\times n}.\qedhere 
    \]
\end{proof}

\begin{theorem}
\label{thm:Asubring}
    $\A$ is a subring of $\C$. 
\end{theorem}

\begin{proof}
    Let $\alpha,\beta\in\A$. By the previous proposition, 
    $\alpha$ is an eigenvalue 
    of an integer matrix $A\in\Z^{n\times n}$, say
    $Av=\alpha v$, 
    $\beta$ is an eigenvalue of an integer matrix 
    $B\in\Z^{m\times m}$, say $Bw=\beta w$. Then
    \[
    (A\otimes I_{m\times m}+I_{n\times n}\otimes B)(v+w)
    =(\alpha+\beta)(v+w), 
    \]
    where $I_{k\times k}$ denotes the $(k\times k)$ identity 
    matrix, and
    \[
    (A\otimes B)(v\otimes w)=(\alpha\beta)v\otimes w.
    \]
    This implies that 
    $\alpha+\beta\in\A$ and $\alpha\beta\in\A$, again 
    by the previous proposition. 
\end{proof}

\begin{theorem}
\label{thm:A}
    Let $G$ be a finite group. If $\chi\in\Char(G)$ and
    $g\in G$, then $\chi(g)\in\A$. 
\end{theorem}

\begin{proof}
    Let $\varphi$ be a representation of $G$ such that 
    $\chi_\rho=\chi$. Since $\varphi_g$ is diagonalizable with
    eigenvalues $\lambda_1,\dots,\lambda_k\in\A$ (because
    $G$ is finite and the $\lambda_j$ are roots of one), 
    \[
    \chi(g)=\trace\varphi_g=\sum_{i=1}^k\lambda_i\in\A. \qedhere
    \]
\end{proof}

\begin{theorem}
    Let $G$ be a finite group, $\chi\in\Irr(G)$ and $g\in G$. 
    If $K$ is the conjugacy class of $g$ in $G$, then
    \[
    \frac{\chi(g)}{\chi(1)}|K|\in\A. 
    \]
\end{theorem}

To prove the theorem we need a lemma. 

\begin{lemma}
    Let $x\in\C$. Then $x\in\A$ if and only if 
    there exist $z_1,\dots,z_k\in\C$ not all zero such that 
    $xz_i=\sum_{j=1}^ka_{ij}z_j$ for some $a_{ij}\in\Z$ and 
    all $i\in\{1,\dots,k\}$. 
\end{lemma}

\begin{proof}
    Let us first prove $\implies$. Let $f=X^k+a_{k-1}X^{k-1}+\cdots+a_1X+a_0\in\Z[X]$
    be such that $f(x)=0$. For $i\in\{1,\dots,k\}$ let 
    $z_i=x^{i-1}$. Then 
    $xz_i=x^i=z_{i+1}$ for all $i\in\{1,\dots,k-1\}$. Moreover, 
    $xz_k=x^k=-a_0-a_1x-\cdots-a_{k-1}x^{k-1}$.
    
    We now prove $\impliedby$. Let $A=(a_{ij})\in\Z^{k\times k}$ and 
    $Z$ be the column vector 
    $Z=\begin{pmatrix}z_1\\\vdots\\z_k\end{pmatrix}$. Note that $Z$ is non-zero. 
    Moreover, $AZ=xZ$, as 
    \[
    (AZ)_i=\sum_{j=1}^ka_{ij}z_j=xz_i=(xZ)_i
    \]
    for all $i$. Thus $x$ is an eigenvalue of $A\in\Z^{k\times k}$ and
    hence $x\in\A$. 
\end{proof}

We now prove the theorem. We will use the following notation: if $\chi$ is a character
of a group $G$ 
and $C$ is a conjugacy class of $G$, then 
$\chi(g)=\chi(xgx^{-1})$ for all $x\in G$. We write 
$\chi(C)$ to denote the value $\chi(g)$ for any $g\in C$. 

\begin{proof}[Proof of Theorem \ref{thm:A}]
    Let $\varphi$ be a representation of $G$ with character $\chi$. 
    Let $C_1,\dots,C_r$ be the conjugacy classes of $G$ 
    and for every $i\in\{1,\dots,r\}$ let 
    \[
    T_i=\sum_{x\in C_i}\varphi_x. 
    \]
    
    \begin{claim}
        $T_i=\left(\frac{|C_i|}{\chi(1)}\chi(C_i)\right)\id$. 
    \end{claim}
    
    We proceed in several steps. First we prove that 
    $T_i=\lambda\id$ for some $\lambda\in\C$. 
    We prove that $T_i$ is a morphism of representations:
    \[
    \varphi_gT_i\varphi_g^{-1}=\sum_{x\in C_i}\varphi_g\varphi_x\varphi_g^{-1}
    =\sum_{x\in C_i}\varphi_{gxg^{-1}}=\sum_{y\in C_i}\varphi_y=T_i.
    \]
    Now Schur's lemma implies that $T_i=\lambda\id$ for some
    $\lambda\in\C$. 
    
    We now prove that 
    \[
    \lambda=\frac{|C_i|\chi(C_i)}{\chi(1)}.
    \]
    To prove
    this we compute $\lambda$:
    \[
    \lambda\chi(1)=\trace(\lambda\id)
    =\trace T_i
    =\sum_{x\in C_i}\trace\varphi_x
    =\sum_{x\in C_i}\chi(x)
    =|C_i|\chi(C_i).
    \]
    From this the claim follows. 
    
    Now we claim that 
    \[
    T_iT_j=\sum_{k=1}^r a_{ijk}T_k
    \]
    for some $a_{ijk}\in\Z_{\geq0}$. In fact, 
    \begin{align*}
        T_iT_j &= \sum_{x\in C_i}\sum_{y\in C_j}\varphi_x\varphi_y
        =\sum_{x\in C_i}\sum_{y\in C_j}\varphi_{xy}
        =\sum_{g\in G}a_{ijg}\varphi_g,
    \end{align*}
    where $a_{ijg}$ is the number of elements $g\in G$ 
    that can be written 
    as $g=xy$ for $x\in C_i$ and $y\in C_j$. 
    
    \begin{claim}
        The $a_{ijg}$ depend only on the conjugacy class of $g$.
    \end{claim}
    
    Let $X_g=\{(x,y)\in C_i\times C_j:g=xy\}$. If $h=kgk^{-1}$, the map
    \[
    X_g\to X_h,\quad (x,y)\mapsto (kxk^{-1},kyk^{-1}),
    \]
    is well-defined. It is bijective with inverse
    \[
    X_h\to X_g,\quad
    (a,b)\mapsto (k^{-1}ak,k^{-1}bk).
    \]
    Hence $|X_g|=|X_h|$. 
    
    Now 
    \begin{align*}
        T_iT_j &= 
        =\sum_{g\in G}a_{ijg}\varphi_g
        =\sum_{k=1}^r\sum_{g\in C_k}a_{ijg}\varphi_g
        =\sum_{k=1}^ra_{ijg}\sum_{g\in C_k}\varphi_g
        =\sum_{k=1}^ra_{ijk}T_k.
    \end{align*}
    Therefore 
    \begin{equation}
        \label{eq:omega}
    \left(\frac{|C_i|}{\chi(1)}\chi(C_i)\right)
    \left(\frac{|C_j|}{\chi(1)}\chi(C_j)\right)
    =\sum_{k=1}^r a_{ijk}\left(\frac{|C_k|}{\chi(1)}\chi(C_k)\right).
    \end{equation}
    By the previous lemma, $x=\frac{|C_j|}{\chi(1)}\chi(C_j)\in\A$.
\end{proof}

\topic{Frobenius' theorem}
\label{degree}

\begin{theorem}[Frobenius]
\index{Frobenius' theorem}
\label{thm:Frobenius_chi(1)}
    Let $G$ be a finite group and $\chi\in\Irr(G)$. 
    Then $\chi(1)$ divides~$|G|$. 
\end{theorem}

\begin{proof}
    Let $\varphi$ be an irreducible representation with character $\chi$. 
    Since $\langle\chi,\chi\rangle=1$, 
    \[
    \frac{|G|}{\chi(1)}=\frac{|G|}{\chi(1)}\langle\chi,\chi\rangle
    =\sum_{g\in G}\frac{\chi(g)}{\chi(1)}\overline{\chi(g)}.
    \]
    Let $C_1,\dots,C_r$ be the conjugacy classes of $G$. 
    Then 
    \[
        \frac{|G|}{\chi(1)}
        =\sum_{i=1}^r\sum_{g\in C_i}\frac{\chi(g)}{\chi(1)}\overline{\chi(g)}
        =\sum_{i=1}^r\left(\frac{|C_i|}{\chi(1)}\chi(C_i)\right)\overline{\chi(C_i)}\in\A\cap\Q=\Z,
    \]
    as $\overline{\chi(C_i)}\in\A$. This implies that $\chi(1)$ divides $|G|$. 
\end{proof}

The character table gives information of the structure of the group. For example,
with the previous result one can easily prove that
groups of order $p^2$ (where $p$ is a prime number) are abelian. 

\begin{exercise}
    Let $p$ and $q$ be prime numbers such that $p<q$.
    If $q\not\equiv1\bmod p$, then a group of order $pq$ is abelian. 
\end{exercise}

Another application:

\begin{theorem}
    Let $G$ be a finite simple group. 
    Then $\chi(1)\ne2$ for all $\chi\in\Irr(G)$. 
\end{theorem}

\begin{proof}
    Let $\chi\in\Irr(G)$ be such that $\chi(1)=2$. Let $\rho\colon G\to\GL_2(\C)$
    be an irreducible representation of $G$ with character $\chi$. Since 
    $G$ is simple, $\ker\rho=\{1\}$. Since $\chi(1)=2$, 
    $G$ is non-abelian and hence $[G,G]=G$. Since 
    $G$ has $(G:[G,G])=1$ degree-one characters, it follows that
    $G$ has only one degree-one character, the trivial one. The composition
    \[
    \begin{tikzcd}
    	G & {\GL_2(\C)} & {\C^{\times}}
    	\arrow["{\rho }", hook, from=1-1, to=1-2]
    	\arrow["{\det }", from=1-2, to=1-3]
    \end{tikzcd}
    \]
    is a degree-one representation, which means that $\det\rho_g=1$ for all $g\in G$. 
    By Frobenius' theorem, $|G|$ is even (because 
    $2=\chi(1)$ divides $|G|$). Let $x\in G$ be such that $|x|=2$ (Cauchy's theorem). 
    Then $|\rho_x|=2$, as $\rho$ is injective. Since $\rho_x$ is diagonalizable, 
    there exists $C\in\GL_2(\C)$ such that
    \[
    C\rho_xC^{-1}=\begin{pmatrix}
    \lambda&0\\
    0&\mu
    \end{pmatrix}
    \]
    for some $\lambda,\mu\in\{-1,1\}$. Since $1=\det\rho_x=\lambda\mu$ and
    $\rho$ is non-trivial, $\lambda=\mu=-1$. In particular, $C\rho_xC^{-1}$ is central
    and hence $\rho_x$ is central. Since $\rho$ is injective, $x$ is central 
    and thus $Z(G)\ne\{1\}$, a contradiction. 
\end{proof}
\chapter{}

\begin{theorem}[Schur]
\index{Schur's theorem}
\label{thm:Schur_chi(1)}
    Let $G$ be a finite group and $\chi\in\Irr(G)$. 
    Then $\chi(1)$ divides $(G:Z(G))$. 
\end{theorem}

We need a lemma. 

\begin{lemma}
    Let $G$ and $G_1$ be finite groups. If $\rho$ is an irreducible
    representation of $G$ and $\rho_1$ is an irreducible representation
    of $G_1$, then 
    $\rho\otimes\rho_1$ is an irreducible representatoin of $G\times G_1$. 
\end{lemma}

\begin{proof}
    Write $\chi=\chi_{\rho}$ and $\chi_1=\chi_{\rho_1}$. Since
    $\chi$ is irreducible, $\langle\chi,\chi\rangle=1$. Similarly, 
    $\langle\chi_1,\chi_1\rangle=1$. Now
    $\rho\otimes\rho_1$ is irreducible, as 
    \begin{align*}
    \langle\chi\chi_1,\chi\chi_1\rangle
    &=\frac{1}{|G\times G_1|}\sum_{(g,g_1)\in G\times G_1}(\chi\chi_1)(g,g_1)\overline{(\chi\chi_1)(g,g_1)}\\
    &=\frac{1}{|G||G_1}\sum_{g\in G}\sum_{g_1\in G}\chi(g)\chi_1(g_1)\overline{\chi(g)}\overline{\chi_1(g_1)}\\
    &=\frac{1}{|G||G_1}\sum_{g\in G}\overline{\chi(g)}\sum_{g_1\in G}\chi(g)\chi_1(g_1)\overline{\chi_1(g_1)}\\
    &=\langle\chi,\chi\rangle\langle\chi_1,\chi_1\rangle=1.\qedhere 
    \end{align*}
\end{proof}

\begin{exercise}
    Let $G$ and $G_1$ be finite groups. 
    Prove that irreducible characters of $G\times G_1$ 
    are of the form $\chi\otimes\chi_1$ for  
    $\chi\in\Irr(G)$ and $\chi_1\in\Irr(G_1)$. 
\end{exercise}

We now prove Schur's theorem. The proof goes back to Tate, it uses the 
\emph{tensor power trick}. See
Tao's blog  
\url{https://terrytao.wordpress.com} for other applications of this powerful
trick. 

\begin{proof}[Proof of Theorem \ref{thm:Schur_chi(1)}]
    Let $\rho\colon G\to\GL(V)$ be an irreducible representation 
    with character $\chi$. Let $z\in Z(G)$. Then $\rho_z$ commutes
    with $\rho_g$ for all $g\in G$. By Schur's lemma, 
    $\rho_z(v)=\lambda(z)v$ for all $v\in V$. Note that
    $\lambda\colon Z(G)\to\C^{\times}$, $z\mapsto\lambda(z)$, 
    is a well-defined group homomorphism, as 
    \[
    \lambda(z_1z_2)v=\rho_{z_1z_2}(v)=\rho_{z_1}\rho_{z_2}(v)
    =\lambda(z_2)\rho_{z_1}(v)=\lambda(z_1)\lambda(z_2)v
    \]
    for all $v\in V$ and $z_1,z_2\in Z(G)$. 
    
    Let $n\in\Z_{\geq1}$. Write $G^n=G\times\cdots\times G$ ($n$-times). Let
    \[
    \sigma\colon G^n\to\GL(V^{\otimes n}),\quad
    (g_1,\dots,g_n)\mapsto \rho_{g_1}\otimes\cdots\otimes\rho_{g_n}.
    \]
    The character of $\sigma$ is $\chi^n$. Moreover, by the previous lemma, 
    $\sigma$ is
    irreducible. We compute:
    \begin{align*}   
    \sigma(z_1,\dots,z_n)(v_1\otimes\cdots\otimes v_n)&=z_1v_1\otimes\cdots\otimes z_nv_n\\
    &=\lambda(z_1)\cdots\lambda(z_n)v_1\otimes\cdots\otimes v_n\\
    &=\lambda(z_1\cdots z_n)v_1\otimes\cdots\otimes v_n.
    \end{align*}
    Let 
    \[
    H=\{(z_1,\dots,z_n)\in Z(G)^n:z_1\cdots z_n=1\}\subseteq G^n.
    \]  
    The subgroup $H$ acts trivially on $V^{\otimes n}$, so there exists
    a representation 
    \[
    \tau\colon G^n/H\to\GL(V^{\otimes n}).
    \]
    Since $\sigma$ is irreducible, so is $\tau$. 
    By Frobenius' theorem, $\chi(1)$ divides $|G|$ 
    and $\chi(1)^n$ divides $|G^n/H|=\frac{|G|^n}{|Z(G)|^{n-1}}$. 
    Write 
    $|G|=\chi(1)s$ and $|G|(G:Z(G))^{n-1}=\chi(1)^nr$ for some $r,s\in\Z$. Let $a$ and $b$ be such that 
    $\gcd(a,b)=1$ and 
    $\frac{a}{b}=\frac{(G:Z(G))}{\chi(1)}$. Then
    \[
    s\left(\frac{a}{b}\right)^{n-1}=s\frac{(G:Z(G))^{n-1}}{\chi(1)^{n-1}}
    =\frac{|G|}{\chi(1)}\frac{(G:Z(G))^{n-1}}{\chi(1)^{n-1}}=r\in\Z.
    \]
    Thus $b^{n-1}$ divides $s$ and hence $b=1$ (because $n$ is arbitrary).  
\end{proof}

\topic{Examples of character tables}


\chapter{}

\topic{McKay's conjecture}
\label{McKay}

Let $G$ be a finite group and let $p$ be a prime number dividing
$|G|$. Write $\Syl_p(G)$ to denote the (non-empty) set of Sylow 
$p$-subgroups of $G$. Recall that 
the \emph{normalizer} of $P$ is the subgroup
\[
N_G(P)=\{g\in G:gPg^{-1}=P\}.
\]

McKay made the following conjecture for the prime $p=2$ and simple groups 
and later generalized by Alperin in~\cite{MR0404417} and independently
by Isaacs in~\cite{MR332945}. 

\begin{conjecture}[McKay]
\index{McKay's conjecture}
\label{conjecture:McKay}
Let $p$ be a prime. If  
$G$ is a finite group and $P\in\Syl_p(G)$, then 
\[
|\{\chi\in\Irr(G): p\nmid \chi(1)\}|
=|\{\psi\in\Irr(N_G(P)): p\nmid\psi(1)\}|.
\]
\end{conjecture}

McKay's conjecture is still open and is a crucial problem in representation theory. 
The conjecture was proved for several classes of groups. Isaacs 
proved the conjecture for solvable groups; see~\cite{MR332945,MR3791517}. 
Malle and Sp\"ath prove the conjecture for $p=2$. 

\begin{theorem}[Malle--Sp\"ath]
\index{Malle--Sp\"ath theorem}
If $G$ is finite and $P\in\Syl_2(G)$,
then 
\[
|\{\chi\in\Irr(G): 2\nmid \chi(1)\}|
=|\{\psi\in\Irr(N_G(P)): 2\nmid\psi(1)\}|.
\]
\end{theorem}

The proof appears in~\cite{MR3549625} and uses the classification of 
finite simple groups. It uses a deep result of 
Isaacs, Malle and Navarro~\cite{MR2336079}. 

We cannot prove Malle--Sp\"ath theorem here. However, 
we can use the computer to prove some particular cases
with the following function: 

\begin{lstlisting}
gap> McKay := function(G, p)
> local N, n, m;
> N := Normalizer(G, SylowSubgroup(G, p));
> n := Number(Irr(G), x->Degree(x) mod p <> 0);
> m := Number(Irr(N), x->Degree(x) mod p <> 0);
> return n = m;
> end;
function( G, p ) ... end
\end{lstlisting}

As a concrete example, let us 
verify McKay's conjecture for the Mathieu simple group 
$M_{11}$ of order 7920. 

\begin{lstlisting}
gap> M11 := MathieuGroup(11);;
gap> PrimeDivisors(Order(M11));
[ 2, 3, 5, 11 ]
gap> McKay(M11,2);
true
gap> McKay(M11,3);
true
gap> McKay(M11,5);
true
gap> McKay(M11,11);
true
\end{lstlisting}

The following conjecture refines McKay's conjecture. It was
formulated by Isaacs and Navarro:

\begin{conjecture}[Isaacs--Navarro]
\index{Isaacs--Navarro conjecture}
\label{conjecture:IsaacsNavarro}
Let $p$ be a prime and $k\in\Z$. 
If $G$ is a finite group and $P\in\Syl_p(G)$,
then
\begin{align*}
|\{\chi\in\Irr(G):&p\nmid \chi(1)\text{ and }\chi(1)\equiv\pm k\bmod p\}|\\
&=|\{\psi\in\Irr(N_G(P)):p\nmid\psi(1)\text{ and }\psi(1)\equiv\pm k\bmod p\}|.
\end{align*}
\end{conjecture}

Isaacs--Navarro conjecture is still open. However, 
it is known to be true for solvable groups, 
sporadic simple groups and 
symmetric groups, see~\cite{MR1935849}. 

\begin{lstlisting}
gap> IsaacsNavarro := function(G, k, p)
> local m, n, N;
> N := Normalizer(G, SylowSubgroup(G, p));
> m := Number(Filtered(Irr(G), x->Degree(x)\
> mod p <> 0), x->Degree(x) mod p in [-k,k] mod p);
> n := Number(Filtered(Irr(N), x->Degree(x)\
> mod p <> 0), x->Degree(x) mod p in [-k,k] mod p);
> return n=m;
> end;
function( G, k, p ) ... end
\end{lstlisting}

It is an exercise to verify Isaacs--Navarro conjecture in some
small groups such the Mathieu simple group $M_{11}$. 

\topic{Commutators}
\label{commutators}

Let $G$ be a finite group with conjugacy classes $C_1,\dots,C_s$. For
$i\in\{1,\dots,s\}$ and $\chi\in\Irr(G)$ let  
\[
\omega_{\chi}(C_i)=\frac{|C_i|\chi(C_i)}{\chi(1)}\in\A.
\]
In the proof of Theorem \ref{thm:B}, Equality \eqref{eq:omega}, 
we obtained
that 
\begin{equation}
\label{eq:again_omega}
\omega_\chi(C_i)\omega_\chi(C_j)=\sum_{k=1}^sa_{ijk}\omega_{\chi}(C_k),
\end{equation}
where $a_{ijk}$ is the number of solutions 
of $xy=z$ with $x\in C_i$, $y\in C_j$ and $z\in C_k$. 

\begin{theorem}[Burnside]
    \index{Burnside's theorem}
    Let $G$ be a finite group with conjugacy classes $C_1,\dots,C_s$. 
    Then
    \[
    a_{ijk}
    =\frac{|C_i||C_j|}{|G|}
    \sum_{\chi\in\Irr(G)}\frac{\chi(C_i)\chi(C_j)\overline{\chi(C_k)}}{\chi(1)}.
    \]
\end{theorem}

\begin{proof}
    By \eqref{eq:again_omega}, 
    \[
    \frac{|C_i||C_j|}{\chi(1)}\chi(C_i)\chi(C_j)
    =\sum_{k=1}^sa_{ijk}|C_k|\chi(C_k).
    \]
    Multiply by $\overline{\chi(C_l)}$ and sum over all
    $\chi\in\Irr(G)$ to obtain 
    \begin{align*}
    |C_i||C_j|\sum_{\chi\in\Irr(G)}\frac{\overline{\chi(C_l)}}{\chi(1)}\chi(C_i)\chi(C_j)
    &=\sum_{\chi\in\Irr(G)}\sum_{k=1}^sa_{ijk}|C_k|\chi(C_k)\overline{\chi(C_l)}\\
    &=\sum_{k=1}^sa_{ijk}|C_k|\sum_{\chi\in\Irr(G)}\chi(C_k)\overline{\chi(C_l)}\\
    &=a_{ijl}|G|,
    \end{align*}
    because 
    \[
    \sum_{\chi\in\Irr(G)}\chi(C_k)\overline{\chi(C_l)}=\begin{cases}
        \frac{|G|}{|C_l|} & \text{if $k=l$},\\
        0 & \text{otherwise}.
        \end{cases}\qedhere
    \]
\end{proof}

\begin{theorem}[Burnside]
\index{Burnside's theorem}
    Let $G$ be a finite group and $g,x\in G$. Then
    $g$ and $[x,y]$ are conjugate for some $y\in G$ if and only if
    \[
    \sum_{\chi\in\Irr(G)}\frac{|\chi(x)|^2\chi(g)}{\chi(1)}>0.
    \]
\end{theorem}

\begin{proof}
    Let $C_1,\dots,C_s$ be the conjugacy classes of $G$. Assume that
    $x\in C_i$ and $g\in C_k$ for some $i$ and $k$. Then
    $C_i^{-1}=\{z^{-1}:z\in C_i\}=C_j$ for some $j$. By Burnside's theorem,
    \[
    a_{ijk}
    =\frac{|C_i|^2}{|G|}\sum_{\chi\in\Irr(G)}\frac{|\chi(C_i)|^2\overline{\chi(C_k)}}{\chi(1)}.
    \]
    We first prove $\impliedby$. Since $a_{ijk}>0$, 
    there exist $u\in C_i$ and $v\in C_j$ such that 
    $g=uv$ (since $zgz^{-1}=u_1v_1$ for some $u_1\in C_i$ and
    $v_1\in C_j$, it follows that 
    $g=(z^{-1}u_1z)(z^{-1}v_1z)$, so take $u=z^{-1}u_1z\in C_i$ and 
    $v=z^{-1}v_1z\in C_j$). If $x$ and $u$ are conjugate, say 
    $u=zxz^{-1}$ for some $z$, then $x^{-1}$ and 
    $v$ are conjugate, as 
    \[
    zxz^{-1}=u\implies zx^{-1}z^{-1}=u^{-1}\in C_i^{-1}=C_j.
    \]
    Let $z_2\in G$ be such that $z_2x^{-1}z_2^{-1}=v$. 
    If $y=z^{-1}z_2$, then $g$ and $[x,y]$ are conjugate, as 
    \[
    g=uv=(zxz^{-1})(z_2x^{-1}z_2^{-1})=(zxyx^{-1}y^{-1})yz_2^{-1}
    =z[x,y]z^{-1}.
    \]
    
    We now prove $\implies$. Let $y\in G$ be such that $g$ and
    $[x,y]$ are conjugate, say $g=z[x,y]z^{-1}$ for some $z\in G$. Let
    $v=yxy^{-1}$. Then 
    $g$ and $xv^{-1}=xyx^{-1}y^{-1}=[x,y]$ are conjugate. In particular, 
    since $g\in C_iC_j$, $a_{ijk}>0$. 
\end{proof}    

\begin{exercise}
    Let $G$ be a finite group, $g\in G$ and $\chi\in\Irr(G)$. 
    Prove that 
    \begin{align*}
        &\sum_{h\in G}\chi([g,h])=\frac{|G|}{\chi(1)}|\chi(g)|^2.
    \shortintertext{Prove also that }
        &\chi(g)\chi(h)=\frac{\chi(1)}{|G|}\sum_{z\in G}\chi(zgz^{-1}h)
    \end{align*}
    holds for all $h\in G$. 
\end{exercise}

We now prove a theorem of Frobenius that 
uses character tables to recognize commutators. For that purpose, 
let 
\[
\tau(g)=|\{(x,y)\in G\times G:[x,y]=g\}|.
\]

\begin{theorem}[Frobenius]
    \index{Frobenius' theorem}
    Let $G$ be a finite group. Then
    \[
    \tau(g)=|G|\sum_{\chi\in\Irr(G)}\frac{\chi(g)}{\chi(1)}.
    \]
\end{theorem}

\begin{proof}
    Let $\chi\in\Irr(G)$. Since $\chi$ is irreducible, 
    \[
    1=\langle\chi,\chi\rangle
    =\frac{1}{|G|}\sum_{z\in G}\chi(z)\overline{\chi(z)}
    =\frac{1}{|G|}\sum_{C}|C|\chi(C)\overline{\chi(C)},
    \]
    where the last sum is taken over all conjugacy classes of $G$. 
    Let $g\in G$ and $C$ be the conjugacy class of $g$. The equation
    $xu^{-1}=g$ with $x\in C$ and $u\in C^{-1}$ has 
    \[
        \frac{|C||C^{-1}|}{|G|}\sum_{\chi\in\Irr(G)}\frac{\chi(C)\chi(C^{-1})\chi(g^{-1})}{\chi(1)}
    \]
    solutions. If $(x,u)$ is a solution of $xu^{-1}=g$, then
    there are $|C_G(x)|$ elements $y$ such that $yxy^{-1}=u$. ($yxy^{-1}=u=y_1xy_1^{-1}$ implies that $y_1^{-1}y\in C_G(x)$ which implies $yC_G(x)=y_1C_G(x)$.) Now 
    $[x,y]=x(yx^{-1}y^{-1})=g$ has 
    \[
    |C|\sum_{\chi}\frac{\chi(C)\chi(C^{-1})\chi(g^{-1})}{\chi(1)}
    \]
    solutions, where the sum is taken over all irreducible characters of $G$. 
    Now we sum over all conjugacy classes of $G$:
    \begin{align*}
        \sum_{C}\sum_{\chi}|C|\frac{\chi(C)\chi(C^{-1})\chi(g^{-1})}{\chi(1)}
        &=\sum_{\chi}\frac{\chi(g^{-1})}{\chi(1)}\left(\sum_C|C|\chi(C)\chi(C^{-1})\right)\\
        &=|G|\sum_{\chi}\frac{\chi(g^{-1})}{\chi(1)}.
    \end{align*}
    From this, the formula follows. 
\end{proof}

Application:

\begin{corollary}
    Let $G$ be a finite group and $g\in G$. Then $g$ 
    is a commutator if and only if 
    \[
    \sum_{\chi\in\Irr(G)}\frac{\chi(g)}{\chi(1)}\ne0.
    \]
\end{corollary}

\topic{Ore's conjecture}
\label{Ore}

In 1951 Ore and independently It\^o
proved that every element of any alternating simple group is a commutator. 
Ore also mentioned that ``it is possible that a similar theorem holds for any simple group of finite order, but it seems that at present we do not have the necessary methods to investigate the question". 

\begin{conjecture}[Ore]
\index{Ore's conjecture}
\label{conjecture:Ore}
    Let $G$ be a finite simple non-abelian group. 
    Then every element of $G$ is a commutator. 
\end{conjecture}

Ore's conjecture was proved in 2010:

\begin{theorem}[Liebeck--O'Brien--Shalev--Tiep]
\index{Liebeck--O'Brien--Shalev--Tiep theorem}
    Every element of a non-abelian finite simple group is a commutator.     
\end{theorem}

The proof appears in~\cite{MR2654085}. It needs about 70 pages and
uses the classification of finite simple groups (CFSG) and character theory.
See \cite{MR3289286} for more information on Ore's conjecture and its proof. 

Although the proof of Ore's conjecture is too complicated for 
this course, we can use the computer to 
prove the conjecture in some particular cases:

\begin{proposition}
    Ore's conjecture is true 
    for sporadic simple groups.
\end{proposition}

\begin{proof}
    Let $G$ be a finite simple group. 
    We know that $g\in G$ is a commutator if and only if 
    $\sum_{\chi\in\Irr(G)}\frac{\chi(g)}{\chi(1)}\ne 0$. Let us write
    a computer script to check whether every element in a group 
    is a commutator. Our
    function needs the character table of a group and returns 
    \lstinline{true} if every element of the group is a commutator and
    \lstinline{false} otherwise. 
\begin{lstlisting}
gap> Ore := function(char) 
> local s,f,k;
> for k in [1..NrConjugacyClasses(char)] do
> s := 0;
> for f in Irr(char) do
> s := s+f[k]/Degree(f);  
> od;
> if s<=0 then
> return false;
> fi;
> od;
> return true;
> end;
function( char ) ... end
\end{lstlisting}
Now we check Ore's conjecture for Mathieu simple groups
and for the Monster group: 
\begin{lstlisting}
gap> Ore(CharacterTable("M11"));
true
gap> Ore(CharacterTable("M12"));
true
gap> Ore(CharacterTable("M22"));
true
gap> Ore(CharacterTable("M23"));
true
gap> Ore(CharacterTable("M24"));
true
gap> Ore(CharacterTable("M"));
true
\end{lstlisting}
It is an exercise to check the conjecture for the other finite sporadic 
simple groups $McL$, $Ru$, $Ly$, $Suz$, $He$, $HN$, $Th$, $Fi_{22}$, $Fi_{23}$, $Fi_{24}'$, $B$, $M$ 
\end{proof}

See \cite{MR3821142} for other applications of character theory. 

\topic{Cauchy--Frobenius--Burnside theorem}

\begin{theorem}[Cauchy--Frobenius--Burnside]
\label{thm:CFB}
\index{Cauchy--Frobenius--Burnside theorem}
    Let $G$ be a finite group that acts on a finite set $X$. 
    If $m$ is the number of orbits, then 
    \[
    m=\frac{1}{|G|}\sum_{g\in G}|\Fix(g)|,
    \]
    where $\Fix(g)=\{x\in X:g\cdot x=x\}$. 
\end{theorem}

\begin{proof}
    Let $X=\{x_1,\dots,x_n\}$ and $V$ be the complex vector space with basis $\{x_1,\dots,x_n\}$. 
    Let $\rho\colon G\to\GL_n(\C)$, $g\mapsto\rho_g$, be the representation
    \[
    (\rho_g)_{ij}=\begin{cases}
        1 & \text{if $g\cdot x_j=x_i$},\\
        0 & \text{otherwise}.
        \end{cases}
    \]
    In particular, $(\rho_g)_{ii}=1$ if $x_i\in\Fix(g)$ and 
    $(\rho_g)_{ii}=0$ if $x_i \notin \Fix(g)$. Thus
    \[
    \chi_\rho(g)=\trace\rho_g=\sum_{i=1}^n(\rho_g)_{ii}=|\Fix(g)|.
    \]
    
    Recall that 
    \begin{gather*}
        V^G=\{v\in V:g\cdot v=v\text{ for all $g\in G$}\}
    \shortintertext{and that}    
        \dim V^G=\frac{1}{|G|}\sum_{z\in G}\chi_{\rho}(z)=\langle\chi_\rho,\chi_1\rangle
    \end{gather*}
    where $\chi_1$ is the trivial character of $G$. 
    
    We can assume that, after a possible re-enumeration,
    $x_1,\dots,x_m$ are the representatives of the orbits 
    of $G$ on $X$. For $i\in\{1,\dots,m\}$, let
    $v_i=\sum_{x\in G\cdot x_i}x$.
    
    \begin{claim}
        $\{v_1,\dots,v_m\}$ is a basis of $V^G$. 
    \end{claim}
    
    If $g\in G$, then $g\cdot v_i=\sum_{x\in G\cdot x_i}g\cdot x=
    \sum_{y\in G\cdot x_i}y=v_i$. Hence $\{v_1,\dots,v_m\}\subseteq V^G$. Moreover, 
    $\{v_1,\dots,v_m\}$ is linearly independent because the $v_j$ are
    orthogonal and non-zero:
    \[
    \langle v_i,v_j\rangle=\begin{cases}
        |G\cdot x_i| & \text{if $i=j$},\\
        0 & \text{otherwise}.
        \end{cases}
    \]
    We now prove that $V^G=\langle v_1,\dots,v_m\rangle$. Let $v\in V^G$.
    Then $v=\sum_{x\in X}\lambda_xx$ for some coefficients $\lambda_x\in\C$.
    If $g\in G$, then $g\cdot v=v$. Since 
    \[
    \sum_{x\in X}\lambda_xx=v=g\cdot v
    =\sum_{x\in X}\lambda_x(g\cdot x)
    =\sum_{x\in X}\lambda_{g^{-1}\cdot x}x,
    \]
    it follows that $\lambda_x=\lambda_{g^{-1}\cdot x}$ for all $x\in X$ and 
    $g\in G$. This means that if $y,z\in X$ and $g\in G$ is such that
    $g\cdot y=z$, then $\lambda_y=\lambda_z$. Thus 
    \[
    v=\sum_{x\in X}\lambda_xx=\sum_{i=1}^m\lambda_{x_i}\sum_{y\in G\cdot x_i}y
    =\sum_{i=1}^m \lambda_{x_i}v_i.
    \]
    
    Hence 
    \[
    m=\dim V^G=\langle\chi_\rho,\chi_1\rangle=\frac{1}{|G|}\sum_{z\in G}\chi_\rho(z)
    =\frac{1}{|G|}\sum_{z\in G}|\Fix(z)|.\qedhere 
    \]
\end{proof}

It is possible to give an alternative short proof of the theorem. For example, 
for transitive actions (i.e. $m=1$), we proceed as follows:
\[
\sum_{g\in G}|\Fix(g)|=\sum_{g\in G}\sum_{\substack{x\in X\\g\cdot x=x}}1
=\sum_{x\in X}\sum_{\substack{g\in G\\g\cdot x=x}}1
=\sum_{x\in X}|G_x|=|G_x||X|=|G|.
\]

\begin{exercise}
\label{xca:CFB}
    Use the previous idea to prove Theorem \ref{thm:CFB}. 
\end{exercise}

\index{Orbital}
\index{Rank}
Let $G$ act on a finite set $X$. Then $G$ acts
on $X\times X$ by
\begin{equation}
    \label{eq:orbitals}
    g\cdot (x,y)=(g\cdot x,g\cdot y).
\end{equation}
The orbits of this action are called
the \textbf{orbitals} of $G$ on $X$. The \textbf{rank} 
of $G$ on $X$ is the number of orbitals. 

\begin{proposition}
    Let $G$ be a group that acts on a finite set $X$.
    The rank of $G$ on $X$ is 
    \[
    \frac{1}{|G|}\sum_{g\in G}|\Fix(g)|^2.
    \]
\end{proposition}

\begin{proof}
    The action \eqref{eq:orbitals} has 
    $\Fix(g)\times\Fix(g)$ as fixed points, as 
    \begin{align*}
        g\cdot (x,y)=(x,y)&\Longleftrightarrow
        (g\cdot x,g\cdot y)=(x,y)\\
        &\Longleftrightarrow g\cdot x=x\text{ and }g\cdot y=y\Longleftrightarrow
        (x,y)\in\Fix(g)\times\Fix(g).
    \end{align*}
    Now the claim follows from Cauchy--Frobenius--Burnside theorem. 
\end{proof}

\begin{definition}
    Let $G$ act on a finite set $X$. 
    We say that $G$ is \textbf{2-transitive} on $X$ 
    if given $x,y\in X$ with $x\ne y$ and 
    $x_1,y_1\in X$ with $x_1\ne y_1$ there exists 
    $g\in G$ such that $g\cdot x=x_1$ and $g\cdot y=y_1$. 
\end{definition}

The symmetric group $\Sym_n$ acts 2-transitively on $\{1,\dots,n\}$. 

\begin{proposition}
    If $G$ is 2-transitive on $X$, then the rank of $G$ on $X$ is two. 
\end{proposition}

\begin{proof}
    The set $\Delta=\{(x,x):x\in X\}$ is an orbital. The complement
    $X\times X\setminus\Delta$ is another orbital: if $x,x_1,y,y_1\in X$
    are such that $x\ne y$ 
    and $x_1\ne y_1$, then there exists $g\in G$ such that 
    $g\cdot x=x_1$ and $g\cdot y=y_1$, so $g\cdot (x,y)=(x_1,y_1)$. 
\end{proof}

\section{Lecture: Week 8}

\subsection{The correspondence theorem}

Let $N$ be a normal subgroup of $G$ 
and 
\[
\pi\colon G\to G/N,\quad 
g\mapsto gN,
\]
be the canonical map. 
If $\widetilde{\rho}\colon G/N\to\GL(V)$ 
is a representation of $G/N$ with 
character
$\widetilde{\chi}$, the composition 
$\rho=\widetilde{\rho}\pi\colon G\to \GL(V)$, $\rho(g)=\widetilde{\rho}(gN)$, 
is a representation of $G$. 
Thus
\[
\chi(g)=\trace{\rho_g}=\trace(\widetilde{\rho}_{gN})=\widetilde{\chi}(gN).
\]
In particular, $\chi(1)=\widetilde{\chi}(1)$. The character $\chi$ 
is the \emph{lifting} to $G$ of the character 
$\widetilde{\chi}$ of $G/N$. 

\begin{proposition}
If $\chi\in\Char(G)$, then 
\[
\ker\chi=\{g\in G:\chi(g)=\chi(1)\}
\]
is a normal subgroup of $G$. 
\end{proposition}

\begin{proof}
Let $\rho\colon G\to\GL_n(\C)$ be a representation with character $\chi$. Then 
$\ker\rho\subseteq\ker\chi$, as $\rho_g=\id$ implies 
$\chi(g)=\trace(\rho_g)=n=\chi(1)$. 

We claim that  
$\ker\chi\subseteq\ker\rho$. If $g\in G$ is such that $\chi(g)=\chi(1)$, since 
$\rho_g$ is diagonalizable, there exist eigenvalues $\lambda_1,\dots,\lambda_n\in\C$ such that
\[
n=\chi(1)=\chi(g)=\sum_{i=1}^n\lambda_i.
\]
Since each $\lambda_i$ is a root of one,  
$\lambda_1=\cdots=\lambda_n=1$. Hence $\rho_g=\id$. 
\end{proof}

\index{Kernel!of a character}
\index{Kernel!of a representation}
If $\chi$ is a character, the subgroup $\ker\chi$ 
is the \emph{kernel} of $\chi$. 

\begin{theorem}[Correspondence theorem]
\index{Correspondence theorem!for characters}
\label{thm:correspondence}
Let $N$ be a normal subgroup of a finite group $G$. There exists
a bijective correspondence 
\[
\Char(G/N) \longleftrightarrow \{\chi\in\Char(G): 
N\subseteq\ker\chi\}
\]
that maps irreducible characters to irreducible characters.
\end{theorem}

\begin{proof}
If $\widetilde{\chi}\in\Char(G/N)$, let $\chi$ be the lifting of $\widetilde{\chi}$ to $G$. If $n\in N$, 
then
\[
\chi(n)=\widetilde{\chi}(nN)=\widetilde{\chi}(N)=\chi(1)
\]
and thus $N\subseteq\ker\chi$. 

If $\chi\in\Char(G)$ is such that $N\subseteq\ker\chi$, let $\rho\colon G\to\GL(V)$ be a representation
with character $\chi$. 
Let $\widetilde{\rho}\colon G/N\to\GL(V)$, $gN\mapsto \rho(g)$. We claim that $\widetilde{\rho}$
is well-defined: 
\[
gN=hN\Longleftrightarrow h^{-1}g\in N\Longrightarrow\rho(h^{-1}g)=\id\Longleftrightarrow \rho(h)=\rho(g).
\]
Moreover, $\widetilde{\rho}$ is a representation, as 
\[
\widetilde{\rho}((gN)(hN))=\widetilde{\rho}(ghN)=\rho(gh)=\rho(g)\rho(h)=\widetilde{\rho}(gN)\widetilde{\rho}(hN).
\]
If $\widetilde{\chi}$ is the character of $\widetilde{\rho}$, then 
$\widetilde{\chi}(gN)=\chi(g)$.

We now prove that $\chi$ is irreducible if and only if 
$\widetilde{\chi}$ is irreducible. If $U$ is a subspace of $V$, then 
\begin{align*}
\text{$U$ is $G$-invariant}
%&\Longleftrightarrow g\cdot U\subseteq U\text{ for all $g\in G$}\\
&\Longleftrightarrow \rho(g)(U)\subseteq U\text{ for all $g\in G$}\\
&\Longleftrightarrow \widetilde{\rho}(gN)(U)\subseteq U\text{ for all $g\in G$}.
\shortintertext{Thus}
\chi\text{ is irreducible }&\Longleftrightarrow
\rho\text{ is irreducible }\\
&\Longleftrightarrow\widetilde{\rho}\text{ is irreducible }\Longleftrightarrow
\widetilde{\chi}\text{ is irreducible }\qedhere.
\end{align*}
\end{proof}

\begin{example}
    Let $G=\Sym_4$ and $N=\{\id,(12)(34),(13)(24),(14)(23)\}$. We know that $N$ is normal in $G$ 
    and that $G/N=\langle a,b\rangle\simeq\Sym_3$, where 
    $a=(123)N$ and $b=(12)N$. 
    The character table of $G/N$ is 
    \begin{center}
		\begin{tabular}{|c|rrr|}
			\hline
			%& $1$ & $3$ & $2$\tabularnewline
			& $N$ & $(12)N$ & $(123)N$ \tabularnewline
			\hline 
			$\widetilde{\chi}_{1}$ & $1$ & $1$ & $1$\tabularnewline
			$\widetilde{\chi}_{2}$ & $1$ & $-1$ & $1$ \tabularnewline
			$\widetilde{\chi}_{3}$ & $2$ & $0$ & $-1$ \tabularnewline
			\hline
		\end{tabular}
	\end{center}
    For each $i\in\{1,2,3\}$ we compute the lifting $\chi_i$ to $G$ of the character  
    $\widetilde{\chi}_i$ of $G/N$. 
    Since $(12)(34)\in N$ and $(13)(1234)=(12)(34)\in N$, 
    \begin{align*}
        \chi( (12)(34) )=\widetilde{\chi}(N),\quad
        \chi( (1234) )=\widetilde{\chi}((13)N)=\widetilde{\chi}((12)N).
    \end{align*}
    Since the characters $\widetilde{\chi_i}$ are irreducibles, 
    the liftings $\chi_i$ are also irreducibles. With this process
    we obtain the following irreducible characters of $G$:
    	\begin{center}
		\begin{tabular}{|c|rrrrr|}
			\hline
			& $1$ & $(12)$ & $(123)$ & $(12)(34)$ & $(1234)$ \tabularnewline
			\hline 
			$\chi_{1}$ & $1$ & $1$ & $1$ & 1 & 1\tabularnewline
			$\chi_{2}$ & $1$ & $-1$ & $1$ & 1 & -1 \tabularnewline
			$\chi_{3}$ & $2$ & $0$ & $-1$ & 2 & 0\tabularnewline
			\hline
		\end{tabular}
	\end{center}
\end{example}

Theorem~\ref{thm:correspondence} has some unexpected applications. For example, the following exercise is elementary but tricky. A simpler solution uses the second orthogonality relation and Theorem~\ref{thm:correspondence}.

\begin{exercise}
\label{xca:centralizer}
    Let $G$ be a finite group, $g\in G$ and $N$ be a normal subgroup of $G$. 
    Prove that $|C_{G/N}(gN)|\leq|C_G(g)|$. 
\end{exercise}



The character table of a group can be used to find the lattice 
of normal subgroups. In particular, the character table detects simple groups. 

\begin{lemma}
    Let $G$ be a finite group and 
    let $g,h\in G$. Then $g$ and $h$ 
    are conjugate if and only if 
    $\chi(g)=\chi(h)$ for all
    $\chi\in\Char(G)$. 
\end{lemma}

\begin{proof}
    If $g$ and $h$ are conjugate, then $\chi(g)=\chi(h)$, as characters are class functions
    of $G$.
    Conversely, if $\chi(g)=\chi(h)$ for all $\chi\in\Char(G)$, then 
    $f(g)=f(h)$ for all class function $f$ of $G$, 
    as characters $G$ generate the space of class functions of $G$. In particular, 
    $\delta(g)=\delta(h)$, where
    \[
    \delta(x)=\begin{cases}
    1 & \text{if $x$ and $g$ are conjugate},\\
    0 & \text{otherwise}.
    \end{cases}
    \]
    This implies that $g$ and $h$ are conjugate.
\end{proof}

As a consequence, we get that 
\begin{equation}
\label{eq:kernels}
\bigcap_{\chi\in\Irr(G)}\ker\chi=\{1\}.
\end{equation}
Indeed, if $g\in\ker\chi$ for all $\chi\in\Irr(G)$, then $g=1$ since 
the lemma implies that $g$ and $1$ are conjugate
because 
$\chi(g)=\chi(1)$ for all $\chi\in\Irr(G)$.

\begin{proposition}
\label{pro:normal}
    Let $G$ be a finite group. 
    If $N$ is a normal subgroup of $G$, 
    then there exist characters
    $\chi_1,\dots,\chi_k\in\Irr(G)$ 
    such that
    \[
    N=\bigcap_{i=1}^k\ker\chi_i.
    \]
\end{proposition}

\begin{proof}
    Apply the previous remark to the group $G/N$ to obtain that 
    \[
    \bigcap_{\widetilde{\chi}\in\Irr(G/N)}\ker\widetilde{\chi}=\{N\}.
    \]
    Assume that $\Irr(G/N)=\{\widetilde{\chi}_1,\dots,\widetilde{\chi}_k\}$. 
    We lift the irreducible characters of $G/N$ to $G$ 
    to obtain (some) irreducible characters $\chi_1,\dots,\chi_k$ 
    of $G$ such that 
    \[
    N\subseteq\ker\chi_1\cap\cdots\cap\ker\chi_k.
    \]
    If $g\in\ker\chi_i$ for all $i\in\{1,\dots,k\}$, then 
    \[
    \widetilde{\chi}_i(N)=\chi_i(1)=\chi_i(g)=\widetilde{\chi}_i(gN)
    \]
    for all $i\in\{1,\dots,k\}$. This implies that
    \[
    gN\in\bigcap_{i=1}^k\ker\widetilde{\chi}_i=\{N\},
    \]
    that is $g\in N$. 
\end{proof}

\index{Group!simple}
Recall that a non-trivial group is \emph{simple} if it contains no non-trivial normal 
proper subgroups. Examples of simple groups are cyclic groups of prime order
and the alternating groups $\Alt_n$ for $n\geq5$. 
As a corollary of Proposition \ref{pro:normal}, 
we can use the character table to detect simple groups.

\begin{proposition}
    Let $G$ be a finite group. Then $G$ is not simple if and only if 
    there exists a non-trivial irreducible character $\chi$ such that
    $\chi(g)=\chi(1)$ 
    for some $g\in G\setminus\{1\}$. 
\end{proposition}

\begin{proof}
    If $G$ is not simple, there exists a normal subgroup $N$ of $G$ such that
    $N\ne G$ and $N\ne\{1\}$. 
    By Proposition \ref{pro:normal}, there exist characters 
    $\chi_1,\dots,\chi_k\in\Irr(G)$
    such that 
    $N=\ker\chi_1\cap\cdots\cap\ker\chi_k$.
    In particular, there exists a non-trivial character
    $\chi_i$ such that $\ker\chi_i\ne\{1\}$. Thus 
    there exists $g\in G\setminus\{1\}$ such that
    $\chi_i(g)=\chi_i(1)$. 
    
    Assume now that there exists a non-trivial irreducible character $\chi$ 
    such that $\chi(g)=\chi(1)$ for some $g\in G\setminus\{1\}$. In particular, $g\in\ker\chi$ 
    and hence $\ker\chi\ne\{1\}$. Since $\chi$ is non-trivial, $\ker\chi\ne G$. 
    Thus $\ker\chi$ is a proper non-trivial normal subgroup of $G$.
\end{proof}

\begin{example}
\index{Mathieu group $M_9$}
    If there exists a group $G$ with
    a character table 
    of the form
    \begin{center}
		\begin{tabular}{|c|rrrrrr|}
			\hline
			$\chi_{1}$ & 1 & 1 & 1 & 1 & 1 & 1\tabularnewline
			$\chi_{2}$ & 1 & 1 & 1 & -1 & 1 & -1 \tabularnewline
			$\chi_{3}$ & 1 & 1 & 1 & 1 & -1 & -1\tabularnewline
		    $\chi_{4}$ & 1 & 1 & 1 & -1 & -1 & 1\tabularnewline
			$\chi_{5}$ & 2 & -2 & 2 & 0 & 0 & 0\tabularnewline
			$\chi_{6}$ & 8 & 0 & -1 & 0 & 0 & 0\tabularnewline
			\hline
		\end{tabular}
	\end{center}
	then $G$ cannot be simple. Note that such a group $G$ would have order $\sum_{i=1}^6\chi_i(1)^2=72$. 
	Mathieu's group $M_{9}$ has precisely this character table! 
\end{example}

\begin{example}
    Let $\alpha=\frac{1}{2}(-1+\sqrt{7}i)$. 
    If there exists a group $G$ with a character table
    of the form
    \begin{center}
		\begin{tabular}{|c|rrrrrr|}
			\hline
			$\chi_{1}$ & 1 & 1 & 1 & 1 & 1 & 1\tabularnewline
			$\chi_{2}$ & 7 & -1 & -1 & 1 & 0 & 0 \tabularnewline
			$\chi_{3}$ & 8 & 0 & 0 & -1 & 1 & 1\tabularnewline
		    $\chi_{4}$ & 3 & -1 & 1 & 0 & $\alpha$ & $\overline{\alpha}$ \tabularnewline
			$\chi_{5}$ & 3 & -1 & 1 & 0 & $\overline{\alpha}$ & $\alpha$\tabularnewline
			$\chi_{6}$ & 6 & 2 & 0 & 0 & 0 & 0\tabularnewline
			\hline
		\end{tabular}
	\end{center}    
	then $G$ is simple. Note that such a group $G$ would have order 
	$\sum_{i=1}^6\chi_i(1)^2=168$. 
	The group  
	\[
	\PSL_2(7)=\SL_2(7)/Z(\SL_2(7))
	\]
	is a simple group that has precisely this character table!  
\end{example}

\subsection{Frobenius' groups}
\label{Frobenius}

If $p$ is a prime number, then
the units $(\Z/p)^{\times}$ 
of $\Z/p$ form a multiplicative group. Moreover, 
$(\Z/p)^{\times}$ 
is cyclic of order $p-1$. 

Let 
\[
G=\left\{\begin{pmatrix}
x & y\\
0 & 1
\end{pmatrix}
:x\in(\Z/p)^\times,\,y\in\Z/p\right\}.
\]
Then $G$ is a group with the usual matrix multiplication
and $|G|=p(p-1)$. 
Let $p$ and $q$ be prime numbers such that $q$ divides $p-1$, 
$z\in\Z$ be an element of multiplicative order $q$ modulo $p$ 
and 
\[
a=\begin{pmatrix}
1&1\\
0&1
\end{pmatrix},
\quad
b=\begin{pmatrix}
z&1\\
0&1
\end{pmatrix},
\quad
H=\langle a,b\rangle.
\]
A direct calculation shows that 
\begin{equation}
\label{eq:pq}
a^p=b^q=\begin{pmatrix}
1&0\\
0&1
\end{pmatrix},
\quad
bab^{-1}=\begin{pmatrix}
1&z\\
0&1
\end{pmatrix}
=a^z.
\end{equation}
Every element of $H$ is of the form $a^ib^j$ for $i\in\{0,\dots,p-1\}$ and  $j\in\{0,\dots,q-1\}$. 
Thus $|H|=pq$. Using~\eqref{eq:pq} we can compute 
the multiplication table of $G$. 

\begin{exercise}
    Let $p$ and $q$ be prime numbers such that $q$ divides $p-1$. Let
    $u,v\in\Z$ be elements of order $q$ modulo $p$. 
    Prove that 
    \[
    \langle a,b:a^p=b^q=1,bab^{-1}=a^u\rangle
    \simeq \langle a,b:a^p=b^q=1,bab^{-1}=a^v\rangle.
    \]
\end{exercise}

The group   
\[
\langle a,b:a^p=b^q=1,bab^{-1}=a^u\rangle,
\]
where $u\in\Z$ has order $q$ modulo $p$, 
is a particular case of a  
\emph{Frobenius group}. 

\begin{exercise}
\label{xca:Frobenius_pq}
    Let $p$ and $q$ be prime numbers such that $p>q$. Let  
    $G$ be a group of order $pq$. Then either $G$ is abelian or
    $q$ divides $p-1$ and 
    \[
    G\simeq \langle a,b:a^p=b^q=1,bab^{-1}=a^u\rangle
    \]
    for some $u\in\Z$ of order
    $q$ modulo $p$. 
\end{exercise}


Using Exercise~\ref{xca:Frobenius_pq}, we can prove, for example, that every group of order $15$ is abelian. 
% We can also show that, up to isomorphism, $\Z/20$ and $F_{5,4}$ are the only groups of order 20.

\begin{definition}
  \index{Frobenius!complement}
  \index{Frobenius!kernel}
  \index{Frobenius!group}
  We say that a finite group $G$ is a 
  \emph{Frobenius group} if $G$ 
  has a non-trivial proper subgroup $H$ such that $H\cap
  xHx^{-1}=\{1\}$ for all $x\in G\setminus H$. In this case, the subgroup 
  $H$ is called a \emph{Frobenius complement}.
\end{definition}

\index{Malnormal subgroup}
A subgroup $H$ such that $gHg^{-1}\cap H=\{1\}$ for all 
$g\not\in H$ is called a \emph{malnormal} subgroup. 
Note that if $H$ is malnormal, then $N_G(H)=H$. 

\begin{exercise}
\label{xca:malnormal}
    Let $G$ be a group and $H$ be a subgroup of $G$. Prove that the following statements are equivalent: 
    \begin{enumerate}
        \item $H$ is malnormal. 
        \item The action of $H$ 
            on $G/H\setminus\{H\}$ by left multiplication is free. 
        \item Any $g\in G\setminus\{1\}$ has zero
            or one fixed point on $G/H$. 
    \end{enumerate}
\end{exercise}

For any group $G$, the subgroups $\{1\}$ and $G$ are malnormal in $G$. Moreover, they are the only subgroups of $G$ that are both normal and malnormal

\begin{exercise}
    Let $G$ be a group. Prove the following statements:
    \begin{enumerate}

        \item If $H$ is malnormal in $G$, then
        $gHg^{-1}$ is malnormal in $G$ for all $g\in G$. 
        \item If $H$ is malnormal in $G$ and 
        $K$ is malnormal in $H$, then $K$ is malnormal
        in $G$. 
        \item The intersection of malnormal
        subgroups is malnormal.
        \item If $H$ is malnormal in $G$ and 
        $S$ is a subgroup of $G$, then 
        $H\cap S$ is malnormal in $S$. 
        
    \end{enumerate}
\end{exercise}

\begin{example}
    Let 
    \[
    G=\left\{\begin{pmatrix}a&b\\0&1\end{pmatrix}:a\in\R^{\times},b\in\R\right\}\quad\text{and}\quad 
    H=\left\{\begin{pmatrix}a&0\\0&1\end{pmatrix}:a\in\R^{\times}\right\}\subseteq G. 
    \]
    Let $g=\begin{pmatrix}x&y\\0&1\end{pmatrix}\in G\setminus H$. Then $y\ne 0$. Since  
    \[
    g\begin{pmatrix}a&0\\0&1\end{pmatrix}g^{-1}
    =\begin{pmatrix}a&-ay+y\\0&1\end{pmatrix},
    \]
    it follows that the subgroup $H$ 
    is malnormal in $G$. 
\end{example}

\begin{exercise}
\label{xca:malnormal_center}
    Let $G$ be a group and $H$ be a non-trivial
    subgroup of $G$. Prove that if $Z(G)\ne\{1\}$, then
    $H$ is not malnormal in $G$. 
\end{exercise}

\begin{bonus}
\label{xca:malnormal_no2torsion}
    Let $G$ be a group with no 2-torsion 
    that contains a normal infinite cyclic group. Prove 
    that $G$ cannot contain a non-trivial proper malnormal subgroup. 
\end{bonus}


\begin{example}
    Let $G$ be a finite group 
    and $P\in\Syl_p(G)$ be such that $|P|=p$ and $N_G(P)=P$. Then $G$ is a Frobenius group
    with complement $P$. 
\end{example}

The previous example shows that 
$\Alt_4$ is a Frobenius group
with complement $\langle(123)\rangle$. Another situation
where the example applies is the dihedral
group 
\[
\D_{2n+1}=\langle r,s:r^{2n+1}=s^2=1,srs=r^{-1}\rangle
\]
of order $2(2n+1)$. It follows that
$\D_{2n+1}$ is a Frobenius
group with complement $\langle s\rangle$. 

\begin{theorem}[Frobenius]
  \label{thm:Frobenius}
  \index{Frobenius!theorem}
  Let $G$ be a Frobenius group with complement $H$. Then
  \[
	N=\left( G\setminus\bigcup_{x\in G}xHx^{-1}\right)\cup\{1\}
  \]
  is a normal subgroup of $G$.
\end{theorem}

\begin{proof}
    Let $1_H$ and $1_G$ be the trivial characters of $H$ and $G$, respectively.  
  For each $\chi\in\Irr(H)$, $\chi\ne1_H$, let 
  $\alpha=\chi-\chi(1)1_H\in\cf(H)$, where $1_H$ denotes the trivial character of $H$. 

  We claim that $\Res_H^G\Ind_H^G\alpha=\alpha$.
  First, $\Ind_H^G\alpha(1)=\alpha(1)=0$. If $h\in H\setminus\{1\}$, then 
  \[
    \Ind_H^G\alpha(h)=\frac{1}{|H|}\sum_{\substack{x\in G\\x^{-1}hx\in H}}\alpha(x^{-1}hx)
    =\frac{1}{|H|}\sum_{x\in H}\alpha(h)=\alpha(h),
  \]
  since, if $x\not\in H$, then $x^{-1}hx\in H$ implies that 
  $h\in H\cap xHx^{-1}=\{1\}$.

  By Frobenius' reciprocity and the definition of $\alpha$, 
  \begin{equation}
    \label{eq:<a,a>=1+chi2}
    \langle\Ind_H^G\alpha,\Ind_H^G\alpha\rangle
    =\langle\alpha,\Res_H^G\Ind_H^G\alpha\rangle=\langle\alpha,\alpha\rangle
    =1+\chi(1)^2.
  \end{equation}
  Again, by Frobenius' reciprocity, 
  \[
  \langle\Ind_H^G\alpha,1_G\rangle
  =\langle\alpha,\Res_H^G1_G\rangle
  =\langle\alpha,1_H\rangle
  =\langle\chi-\chi(1)1_H,1_H\rangle
  =-\chi(1),
  \]
  where $1_G$ is the trivial character of $G$. If we write 
  \[
  \Ind_H^G\alpha=\sum_{\eta\in\Irr(G)}\langle\Ind_H^G\alpha,\eta\rangle\eta
  =\langle\Ind_H^G\alpha,1_G\rangle1_G+\underbrace{\sum_{\substack{1_G\ne\eta\\\eta\in\Irr(G)}}\langle\Ind_H^G\alpha,\eta\rangle\eta}_{\phi},
  \]
  then $\Ind_H^G\alpha=-\chi(1)1_G+\phi$, where $\phi$ is a linear combination of non-trivial 
  irreducible characters of $G$. We compute 
  \[
  1+\chi(1)^2=\langle\Ind_H^G\alpha,\Ind_H^G\alpha\rangle
  =\langle\phi-\chi(1)1_G,\phi-\chi(1)1_G\rangle
  =\langle\phi,\phi\rangle+\chi(1)^2
  \]
  and hence $\langle\phi,\phi\rangle=1$. 
  
  \begin{claim}
  If $\eta\in\Irr(G)$ is such that $\eta\ne 1_G$, then 
  $\langle\Ind_H^G\alpha,\eta\rangle\in\Z$. 
  \end{claim}
  
  By Frobenius' reciprocity, $\langle\Ind_H^G\alpha,\eta\rangle=\langle\alpha,\Res_H^G\eta\rangle$. 
  If we decompose $\Res_H^G\eta$ into irreducibles of $H$, say 
  \[
  \Res_H^G\eta=m_11_H+m_2\chi+m_3\theta_3+\cdots+m_t\theta_t
  \]
  for some $m_1,m_2,\dots,m_t\geq0$, 
  then, since 
  \begin{align*}
  \langle\alpha,1_H\rangle=\langle\chi-\chi(1)1_H,1_H\rangle=-\chi(1),
  &&\langle\alpha,\chi\rangle=\langle\chi-\chi(1)1_H,\chi\rangle=1,
  \end{align*}
  and 
  \[
  \langle\alpha,\theta_j\rangle=\langle\chi-\chi(1)1_H,\theta_j\rangle=0
  \]
  for all $j\in\{3,\dots,t\}$, we conclude that 
  \[
  \langle\Ind_H^G\alpha,\eta\rangle=-m_1\chi(1)+m_2\in\Z.
  \]
  
  \begin{claim}
  $\phi\in\Irr(G)$.
  \end{claim}
  
  Since $\langle\Ind_H^G\alpha,\eta\rangle\in\Z$ for all $\eta\in\Irr(G)$ such that 
  $\eta\ne 1_G$ and 
  \[
  1=\langle\phi,\phi\rangle
  =\sum_{\substack{\eta,\theta\in\Irr(G)\\\eta,\theta\ne1_G}}\langle\Ind_H^G\alpha,\eta\rangle\langle\Ind_H^G\alpha,\theta\rangle\langle\eta,\theta\rangle
  =\sum_{\substack{\eta\ne 1_G\\\eta\in\Irr(G)}}\langle\Ind_H^G\alpha,\eta\rangle^2,
  \]
  there is a unique $\eta\in\Irr(G)$ such that 
  $\langle\Ind_H^G\alpha,\eta\rangle^2=1$ and all the other products are zero, 
  that is 
  $\phi=\pm\eta$ for some $\eta\in\Irr(G)$. Since 
  \[
  \chi-\chi(1)1_H=\alpha=\Res_H^G\Ind_H^G\alpha=\Res_H^G(\phi-\chi(1)1_G)=\Res_H^G\phi-\chi(1)1_H,
  \]
  it follows that $\phi(1)=\Res_H^G\phi(1)=\chi(1)\in\Z_{\geq1}$. Thus $\phi\in\Irr(G)$. 

  \medskip
  We have proved that if $\chi\in\Irr(H)$ is such that $\chi\ne 1_H$, then 
  there exists $\phi_\chi\in\Irr(G)$ such that $\Res_H^G(\phi_\chi)=\chi$. 
  
  \medskip
  We prove that $N$ is equal to 
  \[
	M=\bigcap_{\substack{\chi\in\Irr(H)\\\chi\ne1_H}}\ker\phi_{\chi}.
  \]

  We first prove that $N\subseteq M$. 
  Let $n\in N\setminus\{1\}$ and $\chi\in\Irr(H)\setminus\{1_H\}$. Since $n$ 
  does not belong to a conjugate of 
  $H$, 
  \[
	\Ind_H^G\alpha(n)=\frac{1}{|H|}\sum_{\substack{x\in G\\x^{-1}nx\in H}}\alpha(x^{-1}nx)=0, 
  \]
  as $n\in N$ implies that the set $\{x\in G:x^{-1}nx\in H\}$ is empty. Since 
  \[
  0=\Ind_H^G\alpha(n)
  =\phi_{\chi}(n)-\chi(1)=\phi_{\chi}(n)-\phi_{\chi}(1),
  \]
  we conclude that $n\in\ker\phi_{\chi}$. 
  
  We now prove that $M\subseteq N$. 
  Let $h\in M\cap H$ and $\chi\in\Irr(H)\setminus\{1_H\}$. Then 
  \[
    \phi_{\chi}(h)-\chi(1)=\Ind_H^G\alpha(h)=\alpha(h)=\chi(h)-\chi(1),
  \]
  and $h\in\ker\chi$, as
  \[
    \chi(h)=\phi_{\chi}(h)=\phi_{\chi}(1)=\chi(1).
  \]
  Therefore 
  \[
  h\in\bigcap_{\chi\in\Irr(H)}\ker\chi=\{1\}.
  \]
  By~\eqref{eq:kernels}, the kernels
  of irreducible characters have trivial intersection. 
  We now prove that $M\cap
  xHx^{-1}=\{1\}$ for all $x\in G$. Let $x\in G$ and $m\in M\cap xHx^{-1}$. Since 
  $m=xhx^{-1}$ for some $h\in H$, $x^{-1}mx\in H\cap M=\{1\}$.  This implies that 
  $m=1$.
\end{proof}

There is no proof of Frobenius’ theorem that is  independent of character theory. Purely group-theoretic proofs exist in cases where the Frobenius complement has even order or is solvable; see
\cite[Remark 16.2]{MR1645304}. The Feit--Thompson theorem (which relies heavily on character theory and is significantly more difficult than Frobenius’ theorem) implies that these two cases cover all possibilities. 

In 2013, Terence Tao discovered an \href{https://terrytao.wordpress.com/2013/05/24/a-fourier-analytic-proof-of-frobeniuss-theorem/}{alternative 
Fourier-analytic
proof} of Frobenius’ theorem, though it resembles the original character-theoretic approach.

\begin{optional}
 
\begin{definition}
  \index{Frobenius!kernel}
  Let $G$ be a Frobenius group. The normal subgroup 
  $N$ of Frobenius' theorem is called the \emph{Frobenius kernel}. 
\end{definition}

\begin{corollary}
  Let $G$ be a Frobenius group with complement $H$. 
  Then there exists a normal subgroup $N$ of $G$ 
  such that 
  $G=HN$ and $H\cap N=\{1\}$.
\end{corollary}

\begin{proof}
  Frobenius' theorem yields the subgroup $N$. Since 
  $H\cap gHg^{-1}=\{1\}$ for all $g\in G\setminus H$, 
  it follows that 
  $N_G(H)=H$. It follows that $H$
  has $(G:H)$ conjugates. 
  Let 
  \[
  N=\left( G\setminus\bigcup_{x\in G}xHx^{-1}\right)\cup\{1\}.
  \]
  Then  
  $|N|=|G|-(G:H)(|H|-1)=(G:H)$.
  Since, moreover, $N\cap H=\{1\}$, we conclude that
  \[
  |HN|=|N||H|/|H\cap N|=|N||H|=|G|.
  \]
  Therefore $G=NH$.
\end{proof}

\end{optional}
\section{Lecture: Week 9}
 
\begin{definition}
  \index{Frobenius!kernel}
  Let $G$ be a Frobenius group. The normal subgroup 
  $N$ of Frobenius' theorem is called the \emph{Frobenius kernel}. 
\end{definition}

\begin{proposition}
\label{pro:Frobenius_groups}
  Let $G$ be a Frobenius group with complement $H$. 
  Then there exists a normal subgroup $N$ of $G$ 
  such that 
  $G=HN$ and $H\cap N=\{1\}$.
\end{proposition}

\begin{proof}
  Frobenius' theorem yields the subgroup $N$. Since 
  $H\cap gHg^{-1}=\{1\}$ for all $g\in G\setminus H$, 
  it follows that 
  $N_G(H)=H$. It follows that $H$
  has $(G:H)$ conjugates. 
  Let 
  \[
  N=\left( G\setminus\bigcup_{x\in G}xHx^{-1}\right)\cup\{1\}.
  \]
  Then  
  $|N|=|G|-(G:H)(|H|-1)=(G:H)$.
  Since, moreover, $N\cap H=\{1\}$, we conclude that
  \[
  |HN|=|N||H|/|H\cap N|=|N||H|=|G|.
  \]
  Therefore $G=NH$.
\end{proof}

\begin{optional}
In his doctoral thesis Thompson proved the following result, conjectured
by Frobenius. 

\begin{theorem}[Thompson]
\index{Thompson theorem}
    Let $G$ be a Frobenius group. If $N$ is the Frobenius kernel, then $N$ 
    is nilpotent.
\end{theorem}

See~\cite[Theorem 6.24]{MR2426855} for the proof.
\end{optional}

\subsection{The Cameron--Cohen theorem (again)}

In this section, we use Frobenius’ theorem to strengthen the Cameron--Cohen theorem on derangements (Theorem~\ref{thm:CameronCohen}). To do so, we first require an alternative version of Frobenius’ theorem.

\begin{corollary}[Frobenius]
    \label{cor:Frobenius_combinatorio}
    \index{Frobenius!theorem}
    Let $G$ be a group acting transitively on a finite set $X$. 
    Assume that each $g\in G\setminus\{1\}$ fixes 
    at most one element of 
     $X$. The set $N$ formed by the identity and the derangements 
     of $G$ is a normal subgroup of $G$.
\end{corollary}

\begin{proof}
  Let $x\in X$ and $H=G_x$. We claim that 
  if $g\in G\setminus H$, then $H\cap
  gHg^{-1}=\{1\}$. If $h\in H\cap gHg^{-1}$, then
  $h\cdot x=x$ and $(g^{-1}hg)\cdot
  x=x$. Since $g\cdot x\ne x$, $h$ fixes two elements of
  $X$. Thus 
  $h=1$, as every non-trivial element fixes at most one element of $X$. 

  By Theorem~\ref{thm:Frobenius}, 
  \[
    N=\left(G\setminus\bigcup_{g\in G}gHg^{-1}\right)\cup\{1\}
  \]
  is a subgroup of $G$. Let us compute the elements of $N$. If 
  $h\in\bigcup_{g\in G}gHg^{-1}$, then there exists  $g\in G$ such that $g^{-1}hg\in H$,
  that is $(g^{-1}hg)\cdot x=x$; equivalently, 
  $h\in G_{g\cdot x}$. Therefore, the 
  non-identity elements of $N$ are the elements of $G$
  moving every element of $X$.
\end{proof}

\begin{example}
  Let $F$ be a finite field and $G$ be the group of maps 
  $f\colon F\to F$ of the form 
  $f(x)=ax+b$, $a,b\in F$ with $a\ne0$. The group $G$ acts on 
  $F$ and every 
  $f\ne\id$ fixes at most one element of $F$, as 
  \[
	x=f(x)=ax+b\implies a\ne 1\text{ and } x=b/(1-a).
  \]
  In this case, $N=\{f:f(x)=x+b\,,b\in F\}$ 
  is a subgroup of $G$.
\end{example}

\begin{exercise}
    Prove that Theorem~\ref{thm:Frobenius} can be obtained from
    Corollary~\ref{cor:Frobenius_combinatorio}.
\end{exercise}

Using Frobenius’ theorem (Corollary~\ref{cor:Frobenius_combinatorio}), we can present a refinement of the Cameron–Cohen theorem.

% Wielandt 8.5.4
% 8.5.6 para ver algo de grupos de permutaciones
% 7.1 para ejemplo H(q)
% 10.5.6 (Thompson) N es nilpotente, se usa 10.5.4 

\begin{theorem}[Cameron--Cohen]
\index{Cameron--Cohen!theorem}
\label{thm:CameronCohen>1/n}
    Let $G\leq\Sym_n$ be a transitive subgroup. 
    If $n$ is not the power of a prime number, then
    $c_0>\frac{1}{n}$. 
\end{theorem}

\begin{proof}
    Let us go back to the proof
    of Theorem~\ref{thm:CameronCohen}. Assume that 
    $c_0=1/n$. Then
    \[
    \frac{1}{|G|}\sum_{g\in G}(\chi(g)^2-(n+1)\chi(g)+n)=1
    \]
    and hence $\frac{1}{|G|}\sum_{g\in G}\chi(g)^2=2$. Moreover, 
    since 
    \[
    \frac{1}{|G|}\sum_{g\in G_0}(\chi(g)-1)(\chi(n)-n)
    +\frac{1}{|G|}\sum_{g\in G\setminus G_0}(\chi(g)-1)(\chi(g)-n)=1,
    \]
    it follows that 
    \[
    \sum_{g\in G\setminus G_0}(\chi(g)-1)(\chi(g)-n)=0.
    \]
    Hence $(\chi(g)-1)(\chi(g)-n)=0$
    for all $g\in G\setminus G_0$. 
    
    By Corollary~\ref{cor:Frobenius_combinatorio}, 
    the subset $N=G_0\cup\{\id\}$ is a normal subgroup of $G$. Moreover, $G=N\rtimes H$ for some 
    subgroup $H$ of $G$ of order $n$. Since 
    $n=|H|=|N|-1$, $H$ acts freely and transitively 
    on $N\setminus\{1\}$. 

    We claim that $N$ is a $p$-group for some prime number $p$. Let $n,m\in N\setminus\{1\}$. Since $H$ is transitive on $N\setminus\{1\}$, 
    there exists $h\in H$ such that $h\cdot n=m$. Then
    \[
    |n|=|h\cdot n|=|m|,
    \]
    since for each $h\in H$, the map 
    $x\mapsto h\cdot x$ is an automorphism of $N$. Thus every two elements of $N\setminus\{1\}$ have 
    the same order. Let $p$ be a prime divisor 
    of $|N|$. By Cauchy's theorem, there exists 
    $n\in N$ such that $|n|=p$. Since all non-trivial
    elements of $N$ have the same order, 
    $N$ is a $p$-group. Therefore 
    $n=|N|$ is a power of a prime.
\end{proof}


\begin{exercise}
\label{xca:Frobenius_size20}
Let $G$ be the group of matrices 
of the form $\begin{pmatrix}a&b\\0&1\end{pmatrix}$ where $a,b\in\Z/5$ and $a\ne 0$. Then $|G|=20$. Let 
\[
    h=\begin{pmatrix}
        2\\
        &1
    \end{pmatrix},\quad 
    k=\begin{pmatrix}
        1&1\\
        &1
    \end{pmatrix}.
\]
A direct calculation shows that 
$h^4=1$, $k^5=1$ and $hkh^{-1}=k^2$. Let $H=\langle h\rangle$ 
and $K=\langle k\rangle$. Prove the following statements: 

\begin{enumerate}
    \item Prove that $G=K\rtimes H$.
    \item Find the conjugacy classes of $G$: 
\begin{center}
        \begin{tabular}{cccccc}
             Size & $1$ & $4$ & $5$ & $5$ & $5$\\
             \hline 
             Representative & $1$ & $k$ & $h$ & $h^2$ & $h^3$\\
        \end{tabular}
\end{center}
\item Prove that $G/K$ is cyclic of order four. 
\item Prove that $[G,G]=K$. 
\item Use Theorem~\ref{thm:correspondence} on $G/K$ 
    to find the degree-one characters of $G$. 
\item Let $\chi\in\Irr(K)$ be such that $\chi(k)=\exp(2\pi i/5)$. Prove that 
$\Ind_K^G\chi\in\Irr(G)$. 
\end{enumerate}
\end{exercise}

% \begin{center}   
%         \begin{tabular}{|c|ccccc|}
%              \hline
%              & $1$ & $k$ & $h$ & $h^2$ & $h^3$\\
%              \hline
%              $\chi_1$ & $1$ & $1$ & $1$ & $1$ & $1$\\
%              $\chi_2$ & $1$ & $1$ & $i$ & $-1$ & $-i$\\
%              $\chi_3$ & $1$ & $1$ & $-1$ & $1$ & $-1$\\
%              $\chi_4$ & $1$ & $1$ & $-i$ & $-1$ & $i$\\
%              $\chi_5$ & $4$ & $-1$ & $0$ & $0$ & $0$\\
%              \hline
%         \end{tabular}
%     \end{center} 



\section{Lecture: Week 10}

\subsection{The character table of \texorpdfstring{$\Alt_5$}{A5}}

\index{Character table!of $\Alt_5$}
Let $G=\Alt_5$. 
The group $G$ is a non-abelian simple group of order 60. It has five conjugacy classes, namely

\bigskip 
\begin{center}
    \begin{tabular}{c|ccccc}
         Representative & $\id$  & $(12)(34)$ & $(123)$  & $(12345)$ & $(12354)$\\
         \hline 
         Size & $1$ & $15$ & $20$ & $12$ & $12$ \\
    \end{tabular}
\end{center}
\bigskip 

One can easily get the conjugacy classes of 
$\Alt_5$ with Magma:
\begin{lstlisting}
> A5 := Alt(5);
> ConjugacyClasses(A5);
Conjugacy Classes of group A5
-----------------------------
[1]     Order 1       Length 1
        Id(A5)

[2]     Order 2       Length 15
        (1, 2)(3, 4)

[3]     Order 3       Length 20
        (1, 2, 3)

[4]     Order 5       Length 12
        (1, 2, 3, 4, 5)

[5]     Order 5       Length 12
        (1, 3, 4, 5, 2)    
\end{lstlisting}

Let us see how to obtain all conjugacy classes
of $\Alt_5$ without computers. Let $\sigma\in\Alt_5$ and $C$ be its
conjugacy class in $\Sym_5$. Thus $|C|=(\Sym_5:C_{\Sym_5}(\sigma))$. There are two cases to consider

Assume first that $C_{\Sym_5}(\sigma)\not\subseteq\Alt_5$. Since $\Alt_5$ is a maximal subgroup of $\Sym_5$, it follows that 
$\Alt_5C_{\Sym_5}(\sigma)=\Sym_5$. Using the isomorphism theorems, 
\[
\Sym_5/\Alt_5=\Alt_5C_{\Sym_5(\sigma)}/\Alt_5
\simeq C_{\Sym_5}(\sigma)/(C_{\Sym_5}(\sigma)\cap\Alt_5)
=C_{\Sym_5}(\sigma)/C_{\Alt_5}(\sigma).
\]
Hence 
\[
(\Alt_5:C_{\Alt_5}(\sigma))=\frac{(\Sym_5:C_{\Alt_5}(\sigma))}{(\Sym_5:\Alt_5)}
=\frac{(\Sym_5:C_{\Alt_5}(\sigma))}{(C_{\Sym_5}(\sigma):C_{\Alt_5}(\sigma))}
=(\Sym_5:C_{\Sym_5}(\sigma))=|C|.
\]
Therefore $C$ is the class of $\sigma$ in $\Alt_5$. 

Assume now that $C_{\Sym_5}(\sigma)\subseteq\Alt_5$. Then 
$C_{\Alt_5}(\sigma)=C_{\Sym_5}(\sigma)\cap\Alt_5=C_{\Sym_5}(\sigma)$
and therefore 
\[
(\Alt_5:C_{\Alt_5}(\sigma))=(\Alt_5:C_{\Sym_5}(\sigma))
=\frac12(\Sym_5:C_{\Sym_5}(\sigma))=\frac12|C|.
\]
Thus $C$ splits into two conjugacy classes of $\Alt_5$ of equal size. 

The identity permutation is central. The even permutations 
$(12)(34)$ and $(123)$ both commutes with some odd permutation in $\Sym_5$ (e.g. 
$[(12)(34),(34)]=[(123),(45)]=\id$). Thus these classes do not split
in $\Alt_5$. There are twenty-four 5-cycles in $\Sym_5$. Since $24$ does not
divide $|\Alt_5|=60$, it follows that the class of 5-cycles
splits in $\Alt_5$. As representatives of these classes
we can take $(12345)$ and $(12354)$. 

Since $\Alt_5$ has five conjugacy classes, $|\Irr(G)|=5$. We already know one irreducible character of $G$, namely the trivial character 
$\Tchar_G$. 

Let $H=\Alt_4$. We compute $\Ind_H^G\Tchar_H$, where $\Tchar_H$ is the trivial character of $H$. 
By Corollary~\ref{cor:reciprocity}, 
\[
\left(\Ind_H^G\Tchar_H\right)(\id) = 5.
\]
And a direct calculation shows
\begin{align*}
    &\left(\Ind_H^G\Tchar_H\right)((12)(34)) = 1,\\
    &\left(\Ind_H^G\Tchar_H\right)((123)) = 2,\\
    &\left(\Ind_H^G\Tchar_H\right)((12345)) = 0\\ 
    &\left(\Ind_H^G\Tchar_H\right)((12354)) = 0.
\end{align*}

By Frobenius' reciprocity,
\begin{align*}
    \langle\Ind_H^G\Tchar_H,\Tchar_G\rangle = \langle\Tchar_H,\Res_H^G\Tchar_G\rangle
    = \langle\Tchar_H,\Tchar_H\rangle=
    1.
\end{align*}

Let $\chi_2=\Ind_H^G\Tchar_H-\Tchar_G$. Since 
\[
\langle\Ind_H^G\Tchar_H-\Tchar_G,\Ind_H^G\Tchar_H-\Tchar_G\rangle=1, 
\]
it follows that $\chi_2\in\Irr(G)$. 

\begin{exercise}
\label{xca:A5_chi2}
    Use Proposition~\ref{pro:2transitive} to derive (once again) the values of $\chi_2$.
\end{exercise}

So far we have the 
following table: 

\bigskip 
\begin{center}
        \begin{tabular}{|c|ccccc|}
        \hline  
         & $\id$ & $(12)(34)$ & $(123)$ & $(12345)$ & $(12354)$\\
        \hline 
        $\Tchar_G$ & $1$ & $1$ & $1$ & $1$ & $1$\\
        $\chi_2$ & $4$ & $0$ & $1$ & $-1$ & $-1$\\
        $\chi_3$ & $n_3$ & $\cdot$ & $\cdot$ & $\cdot$& $\cdot$\\
        $\chi_4$ & $n_4$ & $\cdot$ & $\cdot$ & $\cdot$& $\cdot$\\
        $\chi_5$ & $n_5$ & $\cdot$ & $\cdot$ & $\cdot$& $\cdot$\\
        \hline 
    \end{tabular}
\end{center}
\bigskip 

As $G$ is simple non-abelian, 
$|G/[G,G]|=1$. It follows that
$\Tchar_G$ is the only linear character of $G$. Moreover, 
$\chi_j(1)\geq3$ by Theorem~\ref{thm:simple}. Since 
\[
60=1+16+n_3^2+n_4^2+n_5^2
\]
and each $n_j$ divides $|G|=60$ 
(see Theorem \ref{thm:Frobenius_chi(1)}), it follows that 
$n_j\in\{3,4,5,6\}$. If some $n_j=6$, say without
loss of generality $n_3=6$, then 
\[
7=43-36=n_2^2+n_3^2, 
\]
a contradiction. Thus $n_j\in\{3,4,5\}$ for 
all $j\in\{3,4,5\}$. Without loss of generality, 
we may assume that $n_3=n_4=3$ and $n_5=5$. 

\bigskip 
\begin{center}
        \begin{tabular}{|c|ccccc|}
        \hline  
         & $\id$ & $(12)(34)$ & $(123)$ & $(12345)$ & $(12354)$\\
        \hline 
        $\Tchar_G$ & $1$ & $1$ & $1$ & $1$ & $1$\\
        $\chi_2$ & $4$ & $0$ & $1$ & $-1$ & $-1$\\
        $\chi_3$ & $3$ & $\cdot$ & $\cdot$ & $\cdot$& $\cdot$\\
        $\chi_4$ & $3$ & $\cdot$ & $\cdot$ & $\cdot$& $\cdot$\\
        $\chi_5$ & $5$ & $\cdot$ & $\cdot$ & $\cdot$& $\cdot$\\
        \hline 
    \end{tabular}
\end{center}
\bigskip 

The group $\Alt_5$ acts on the set $Y$ of subsets 
of $\{1,2,\dots,5\}$ of two elements, namely
\[
g\cdot \{a,b\}=\{g\cdot a,g\cdot b\}.
\]
Note that $|Y|=\binom{5}{2}=10$. Moreover, 
this action is transitive. Let us compute 
the character $\psi$ of the corresponding 
$\C\Alt_5$-module and the difference 
$\psi-\Tchar_G$ (We know $\psi$ counts
fixed points.)

\bigskip 
\begin{center}
        \begin{tabular}{|c|ccccc|}
        \hline  
         & $\id$ & $(12)(34)$ & $(123)$ & $(12345)$ & $(12354)$\\
        \hline 
        $\psi$ & $10$ & $2$ & $1$ & $0$ & $0$\\
        $\psi-\Tchar_G$ & $9$ & $1$ & $0$ & $-1$ & $-1$\\
        \hline 
    \end{tabular}
\end{center}
\bigskip 

The identity, of course, fixes all the ten elements
of $Y$. The permutation 
$(12)(34)$ fixed two two-elements subsets, namely
$\{1,2\}$ and $\{3,4\}$. The permutation 
$(123)$ fixes only one two-elements subset, namely
$\{4,5\}$. Finally, $(12345)$ and 
$(12354)$ fix no two-element subsets. 

Now we compute 
\[
\langle \psi-\Tchar_G,\psi-\Tchar_G\rangle=2
\]
and hence $\psi-\Tchar_G$ is the sum of two irreducible
characters (see Exercise~\ref{xca:n_irreducible}). Since
\[
\langle \psi-\Tchar_G,\chi_2\rangle=1,
\]
it follows that $\psi-\Tchar_G-\chi_2\in\Irr(G)$. Let 
$\chi_5=\psi-\Tchar_G-\chi_2$. Then

\bigskip 
\begin{center}
        \begin{tabular}{|c|ccccc|}
        \hline  
         & $\id$ & $(12)(34)$ & $(123)$ & $(12345)$ & $(12354)$\\
        \hline 
        $\Tchar_G$ & $1$ & $1$ & $1$ & $1$ & $1$\\
        $\chi_2$ & $4$ & $0$ & $1$ & $-1$ & $-1$\\
        $\chi_3$ & $3$ & $\cdot$ & $\cdot$ & $\cdot$& $\cdot$\\
        $\chi_4$ & $3$ & $\cdot$ & $\cdot$ & $\cdot$& $\cdot$\\
        $\chi_5$ & $5$ & $1$ & $-1$ & $0$& $0$\\
        \hline 
    \end{tabular}
\end{center}
\bigskip 

Let $K=\langle(12345)\rangle$ and 
$\eta\in\Irr(K)$ be such that $\eta((12345))=\zeta$, where
$\zeta=\exp(2\pi i/5)$ is a primitive $5$-th root of one. We can then compute 
$\Ind_K^G\eta$. 

\bigskip 
\begin{center}
        \begin{tabular}{|c|ccccc|}
        \hline  
         & $\id$ & $(12)(34)$ & $(123)$ & $(12345)$ & $(12354)$\\
         \hline 
         $\Ind_K^G\eta$ & $12$ & $0$ & $0$ & $\zeta^2+\zeta^3$ & $\zeta+\zeta^4$\\
         \hline 
\end{tabular}
\end{center}
\bigskip 

Since 
$\langle\Ind_K^G\eta,\chi_2\rangle=1=\langle\Ind_H^G\eta,\chi_5\rangle$,
it follows that 
\bigskip 
\begin{center}
        \begin{tabular}{|c|ccccc|}
        \hline  
         & $\id$ & $(12)(34)$ & $(123)$ & $(12345)$ & $(12354)$\\
         \hline 
         $\Ind_K^G\eta-\chi_2-\chi_5$ & $3$ & $-1$ & $0$ & $-\zeta-\zeta^4$ & $-\zeta^2-\zeta^3$\\
         \hline 
\end{tabular}
\end{center}
\bigskip 
Let $\chi_3=\Ind_K^G\eta-\chi_2-\chi_5$. Then $\chi_3\in\Irr(G)$, because it is
not the sum of three copies of the trivial character. Thus this is how our character table looks like: 
\bigskip
\begin{center}
        \begin{tabular}{|c|ccccc|}
        \hline  
         & $\id$ & $(12)(34)$ & $(123)$ & $(12345)$ & $(12354)$\\
        \hline 
        $\Tchar_G$ & $1$ & $1$ & $1$ & $1$ & $1$\\
        $\chi_2$ & $4$ & $0$ & $1$ & $-1$ & $-1$\\
        $\chi_3$ & $3$ & $-1$ & $0$ & $-\zeta-\zeta^4$ & $-\zeta^2-\zeta^3$\\
        $\chi_4$ & $3$ & $\cdot$ & $\cdot$ & $\cdot$& $\cdot$\\
        $\chi_5$ & $5$ & $1$ & $-1$ & $0$& $0$\\
        \hline 
    \end{tabular}
\end{center}
\bigskip 

\begin{exercise}
    Use the orthogonality relations
    to compute the missing row of the character table
    of $\Alt_5$. 
\end{exercise}

The previous exercise finishes the calculation
of the character table of $\Alt_5$; see Table~\ref{tab:A5}. 

\begin{table}[h]
\caption{The character table of $\Alt_5$.}
\label{tab:A5}
        \begin{tabular}{|c|ccccc|}
        \hline  
        & $1$ & $15$ & $20$ & $12$ & $12$ \\
         & $\id$ & $(12)(34)$ & $(123)$ & $(12345)$ & $(12354)$\\
        \hline 
        $\chi_1$ & $1$ & $1$ & $1$ & $1$ & $1$\\
        $\chi_2$ & $4$ & $0$ & $1$ & $-1$ & $-1$\\
        $\chi_3$ & $3$ & $-1$ & $0$ & $-\zeta-\zeta^4$ & $-\zeta^2-\zeta^3$\\
        $\chi_4$ & $3$ &  $-1$ & $0$ & $-\zeta^2-\zeta^3$ & $-\zeta-\zeta^4$ \\
        $\chi_5$ & $5$ & $1$ & $-1$ & $0$& $0$\\
        \hline 
    \end{tabular}
\end{table}

One last observation: 
Since $\zeta=\exp(2\pi i/5)$, it follows
that 
\[
-\zeta-\zeta^4=\frac{1-\sqrt{5}}{2},
\quad 
-\zeta^2-\zeta^3=\frac{1+\sqrt{5}}{2}.
\]

% Let us see what Magma says:

% \begin{lstlisting}
% > CharacterTable(Alt(5));


% Character Table
% ---------------


% ---------------------------
% Class |   1  2  3    4    5
% Size  |   1 15 20   12   12
% Order |   1  2  3    5    5
% ---------------------------
% p  =  2   1  1  3    5    4
% p  =  3   1  2  1    5    4
% p  =  5   1  2  3    1    1
% ---------------------------
% X.1   +   1  1  1    1    1
% X.2   +   3 -1  0   Z1 Z1#2
% X.3   +   3 -1  0 Z1#2   Z1
% X.4   +   4  0  1   -1   -1
% X.5   +   5  1 -1    0    0


% Explanation of Character Value Symbols
% --------------------------------------

% # denotes algebraic conjugation, that is,
% #k indicates replacing the root of unity w by w^k

% Z1     = (CyclotomicField(5: Sparse := true)) ! [ RationalField() | 0, 0, -1, -1 ]    
% \end{lstlisting}

\subsection{Mackey's theorem}

We begin with two routine exercises. 

%\begin{exercise}
%$    Let $G$ be a finite group and $\chi\in\Char(G)$. 
%$    Let $f\colon G\to H$ be an isomorphism of 
%$    groups. Then 
%$    \[
%$    (f\cdot )(f(g))
%$    
%\end{exercise}

\begin{exercise}
\label{xca:conjugate_chars1}
Let $G$ be a finite group and $N$ be a normal subgroup
of $G$. Prove that $G$ acts on $\Irr(N)$ via 
\[
(g\cdot\theta)(n)=\theta(g^{-1}ng),\quad 
g\in G,\theta\in\Irr(N),n\in N.
\]
\end{exercise}

\begin{exercise}
\label{xca:conjugate_chars2}
Let $G$ be a finite group and $N$ be a normal subgroup of $G$. 
Let $\chi\in\cf(G)$, $\theta\in\cf(N)$ and $g\in G$. Prove that
\[
\langle\Res_N^G\chi,g\cdot\theta\rangle=\langle\Res_N^G\chi,\theta\rangle.
\]
% for all $\chi\in\cf(G)$.
%     \begin{enumerate}
%         \item $\langle g\cdot\alpha,g\cdot\beta\rangle=\langle\alpha,\beta\rangle$.
%         \item $\langle\Res_N^G\chi,g\cdot\alpha\rangle=\langle\Res_N^G\chi,\alpha\rangle$ for all $\chi\in\cf(G)$. 
%     \end{enumerate}
\end{exercise}


\begin{definition}
    Two representations of a finite group 
    are said to be \emph{disjoint} if they have no common 
    irreducible constituent. 
\end{definition}

\begin{exercise}
\label{xca:disjoint}
    Prove that 
    two representations are disjoint if and only if their characters are orthogonal. 
\end{exercise}

Let $H$ and $K$ be subgroups of a group $G$. 
The group $H\times K$ acts on $G$ via $(h,k)\cdot g=hgk^{-1}$. 
The orbit of $g$ under this action is 
the \emph{double coset} 
\[
HgK=\{(h,k)\cdot g:h\in H,k\in K\}
=\{hgk^{-1}:h\in H,k\in K\}
\]
with representative $g$. 

\begin{theorem}[Mackey]
\label{thm:Mackey}
\index{Mackey!theorem}
    Let $G$ be a finite group and $H$ and $K$ be 
    subgroups of $G$. Let $S$ be a complete set 
    of representatives of double $(H,K)$-cosets. If $\alpha\in\cf(K)$, then 
    \[
    \Res_H^G\Ind_K^G\alpha=\sum_{s\in S}\Ind_{H\cap sKs^{-1}}^H\Res_{H\cap sKs^{-1}}^{sKs^{-1}}(s\cdot f).
    \]
\end{theorem}

\begin{proof}
    For $s\in S$, let $X(s)$ be a left transversal 
    for $H\cap sKs^{-1}$ on $H$. Then 
    \[
    H=\bigcup_{x\in X(s)}x(H\cap sKs^{-1}),
    \]
    where the union is disjoint. 

    \begin{claim}
        $HsK=\bigcup_{x\in X(s)}xsK$, where the union is disjoint.
    \end{claim} 
    
    Let $z\in HsK$. Then $z=hsk$ for some $h\in H$ and $k\in K$. Since $h\in x(H\cap sKs^{-1})$ 
    for some $x\in X(s)$, 
    \[
    z=hsk\in x(H\cap sKs^{-1})sK\subseteq xsK. 
    \]
    Conversely, let $z\in xsK$ for some $x\in X(s)\subseteq H$. Then $z\in xsK\subseteq HsK$. To see that the union is disjoint, 
    suppose that $xsK=x_1sK$ for some $x,x_1\in X(s)$. Then 
    $x_1^{-1}x\in sKs^{-1}\cap H$. Thus $x(sKs^{-1}\cap H)=x_1(sKs^{-1}\cap H)$ and hence $x=x_1$, because $X(s)$ 
    is a left transversal for $sKs^{-1}\cap H$ in $H$. 

    \bigskip 
    Let $T(s)=\{xs:x\in X(s)\}$ and 
    \[
    T=\bigcup_{s\in S}T(s)
    \]
    To see that the union is disjoint, we proceed as follows. 
    Let $xs=x_1s_1$ for some $s,s_1\in S$, $x\in X(s)$ and $x_1\in X(s_1)$. 
    Since $x^{-1}x_1\in H$ and 
    $HsK=Hx^{-1}x_1s_1K=Hs_1K$, $s=s_1$ and hence $x=x_1$. 

    Then
    \[
    G=\bigcup_{s\in S}HsK
    =\bigcup_{s\in S}\bigcup_{x\in X(s)}xsK
    =\bigcup_{s\in S}\bigcup_{t\in T(s)}tK
    =\bigcup_{t\in T}tK.
    \]
    Since the unions are disjoint, 
    it follows that $T$ is a left transversal of $K$ in $G$. 

    For $h\in H$, 
    \begin{align*}
        (\Ind_K^G\alpha)(h) &= \sum_{t\in T}\alpha^0(t^{-1}ht)\\
        &=\sum_{s\in S}\sum_{t\in T(s)}\alpha^0(t^{-1}ht)\\
        &=\sum_{s\in S}\sum_{x\in X(s)}\alpha^0(s^{-1}x^{-1}hxs)\\
        &=\sum_{s\in S}\sum_{\substack{x\in X(s)\\x^{-1}hx\in sKs^{-1}}}(s\cdot \alpha)(x^{-1}hx)\\
        &=\sum_{s\in S}\sum_{\substack{x\in X(s)\\x^{-1}hx\in H\cap sKs^{-1}}}\Res_{H\cap sKs^{-1}}^{sKs^{-1}}(s\cdot \alpha)(\underbrace{x^{-1}hx)}_{\in H\cap sKs^{-1}}\\
        &=\sum_{s\in S}\Ind_{H\cap sKs^{-1}}^H\Res_{H\cap sKs^{-1}}^{sKs^{-1}}(s\cdot \alpha)(h).\qedhere 
    \end{align*}
\end{proof}

\begin{theorem}[Mackey's irreducibility criterion]
\label{thm:Mackey_irreducibility}
\index{Mackey!irreducibility criterion}
Let $H$ be a subgroup of a finite group $G$ and $\chi\in\Char(H)$.
Then $\Ind_H^G\chi\in\Irr(G)$ if and only if $\chi\in\Irr(H)$ and $\Res_{H\cap sHs^{-1}}^H\chi$ and 
$\Res_{H\cap sHs^{-1}}^{sHs^{-1}}(s\cdot\chi)$ are disjoint for all $s\not\in H$. 
\end{theorem}
\begin{proof}
    Let $S$ be a complete set of representatives of $(H,H)$-double cosets. Without loss of generality, 
    we may assume that $1\in S$. Note that if $s=1$, then 
    $H\cap sHs^{-1}=H$ and $s\cdot\chi=\chi$. By Mackey's theorem, 
    \begin{align*}
        \Res_H^G\Ind_H^G\chi 
        &=\sum_{s\in S}\Ind_{H\cap sHs^{-1}}^H\Res_{H\cap sHs^{-1}}^{sHs^{-1}}(s\cdot\chi)\\
        &=\chi+\sum_{1\ne s\in S}\Ind_{H\cap sHs^{-1}}^H\Res_{H\cap sHs^{-1}}^{sHs^{-1}}(s\cdot\chi).
    \end{align*}
    By Frobenius' reciprocity, 
    \begin{align*}
        \langle\Ind_H^G\chi,\Ind_H^G\chi\rangle
        &=\langle\Res_H^G\Ind_H^G\chi,\chi\rangle\\
        &=\underbrace{\langle\chi,\chi\rangle}_{\geq1}+\sum_{1\ne s\in S}\underbrace{\langle\Ind_{H\cap sHs^{-1}}^H\Res_{H\cap sHs^{-1}}^{sHs^{-1}}(s\cdot\chi),\chi\rangle}_{\geq0}.
    \end{align*}

    If $\chi\in\Irr(H)$ and $\Res_{H\cap sHs^{-1}}^H\chi$ and 
    $\Res_{H\cap sHs^{-1}}^{sHs^{-1}}(s\cdot\chi)$ are disjoint for all $s\not\in H$, then 
    $\langle\Ind_H^G\chi,\Ind_H^G\chi\rangle=1$ and hence $\Ind_H^G\chi\in\Irr(G)$. 

    Conversely, if $\Ind_H^G\chi\in\Irr(G)$, then $\langle\Ind_H^G\chi,\Ind_H^G\chi\rangle=1$. Thus 
    $\langle\chi,\chi\rangle=1$ and 
    \[
    \langle\Res_{H\cap sHs^{-1}}^{sHs^{-1}}(s\cdot\chi),\Res_{H\cap sHs^{-1}}^H\chi\rangle=
    \langle\Ind_{H\cap sHs^{-1}}^H\Res_{H\cap sHs^{-1}}^{sHs^{-1}}(s\cdot\chi),\chi\rangle=0
    \]
    for all $s\in S\setminus\{1\}$. As every element $s\notin H$ 
    could serve as a representative of an $(H,H)$-double coset, the claim follows.
\end{proof}

Theorem~\ref{thm:Mackey_irreducibility} takes a particularly elegant form when the subgroup is normal.

\begin{exercise}
\label{xca:Mackey}
    Let $H$ be a normal subgroup of a finite group $G$ and $\chi\in\Char(H)$.
    Then $\Ind_H^G\chi\in\Irr(G)$ if and only if $\chi\in\Irr(H)$ and $\chi\ne s\cdot\chi$ 
    for all $s\not\in H$. 
\end{exercise}

\begin{example}
\label{exa:p(p-1)}
    For a prime number $p\geq3$, let 
    \[
        G=\left\{\begin{pmatrix}a&b\\0&1\end{pmatrix}:0\ne a\in\Z/p,\,b\in\Z/p\right\}\text{ and }
        H=\left\{\begin{pmatrix}1&b\\0&1\end{pmatrix}:b\in\Z/p\right\}.
    \]
    Then $|G|=p(p-1)$, $H$ is a normal subgroup of $G$ and $|G/H|=p-1$. Let 
    \[
    \chi\colon H\to\C^{\times},\quad\begin{pmatrix}1&b\\0&1\end{pmatrix}\mapsto \exp(2\pi ib/p).
    \]
    Then $\chi$ is a group homomorphism. For each $a\in\Z/p\setminus\{0,1\}$, let $s(a)=\begin{pmatrix}a&0\\0&1\end{pmatrix}$. Then 
    \[
    (s(a)\cdot\chi)\begin{pmatrix}1&b\\0&1\end{pmatrix}=\chi\begin{pmatrix}1&a^{-1}b\\0&1\end{pmatrix}=\exp(2\pi ia^{-1}b/p)
    \ne \exp(2\pi ib/p).
    \]
    Hence $s(a)\cdot\chi\ne\chi$ for all $a\in\Z/p\setminus\{0,1\}$. By Exercise~\ref{xca:Mackey}, 
    $\Ind_H^G\chi\in\Irr(G)$ and 
    \[
    \deg\Ind_H^G\chi=(\Ind_H^G\chi)(1)=(G:H)\chi(1)=p-1.
    \]
    
    Since 
    $|G|-(p-1)^2=p-1$, 
    we still need additional irreducible characters to fully determine $\Irr(G)$. 
    The group $G/H$ is cyclic of order $p-1$, so it has $p-1$ irreducible characters, all of degree one. 
    These irreducible characters lift to irreducible characters of $G$ (see Theorem~\ref{thm:correspondence}). 
\end{example}

\begin{bonus}
\label{xca:p(p-1)}
    Find the character table of the group of Example~\ref{exa:p(p-1)}. 
\end{bonus}
\section{}

\subsection{Solvable groups and Burnside's theorem}

\index{Derived series}
For a group $G$ let 
$G^{(0)}=G$ and 
$G^{(i+1)}=[G^{(i)},G^{(i)}]$ for $i\geq0$.
The \emph{derived series} of $G$ is the sequence
\[
G=G^{(0)}\supseteq G^{(1)}\supseteq G^{(2)}\supseteq\cdots
\]
Each $G^{(i)}$ is a characteristic subgroup of $G$. We say that 
$G$ is \emph{solvable} if $G^{(n)}=\{1\}$ for some $n$.  

\begin{example}
	Abelian groups are solvable. 
\end{example}

\begin{example}
	The group $\SL_2(3)$ is solvable, as the derived series is 
	\[
	\SL_2(3)\supseteq Q_8\supseteq C_4\supseteq C_2\supseteq \{1\}.
	\]
	Here is the what the computer says:
\begin{lstlisting}
gap> IsSolvable(SL(2,3));
true
gap> List(DerivedSeries(SL(2,3)),StructureDescription);
[ "SL(2,3)", "Q8", "C2", "1" ]
\end{lstlisting}
\end{example}

\begin{example}
	Non-abelian simple groups cannot be solvable. 
\end{example}

\begin{exercise}
	\label{xca:solvable}
	Let $G$ be a group. Prove the following statements:
	\begin{enumerate}
		\item A subgroup $H$ of $G$ is solvable, when $G$ is solvable.
		\item Let $K$ be a normal subgroup of $G$. 
		    Then $G$ is solvable if and only if $K$ and $G/K$ are solvable.
	\end{enumerate}
\end{exercise}

\begin{example}
	For $n\geq5$ the group $\Alt_n$ is not solvable. It follows that 
	$\Sym_n$ is not solvable for $n\geq5$. 
\end{example}

\begin{exercise}
\label{xca:pgroups_solvable}
	Let $p$ be a prime number. Prove that 
	finite $p$-groups are solvable.
\end{exercise}

\begin{theorem}[Burnside]
	\index{Burnside's theorem}
	\label{thm:Burnside_auxiliar}
	Let $G$ be a finite group. If $\phi\colon G\to\GL_n(\C)$ is a representation
	with character $\chi$ and $C$ is a conjugacy class of $G$ such that 
	$\gcd(|C|,n)=1$, then for every $g\in C$ either 
	$\chi(g)=0$ or $\phi_g$ is a scalar matrix. 
\end{theorem}

We need a lemma.

\begin{lemma}
	\label{lem:4Burnside}
	Let $\epsilon_1,\dots,\epsilon_n$ be roots of one such that 
	$(\epsilon_1+\cdots+\epsilon_n)/n\in\A$. Then either 
	$\epsilon_1=\cdots=\epsilon_n$ or 
	$\epsilon_1+\cdots+\epsilon_n=0$.
\end{lemma}

\begin{proof}
	Let $\alpha=(\epsilon_1+\cdots+\epsilon_n)/n$.
	If the $\epsilon_j$s are not all equal, then $\|\alpha\|<1$. Moreover, 
	$\|\beta\|<1$ for every algebraic conjugate $\beta$ of $\alpha$. Since the product 
	of the algebraic conjugates of $\alpha$ is an integer of absolute value 
	$<1$, it follows that it is zero. 
\end{proof}

Now we prove the theorem.

\begin{proof}[Proof of Theorem \ref{thm:Burnside_auxiliar}]
	Let $\epsilon_1,\dots,\epsilon_n$ be the eigenvalues of $\phi_g$. By assumption, 
	$\gcd(|C|,n)=1$, there exist $a,b\in\Z$ such that $a|C|+bn=1$. Since 
	$|C|\chi(g)/n\in\A$, after multiplying by $\chi(g)/n$ we obtain that  
	\[
		a|C|\frac{\chi(g)}{n}+b\chi(g)=\frac{\chi(g)}{n}=\frac{1}{n}(\epsilon_1+\cdots+\epsilon_n)\in\A.
	\]
	The previous lemma implies that there are two cases to consider: 
	either $\epsilon_1=\cdots=\epsilon_n$ or $\epsilon_1+\cdots+\epsilon_n=0$. In the first
	case, since $\phi_g$ is diagonalizable, $\phi_g$ is a scalar matrix. 
	In the second case, $\chi(g)=0$.
\end{proof}

\begin{theorem}[Burnside]
	\index{Burnside's theorem}
	Let $p$ be a prime number. If $G$ is a finite group and 
	$C$ is a conjugacy class of $G$ with $p^k>1$ elements, then $G$ 
	is not simple.
\end{theorem}

\begin{proof}
	Let $g\in C\setminus\{1\}$. Column orthogonality implies that 
	\begin{equation}
	\label{eq:Burnside}
	\begin{aligned}
		0&=\sum_{\chi\in\Irr(G)}\chi(1)\chi(g)\\
		&=\sum_{p\mid\chi(1)}\chi(1)\chi(g)+\sum_{p\nmid\chi(1):\chi\ne\chi_1}\chi(1)\chi(g)+1,
	\end{aligned}
	\end{equation}
	where the one corresponds to the trivial representation of
	$G$.
	
	Look at this equation modulo $p$. If $\chi(g)=0$ for all
	$\chi\in\Irr(G)$
	such that $\chi\ne\chi_1$ and $p\nmid\chi(1)$, then
	\[
	-\frac{1}{p}=\sum\frac{\chi(1)}{p}\chi(g)\in\A\cap\Q=\Z,
	\]
	where the sum is taken over all non-trivial irreducibles
	of $G$ of degree divisible by $p$, a contradiction. Hence 
	there exists an irreducible non-trivial representation 
	$\phi$ with character $\chi$ such that $p$ does not divide
	$\chi(1)$ and $\chi(g)\ne0$. By the previous theorem, 
	$\phi_g$ is a scalar matrix. If $\phi$ is faithful, then 
	$g$ is a non-trivial central element, a contradiction since 
    $|C|>1$. If $\phi$ is not faithful, then 
    $G$ is not simple (because 
	$\ker\phi$ is a non-trivial proper normal subgroup of $G$). 
\end{proof}

\begin{theorem}[Burnside]
  \index{Burnside's theorem}
  Let $p$ and $q$ be prime numbers. If $G$ has order $p^aq^b$, then $G$ is solvable.
\end{theorem}

\begin{proof}
	If $G$ is abelian, then it is solvable.
	Suppose now $G$ is non-abelian.
	Let us assume that the theorem is not true. Let $G$ be a group
	of minimal order $p^aq^b$
	that is not solvable. Since $|G|$ is minimal, $G$ is a non-abelian simple group.
	By the previous theorem, 
	$G$ has no conjugacy classes of size $p^k$ nor 
	conjugacy classes of size $q^l$ with $k,l\geq1$. The size
	of every conjugacy class of $G$ is one or divisible by $pq$. 
	Note that, since $G$ is a non-abelian simple group,
	the center of $G$ is trivial.
	Thus there is only one conjugacy class of size one.
	By the class
	equation,
	\[
		|G|=1+\sum_{C:|C|>1}|C|\equiv 1 \bmod pq,
	\]
	where the sum is taken over all conjugacy classes 
	with more than one element, a contradiction.
\end{proof}

Some generalizations of Burnside's theorem. 

\begin{theorem}[Kegel--Wielandt]
    \index{Kegel--Wielandt's theorem}
    \label{thm:KegelWielandt}
    If $G$ is a finite group and there are nilpotent subgroups 
    $A$ and $B$ of $G$ such that 
    $G=AB$, then $G$ is solvable.
\end{theorem}

See~\cite[Theorem 2.4.3]{MR1211633} for the proof.


Another generalization of Burnside's theorem
is based on \emph{word maps}. A word map
of a group $G$ is a map 
\[
G^k\to G,\quad 
(x_1,\dots,x_k)\mapsto w(x_1,\dots,x_k)
\]
for some word $w(x_1,\dots,x_k)$ of the free group $F_k$ of rank $k$. 
Some word maps are surjective in certain families of groups. For example, 
Ore's conjecture is precisely the surjectivity of the word map
$(x,y)\mapsto [x,y]=xyx^{-1}y^{-1}$ in every finite non-abelian simple 
group. 

\begin{theorem}[Guralnick--Liebeck--O'Brien--Shalev--Tiep]
    Let $a,b\geq0$, $p$ and $q$ be prime numbers and $N=p^aq^b$. The map 
    $(x,y)\mapsto x^Ny^N$ is surjective in every finite simple group. 
\end{theorem}

The proof appears in~\cite{MR3827208}. 

The theorem implies Burnside's theorem. Let $G$ be a group of order
$N=p^aq^b$. Assume that $G$ is not solvable. 
Fix a composition series of $G$. There is a non-abelian factor $S$ 
of order that divides $N$. Since 
$S$ is simple non-abelian and $s^N=1$, it follows that the word map
$(x,y)\mapsto x^Ny^N$ has trivial image in $S$, a contradiction 
to the theorem. 

\subsection{Feit--Thompson theorem}

\begin{theorem}[Feit--Thompson]
    \index{Feit--Thompson theorem}
    Groups of odd order are solvable. 
\end{theorem}

The proof of Feit--Thompson theorem is extremely hard. 
It occupies a full volume of the 
\emph{Pacific Journal of Mathematics}~\cite{MR166261}. 
A formal verification of the proof 
(based on the computer software Coq) 
was announced in~\cite{MR3111271}.  

Back in the day it was believed that if a certain divisibility 
conjecture is true, 
the proof of Feit--Thompson theorem 
could be simplified. 

\begin{conjecture}[Feit--Thompson]
\index{Feit--Thompson conjecture}
    There are no prime numbers $p$ and $q$ such that
    $\frac {p^{q}-1}{p-1}$ divides $\frac{q^{p} - 1}{q - 1}$. 
\end{conjecture}

The conjecture remains open. However, now we know that 
proving the conjecture will not simplify further
the proof of Feit--Thompson theorem. 

In 2012 Le proved that the conjecture is true for $q=3$, see 
\cite{MR2900154}. 


In~\cite{MR297686} 
Stephens proved that a certain stronger version of the conjecture 
does not hold, as the integers 
$\frac {p^{q}-1}{p-1}$ and $\frac{q^{p} - 1}{q - 1}$ 
could have common factors. In fact, if $p=17$ and $q=3313$, 
then 
\[
\gcd\left(\frac {p^{q}-1}{p-1},\frac{q^{p} - 1}{q - 1}\right)=112643.
\]
Nowadays we can check this easily in almost every desktop computer:
\begin{lstlisting}
gap> Gcd((17^3313-1)/16,(3313^17-1)/3312);
112643
\end{lstlisting}
No other counterexamples have been found of Stephen’s stronger version of the conjecture.




\section{Lecture: Week 12}

\subsection{Kronecker's theorem}

We begin with a classical theorem of Kronecker on algebraic integers. Recall 
that $\alpha\in\C$ is an \emph{algebraic integer} if there is a monic
polynomial $f\in\Z[X]$ such that $f(\alpha)=0$ (see Definition~\ref{def:algebraic_integer}). Let $\A$ 
be the set of algebraic integers. 

\begin{exercise}
    \label{xca:irreducible}
    Let $\alpha\in\A$. Prove that there exists a monic polynomial $f\in\Z[X]$, 
    irreducible $f\in\Q[X]$ such that $f(\alpha)=0$. 
\end{exercise}

The polynomial of Exercise~\ref{xca:irreducible} is called the \emph{minimal polynomial} of $\alpha$. 

\begin{exercise}
\label{xca:distinct}
    Let $\alpha\in\A$. Prove that the roots of the
    minimal polynomial of $\alpha$ are pairwise distinct. 
\end{exercise}

The \emph{conjugates} of $\alpha$ are the roots of the minimal polynomial of $\alpha$. 

Recall that for an $n\times n$ matrix $A=(a_{ij})$, its \emph{norm} is defined 
as the maximum absolute row sum of the matrix, that is 
\[
\|A\|_\infty=\max_{1\leq i\leq n}\sum_{j=1}^n|a_{ij}|.
\]
For $A,B\in\C^{n\times n}$ and $\lambda\in\C$, 
the following properties hold: 
\begin{enumerate}
    \item $\|A\|_\infty\geq0$.
    \item $\|A\|_\infty=0$ if and only if $A$ is the $n\times n$ zero matrix. 
    \item $\lambda\|A\|_\infty=|\lambda|\|A\|_\infty$. 
    \item $\|A+B\|_\infty\leq\|A\|_\infty+\|B\|_\infty$. 
    \item $\|AB\|_\infty\leq\|A\|_\infty\|B\|_\infty$. 
\end{enumerate}

\begin{theorem}[Kronecker]
\index{Kronecker theorem}
\label{thm:Kronecker}
Let $\alpha\in\A$. Assume that all the conjugates of $\alpha$ 
have absolute value at most one. Then either $\alpha=0$ or $\alpha$ is a root of one. 
\end{theorem}

\begin{proof}
    Assume that $\alpha\ne 0$. 
    Let $f\in\Z[X]$ be the minimal polynomial of $\alpha$, say 
    \[
    f=X^n+a_{n-1}X^{n-1}+\cdots+a_1X+a_0
    \]
    for integers $a_0,a_1,\dots,a_{n-1}\in\Z$. Then $f(0)\ne0$ because $f$ is irreducible in $\Q[X]$ (see Exercise~\ref{xca:irreducible}).  
    Let 
    \[
    F=\begin{pmatrix}
        0 & 0 & \cdots & 0 & -a_0\\
        1 & 0 & \cdots & 0 & -a_1\\
        0 & 1 & \cdots & 0 & -a_2\\
        \vdots & \vdots & \ddots & \vdots & \vdots\\
        0 & 0 & \cdots & 1 & -a_{n-1}
    \end{pmatrix}\in\Z^{n\times n}
    \]
    be the \emph{companion matrix} of $f$. The characteristic polynomial and the minimal polynomial of 
    the matrix $F$ are equal to $f$. Moreover, the roots of $f$ are the eigenvalues of $F$. Since 
    all the roots of $f$ are distinct, all the eigenvalues of $F$ are different. Thus 
    $F$ is diagonalizable, so there exists $P\in\GL_n(\C)$ such that $F=PDP^{-1}$, where
    $D$ is the $n\times n$ diagonal matrix with diagonal 
    entries $\alpha=\alpha_1,\alpha_2,\dots,\alpha_n$, 
    the roots of $f$ (i.e., the conjugates of $\alpha$), 
    so all with absolute value at most one. Thus 
    $\|D\|_\infty\leq 1$. 
    Since $0\not\in\{\alpha_1,\dots,\alpha_n\}$, the matrix
    $F$ is invertible. Moreover, 
    \[
    F^k=(PDP^{-1})^k=PD^kP^{-1}
    \]
    for all $k\geq1$. Note that the set 
    $X=\{F^k:k\geq 1\}\subseteq M_n(\Z)$ is bounded in $M_n(\C)$, 
    as 
    \[
    \|F^k\|_\infty=\|PD^kP^{-1}\|_\infty\leq 
    \|P\|_\infty\|D\|_\infty^k\| P^{-1}\|_\infty
    \leq \underbrace{\|P\|_\infty\| P^{-1}\|_\infty}_{\text{This is independent of $k$}}.
    \]
    Thus $X$ is finite. In particular, there are integers $i<j$ such that 
    $F^i=F^j$. Since $F$ is invertible, $F^{j-i}$ is the $n\times n$ 
    identity matrix. Since $\alpha$ is an eigenvalue of $F$, it follows
    that $\alpha^{j-i}=1$.  
    % Let $\{e_1,\dots,e_n\}$ be the standard basis of $\C^{n\times1}$. 
    % A direct calculation shows that $Fe_{j}=e_{j+1}$ for all $j\in\{2,\dots,n-1\}$ and 
    % \begin{align*}
    %     Fe_n&=-a_0e_1-a_1e_2-\cdots-a_{n-1}e_n.
    % \end{align*} 
\end{proof}

The proof of the theorem presented here goes back to Greiter~\cite{MR514044}. 
Kronecker’s original proof is somewhat similar, relying on 
Vieta’s formulas and estimates involving binomial coefficients; see~\cite{MR1834706}.


\subsection{Solvable groups and Burnside's theorem}

\index{Derived series}
For a group $G$ let 
$G^{(0)}=G$ and 
$G^{(i+1)}=[G^{(i)},G^{(i)}]$ for $i\geq0$.
The \emph{derived series} of $G$ is the sequence
\[
G=G^{(0)}\supseteq G^{(1)}\supseteq G^{(2)}\supseteq\cdots
\]
Each $G^{(i)}$ is a characteristic subgroup of $G$. We say that 
$G$ is \emph{solvable} if $G^{(n)}=\{1\}$ for some $n$.  

\begin{example}
	Abelian groups are solvable. 
\end{example}

\begin{example}
	The group $\SL_2(3)$ is solvable. 
	Let us see what the computer says:
\begin{lstlisting}
> G := SL(2,3);;
> IsSolvable(G);
true
> [GroupName(x) : x in DerivedSeries(G)];
[ SL(2,3), Q8, C2, C1 ]
\end{lstlisting}
\end{example}

\begin{example}
	Non-abelian simple groups cannot be solvable. 
\end{example}

For $n\geq5$, the group $\Alt_n$ is not solvable.

\begin{exercise}
	\label{xca:solvable}
	Let $G$ be a group. Prove the following statements:
	\begin{enumerate}
		\item A subgroup $H$ of $G$ is solvable, when $G$ is solvable.
		\item Let $K$ be a normal subgroup of $G$. 
		    Then $G$ is solvable if and only if $K$ and $G/K$ are solvable.
	\end{enumerate}
\end{exercise}

For $n\geq5$, the group $\Sym_5$ is not solvable. 

\begin{exercise}
\label{xca:pgroups_solvable}
	Let $p$ be a prime number. Prove that 
	finite $p$-groups are solvable.
\end{exercise}

Exercises~\ref{xca:solvable} and~\ref{xca:pgroups_solvable} may be omitted if the reader is already familiar with solvable groups.

\begin{theorem}[Burnside]
	\index{Burnside theorem}
	\label{thm:Burnside_auxiliar}
	Let $G$ be a finite group. If $\phi\colon G\to\GL_n(\C)$ is a representation
	with character $\chi$ and $C$ is a conjugacy class of $G$ such that 
	$\gcd(|C|,n)=1$, then for every $g\in C$ either 
	$\chi(g)=0$ or $\phi_g$ is a scalar matrix. 
\end{theorem}

% We need a lemma.

% \begin{lemma}
% 	\label{lem:4Burnside}
% 	Let $\epsilon_1,\dots,\epsilon_n$ be roots of one such that 
% 	$(\epsilon_1+\cdots+\epsilon_n)/n\in\A$. Then either 
% 	$\epsilon_1=\cdots=\epsilon_n$ or 
% 	$\epsilon_1+\cdots+\epsilon_n=0$.
% \end{lemma}

% \begin{proof}
% 	Let $\alpha=(\epsilon_1+\cdots+\epsilon_n)/n$.
% 	If the $\epsilon_j$s are not all equal, then $\|\alpha\|<1$. Moreover, 
% 	$\|\beta\|<1$ for every algebraic conjugate $\beta$ of $\alpha$. Since the product 
% 	of the algebraic conjugates of $\alpha$ is an integer of absolute value 
% 	$<1$, it follows that it is zero. 
% \end{proof}

Now we prove the theorem.

\begin{proof}[Proof of Theorem \ref{thm:Burnside_auxiliar}]
	Let $\epsilon_1,\dots,\epsilon_n$ be the eigenvalues of $\phi_g$. Then 
    $\epsilon_1,\dots,\epsilon_n$ are roots of one. 
    By assumption, 
	$\gcd(|C|,n)=1$, there exist $a,b\in\Z$ such that $a|C|+bn=1$. Since 
	$|C|\chi(g)/n\in\A$, after multiplying by $\chi(g)/n$ we obtain that  
	\[
		a|C|\frac{\chi(g)}{n}+b\chi(g)=\frac{\chi(g)}{n}=\frac{1}{n}(\epsilon_1+\cdots+\epsilon_n)\in\A.
	\]
    Let $\alpha_1=\chi(g)/n\in\A$ and $\alpha_2,\dots,\alpha_n$ be its conjugates. Since $|\alpha_1|\leq 1$ 
    and $\alpha_2,\dots,\alpha_n$ are conjugates of $\alpha_1$, it follows that  
    $|\alpha_j|\leq 1$ for all $j\in\{1,\dots,n\}$. By Kronecker's theorem, 
    either $\alpha_1=0$ or $\alpha_1$ is a root of one. If $\alpha_1=0$, then $\chi(g)=0$. If 
    $\alpha_1$ is a root of one, then 
    \[
    1=|\alpha_1|=\frac{|\chi(g)|}{n}=\frac1{n}.
    \]
    Thus $|\chi(g)|=n=\chi(1)$. This means that $g\in\Z(\chi)$. By Exercise~\ref{xca:center}, 
    $\phi_g$ is a scalar matrix. 
	% The previous lemma implies that there are two cases to consider: 
	% either $\epsilon_1=\cdots=\epsilon_n$ or $\epsilon_1+\cdots+\epsilon_n=0$. In the first
	% case, since $\phi_g$ is diagonalizable, $\phi_g$ is a scalar matrix. 
	% In the second case, $\chi(g)=0$.
\end{proof}

\begin{theorem}[Burnside]
	\index{Burnside theorem}
    \label{thm:pq_notsimple}
	Let $p$ be a prime number. If $G$ is a finite group and 
	$C$ is a conjugacy class of $G$ with $p^k>1$ elements, then $G$ 
	is not simple.
\end{theorem}

\begin{proof}
	Let $g\in C\setminus\{1\}$. Column orthogonality implies that 
	\begin{equation}
	\label{eq:Burnside}
	\begin{aligned}
		0&=\sum_{\chi\in\Irr(G)}\chi(1)\chi(g)\\
		&=\sum_{p\mid\chi(1)}\chi(1)\chi(g)+\sum_{p\nmid\chi(1):\chi\ne\chi_1}\chi(1)\chi(g)+1,
	\end{aligned}
	\end{equation}
	where the one corresponds to the trivial representation of
	$G$.
	
	Look at this equation modulo $p$. If $\chi(g)=0$ for all
	$\chi\in\Irr(G)$
	such that $\chi\ne\chi_1$ and $p\nmid\chi(1)$, then
	\[
	-\frac{1}{p}=\sum\frac{\chi(1)}{p}\chi(g)\in\A\cap\Q=\Z,
	\]
	where the sum is taken over all non-trivial irreducibles
	of $G$ of degree divisible by $p$, a contradiction. Hence 
	there exists an irreducible non-trivial representation 
	$\phi$ with character $\chi$ such that $p$ does not divide
	$\chi(1)$ and $\chi(g)\ne0$. By the previous theorem, 
	$\phi_g$ is a scalar matrix. If $\phi$ is faithful, then 
	$g$ is a non-trivial central element, a contradiction since 
    $|C|>1$. If $\phi$ is not faithful, then 
    $G$ is not simple (because 
	$\ker\phi$ is a non-trivial proper normal subgroup of $G$). 
\end{proof}

\begin{theorem}[Burnside]
  \index{Burnside $p^aq^b$-theorem}
  \label{thm:pq}
  Let $p$ and $q$ be prime numbers. If $G$ has order $p^aq^b$, then $G$ is solvable.
\end{theorem}

\begin{proof}
	If $G$ is abelian, then it is solvable.
	Suppose now $G$ is non-abelian.
	Let us assume that the theorem is not true. Let $G$ be a group
	of minimal order $p^aq^b$
	that is not solvable. Since $|G|$ is minimal, $G$ is a non-abelian simple group.
	By the previous theorem, 
	$G$ has no conjugacy classes of size $p^k$ nor 
	conjugacy classes of size $q^l$ with $k,l\geq1$. The size
	of every conjugacy class of $G$ is one or divisible by $pq$. 
	Note that, since $G$ is a non-abelian simple group,
	the center of $G$ is trivial.
	Thus there is only one conjugacy class of size one.
	By the class
	equation,
	\[
		|G|=1+\sum_{C:|C|>1}|C|\equiv 1 \bmod pq,
	\]
	where the sum is taken over all conjugacy classes of $G$ 
	with more than one element, a contradiction.
\end{proof}

\subsection{Some generalizations of Burnside's theorem}

If the reader does not know what nilpotent groups are, this section can be safely omitted.

\begin{theorem}[Kegel--Wielandt]
    \index{Kegel--Wielandt theorem}
    \label{thm:KegelWielandt}
    If $G$ is a finite group and there are nilpotent subgroups 
    $A$ and $B$ of $G$ such that 
    $G=AB$, then $G$ is solvable.
\end{theorem}

See~\cite[Theorem 2.4.3]{MR1211633} for the proof.

\begin{exercise}
    Prove that Theorem~\ref{thm:KegelWielandt} 
    implies Theorem~\ref{thm:pq}.
\end{exercise}

Another generalization of Burnside's theorem
is based on \emph{word maps}. A word map
of a group $G$ is a map 
\[
G^k\to G,\quad 
(x_1,\dots,x_k)\mapsto w(x_1,\dots,x_k)
\]
for some word $w(x_1,\dots,x_k)$ of the free group $F_k$ of rank $k$. 
Some word maps are surjective in certain families of groups. For example, 
Ore's conjecture is precisely the surjectivity of the word map
$(x,y)\mapsto [x,y]=xyx^{-1}y^{-1}$ in every finite non-abelian simple 
group. 

\begin{theorem}[Guralnick--Liebeck--O'Brien--Shalev--Tiep]
    Let $a,b\geq0$, $p$ and $q$ be prime numbers and $N=p^aq^b$. The map 
    $(x,y)\mapsto x^Ny^N$ is surjective in every finite simple group. 
\end{theorem}

The proof appears in~\cite{MR3827208}. 

The theorem implies Burnside's theorem. Let $G$ be a group of order
$N=p^aq^b$. Assume that $G$ is not solvable. 
Fix a composition series of $G$. There is a non-abelian factor $S$ 
of order that divides $N$. Since 
$S$ is simple non-abelian and $s^N=1$, it follows that the word map
$(x,y)\mapsto x^Ny^N$ has trivial image in $S$, a contradiction 
to the theorem. 

\subsection{The Feit--Thompson theorem}

\begin{theorem}[Feit--Thompson]
    \index{Feit--Thompson!theorem}
    Groups of odd order are solvable. 
\end{theorem}

The proof of Feit--Thompson theorem is extremely hard. 
It occupies a full volume of the 
\emph{Pacific Journal of Mathematics}~\cite{MR166261}. 
A formal verification of the proof 
(based on the computer software Coq) 
was announced in~\cite{MR3111271}.  

Back in the day it was believed that if a certain divisibility 
conjecture is true, 
the proof of Feit--Thompson theorem 
could be simplified. 

\begin{conjecture}[Feit--Thompson]
\index{Feit--Thompson!conjecture}
    There are no prime numbers $p$ and $q$ such that
    $\frac {p^{q}-1}{p-1}$ divides $\frac{q^{p} - 1}{q - 1}$. 
\end{conjecture}

The conjecture remains open. However, now we know that 
proving the conjecture will not simplify further
the proof of Feit--Thompson theorem. 

In 2012 Le proved that the conjecture is true for $q=3$, see 
\cite{MR2900154}. 


In~\cite{MR297686} 
Stephens proved that a certain stronger version of the conjecture 
does not hold, as the integers 
$\frac {p^{q}-1}{p-1}$ and $\frac{q^{p} - 1}{q - 1}$ 
could have common factors. In fact, if $p=17$ and $q=3313$, 
then 
\[
\gcd\left(\frac {p^{q}-1}{p-1},\frac{q^{p} - 1}{q - 1}\right)=112643.
\]
Nowadays we can check this easily in almost every desktop computer:
% \begin{lstlisting}
% gap> Gcd((17^3313-1)/16,(3313^17-1)/3312);
% 112643
% \end{lstlisting}
\begin{lstlisting}
> p := 17; 
> q := 3313;
> bool, a := IsCoercible(Integers(), (p^q-1)/(p-1));
> bool, b := IsCoercible(Integers(), (q^p-1)/(q-1));
> Gcd(a,b);
112643    
\end{lstlisting}
No other counterexamples have been found of Stephen’s stronger version of the conjecture.



\subsection{The character table of \texorpdfstring{$\Alt_5$}{A5}}

\index{Character table!of $\Alt_5$}
Let $G=\Alt_5$. 
The group $G$ is a non-abelian simple group of order 60. It has five conjugacy classes, namely

\bigskip 
\begin{center}
    \begin{tabular}{c|ccccc}
         Representative & $\id$  & $(12)(34)$ & $(123)$  & $(12345)$ & $(12354)$\\
         \hline 
         Size & $1$ & $15$ & $20$ & $12$ & $12$ \\
    \end{tabular}
\end{center}
\bigskip 

One can easily get the conjugacy classes of 
$\Alt_5$ with Magma:
\begin{lstlisting}
> A5 := Alt(5);
> ConjugacyClasses(A5);
Conjugacy Classes of group A5
-----------------------------
[1]     Order 1       Length 1
        Id(A5)

[2]     Order 2       Length 15
        (1, 2)(3, 4)

[3]     Order 3       Length 20
        (1, 2, 3)

[4]     Order 5       Length 12
        (1, 2, 3, 4, 5)

[5]     Order 5       Length 12
        (1, 3, 4, 5, 2)    
\end{lstlisting}

Let us see how to obtain all conjugacy classes
of $\Alt_5$ without computers. Let $\sigma\in\Alt_5$ and $C$ be its
conjugacy class in $\Sym_5$. Thus $|C|=(\Sym_5:C_{\Sym_5}(\sigma))$. There are two cases to consider

Assume first that $C_{\Sym_5}(\sigma)\not\subseteq\Alt_5$. Since $\Alt_5$ is a maximal subgroup of $\Sym_5$, it follows that 
$\Alt_5C_{\Sym_5}(\sigma)=\Sym_5$. Using the isomorphism theorems, 
\[
\Sym_5/\Alt_5=\Alt_5C_{\Sym_5(\sigma)}/\Alt_5
\simeq C_{\Sym_5}(\sigma)/(C_{\Sym_5}(\sigma)\cap\Alt_5)
=C_{\Sym_5}(\sigma)/C_{\Alt_5}(\sigma).
\]
Hence 
\[
(\Alt_5:C_{\Alt_5}(\sigma))=\frac{(\Sym_5:C_{\Alt_5}(\sigma))}{(\Sym_5:\Alt_5)}
=\frac{(\Sym_5:C_{\Alt_5}(\sigma))}{(C_{\Sym_5}(\sigma):C_{\Alt_5}(\sigma))}
=(\Sym_5:C_{\Sym_5}(\sigma))=|C|.
\]
Therefore $C$ is the class of $\sigma$ in $\Alt_5$. 

Assume now that $C_{\Sym_5}(\sigma)\subseteq\Alt_5$. Then 
$C_{\Alt_5}(\sigma)=C_{\Sym_5}(\sigma)\cap\Alt_5=C_{\Sym_5}(\sigma)$
and therefore 
\[
(\Alt_5:C_{\Alt_5}(\sigma))=(\Alt_5:C_{\Sym_5}(\sigma))
=\frac12(\Sym_5:C_{\Sym_5}(\sigma))=\frac12|C|.
\]
Thus $C$ splits into two conjugacy classes of $\Alt_5$ of equal size. 

The identity permutation is central. The even permutations 
$(12)(34)$ and $(123)$ both commutes with some odd permutation in $\Sym_5$ (e.g. 
$[(12)(34),(34)]=[(123),(45)]=\id$). Thus these classes do not split
in $\Alt_5$. There are twenty-four 5-cycles in $\Sym_5$. Since $24$ does not
divide $|\Alt_5|=60$, it follows that the class of 5-cycles
splits in $\Alt_5$. As representatives of these classes
we can take $(12345)$ and $(12354)$. 

Since $\Alt_5$ has five conjugacy classes, $|\Irr(G)|=5$. Assume that 
\[
\Irr(G)=\{\chi_1,\chi_2,\chi_3,\chi_4,\chi_5\}, 
\]
where $\chi_1$ is the trivial character. 

Let $H=\Alt_4$. We compute $\Ind_H^G\chi_1$. By Corollary~\ref{cor:reciprocity}, 
\[
\left(\Ind_H^G\chi_1\right)(\id) = 5.
\]
And a direct calculation shows
\begin{align*}
    &\left(\Ind_H^G\chi_1\right)((12)(34)) = 1,\\
    &\left(\Ind_H^G\chi_1\right)((123)) = 2,\\
    &\left(\Ind_H^G\chi_1\right)((12345)) = 0\\ 
    &\left(\Ind_H^G\chi_1\right)((12354)) = 0.
\end{align*}

Now, using Frobenius' reciprocity and the fact that 
$\Res_H^G\chi_1$ is the trivial character of $H$, 
\begin{align*}
    \langle\Ind_H^G\chi_1,\chi_1\rangle = \langle\chi_1,\Res_H^G\chi_1\rangle=1.
\end{align*}

Let $\chi_2=\Ind_H^G\chi_1-\chi_1$. Since 
\[
\langle\Ind_H^G\chi_1-\chi_1,\Ind_H^G\chi_1-\chi_1\rangle=1, 
\]
it follows that $\chi_2\in\Irr(G)$. 

\begin{exercise}
\label{xca:A5_chi2}
    Use Proposition~\ref{pro:2transitive} to derive (once again) the values of $\chi_2$.
\end{exercise}

So far we have the 
following table: 

\bigskip 
\begin{center}
        \begin{tabular}{|c|ccccc|}
        \hline  
         & $\id$ & $(12)(34)$ & $(123)$ & $(12345)$ & $(12354)$\\
        \hline 
        $\chi_1$ & $1$ & $1$ & $1$ & $1$ & $1$\\
        $\chi_2$ & $4$ & $0$ & $1$ & $-1$ & $-1$\\
        $\chi_3$ & $n_3$ & $\cdot$ & $\cdot$ & $\cdot$& $\cdot$\\
        $\chi_4$ & $n_4$ & $\cdot$ & $\cdot$ & $\cdot$& $\cdot$\\
        $\chi_5$ & $n_5$ & $\cdot$ & $\cdot$ & $\cdot$& $\cdot$\\
        \hline 
    \end{tabular}
\end{center}
\bigskip 

As $G$ is simple non-abelian, 
$|G/[G,G]|=1$. It follows that
$\chi_1$ is the only linear character of $G$. Moreover, 
$\chi_j(1)\geq3$ by Theorem~\ref{thm:simple}. Since 
\[
60=1+16+n_3^2+n_4^2+n_5^2
\]
and each $n_j$ divides $|G|=60$ 
(see Theorem \ref{thm:Frobenius_chi(1)}), it follows that 
$n_j\in\{3,4,5,6\}$. If some $n_j=6$, say without
loss of generality $n_3=6$, then 
\[
7=43-36=n_2^2+n_3^2, 
\]
a contradiction. Thus $n_j\in\{3,4,5\}$ for 
all $j\in\{3,4,5\}$. Without loss of generality, 
we may assume that $n_3=n_4=3$ and $n_5=5$. 

\bigskip 
\begin{center}
        \begin{tabular}{|c|ccccc|}
        \hline  
         & $\id$ & $(12)(34)$ & $(123)$ & $(12345)$ & $(12354)$\\
        \hline 
        $\chi_1$ & $1$ & $1$ & $1$ & $1$ & $1$\\
        $\chi_2$ & $4$ & $0$ & $1$ & $-1$ & $-1$\\
        $\chi_3$ & $3$ & $\cdot$ & $\cdot$ & $\cdot$& $\cdot$\\
        $\chi_4$ & $3$ & $\cdot$ & $\cdot$ & $\cdot$& $\cdot$\\
        $\chi_5$ & $5$ & $\cdot$ & $\cdot$ & $\cdot$& $\cdot$\\
        \hline 
    \end{tabular}
\end{center}
\bigskip 

The group $\Alt_5$ acts on the set $Y$ of subsets 
of $\{1,2,\dots,5\}$ of two elements, namely
\[
g\cdot \{a,b\}=\{g\cdot a,g\cdot b\}.
\]
Note that $|Y|=\binom{5}{2}=10$. Moreover, 
this action is transitive. Let us compute 
the character $\psi$ of the corresponding 
$\C\Alt_5$-module and the difference 
$\psi-\chi_1$ (We know $\psi$ counts
fixed points.)

\bigskip 
\begin{center}
        \begin{tabular}{|c|ccccc|}
        \hline  
         & $\id$ & $(12)(34)$ & $(123)$ & $(12345)$ & $(12354)$\\
        \hline 
        $\psi$ & $10$ & $2$ & $1$ & $0$ & $0$\\
        $\psi-\chi_1$ & $9$ & $1$ & $0$ & $-1$ & $-1$\\
        \hline 
    \end{tabular}
\end{center}
\bigskip 

The identity, of course, fixes all the ten elements
of $Y$. The permutation 
$(12)(34)$ fixed two two-elements subsets, namely
$\{1,2\}$ and $\{3,4\}$. The permutation 
$(123)$ fixes only one two-elements subset, namely
$\{4,5\}$. Finally, $(12345)$ and 
$(12354)$ fix no two-element subsets. 

Now we compute 
\[
\langle \psi-\chi_1,\psi-\chi_1\rangle=2
\]
and hence $\psi-\chi_1$ is the sum of two irreducible
characters (see Exercise~\ref{xca:n_irreducible}). Since
\[
\langle \psi-\chi_1,\chi_2\rangle=1,
\]
it follows that $\psi-\chi_1-\chi_2\Irr(G)$. Let 
$\chi_5=\psi-\chi_1-\chi_2$. Then

\bigskip 
\begin{center}
        \begin{tabular}{|c|ccccc|}
        \hline  
         & $\id$ & $(12)(34)$ & $(123)$ & $(12345)$ & $(12354)$\\
        \hline 
        $\chi_1$ & $1$ & $1$ & $1$ & $1$ & $1$\\
        $\chi_2$ & $4$ & $0$ & $1$ & $-1$ & $-1$\\
        $\chi_3$ & $3$ & $\cdot$ & $\cdot$ & $\cdot$& $\cdot$\\
        $\chi_4$ & $3$ & $\cdot$ & $\cdot$ & $\cdot$& $\cdot$\\
        $\chi_5$ & $5$ & $1$ & $-1$ & $0$& $0$\\
        \hline 
    \end{tabular}
\end{center}
\bigskip 

Let $K=\langle(12345)\rangle$ and 
$\eta\in\Irr(K)$ be such that $\eta((12345))=\zeta$, where
$\zeta=\exp(2\pi i/5)$ is a primitive $5$-th root of one. We can then compute 
$\Ind_K^G$. 

\bigskip 
\begin{center}
        \begin{tabular}{|c|ccccc|}
        \hline  
         & $\id$ & $(12)(34)$ & $(123)$ & $(12345)$ & $(12354)$\\
         \hline 
         $\Ind_K^G\psi$ & $12$ & $0$ & $0$ & $\zeta^2+\zeta^3$ & $\zeta+\zeta^4$\\
         \hline 
\end{tabular}
\end{center}
\bigskip 

Since 
$\langle\Ind_K^G\psi,\chi_2\rangle=1=\langle\Ind_H^G,\chi_5\rangle$,
it follows that 
\bigskip 
\begin{center}
        \begin{tabular}{|c|ccccc|}
        \hline  
         & $\id$ & $(12)(34)$ & $(123)$ & $(12345)$ & $(12354)$\\
         \hline 
         $\Ind_K^G\psi-\chi_2-\chi_5$ & $3$ & $-1$ & $0$ & $-\zeta-\zeta^4$ & $-\zeta^2-\zeta^3$\\
         \hline 
\end{tabular}
\end{center}
\bigskip 
Let $\chi_3=\Ind_K^G\psi-\chi_2-\chi_5$. Then $\chi_3\in\Irr(G)$, because it is
not the sum of three copies of the trivial character. 
\bigskip
\begin{center}
        \begin{tabular}{|c|ccccc|}
        \hline  
         & $\id$ & $(12)(34)$ & $(123)$ & $(12345)$ & $(12354)$\\
        \hline 
        $\chi_1$ & $1$ & $1$ & $1$ & $1$ & $1$\\
        $\chi_2$ & $4$ & $0$ & $1$ & $-1$ & $-1$\\
        $\chi_3$ & $3$ & $-1$ & $0$ & $-\zeta-\zeta^4$ & $-\zeta^2-\zeta^3$\\
        $\chi_4$ & $3$ & $\cdot$ & $\cdot$ & $\cdot$& $\cdot$\\
        $\chi_5$ & $5$ & $1$ & $-1$ & $0$& $0$\\
        \hline 
    \end{tabular}
\end{center}
\bigskip 

\begin{exercise}
    Use the orthogonality relations
    to compute the missing row of the character table
    of $\Alt_5$. 
\end{exercise}

The previous exercise finishes the calculation
of the character table of $\Alt_5$; see Table~\ref{tab:A5}. 


\begin{table}[h]
\caption{The character table of $\Alt_5$.}
\label{tab:A5}
        \begin{tabular}{|c|ccccc|}
        \hline  
        & $1$ & $15$ & $20$ & $12$ & $12$ \\
         & $\id$ & $(12)(34)$ & $(123)$ & $(12345)$ & $(12354)$\\
        \hline 
        $\chi_1$ & $1$ & $1$ & $1$ & $1$ & $1$\\
        $\chi_2$ & $4$ & $0$ & $1$ & $-1$ & $-1$\\
        $\chi_3$ & $3$ & $-1$ & $0$ & $-\zeta-\zeta^4$ & $-\zeta^2-\zeta^3$\\
        $\chi_4$ & $3$ &  $-1$ & $0$ & $-\zeta^2-\zeta^3$ & $-\zeta-\zeta^4$ \\
        $\chi_5$ & $5$ & $1$ & $-1$ & $0$& $0$\\
        \hline 
    \end{tabular}
\end{table}

One last observation: 
Since $\zeta=\exp(2\pi i/5)$, it follows
that 
\[
-\zeta-\zeta^4=\frac{1-\sqrt{5}}{2},
\quad 
-\zeta^2-\zeta^3=\frac{1+\sqrt{5}}{2}.
\]

% Let us see what Magma says:

% \begin{lstlisting}
% > CharacterTable(Alt(5));


% Character Table
% ---------------


% ---------------------------
% Class |   1  2  3    4    5
% Size  |   1 15 20   12   12
% Order |   1  2  3    5    5
% ---------------------------
% p  =  2   1  1  3    5    4
% p  =  3   1  2  1    5    4
% p  =  5   1  2  3    1    1
% ---------------------------
% X.1   +   1  1  1    1    1
% X.2   +   3 -1  0   Z1 Z1#2
% X.3   +   3 -1  0 Z1#2   Z1
% X.4   +   4  0  1   -1   -1
% X.5   +   5  1 -1    0    0


% Explanation of Character Value Symbols
% --------------------------------------

% # denotes algebraic conjugation, that is,
% #k indicates replacing the root of unity w by w^k

% Z1     = (CyclotomicField(5: Sparse := true)) ! [ RationalField() | 0, 0, -1, -1 ]    
% \end{lstlisting}
%\chapter{}

\topic{Lie algebras}

\begin{definition}
    \index{Lie algebra}
    Let $K$ be a field. 
    A \textbf{Lie algebra} (over $K$) is a $K$-vector space
    $L$ together with a bilinear map 
    $L\times L\to L$, $(x,y)\mapsto [x,y]$,
    such that
    \begin{align}
        \label{eq:[xx]=0}&[x,x]=0\quad\text{for all $x\in L$},\\ 
        \label{eq:Jacobi}&[x,[y,z]]+[y,[z,x]]+[z,[x,y]]=0\quad\text{for all $x,y,z\in L$}.
    \end{align}
\end{definition}

\index{Jacobi identity}
Equality \eqref{eq:Jacobi} is known as the \textbf{Jacobi identity}. 

\begin{exercise}
    Prove that \eqref{eq:[xx]=0} implies $[x,y]=-[y,x]$ for all
    $x,y\in L$. 
\end{exercise}

\index{Abelian Lie algebra}
A Lie algebra $L$ is said to be \textbf{abelian} if $[x,y]=0$ for 
all $x,y\in L$. 

\begin{exercise}
    If $L$ and $L_1$ are Lie algebras, then 
    $L\oplus L_1$ is a Lie algebra with
    $[(x,x_1),(y,y_1)]=([x,y),(x_1,y_1)]$ for $x,y\in L$ and
    $x_1,y_1\in L_1$. 
\end{exercise}

\begin{exercise}
    Prove that $\R^3$ with the usual vector product 
    \[
    [(x_1,x_2,x_3),(y_1,y_2,y_3)]=(x_2y_3-x_3y_2,x_3y_1-x_1y_3,x_1y_2-x_2y_1)
    \]
    is a (real) Lie algebra.     
\end{exercise}

We will main work with finite-dimensional complex Lie algebras.

\begin{example}[general linear Lie algebra]
\index{General linear Lie algebra}
    Let $V$ be a finite-dimensional vector space and 
    $\gl(V)$ be the set of linear maps $V\to V$. Then 
    $\gl(V)$ with $[x,y]=xy-yx$ is a Lie algebra. 
\end{example}

A matrix version of the previous example: We write $\gl(n,\C)$ 
to denote the vector space of all $n\times n$ complex 
matrices with Lie bracket $[x,y]=xy-yx$. The vector space
$\gl(n,\C)$ has a basis $\{e_{ij}:1\leq i,j\leq n\}$, where
\[
(e_{ij})_{kl}=\begin{cases} 
    1 & \text{if $(i,j)=(k,l)$},\\
    0 & \text{otherwise}.
    \end{cases}
\]

\begin{exercise}
    Compute $[e_{ij},e_{ik}]$.
\end{exercise}

\begin{example}[special linear Lie algebra]
\index{Special linear Lie algebra}
    Let $\sl(n,\C)$ be the subspace of $\gl(n,\C)$ consisting
    of all matrices with trace zero. 
\end{example}

\begin{exercise}
    Find a basis of $\sl(n,\C)$. 
\end{exercise}

We discuss a particular important case,
\[
\sl(2,\C)=\left\{\begin{pmatrix}
    a & b\\
    c & -a
    \end{pmatrix}:a,b,c\in\C\right\}
\]
Note that 
$e=\begin{pmatrix}
        0&1\\
        0&0\end{pmatrix}$, $h=\begin{pmatrix}
        1&0\\
        0&-1\end{pmatrix}$ and $f=\begin{pmatrix}0&0\\1&0\end{pmatrix}$ 
        is an ordered basis for $\sl(2,\C)$. In this basis,
\[
[h,e]=2e,\quad
[h.f]=-2f,\quad
[e,f]=h.
\]

%\begin{example}
%    Let $\bl(n,\C)$ be the subspace of all upper triangular matrices
%    in $\gl(n,\C)$. Then $\bl(n,\C)$ is a Lie algebra. 
%\end{example}

\begin{definition}
    \index{Lie!subalgebra}
    A Lie \textbf{subalgebra} of $L$ is a vector space $L_1$ of $L$ 
    such that $[x,y]\in L_1$ for all $x,y\in L_1$. 
\end{definition}

Of course, $\sl(n,\C)$ is a subalgebra of $\gl(n,\C)$. 

\begin{definition}
\index{Ideal!of a Lie algebra}
    An \textbf{ideal} of a Lie algebra $L$ is a subspace $I$ of $L$ 
    such that $[x,y]\in I$ for all $x\in L$ and $y\in I$. 
\end{definition}

Trivial examples of ideals of a Lie algebra $L$ are
$\{0\}$ and $L$.

\begin{example}
\index{Center!of a Lie algebra}
    Let $L$ be a Lie algebra. Then 
    the \textbf{center} 
    \[
    Z(L)=\{x\in L:[x,y]=0\text{ for all $y\in L$}\}.
    \]
    is an ideal of $L$. 
\end{example}

\begin{example}
\index{Derived algebra!of a Lie algebra}
    Let $L$ be a Lie algebra. 
    The \textbf{derived algebra} $[L,L]$
    consists of all linear combinations of commutators $[x,y]$ 
    is an ideal of $L$. 
\end{example}

\begin{exercise}
    Compute $Z(\sl(n,\C))$. 
\end{exercise}

\begin{exercise}
    Prove that $\sl(2,\C)$ has no non-trivial ideals. 
\end{exercise}

One easily checks that $\sl(n,\C)$ is an ideal of $\gl(n,\C)$. In fact, 
an ideal is always a subalgebra. The converse is not true. 
Can you find an example?

\begin{definition}
\index{Homomorphism!of Lie algebras}
    Let $L$ and $L_1$ be Lie algebras. A map $f\colon L\to L_1$ is a 
    \textbf{Lie algebra homomorphism} if $f([x,y])=[f(x),f(y)]$ for all
    $x,y\in L$. 
\end{definition}

As usual, an isomorphism between Lie algebras will be
a bijective homomorphism of Lie algebras. 

\begin{example}
    Let $L$ and $L_1$ be Lie algebras. The canonical injections
    $L\to L\oplus L_1$ and $L_1\to L_\oplus L_1$ and
    the canonical surjections $L\oplus L_1\to L$ and 
    $L\oplus L_1\to L_1$ are Lie algebras homomorphisms.  
\end{example}

\begin{example}
    Let $L$ be a Lie algebra. The \textbf{opposite Lie algebra} 
    $L^{\op}$ is the vector space $L$ with 
    $[x,y]^{\op}=-[x,y]$. Then $L\to L^{\op}$, $x\mapsto -x$, 
    is an isomorphism of Lie algebras.
\end{example}

\begin{exercise}
    Let $f\colon L\to L_1$ be a Lie algebra homomorphism. Prove
    that the \textbf{kernel} of $f$, 
    $\ker f=\{x\in L:f(x)=0\}$ is an ideal
    of $L$, and that the \textbf{image} of $f$ 
    is a subalgebra of $L_1$. 
\end{exercise}

\begin{example}
\index{Adjoint homomorphism}
    Let $L$ be a Lie algebra. 
    The \textbf{adjoint homomorphism} is the map 
    \[
    \ad\colon L\to\gl(L),\quad
    (\ad x)(y)=[x,y].
    \]
\end{example}

Let $L$ be a Lie algebra and $I$ be an ideal of $L$. Then 
the quotient vector space $L/I$ is a Lie algebra
with $[x+I,y+I]=[x,y]+I$. The canonical map 
$L\to L/I$, $x\mapsto x+I$, 
is a surjective Lie algebra homomorphism. 

\begin{exercise}
    Let $f\colon L\to L_1$ be a Lie algebra homomorphism.
    Prove that $f/\ker f\simeq f(L)$. 
\end{exercise}

\begin{definition}
\index{Simple Lie algebra}
    A Lie algebra $L$ is said to be \textbf{simple} if 
    $[L,L]\ne\{0\}$ and $\{0\}$ and $L$ are the only ideals of $L$. 
\end{definition}

If $L$ is a simple Lie algebra, then $Z(L)=\{0\}$ and $L=[L,L]$. 

\begin{exercise}
    Prove that every simple Lie algebra is isomorphic to 
    a linear Lie algebra. 
\end{exercise}

\topic{Representations of Lie algebras}

\begin{definition}
\index{Representation!of a Lie algebra}
    A \textbf{representation} of a Lie algebra $L$ 
    is a Lie homomorphism $\rho\colon L\to\gl(V)$, where $V$ is a vector space. 
\end{definition}

If $L\to\gl(V)$ is a representation of a Lie algebra $L$, 
fixing a basis for $V$ 
we obtain a \textbf{matrix representation}
$L\to\gl(n,\C)$. 

\begin{example}
    Let $L$ be a Lie algebra. 
    The map $\ad\colon L\to\gl(L)$, $x\mapsto (\ad x)$, is a lie 
    homomorphism. 
\end{example}

\begin{definition}
Let $L$ be a Lie algebra. 
A (left) Lie $L$-module is a vector space $V$ 
together with a map $L\times V\to V$, $(x,v)\mapsto xv$, 
such that $(x,v)\mapsto xv$ is bilinear 
and 
\[
    [x,y]v=x(yv)-y(xv)
\]
for all $x,y\in L$ and $v\in V$. 
\end{definition}

As it happens in the case of groups, Lie modules are
in bijective correspondence with representations. 

\begin{example}
    Let $L$ be a subalgebra of $\gl(V)$. Then 
    $L$ is an $L$-module. 
\end{example}

\begin{definition}
    Let $L$ be a Lie algebra and $V$ be a Lie $L$-module. 
    A \textbf{submodule} of $V$ is a subspace $W$ 
    such that $xw\in W$ for all $x\in L$ and $w\in W$. 
\end{definition}

\begin{example}
We know that $L$ is an $L$-module 
with the adjoint representation. The submodules of $L$ are
the ideals of $L$. 
\end{example}

If $W$ is a submodule of $V$, then $V/W$ 
with $x(v+W)=xv+W$ is a module. 

\begin{definition}
    Let $L$ be a Lie algebra. An $L$-module $V$ 
    is said to be \textbf{simple} (or irreducible) 
    if $V\ne \{0\}$ and it has no submodules other than $\{0\}$ and $V$. 
\end{definition}

One-dimensional modules are simple. In particular, 
the trivial module is always simple. 

\begin{example}
    Let $L$ be a simple Lie algebra (e.g. $\sl(2,\C)$). Then 
    the adjoint representation is irreducible, that is $L$ is a simple $L$-module.  
\end{example}

\begin{definition}
    Let $L$ be a Lie algebra and $V$ be an $L$-module. We say 
    that $V$ is \textbf{indecomposable} if 
    there are no non-zero submodules $U$ and $W$ such that
    $V=U\oplus W$.
\end{definition}

Clearly, irreducible modules are indecomposable. 
The converse is not true.

\begin{definition}
    Let $L$ be a Lie algebra and $V$ be an $L$-module. We say that
    $V$ is \textbf{completely reducible} if $V=S_1\oplus\cdots\oplus S_k$
    for simple modules $S_1,\dots,S_k$. 
\end{definition}

\begin{exercise}
    Let $\mathfrak{b}(n,\C)$ be the set of $n\times n$ 
    upper triangular 
    matrices in $\gl(n,\C)$. Prove that $V=\C^n$ is indecomposable,  
    not irreducible. 
\end{exercise}

\begin{definition}
    Let $L$ be a Lie algebra and $f\colon V\to W$ be a map. 
    We say that $f$ is an $L$-module \textbf{homomorphism} if
    $f(xv)=xf(v)$ for all $x\in L$ and $v\in V$. 
\end{definition}

As usual, an isomorphism is a bijective module homomorphism. 

\begin{exercise}
    State and prove the isomorphism theorems for modules
    over Lie algebras. 
\end{exercise}

\begin{exercise}[Schur lemma]
    Let $L$ be a Lie algebra.
    \begin{enumerate}
        \item Let $S$ and $T$ be simple $L$-modules.
            Prove that a non-zero homomorphism $f\colon S\to T$ 
            is an isomorphism.
        \item Let $S$ be a finite-dimensional simple $L$-module. 
            Prove that if $f\colon S\to S$ is a homomorphism, then
                $f=\lambda\id$ for some $\lambda\in\C$. 
    \end{enumerate} 
\end{exercise}

As an example, if $V$ is a simple module, then $z$
acts by scalar multiplication on $V$, that is
$zv=\lambda v$ for some $\lambda\in\C$. 

\topic{Representations of $\sl(2,\C)$}

Consider the polynomial ring $\C[X,Y]$ in two commuting variables
$X$ and $Y$. Let $V_d$ be the subspace of homogeneous polynomials
of degree $d$. Then 
\[
\dim V_d=\begin{cases}
    1 & \text{if $d=0$},\\
    d+1 & \text{otherwise},
    \end{cases}
\]
as a basis of $V_d$ is given 
by $\{X^d,X^{d-1}Y,X^{d-2}Y^2,\dots,XY^{d-1},Y^d\}$. 

\begin{exercise}
    Prove that $\varphi\colon\sl(2,\C)\to\gl(V_d)$, 
    \begin{align}
        \varphi(e)=X\frac{\partial}{\partial Y},
        &&
        \varphi(f)=Y\frac{\partial}{\partial X},
        &&
        \varphi(h)=X\frac{\partial}{\partial X}-Y\frac{\partial}{\partial Y},
    \end{align}
    is a representation of $\sl(2,\C)$. 
    This means that 
    \begin{gather*}
    \varphi(e)(X^aY^b)=bX^{a+1}Y^{b-1},
    \quad
    \varphi(f)(X^aY^b)=aX^{a-1}Y^{b+1},
    \shortintertext{and that}
    \varphi(h)(X^aY^b)=(a-b)X^aY^b.
    \end{gather*}
\end{exercise}

In the basis $\{X^d,X^{d-1}Y,X^{d-2}Y^2,\dots,XY^{d-1},Y^d\}$, 
\begin{align*}
\varphi(e)=\left(\begin{smallmatrix}
0 & 1 & 0 & \cdots & 0\\
0 & 0 & 2 & \cdots & 0\\
\vdots & \vdots & \vdots & \ddots & \vdots\\
0 & 0 & 0 & \cdots & d\\
0 & 0 & 0 & \cdots & 0
\end{smallmatrix}\right),
&& 
\varphi(f)=\left(\begin{smallmatrix}
0 & 0 & \cdots & 0 & 0\\
d & 0 & \cdots & 0 & 0\\
0 & d-1 & \cdots & 0 & 0\\
\vdots & \vdots & \ddots & \vdots & \vdots\\
0 & 0 & \cdots & 1 & 0
\end{smallmatrix}\right),
&&
\varphi(h)=\left(\begin{smallmatrix}
d & 0 & \cdots & 0 & 0\\
0 & d-2 & \cdots & 0 & 0\\
\vdots & \vdots & \ddots & \vdots & \vdots\\
0 & 0 & \cdots & -d+2 & 0\\
0 & 0 & \cdots & 0 & -d
\end{smallmatrix}\right).
\end{align*}

\begin{exercise}
    Prove that $V_d$ is generated (as an $\sl(2,\C)$-module) by
    $X^aY^b$ for some $a$ and $b$ such that $a+b=d$. 
\end{exercise}

The following exercise is important:

\begin{exercise}
    Prove that each $V_d$ is a simple $\sl(2,\C)$-module.
\end{exercise}

Now we prove one of the main results of this section. 

\begin{theorem}
\label{thm:irreducibles_sl2}
    Let $V$ be a finite-dimensional 
    simple $\sl(2,\C)$-module. Then $V\simeq V_d$ for some $d$. 
\end{theorem}

We use the notation $e^2v=e(ev)$.

We need some lemmas. 

\begin{lemma}
    Let $V$ be an $\sl(2,\C)$-module and $v\in V$ be 
    an eigenvector of $h$ with eigenvalue $\lambda$. 
    \begin{enumerate}
        \item Either $ev=0$ or $ev$ is an eigenvector of $h$ with
            eigenvalue $\lambda+2$.
        \item Either $fv=0$ or $fv$ is an eigenvector of $h$ with
            eigenvalue $\lambda-2$.
    \end{enumerate} 
\end{lemma}

\begin{proof}
    We only prove 1): $h(ev)=e(hv)+[h,e]v=e(\lambda v)+2ev=(\lambda+2)ev$.
\end{proof}

\begin{lemma}
    Let $V$ be a finite-dimensional $\sl(2,\C)$-module.
    There exists an eigenvector 
    $w\in V$ of $h$ such that $ew=0$. 
\end{lemma}

\begin{proof}
    The linear map $h\colon V\to V$ has at least one eigenvector 
    $v$ with eigenvalue $\lambda$. If the elements 
    $v,ev,e^2v,\dots$ are non-zero, they are linearly independent, as they 
    form a sequence of eigenvectors of $h$ with different eigenvalues.  
    As $\dim V<\infty$, it follows that there exists $k$ 
    such that $e^kv\ne 0$ and $e^{k+1}v=0$. Let $w=e^kv\ne 0$. 
    Then
    $hw=(\lambda+2k)w$ and $ew=0$. 
\end{proof}

Now we prove the theorem.

\begin{proof}[Proof of Theorem \ref{thm:irreducibles_sl2}]
    By the previous lemma, there exists an eigenvector $w$ 
    of $h$ of eigenvalue $\lambda$ such that $ew=0$. Since $V$ is finite-dimensional, 
    after considering
    the sequence $w,fw,f^2w\dots$ we find that there
    exists $d\geq0$ such that 
    $f^dw\ne 0$ and $f^{d+1}w=0$. 
    
    \begin{claim}
        $\{w,fw,\dots,f^dw\}$ is a basis of a submodule of $V$.
    \end{claim}
    
    The elements are linearly independent, as they are eigenvectors of $h$ with 
    different eigenvalues. The subspace $W=\langle w,fw,\dots,f^dw\rangle$ 
    is invariant under $h$ and $f$. Let us prove that $W$ is invariant 
    under $e$, that is $eW\subseteq W$. We need to prove that
    $e(f^kw)\in W$ for all $k$. We proceed by induction on $k$. 
    The case $k=0$ is trivial. 
    If the claim holds for some $k$, then $ef^kw\in W$ by 
    the inductive hypothesis. Thus 
    \[
    e(f^{k+1}w)=ef(f^{k}w)=(fe+h)f^{k}w\in W,
    \]
    as $hf^kw\in W$. 
    
    \begin{claim}
        $\lambda=d$. 
    \end{claim}
    
    The matrix of $h$ with respect to $\{w,fw,\dots,f^dw\}$ 
    is diagonal with trace 
    \[
    \lambda+(\lambda-2)+\cdots+(\lambda-2d)=(d+1)(\lambda-d).
    \]
    Since $[e,f]=h$ has trace zero, it follows that $\lambda=d$. 
    
    \begin{claim}
        $V\simeq V_d$.
    \end{claim}
    
    The vector spaces are isomorphic, as $V$ has basis $\{w,fw,\dots,f^dw\}$ and 
    $V_d$ has basis $\{X^d,fX^d,\dots,f^dX^d\}$, where
    $f^kX^d\in\C X^{d-k}Y^k$. The eigenvalues of $h$ on $f^kw$ 
    are the same as the eigenvalues of $h$ on $f^kX^d$. Let 
    \[
    \varphi\colon V\to V_d,\quad
    f^kw\mapsto f^kX^d.
    \]
    This bijective linear map commutes with the action of $h$ and $f$. 
    It also satisfies $\varphi(ew)=...$
    and 
    \[
    \varphi(hf^kw)=...
    \varphi(ef^kw)=...
    \]
\end{proof}

We now summarize our results in terms of \textbf{highest weight vectors}
and \textbf{highest weights}.

\begin{corollary}
    Let $V$ be a finite-dimensional $\sl(2,\C)$-module and $w\in V$ 
    (a highest vector of $V$)
    be an eigenvector of $h$ 
    such that $ew=0$. Then $hw=dw$ for some
    non-negative integer $d$ (a highest weight). 
    Moreover, the submodule
    of $V$ generated by $w$ is isomorphic to $V_d$. 
\end{corollary}

\begin{proof}
    
\end{proof}

The following result is a particular
case of Weyl's theorem in the context of $\sl(2,\C)$-modules. 

\begin{theorem}
    Any finite-dimensional $\sl(2,\C)$-module is absolutely reducible. 
\end{theorem}

\begin{proof}
    Let $V$ be a finite-dimensional $\sl(2,\C)$-module. 
    We proceed in several steps.
    
    \begin{claim}
        The elmenet $Z=\frac12h^2+h+2fe$ commutes with every $X\in\sl(2,\C)$. 
    \end{claim}
    
    We first compute
    \begin{equation}
        \begin{aligned}
            \label{eq:Casimir}
            ZX-XZ &= \frac12h^2X-\frac12Xh^2+[h,X]+2feX-2Xfe\\
            &=\frac12h[e,X]-\frac12[X,h]h+[h,X]+2f[e,X]-2[X,f]e.
        \end{aligned}
    \end{equation}
    Now one checks that 
    Equation \eqref{eq:Casimir} is zero if $X\in\{h,e,f\}$ and
    the claim follows. 
    %See \cite[Theorem V.4.6]{MR1321145} or \cite[Theorem 1.67]{MR1920389}. 
    
    \begin{claim}
        If $\dim V=n+1$, then $Z$ acts as the scalar 
        $\frac12n^2+n$, which is not zero unless $V$ is the trivial module. 
    \end{claim}
    
    By Schur's lemma, $Z$ acts by a scalar. Since $V$ is a simple
    $\sl(2,\C)$-module of dimension $n+1$, $V\simeq V_{n+1}$. In particular, 
    $hv_0=nv_0$ and $ev_0=0$. 
    
    \begin{claim}
        Let $U\subseteq V$ be a submodule of codimension one. Then 
        there exists a submodule $W$ of $V$ such that $V=U\oplus W$ 
        and $\dim W=1$. 
    \end{claim}
    
    We split the proof of the claim into several steps. 
    
    First we assume that
    $\dim U=1$. The quotient module $V/U$ is one-dimensional and hence simple. 
    Thus $\sl(2,\C)V\subseteq U$ and $\sl(2,\C)U=\{0\}$. Hence 
    \[
    [X,Y]V\subseteq XYV-YXV\subseteq XU+YU=\{0\}.
    \]
    Since $\sl(2,\C)=[\sl(2,\C),\sl(2,\C)]$, we conclude that
    $\sl(2,\C)V=\{0\}$. Thus any complement of $U$ will serve as $W$. 
    
    \medskip
    We now finish the proof of the theorem. 
\end{proof}

\topic{Enveloping algebras}


%\section{Lie algebras}

\begin{definition}
    \index{Lie algebra}
    Let $K$ be a field. 
    A (complex) \emph{Lie algebra} is a complex vector space
    $L$ together with a bilinear map 
    $L\times L\to L$, $(x,y)\mapsto [x,y]$,
    such that
    \begin{align}
        \label{eq:[xx]=0}&[x,x]=0\quad\text{for all $x\in L$},\\ 
        \label{eq:Jacobi}&[x,[y,z]]+[y,[z,x]]+[z,[x,y]]=0\quad\text{for all $x,y,z\in L$}.
    \end{align}
\end{definition}

\index{Jacobi identity}
Equality \eqref{eq:Jacobi} is known as the \emph{Jacobi identity}. 

\begin{exercise}
    Prove that \eqref{eq:[xx]=0} implies $[x,y]=-[y,x]$ for all
    $x,y\in L$. 
\end{exercise}

\index{Abelian Lie algebra}
A Lie algebra $L$ is said to be \emph{abelian} if $[x,y]=0$ for 
all $x,y\in L$. 

\begin{exercise}
    If $L$ and $L_1$ are Lie algebras, then 
    $L\oplus L_1$ is a Lie algebra with
    $[(x,x_1),(y,y_1)]=([x,y),(x_1,y_1)]$ for $x,y\in L$ and
    $x_1,y_1\in L_1$. 
\end{exercise}

\begin{exercise}
    Prove that $\R^3$ with the usual vector product 
    \[
    [(x_1,x_2,x_3),(y_1,y_2,y_3)]=(x_2y_3-x_3y_2,x_3y_1-x_1y_3,x_1y_2-x_2y_1)
    \]
    is a (real) Lie algebra.     
\end{exercise}

We will mainly work with finite-dimensional complex Lie algebras.

\begin{example}[general linear Lie algebra]
\index{General linear Lie algebra}
    Let $V$ be a finite-dimensional vector space and 
    $\gl(V)$ be the set of linear maps $V\to V$. Then 
    $\gl(V)$ with $[x,y]=xy-yx$ is a Lie algebra. 
\end{example}

A matrix version of the previous example: We write $\gl(n,\C)$ 
to denote the vector space of all $n\times n$ complex 
matrices with Lie bracket $[x,y]=xy-yx$. The vector space
$\gl(n,\C)$ has a basis $\{e_{ij}:1\leq i,j\leq n\}$, where
\[
(e_{ij})_{kl}=\begin{cases} 
    1 & \text{if $(i,j)=(k,l)$},\\
    0 & \text{otherwise}.
    \end{cases}
\]

\begin{exercise}
    Compute $[e_{ij},e_{ik}]$.
\end{exercise}

\begin{example}[special linear Lie algebra]
\index{Special linear Lie algebra}
    Let $\sl(n,\C)$ be the subspace of $\gl(n,\C)$ consisting
    of all matrices with trace zero. 
\end{example}

\begin{exercise}
    Find a basis of $\sl(n,\C)$. 
\end{exercise}

%\begin{example}
%    Let $\bl(n,\C)$ be the subspace of all upper triangular matrices
%    in $\gl(n,\C)$. Then $\bl(n,\C)$ is a Lie algebra. 
%\end{example}

\begin{definition}
    \index{Lie!subalgebra}
    A Lie \emph{subalgebra} of $L$ is a vector space $L_1$ of $L$ 
    such that $[x,y]\in L_1$ for all $x,y\in L_1$. 
\end{definition}

Of course, $\sl(n,\C)$ is a subalgebra of $\gl(n,\C)$. 

\begin{definition}
\index{Ideal!of a Lie algebra}
    An \emph{ideal} of a Lie algebra $L$ is a subspace $I$ of $L$ 
    such that $[x,y]\in I$ for all $x\in L$ and $y\in I$. 
\end{definition}

Trivial examples of ideals of a Lie algebra $L$ are
$\{0\}$ and $L$.

\begin{example}
\index{Center!of a Lie algebra}
    Let $L$ be a Lie algebra. Then 
    the \emph{center} 
    \[
    Z(L)=\{x\in L:[x,y]=0\text{ for all $y\in L$}\}.
    \]
    is an ideal of $L$. 
\end{example}

\begin{example}
\index{Derived algebra!of a Lie algebra}
    Let $L$ be a Lie algebra. 
    The \emph{derived algebra} $[L,L]$
    consists of all linear combinations of commutators $[x,y]$ 
    is an ideal of $L$. 
\end{example}

\begin{exercise}
    Compute $Z(\sl(n,\C))$. 
\end{exercise}

\begin{exercise}
    Prove that $\sl(2,\C)$ has no non-trivial ideals. 
\end{exercise}

One easily checks that $\sl(n,\C)$ is an ideal of $\gl(n,\C)$. In fact, 
an ideal is always a subalgebra. The converse is not true. 
Can you find an example?

\begin{definition}
\index{Homomorphism!of Lie algebras}
    Let $L$ and $L_1$ be Lie algebras. A map $f\colon L\to L_1$ is a 
    \emph{Lie algebra homomorphism} if $f([x,y])=[f(x),f(y)]$ for all
    $x,y\in L$. 
\end{definition}

As usual, an isomorphism between Lie algebras will be
a bijective homomorphism of Lie algebras. 

\begin{example}
    Let $L$ and $L_1$ be Lie algebras. The canonical injections
    $L\to L\oplus L_1$ and $L_1\to L_\oplus L_1$ and
    the canonical surjections $L\oplus L_1\to L$ and 
    $L\oplus L_1\to L_1$ are Lie algebras homomorphisms.  
\end{example}

\begin{example}
    Let $L$ be a Lie algebra. The \emph{opposite Lie algebra} 
    $L^{\op}$ is the vector space $L$ with 
    $[x,y]^{\op}=-[x,y]$. Then $L\to L^{\op}$, $x\mapsto -x$, 
    is an isomorphism of Lie algebras.
\end{example}

\begin{exercise}
    Let $f\colon L\to L_1$ be a Lie algebra homomorphism. Prove
    that the \emph{kernel} of $f$, 
    $\ker f=\{x\in L:f(x)=0\}$ is an ideal
    of $L$, and that the \emph{image} of $f$ 
    is a subalgebra of $L_1$. 
\end{exercise}

\begin{example}
\index{Adjoint homomorphism}
    Let $L$ be a Lie algebra. 
    The \emph{adjoint homomorphism} is the map 
    \[
    \ad\colon L\to\gl(L),\quad
    (\ad x)(y)=[x,y].
    \]
\end{example}

Let $L$ be a Lie algebra and $I$ be an ideal of $L$. Then 
the quotient vector space $L/I$ is a Lie algebra
with $[x+I,y+I]=[x,y]+I$. The canonical map 
$L\to L/I$, $x\mapsto x+I$, 
is a surjective Lie algebra homomorphism. 

\begin{exercise}
    Let $f\colon L\to L_1$ be a Lie algebra homomorphism.
    Prove that $f/\ker f\simeq f(L)$. 
\end{exercise}

\begin{definition}
\index{Simple Lie algebra}
    A Lie algebra $L$ is said to be \emph{simple} if 
    $[L,L]\ne\{0\}$ and $\{0\}$ and $L$ are the only ideals of $L$. 
\end{definition}

If $L$ is a simple Lie algebra, then $Z(L)=\{0\}$ and $L=[L,L]$. 

\begin{exercise}
    Prove that every simple Lie algebra is isomorphic to 
    a linear Lie algebra. 
\end{exercise}

\subsection{Representations of Lie algebras}

\begin{definition}
\index{Representation!of a Lie algebra}
    A \emph{representation} of a Lie algebra $L$ 
    is a Lie homomorphism $\rho\colon L\to\gl(V)$, where $V$ is a vector space. 
\end{definition}

If $L\to\gl(V)$ is a representation of a Lie algebra $L$, 
fixing a basis for $V$ 
we obtain a \emph{matrix representation}
$L\to\gl(n,\C)$. 

\begin{example}
    Let $L$ be a Lie algebra. 
    The map $\ad\colon L\to\gl(L)$, $x\mapsto (\ad x)$, is a lie 
    homomorphism. 
\end{example}

\begin{definition}
Let $L$ be a Lie algebra. 
A (left) Lie $L$-module is a vector space $V$ 
together with a map $L\times V\to V$, $(x,v)\mapsto xv$, 
such that $(x,v)\mapsto xv$ is bilinear 
and 
\[
    [x,y]v=x(yv)-y(xv)
\]
for all $x,y\in L$ and $v\in V$. 
\end{definition}

As it happens in the case of groups, Lie modules are
in bijective correspondence with representations. 

\begin{example}
    Let $L$ be a subalgebra of $\gl(V)$. Then 
    $L$ is an $L$-module. 
\end{example}

\begin{definition}
    Let $L$ be a Lie algebra and $V$ be a Lie $L$-module. 
    A \emph{submodule} of $V$ is a subspace $W$ 
    such that $xw\in W$ for all $x\in L$ and $w\in W$. 
\end{definition}

\begin{example}
We know that $L$ is an $L$-module 
with the adjoint representation. The submodules of $L$ are
the ideals of $L$. 
\end{example}

If $W$ is a submodule of $V$, then $V/W$ 
with $x(v+W)=xv+W$ is a module. 

\begin{definition}
    Let $L$ be a Lie algebra. An $L$-module $V$ 
    is said to be \emph{simple} (or irreducible) 
    if $V\ne \{0\}$ and it has no submodules other than $\{0\}$ and $V$. 
\end{definition}

One-dimensional modules are simple. In particular, 
the trivial module is always simple. 

\begin{example}
    Let $L$ be a simple Lie algebra (e.g. $\sl(2,\C)$). Then 
    the adjoint representation is irreducible, that is $L$ is a simple $L$-module.  
\end{example}

\begin{definition}
    Let $L$ be a Lie algebra and $V$ be an $L$-module. We say 
    that $V$ is \emph{indecomposable} if 
    there are no non-zero submodules $U$ and $W$ such that
    $V=U\oplus W$.
\end{definition}

Clearly, irreducible modules are indecomposable. 
The converse is not true.


\begin{exercise}
    Let $\mathfrak{b}(n,\C)$ be the set of $n\times n$ 
    upper triangular 
    matrices in $\gl(n,\C)$. Prove that $V=\C^n$ is indecomposable,  
    not irreducible. 
\end{exercise}

\begin{definition}
\index{Homomorphism!of modules over Lie algebras}
    Let $L$ be a Lie algebra and $f\colon V\to W$ be a map. 
    We say that $f$ is an $L$-module \emph{homomorphism} if
    $f(xv)=xf(v)$ for all $x\in L$ and $v\in V$. 
\end{definition}

As usual, an isomorphism is a bijective module homomorphism. 

\begin{exercise}
    State and prove the isomorphism theorems for modules
    over Lie algebras. 
\end{exercise}

\begin{exercise}[Schur's lemma]
\index{Schur's lemma for Lie algebras}
    Let $L$ be a Lie algebra.
    \begin{enumerate}
        \item Let $S$ and $T$ be simple $L$-modules.
            Prove that a non-zero homomorphism $f\colon S\to T$ 
            is an isomorphism.
        \item Let $S$ be a finite-dimensional simple $L$-module. 
            Prove that if $f\colon S\to S$ is a homomorphism, then
                $f=\lambda\id$ for some $\lambda\in\C$. 
    \end{enumerate} 
\end{exercise}

As an example, if $V$ is a simple module, then $z$
acts by scalar multiplication on $V$, that is
$zv=\lambda v$ for some $\lambda\in\C$. 

%\section{Representations of $\sl(2,\C)$}

We discuss a particularly essential Lie algebra: 
\[
\sl(2,\C)=\left\{\begin{pmatrix}
    a & b\\
    c & -a
    \end{pmatrix}:a,b,c\in\C\right\}.
\]
Note that 
$e=\begin{pmatrix}
        0&1\\
        0&0\end{pmatrix}$, $h=\begin{pmatrix}
        1&0\\
        0&-1\end{pmatrix}$ and $f=\begin{pmatrix}0&0\\1&0\end{pmatrix}$ 
        is an ordered basis for $\sl(2,\C)$. In this basis,
\[
[h,e]=2e,\quad
[h.f]=-2f,\quad
[e,f]=h.
\]

Let $V$ be an $\sl(2,\C)$-module. Note that we do not assume that $V$ is simple. 
For $\lambda\in\C$ an eigenvalue of the action of $h$, let
\[
V_{\lambda}=\{v\in V:h\cdot v=\lambda v\}
\]
be the \emph{weight space} of $\lambda$. 
If $\lambda$ is not an eigenvector of $h$, 
we set $V_{\lambda}=\{0\}$. A \emph{weight} of $V$  
is a scalar $\lambda$ such that $V_{\lambda}\ne\{0\}$. 

\begin{lemma}
    Let $V$ be an $\sl(2,\C)$-module and $v\in V_{\lambda}$. 
    \begin{enumerate}
        \item Either $e\cdot v=0$ or $e\cdot v$ is an eigenvector of $h$ with
            eigenvalue $\lambda+2$.
        \item Either $f\cdot v=0$ or $f\cdot v$ is an eigenvector of $h$ with
            eigenvalue $\lambda-2$.
    \end{enumerate} 
\end{lemma}

\begin{proof}
    We only prove 1):
    \[h\cdot(e\cdot v)=e\cdot(h\cdot v)+[h,e]\cdot v=e\cdot(\lambda v)+2e\cdot v=(\lambda+2)e\cdot v.
    \]
    The second formula is left as an exercise. 
\end{proof}

\begin{lemma}
\label{lem:maximal_weight}
    Let $V$ be a finite-dimensional $\sl(2,\C)$-module.
    There exists an eigenvector 
    $w\in V$ of $h$ such that $e\cdot w=0$. 
\end{lemma}

\begin{proof}
    The linear map induced by the action of $h$ has at least one eigenvector 
    $v$ with eigenvalue $\lambda$. If the elements 
    $v,e\cdot v,e^2\cdot v,\dots$ are non-zero, they are linearly independent, as they 
    form a sequence of eigenvectors of $h$ with different eigenvalues.  
    As $\dim V<\infty$, it follows that there exists $k$ 
    such that $e^k\cdot v\ne 0$ and $e^{k+1}\cdot v=0$. 
    Let $w=e^k\cdot v\ne 0$. 
    Then
    $h\cdot w=(\lambda+2k)\cdot w$ and $e\cdot w=0$. 
\end{proof}

A vector $v\in V$ such that $V_{\lambda}\ne\{0\}$ and
$V_{\lambda+2}=\{0\}$ will be called a \emph{highest weight vector} 
of weight $\lambda$. 

\begin{lemma}
\label{lem:basis}
    Let $V$ be a finite-dimensional simple $\sl(2,\C)$-module
    and let $w$ be a highest vector of weight $\lambda$. Let 
    $k$ be such that $f^k\cdot w\ne 0$ and $f^{k+1}\cdot w=0$.
    Then $\{w,f\cdot w,\dots,f^{k}\cdot w\}$ is a basis of $V$.
    Moreover, $\lambda=k$.
\end{lemma}

\begin{proof}
    The elements $w,f\cdot w,\dots,f^{k}\cdot w$ 
    are linearly independent, as they are eigenvectors of $h$ with 
    different eigenvalues. Since $V$ is simple, it is enough 
    to prove that 
    the non-zero subspace $W=\langle w,f\cdot w,\dots,f^k\cdot w\rangle$ 
    is a submodule of $V$. This subspace 
    is invariant under the action of 
    $h$ and $f$. One easily proves by induction that 
    \[
    (hf^j)\cdot w=(\lambda-2j)f^jw
    \]
    for all $j\geq0$. 
    Let us prove that $W$ is invariant 
    under the action of $e$, that is, $e\cdot W\subseteq W$. 
    We claim that 
    \[
    (ef^{j})\cdot w=j(\lambda-j+1)f^{j-1}\cdot w\in W
    \]
    for all $j$. We proceed by induction on $j$. Note that 
    the case $j=0$ is trivial, as $e\cdot w=0$. 
    The case $j=1$ is easy:
    \[
    (ef)\cdot w=(h+fe)\cdot w=h\cdot w+f\cdot (e\cdot w)=\lambda w.
    \]
    If the claim holds for some $j$, by using 
    the inductive hypothesis, 
    \begin{align*}
        e\cdot(f^{j+1}\cdot w)&=(ef)\cdot (f^{j}\cdot w)\\
        &=(fe+h)\cdot (f^{j}\cdot w)\\
        &=h\cdot (f^j\cdot w)+j(\lambda-j+1)(f^j\cdot w)\\
        &=(j+1)(\lambda-j)(f^{j}\cdot w).
    \end{align*}
    
    We now compute $\lambda$. 
    The matrix of the action of $h$ in the basis 
    $\{w,f\cdot w,\dots,f^k\cdot w\}$ 
    is diagonal with trace 
    \[
    \lambda+(\lambda-2)+\cdots+(\lambda-2k)=(k+1)(\lambda-k).
    \]
    Since $[e,f]=h$ has trace zero, it follows that $\lambda=k$. 
\end{proof}

We now summarize what we know about 
simple $\sl(2,\C)$-modules.

\begin{theorem}
    Let $V$ be a finite-dimensional simple $\sl(2,\C)$-module.
    Then 
    $V$ is the direct sum of one-dimensional weight spaces
    \begin{equation}
    \label{eq:sl2_decomposition}
        V=V_{\lambda}\oplus V_{\lambda-2}\oplus\cdots\oplus V_{-\lambda+2}\oplus V_{-\lambda}.
    \end{equation}
    In particular, $\dim V=\lambda+1$. 
    The set $\{w,f\cdot w,\dots,f^\lambda\cdot w\}$ 
    is a basis of $V$ and 
    \begin{equation}
        \label{eq:sl2_module}
        \begin{aligned}
            &h\cdot (f^j\cdot w)=(\lambda-2j)f^j\cdot w,\\
            &f\cdot (f^j\cdot w)=f^{j+1}\cdot w,\\
            &e\cdot (f^j\cdot w)=j(\lambda-j+1)f^{j-1}\cdot w.
        \end{aligned}
    \end{equation}
\end{theorem}

\begin{proof}
    By Lemma \ref{lem:maximal_weight}, there exists an eigenvector $w$ of $h$ 
    with eigenvalue $\lambda$ such that $e\cdot w=0$. By Lemma \ref{lem:basis}, 
    the set 
    $\{w,f\cdot w,\dots,f^k\cdot w\}$ is a basis of $V$ and
    the formulas of \eqref{eq:sl2_module} follow. For $j\in\{0,\dots,k\}$
    the complex vector space generated by $f^j\cdot w$ 
    is a one-dimensional weight space of weight $\lambda-2j$. Thus \eqref{eq:sl2_decomposition}
    follows. 
\end{proof}

In particular, there exists at most one simple $\sl(2,(\C)$-module (up to isomorphism) 
of each possible dimension $\lambda+1$ for $\lambda+1$. 

\begin{exercise}
    Prove that two 
    finite-dimensional $\sl(2,\C)$ generated by
    highest weight vectors of the same weight are isomorphic. 
\end{exercise}

Consider the polynomial ring $\C[X,Y]$ in two commuting variables
$X$ and $Y$. Let $V_d$ be the subspace of homogeneous polynomials
of degree $d$. Then $\dim V_d=d+1$ 
as a basis of $V_d$ is given 
by $\{X^d,X^{d-1}Y,X^{d-2}Y^2,\dots,XY^{d-1},Y^d\}$. 

\begin{exercise}
    Prove that $\varphi\colon\sl(2,\C)\to\gl(V_d)$, 
    \begin{align}
        \varphi(e)=X\frac{\partial}{\partial Y},
        &&
        \varphi(f)=Y\frac{\partial}{\partial X},
        &&
        \varphi(h)=X\frac{\partial}{\partial X}-Y\frac{\partial}{\partial Y},
    \end{align}
    is a representation of $\sl(2,\C)$. 
    This means that 
    \begin{gather*}
    \varphi(e)(X^aY^b)=bX^{a+1}Y^{b-1},
    \quad
    \varphi(f)(X^aY^b)=aX^{a-1}Y^{b+1},
    \shortintertext{and that}
    \varphi(h)(X^aY^b)=(a-b)X^aY^b.
    \end{gather*}
\end{exercise}

In the basis $\{X^d,X^{d-1}Y,X^{d-2}Y^2,\dots,XY^{d-1},Y^d\}$, 
\begin{align*}
\varphi(e)=\left(\begin{smallmatrix}
0 & 1 & 0 & \cdots & 0\\
0 & 0 & 2 & \cdots & 0\\
\vdots & \vdots & \vdots & \ddots & \vdots\\
0 & 0 & 0 & \cdots & d\\
0 & 0 & 0 & \cdots & 0
\end{smallmatrix}\right),
&& 
\varphi(f)=\left(\begin{smallmatrix}
0 & 0 & \cdots & 0 & 0\\
d & 0 & \cdots & 0 & 0\\
0 & d-1 & \cdots & 0 & 0\\
\vdots & \vdots & \ddots & \vdots & \vdots\\
0 & 0 & \cdots & 1 & 0
\end{smallmatrix}\right),
&&
\varphi(h)=\left(\begin{smallmatrix}
d & 0 & \cdots & 0 & 0\\
0 & d-2 & \cdots & 0 & 0\\
\vdots & \vdots & \ddots & \vdots & \vdots\\
0 & 0 & \cdots & -d+2 & 0\\
0 & 0 & \cdots & 0 & -d
\end{smallmatrix}\right).
\end{align*}

The previous lemma allows us to describe the actions of 
$e$, $f$ and $h$ on $V$ with a very nice picture:
\[
\begin{tikzcd}
\{0\} 
& V_{-\lambda} \arrow[out=90,in=120,loop]\arrow[l,bend left,"f"]\arrow[r,bend left,"e"]
& V_{-\lambda+2} \arrow[out=90,in=120,loop]\arrow[l,bend left,"f"]\arrow[r,bend left,"e"]
& \cdots\arrow[l,bend left,"f"]\arrow[r,bend left,"e"]
& V_{\lambda-2} \arrow[out=90,in=120,loop]\arrow[l,bend left,"f"]\arrow[r,bend left,"e"]
& V_{\lambda} \arrow[out=90,in=120,loop]\arrow[r,bend left,"e"]\arrow[l,bend left,"f"]
& \{0\}
\end{tikzcd}
\]

\begin{exercise}
    Prove that $V_d$ is generated (as an $\sl(2,\C)$-module) by
    $X^aY^b$ for some $a$ and $b$ such that $a+b=d$. 
\end{exercise}

\begin{exercise}
    Prove that each $V_d$ is a simple $\sl(2,\C)$-module.
\end{exercise}

\begin{exercise}
    Prove that any $V$ finite-dimensional simple 
    $\sl(2,\C)$-module is isomorphic to $V_d$ for some $d$. 
\end{exercise}

\begin{definition}
\index{Completely reducible}
    Let $L$ be a Lie algebra and $V$ be an $L$-module. We say that
    $V$ is \emph{completely reducible} if $V=S_1\oplus\cdots\oplus S_k$
    for simple modules $S_1,\dots,S_k$. 
\end{definition}

The following result is a particular
case of Weyl's theorem in the context of $\sl(2,\C)$-modules. 

\begin{theorem}
    Any finite-dimensional $\sl(2,\C)$-module is absolutely reducible. 
\end{theorem}

\begin{proof}
    See \cite[Exercises 8.6 and 9.15]{MR2218355}.
\end{proof}

% \begin{proof}
%     Let $V$ be a finite-dimensional $\sl(2,\C)$-module. 
%     We proceed in several steps.
    
%     \begin{claim}
%         The elemenet $Z=\frac12h^2+h+2fe$ commutes with every $X\in\sl(2,\C)$. 
%     \end{claim}
    
%     We first compute
%     \begin{equation}
%         \begin{aligned}
%             \label{eq:Casimir}
%             ZX-XZ &= \frac12h^2X-\frac12Xh^2+[h,X]+2feX-2Xfe\\
%             &=\frac12h[e,X]-\frac12[X,h]h+[h,X]+2f[e,X]-2[X,f]e.
%         \end{aligned}
%     \end{equation}
%     Now one checks that 
%     Equation \eqref{eq:Casimir} is zero if $X\in\{h,e,f\}$ and
%     the claim follows. 
%     %See \cite[Theorem V.4.6]{MR1321145} or \cite[Theorem 1.67]{MR1920389}. 
    
%     \begin{claim}
%         If $\dim V=n+1$, then $Z$ acts as the scalar 
%         $\frac12n^2+n$, which is not zero unless $V$ is the trivial module. 
%     \end{claim}
    
%     By Schur's lemma, $Z$ acts by a scalar. Since $V$ is a simple
%     $\sl(2,\C)$-module of dimension $n+1$, $V\simeq V_{n+1}$. In particular, 
%     $hv_0=nv_0$ and $ev_0=0$. 
    
%     \begin{claim}
%         Let $U\subseteq V$ be a submodule of codimension one. Then 
%         there exists a submodule $W$ of $V$ such that $V=U\oplus W$ 
%         and $\dim W=1$. 
%     \end{claim}
    
%     We split the proof of the claim into several steps. 
    
%     First, we assume that
%     $\dim U=1$. The quotient module $V/U$ is one-dimensional and hence simple. 
%     Thus $\sl(2,\C)V\subseteq U$ and $\sl(2,\C)U=\{0\}$. Hence 
%     \[
%     [X,Y]V\subseteq XYV-YXV\subseteq XU+YU=\{0\}.
%     \]
%     Since $\sl(2,\C)=[\sl(2,\C),\sl(2,\C)]$, we conclude that
%     $\sl(2,\C)V=\{0\}$. Thus any complement of $U$ will serve as $W$. 
    
%     \medskip
%     We now finish the proof of the theorem. 
% \end{proof}

\subsection{Enveloping algebras}

Let $L$ be a finite-dimensional Lie algebra with
basis $\{x_1,\dots,x_n\}$. Write
\[
[x_i,x_j]=\sum_{k=1}^n c^k_{ij}x_k
\]
for scalars $c_{ij}^k\in\C$. 
These scalars are called 
the \emph{structure constants} of $L$. 

\index{Universal enveloping algebra}
The \emph{universal enveloping algebra} of $L$ 
is the associative algebra $U(L)$ with generators 
$x_1,\dots,x_n$ and relations 
\[
x_ix_j-x_jx_i=\sum_{k=1}^n c_{ij}^kx_k.
\]

Our definition depends on the choice of the basis of the Lie algebra $L$. 
However, it is possible to define $U(L)$ as the quotient 
of the tensor algebra $T(L)$ by the ideal $I$ 
generated by $x\otimes y-y\otimes x-[x,y]$ for all $x,y\in L$.  We refer 
to \cite[Chapter V]{MR1321145} for more details. 

\begin{example}
    If $L$ is an abelian Lie algebra, then $U(L)$ is 
    the symmetric algebra $S(L)$. 
\end{example}

\begin{example}
    The universal enveloping algebra $U(\sl(2,\C))$ 
    is the algebra with generators $e,f,h$ and relations 
    \[
    ef-fe=h,\quad
    hf-fh=-2f,\quad
    he-eh=2e.
    \]
\end{example}


The universal enveloping algebra satisfies a \emph{universal property}. 
Let $L$ be a Lie algebra. If $A$ is an associative algebra, then
$A$ has the structure of a Lie algebra with
bracket 
$[a,b]=ab-ba$. If $f\colon L\to A$ is a homomorphism of Lie algebras, 
then there exists a unique algebra homomorphism
$\varphi\colon U(L)\to A$ such that 
\[
\begin{tikzcd}
	L & A \\
	& U(L)
	\arrow["f", from=1-1, to=1-2]
	\arrow["{\iota }"', from=1-1, to=2-2]
	\arrow["\varphi"', dashed, from=2-2, to=1-2]
\end{tikzcd}
\]
commutes, where $\iota\colon L\to U(L)$ 
denotes the canonical map. 

%\begin{exercise}
%    Let $L$ and $L_1$ be Lie algebras. Prove that 
%    $U(L\oplus L_1)\simeq U(L)\otimes U(L_1)$. 
%\end{exercise}

\begin{theorem}[Poincar\'e--Birkhoff--Witt]
    \index{Poincar\'e--Birkhoff--Witt theorem}
    \index{PBW theorem}
    \label{thm:PBW}
    Let $L$ be a finite-dimensional Lie algebra and 
    $\{x_1,\dots,x_n\}$ be an ordered basis of $L$. Then 
    \[
    \{x_1^{a_1}\cdots x_n^{a_n}:a_1,\dots,a_n\geq0\}
    \]
    is a basis of $U(L)$. 
\end{theorem}

See for example \cite[\S17.4]{MR499562}. 

\begin{example}
    The set $\{e^ih^jf^k:i,j,k\geq0\}$ is a basis of $U(\sl(2,\C))$. A proof of this 
    fact that does not use Theorem \ref{thm:PBW} 
    can be done using Ore's extensions; see \cite[Proposition 3.2]{MR1321145}.
\end{example}


\begin{exercise}
    Let $L$ be a finite-dimensional Lie algebra and $U(L)$ be its
    universal enveloping algebra. Prove that there exists a bijective
    correspondence between (simple) $L$-modules and (simple) $U(L)$-modules. 
\end{exercise}

%\include{A03}
%\section{The theorems of Wedderburn and Artin--Wedderburn}

In this note, all algebras are unitary algebras over a fixed field $K$. 

A \textbf{left ideal} (resp. right ideal) 
of an algebra
$A$ is an additive subgroup $I$ of $A$ such that 
$ax\in I$ (resp $xa\in I$) 
for all $a\in A$ and $x\in I$. An \textbf{ideal} of $A$ 
is a subset that is both a left and a right ideal of $A$. 

\begin{exercise}
Prove that every left ideal $I$ of an algebra 
$A$ is a subspace of $A$ with $\lambda x=(\lambda 1_A)x$ 
for all $\lambda\in K$ and $x\in I$.
\end{exercise}

For left ideals $X$ and $Y$ of 
an algebra $A$, let $XY$ be the additive subgroup of $A$ 
generated by $\{xy:x\in X,y\in Y\}$, that is 
\[
XY=\left\{\sum_{i=1}^n:n\geq0,\,x_1,\dots,x_n\in X,\,y_1,\dots,y_n\in Y\right\},
\]
with the usual convention that the empty sum is equal to zero. 
    
\begin{exercise}
    Let $A$ be an algebra and $X$ and $Y$ be left ideals of $A$. 
    Prove the following statements:
    \begin{enumerate}
        \item $XY$ is a left ideal of $A$.
        \item $AX=X$ and $XR$ is a right ideal of $A$. 
        \item If $Z$ is a left ideal of $A$, then $(XY)Z=X(YZ)$. 
    \end{enumerate}
\end{exercise}

\begin{definition}
An algebra $A$ is said to be 
\textbf{semiprime} if $I^2\ne\{0\}$ for every non-zero ideal $I$ of~$A$.	
\end{definition}

\begin{definition}
A non-zero left ideal $I$ of $A$ is \textbf{minimal} if 
$J\subseteq I$ for some non-zero 
left ideal $J$ of $A$ implies $J=I$. 
\end{definition}

\begin{exercise}
\label{xca:minimal}
    Let $A$ be a finite-dimensional algebra and $I$ be a non-zero left
    ideal of $A$. Prove that $I$ contains a minimal left ideal of $A$. 
\end{exercise}

\begin{definition}
    An element $e$ of an algebra $A$ is said to be 
    \textbf{idempotent} if $e^2=e$. 
\end{definition}

Trivial examples of idempotents of an algebra $A$ 
are $0_A$ and $1_A$. The
matrix 
$\begin{pmatrix}1&0\\0&0\end{pmatrix}$ 
is a non-trivial idempotent of the algebra $M_2(\R)$ of real $2\times 2$ matrices. 

\begin{exercise}
    Let $e$ and $f$ be idempotents of an algebra $A$. 
    Prove that $eAe\subseteq fAf$ if and only if $ef=f=fe$.
\end{exercise}

For the proof of Wedderburn's theorem we shall need some lemmas. 

\begin{lemma}[Brauer]
\label{lem:Brauer}
Let $A$ be an algebra and $K$ be a non-zero 
minimal left ideal of $A$ such that $K^2\ne\{0\}$. 
Then $K=Ae$ for some idempotent $e\in A$ and
$eAe$ is a division algebra.
\end{lemma}

\begin{proof}
Since $K^2$ is a non-zero left ideal of $A$ and 
$K^2\subseteq K$, the minimality of $K$ implies that $K^2=K$. In particular, 
there exists $u\in K$ such that $\{0\}\ne Ku\subseteq K$. By the minimality of $K$, 
$Ku=K$. In particular, $u=eu$ for some $e\in K$. 
Let 
\[
L=\{x\in K:xu=0\}\subseteq K.
\] 
We claim that $xe-x\in L$ for all $x\in K$. In fact, 
if $x\in K$, then 
\[
(xe-x)u=x(eu)-xu=xu-xu=0.
\]
Note that $L$ is a left ideal of $A$. Moreover, $L\ne K$ (otherwise, 
if $L=K$, then $\{0\}\ne Ku=\{0\}$, a contradiction). 
Since $K$ is a minimal left ideal, 
$L=\{0\}$. Thus $xe=e$ for all $x\in K$. In particular, $e^2=e$. 

Since $Ae\subseteq AK\subseteq K$ and $Ae$ is a non-zero 
left ideal of $A$, 
the minimality of $K$ implies that $Ae=K$. 

It is an exercise to show that $eAe$ is an algebra. 

Let us prove
that every non-zero element $x$ of $eAe$ is invertible. Since  
\[
\{0\}\ne Ax\subseteq A(eAe)\subseteq (AeA)e\subseteq AK\subseteq K
\]
and $Ax$ is a left ideal of $A$, the minimality of $K$ implies that 
$Ax=K$. Thus $e=yx$ for some $y\in A$. Write $x=eae$ with $a\in A$. Then 
$ex=e(eae)=e^2ae=eae=x$ and 
$xe=x$. Now 
\[
(eye)x=(ey)(ex)=e(yx)=e^2=e.
\]
Since $0\ne eye\in eAe$, there exists $z\in eAe$ such that 
$z(eye)=e$. Thus 
\[
z=ze=z((eye)x)=(z(eye))x=ex=x
\]
and $x$ is invertible. 
\end{proof}

% Observemos que si $e$ y $g$ son idempotentes, entonces 
% \[
% eAe\subseteq gAg\Longleftrightarrow eg=g=ge.
% \]
% Si $eg=e=ge$, entonces $eAe=(ge)A(ge)\subseteq gAg$. Recíprocamente, si $eAe\subseteq gAg$, entonces
% $e=e^2\subseteq gAg$, digamos $e=gag$ para algún $a\in A$. Esto implica que 
% \[
% eg=(gag)g=gag^2=gag=e,\quad
% ge=g(gag)=g^2ag=gag=e.
% \]  

We now present Henderson's proof~\cite{MR184969}
of Wedderburn's theorem. 

\begin{theorem}[Wedderburn]
    Let $A$ be a finite-dimensional simple algebra. 
    Then 
    \[
    A\simeq M_n(D)
    \]
    for some $n\geq1$ and 
    some division algebra $D$. 
\end{theorem}

\begin{proof}
    Let $K$ be a minimal left ideal of $A$. Since $KA$ is a non-zero 
    ideal of $A$ and $A$ is simple, 
    $KA=A$. Moreover,  
	\[
	A=A^2=(KA)^2=KAKA\subseteq K^2A
	\]
	and therefore $K^2\ne\{0\}$. By Lemma~\ref{lem:Brauer}, 
    $K=Ae$ for some idempotent $e\in K$ and 
    $D=eAe$ is a division algebra. 

    We claim that the map 
    \[
    K\times D\to K,\quad (x,\delta)\mapsto x\delta, 
    \]
    turns $K$ into a right $D$-module. For example, since $K=Ae$, 
    if $x=ae\in K=Ae$, then $xe=(ae)e=ae^2=ae=x$. 
    
    We claim that the map 
    $\varphi_a\colon K\to K$, $x\mapsto ax$, is a $D$-module homomorphism. 
    \[
	\varphi_a(x\delta)=a(x\delta)=(ax)\delta=\varphi_a(x)\delta.
	\]  
    Since $\varphi_{ab}(x)=\varphi_a(\varphi_b(x))$ 
    for all 
    $a,b\in A$ and $x\in K$, the map 
    \[
    \varphi\colon A\to\End_D(K),\quad a\mapsto\varphi_a,
    \]
    is an algebra homomorphism. 

    We claim that $\varphi$ is injective. Let $a\in A$ be such that 
    $\varphi_a=0$. Then 
    \[
	0=\varphi_a(K)=aK=aAe.
	\]
    Since $A$ is simple and $AeA$ is a non-zero ideal of $A$, 
    we get $AeA=A$. Then 
    \[
    0=aAe=(aAe)(Ae)=(aA)(eAe)=aA
    \]
    and therefore $a=0$. 

    We claim that $\varphi$ is surjective. Since 
    $A=AeA$ by the simplicity of $A$,
    \[
	1=a_1eb_1+\cdots+a_neb_n
	\]
    for some $a_1,\dots,a_n,b_1,\dots,b_n\in A$. 
    We claim that for every $\alpha\in\End_D(K)$
    there exists $a\in A$ such that 
    $\alpha=\varphi_a$.  
	Let $a=\sum_{i=1}^n\alpha(a_ie)eb_i$ and $x\in A$. Since 
    \[
    x=a_1eb_1x+\cdots+a_neb_nx=\sum_{i=1}^n(a_ie)(eb_ix), 
    \]
    and $e(b_ix)e\in D$ for all $i\in\{1,\dots,n\}$, using 
    that $\alpha$ is a $D$-module homomorphism we get 
    \begin{align*}
        \alpha(xe)&=\alpha(a_1eb_1xe+\cdots+a_neb_nxe)\\
        &=\alpha(a_1e)eb_1xe+\cdots+\alpha(a_ne)eb_nxe\\
        &=\left(\alpha(a_1e)eb_1+\cdots+\alpha(a_ne)eb_n\right)xe\\
        &=\varphi_a(xe).
    \end{align*}
    Thus $A\simeq\End_D(K)$. 
    We now note that the set 
    \[
	\{\alpha\in\End_D(K):\dim\alpha(K)<\infty\}
	\]
    is a non-zero proper ideal of $\End_D(K)\simeq A$, a contradiction
    because $A$ is simple. We conclude that 
     $A\simeq\End_D(K)\simeq M_n(D)$ for some $n\geq1$. 
\end{proof}

\section{The Artin--Wedderburn theorem}

Before going into the Artin-Wedderburn theorem, 
we need some preliminaries. 

\begin{lemma}
    \label{lem:idempotents}
    Let $A$ be a finite-dimensional semiprime algebra. Every 
    non-zero left ideal contains a non-zero idempotent. 
\end{lemma}

\begin{proof}
Let $J$ be a non-zero left ideal of $A$. 
Since $A$ is finite-dimensional, there exists a non-zero 
minimal left ideal $I$ contained in $J$ (Exercise~\ref{xca:minimal})
Note that $IA$ is a non-zero ideal of $A$. 
Since $A$ 
is semiprime, $(IA)^2\ne\{0\}$. Then 
\[
(IA)^2=IAIA\subseteq I^2A
\]
and hence 
$I^2\ne\{0\}$, otherwise, $(IA)^2=\{0\}$. By Brauer's lemma, there exists an idempotent 
$e\in I$ such that $I=Ae$. 
\end{proof}

Recall that a proper (left) ideal $M$ of an algebra $A$ is said to be \textbf{maximal} 
if the only ideals $K$ such that $M\subseteq K\subseteq A$ are $K=M$ and $K=A$. Every
finite-dimensional algebra has maximal ideals. 

\begin{exercise}
\label{xca:maximal}
    Let $M$ be a maximal ideal of an algebra $A$. Prove
    that $A/M$ is a simple algebra. 
\end{exercise}

We now present Nicholson's proof of the  
Artin--Wedderburn theorem of~\cite{MR1244013}. We shall need a lemma.

\begin{lemma}
    \label{lem:ArtinWedderburn}
    Let $A$ be an algebra, $M$ be a maximal ideal of $A$ and $K$ be a non-zero ideal of $A$ 
    such that $K\cap M=\{0\}$. Then $K$ and $M$ are unitary algebras, $K$ 
    is simple and $A\simeq K\times M$. 
\end{lemma}

\begin{proof}
    Since $K+M=\{k+m:k\in K,\,m\in M\}$ is an ideal of $A$ containing $M$ and   
    $M$ is maximal, either $K+M=M$ or $K+M=A$. If $K+M=M$, then $K\subseteq K+M$ and
    hence $K=K\cap M=\{0\}$, a contradiction. Thus $K+M=A$. Let $e\in K$ and $f\in M$ be such 
    that $1=e+f$. If $k\in K$, then 
    $kf=fk=0$, as $K$ is an ideal and $K\cap M=\{0\}$. In particular, 
    \[
    k=k1=ke+kf=ke,\quad 
    k=1k=ek+fk=ek.
    \]
    for all $k\in K$. Thus $e$ is the unit of $K$. Similarly, $f$ is the unit of $M$. 

    Let $\varphi\colon K\times M\to A$, $(k,m)\mapsto k+m$. A direct calculation shows that
    $\varphi$ is an algebra isomorphism. Moreover, 
    the map $A\to K$, $k+m\mapsto k$, is a surjective algebra homomorphism 
    with kernel $M$. By the first isomorphism theorem, $A/M\simeq K$. Since $M$ is maximal, 
    $K$ is then simple. 
\end{proof}

\begin{theorem}[Artin--Wedderburn]
    Let $A$ be a semiprime finite-dimensional algebra. Then 
    \[
    A\simeq\prod_{i=1}^kM_{n_i}(D_i)
    \]
    for some $n_1,\dots,n_k\geq0$ and some 
    division algebras $D_1,\dots,D_k$. 
\end{theorem}

\begin{proof}
    We proceed by induction on $\dim A$. If $\dim A=1$, the result trivially holds, as 
    $A\simeq F$. Assume
    then that $\dim A>1$ and that the result holds for algebras
    of dimension $<\dim A$. Let $M$ be a maximal ideal of $A$ and 
    $K=\{a\in A:Ma=0\}$. Then $K$ is an ideal of $A$. 
    Moreover, $K\cap M$ is an ideal of $A$
    such that 
    \[
    (K\cap M)^2\subseteq MK=\{0\}.
    \]
    Since $A$ is semiprime, $K\cap M=\{0\}$. 
 
    Assume first that $M=\{0\}$. Then $A\simeq A/M$ is a simple 
    algebra (Exercise~\ref{xca:maximal}). By Wedderburn's theorem, 
    and $A\simeq M_n(D)$ for some division algebra $D$. 
  
    Assume now that $M\ne\{0\}$. By Lemma~\ref{lem:idempotents}, 
    $M$ contains non-zero idempotents. 
    Let $e\in M$ be a non-zero idempotent
    such that $Ae$ has maximal dimension. Since $M$ is maximal, $e\ne 1$. We claim that 
    $e-1\in K$, that is $M(1-e)=\{0\}$. If not, 
    since $M(1-e)$ is a left ideal of $A$,  
    there exists a non-zero idempotent $f\in M(1-e)$  
    by Lemma~\ref{lem:idempotents}. 
    In particular, 
    $f=m(1-e)$ for some $m\in M$ and hence 
    $fe=m(1-e)e=0$. Let $g=e+f-ef\in M$. Then
    $g$ is an idempotent:
    \begin{align*}
    g^2 & =(e+f-ef)(e+f-ef)\\
    &=e^2+ef-e^2f+fe+f^2-fef-efe-ef^2+efef\\
    &=e+f-ef\\
    &=g.
    \end{align*}
    Moreover, 
	\begin{align*}
	&eg=e(e+f-ef)=e^2+ef-e^2f=e+ef-ef=e,\\
	&ge=(e+f-ef)e=e^2+fe-efe=e,
	\end{align*}
    Thus $e=eg\in Rg$ and therefore $Re\subseteq Ag$. The maximality of  
    $\dim Ae$ implies that $Ae=Ag$. Let $a\in A$ be such that $g=ae$. 
    Then
    \[
    e=ge=(ae)e=g
    \]
    and thus $f=ef$. This implies that
    \[
    f=f^2=e(fe)f=0,
    \]
    a contradiction. Therefore $0\ne e-1\in K$ and $K$ is non-zero. 
    By Lemma~\ref{lem:ArtinWedderburn}, $K$ and $M$ are unitary 
    algebras, $K$ is simple and $A\simeq K\times M$. By Wedderburn's theorem,
    $A\simeq M_{n}(D)$ for some division algebra $D$. Since 
    $\dim M<\dim A$, the inductive hypothesis implies that 
    $M\simeq M_{n_1}(D_1)\times M_{n_k}(D_k)$ for
    division algebras $D_1,\dots,D_k$. 
\end{proof}
d
%\include{A05}
%\chapter*{Some topics for a final project}

\subsection*{Staircase groups}

This topic describes a situation similar to that of \S\ref{Kolchin}, but
more general. See \cite[Chapter 5]{MR1369573}.

\subsection*{Kegel--Wielandt's theorem}

Prove Kegel--Wielandt's theorem, see 
\cite[Theorem 2.13]{MR1211633}.

\subsection*{The Drinfeld double of a finite group}

See \cite[Chapter IX]{MR1321145} and 
\cite[Chapter 8]{MR3752618}.

\subsection*{Ito's theorem}

The theorem states that if $\chi$ is an irreducible character
of a finite group $G$, then $\chi(1)$ divides 
$(G:A)$ for every normal abelian subgroup $A$ of $G$. 
See \cite[\S8.1]{MR0450380}. 

%\chapter*{Some solutions}

\begin{sol}{xca:deg2}
  Assume that $\phi$ is not irreducible. There exists a proper non-zero $G$-invariant 
  subspace $W$ of $V$. Thus $\dim W=1$. Let $w\in W\setminus\{0\}$.
  For each $g\in G$, $\phi_g(w)\in W$. Thus $\phi_g(w)=\lambda w$ for some 
  $\lambda$. This means that $w$ is a common eigenvector for all the $\phi_g$.
  Conversely, if $\phi$ admits a common eigenvector $v\in V$, then 
  the subspace generated by $v$ is $G$-invariant.
\end{sol}

\bibliographystyle{abbrv}
\bibliography{refs}
\printindex

\end{document}

