\documentclass[12pt]{amsproc}

\newcommand{\course}{Representation theory of algebras}
\usepackage[foot]{amsaddr} 
\usepackage{hyperref}
\usepackage{listings}
\usepackage{tikz-cd}
\usepackage{datetime}
\usepackage{amssymb}
\usepackage{mathtools}
\usepackage{stmaryrd}
\usepackage{mdframed}
\usepackage{textcomp}
\usepackage{colortbl}
\usepackage[most]{tcolorbox}
%\usepackage{cleveref}

%\usepackage{fontspec}
%\setmainfont{Verdana}

\hypersetup{
  final, 
  colorlinks, 
  linkcolor=-red!55!green!50, 
  citecolor=blue!50!red,
  urlcolor=green!30!black
}

\swapnumbers

% For Dyslexic replace the next three lines by 
%\usepackage{fontspec}
%\setmainfont{OpenDyslexic}
%\usepackage{mathptmx}
%\usepackage{newtxtext}

\usepackage[margin=1in,footskip=.25in]{geometry}

\overfullrule=1mm

\renewcommand\emph[1]{\textcolor{blue!50!red}
{\bfseries #1}}

\renewcommand\thesection{\arabic{section}}
\renewcommand\thesubsection{\arabic{section}.\arabic{subsection}}


% \lstdefinelanguage{Julia}%
%   {morekeywords={abstract,break,case,catch,const,continue,do,else,elseif,%
%       end,export,false,for,function,immutable,import,importall,if,in,%
%       macro,module,otherwise,quote,return,switch,true,try,type,typealias,%
%       using,while},%
%    sensitive=true,%
%    alsoother={$},%
%    morecomment=[l]\#,%
%    morecomment=[n]{\#=}{=\#},%
%    morestring=[s]{"}{"},%
%    morestring=[m]{'}{'},%
% }[keywords,comments,strings]%

% \definecolor{background}{HTML}{F5F5F5}
% \definecolor{jlstring}{HTML}{880000}%          % julia's strings
% \definecolor{jlbase}{HTML}{444444}%            % julia's base color
% \definecolor{jlkeyword}{HTML}{444444}%         % julia's keywords
% \definecolor{jlliteral}{HTML}{78A960}%         % julia's literals
% \definecolor{jlbuiltin}{HTML}{397300}%         % julia's built-ins
% \definecolor{jlmacros}{HTML}{1F7199}%          % julia's macros
% \definecolor{jlfunctions}{HTML}{444444}%       % julia's functions
% \definecolor{jlcomment}{HTML}{888888}%         % julia's comments
% \definecolor{jlstring}{HTML}{880000}%          % julia's strings


% \lstset{%
%     language         = Julia,
%     basicstyle       = \color{jlstring}\ttfamily\scriptsize,
%     backgroundcolor  = \color{background},
%     keywordstyle     = \color{jlkeyword},
%     stringstyle      = \color{jlstring},
%     commentstyle     = \color{jlcomment},
%     showstringspaces = false,
%     columns=fixed,
% }

\newcommand{\TOCpart}[1]{\addtocontents{toc}
{\bigskip\noindent\textcolor{green!40!blue}{\textbf{#1.}}}}

% To put some text with a light pink background
\newtcolorbox{eqbox}[1][]
{
boxrule=0pt,frame hidden,colback=red!5!white,
  left=.3em, right=.3em, top=.3em, bottom=.3em,
  beforeafter skip balanced=.4\baselineskip plus 2pt,
  before upper={\parindent4mm\noindent},
colframe=blue!55!white
}
\newenvironment{optional}[1][]{
\begin{eqbox}
}{
\end{eqbox}
}


\lstset{
    language = Magma,
    basicstyle=\ttfamily\small,
    commentstyle=\small,
    backgroundcolor = \color{gray!20!white},
    showstringspaces = false,
    columns=fixed,
}

%\renewcommand\sectionname{Lecture}
\renewcommand\subsectionname{\S}

% para enumerar
\renewcommand{\labelenumi}{\textbf{\arabic{enumi})}}

\usepackage[most]{tcolorbox}

%\newtcolorbox{mybox}%[colback=red!5!white,colframe=red!75!black]
% enhanced,
% boxrule=0pt,frame hidden,
% %borderline west={4pt}{0pt}{black},
% %colback={gray!20},
% sharp corners,
% left=.5cm
% %left=18.0pt
% }

\makeindex             

\newcommand{\Irr}{\operatorname{Irr}}
\newcommand{\Ann}{\operatorname{Ann}}
\newcommand{\op}{\operatorname{op}}
\newcommand{\Gal}{\operatorname{Gal}}
\newcommand{\supp}{\operatorname{supp}}
\newcommand{\Q}{\mathbb{Q}}
\newcommand{\Z}{\mathbb{Z}}
\newcommand{\F}{\mathbb{F}}
\newcommand{\R}{\mathbb{R}}
\newcommand{\B}{\mathbb{B}}
\newcommand{\C}{\mathbb{C}}
\newcommand{\D}{\mathbb{D}}
\newcommand{\rank}{\operatorname{rank}}
\newcommand{\norm}{\operatorname{norm}}
\newcommand{\Hom}{\operatorname{Hom}}
\newcommand{\Syl}{\mathrm{Syl}}
\newcommand{\id}{\operatorname{id}}
\newcommand{\Aut}{\operatorname{Aut}}
\newcommand{\Inn}{\operatorname{Inn}}
\newcommand{\End}{\operatorname{End}}
\newcommand{\Alt}{\mathbb{A}}
\newcommand{\Sym}{\mathbb{S}}
\newcommand{\lcm}{\operatorname{lcm}}
\newcommand{\trace}{\operatorname{trace}}
\newcommand{\sgn}{\operatorname{sign}}
\newcommand{\ch}{\operatorname{char}}
\newcommand{\im}{\operatorname{im}}
\newcommand{\Ret}{\operatorname{Ret}}
\newcommand{\GL}{\mathbf{GL}}
\newcommand{\SL}{\mathbf{SL}}
\newcommand{\PSL}{\mathbf{PSL}}
\newcommand{\PGL}{\mathbf{PGL}}
\newcommand{\Fix}{\operatorname{Fix}}
\newcommand{\Aff}{\operatorname{Aff}}
\newcommand{\Soc}{\operatorname{Soc}}
\newcommand{\Core}{\operatorname{Core}}
\newcommand{\legendre}[2]{\left(\frac{#1}{#2}\right)}
\newcommand{\Fun}{\operatorname{Fun}}
\newcommand{\Res}{\operatorname{Res}}
\newcommand{\Ind}{\operatorname{Ind}}
\newcommand{\cp}{\operatorname{cp}}
\newcommand{\cf}{\operatorname{cf}}
\newcommand{\Char}{\operatorname{Char}}
\newcommand{\gl}{\mathfrak{gl}}
\renewcommand{\sl}{\mathfrak{sl}}
\newcommand{\ad}[1]{\operatorname{ad}\,#1}
\newcommand{\A}{\mathbb{A}}


% column vector
\newcount\colveccount
\newcommand*\colvec[1]{
\global\colveccount#1
\begin{pmatrix}
        \colvecnext
        }
        \def\colvecnext#1{
        #1
        \global\advance\colveccount-1
        \ifnum\colveccount>0
        \\
        \expandafter\colvecnext
        \else
\end{pmatrix}
\fi
}

\newtheorem{theorem}{Theorem}[section]
\newtheorem{lemma}[theorem]{Lemma}
\newtheorem{proposition}[theorem]{Proposition}
\newtheorem{corollary}[theorem]{Corollary}

\theoremstyle{definition}
\newtheorem{definition}[theorem]{Definition}
\newtheorem{example}[theorem]{Example}
%\newtheorem{examples}[theorem]{Examples}
\newtheorem{xca}[theorem]{Exercise}
\newtheorem{bxca}[theorem]{Bonus exercise}
%\newtheorem{exa}[theorem]{Example}
\newtheorem{remark}[theorem]{Remark}
\newtheorem{que}[theorem]{Question}
\newtheorem{conj}[theorem]{Conjecture}
\newtheorem{open}[theorem]{Open problem}
\newtheorem{convention}[theorem]{Convention}

%\newtheorem{exercise}[theorem]{Exercise}

\theoremstyle{remark}
\newtheorem*{claim}{Claim}

\newenvironment{sol}[1]
{\renewcommand{\qedsymbol}{}\begin{proof}[\ref{#1}]}
  {\end{proof}}

% \newtcolorbox{mybox}{
% enhanced,
% boxrule=0pt,frame hidden,
% %borderline west={4pt}{0pt}{black},
% %colback={gray!20},
% sharp corners,
% left=.5cm
% %left=18.0pt
% }
% \newenvironment{exercise}
%   {\begin{mybox}\begin{xca}}
%   {\end{xca}\end{mybox}}

\newenvironment{problem}
{\begin{tcolorbox}[boxrule=0pt,frame hidden,colback=blue!5!white,
  left=.3em, right=.3em, top=-.2em, bottom=.3em,
  beforeafter skip balanced=.4\baselineskip plus 2pt,
  before upper={\parindent4mm\noindent},
colframe=blue!55!white]\begin{open}}{\end{open}\end{tcolorbox}}

\newenvironment{question}
{\begin{tcolorbox}[boxrule=0pt,frame hidden,colback=blue!5!white,
  left=.3em, right=.3em, top=-.2em, bottom=.3em,
  beforeafter skip balanced=.4\baselineskip plus 2pt,
  before upper={\parindent4mm\noindent},
colframe=blue!55!white]\begin{que}}{\end{que}\end{tcolorbox}}

\newenvironment{conjecture}
{\begin{tcolorbox}[boxrule=0pt,frame hidden,colback=blue!5!white,
  left=.3em, right=.3em, top=-.2em, bottom=.3em,
  beforeafter skip balanced=.4\baselineskip plus 2pt,
  before upper={\parindent4mm\noindent},
colframe=blue!55!white]\begin{conj}}{\end{conj}\end{tcolorbox}}


\newenvironment{exercise}
{\begin{tcolorbox}[boxrule=0pt,frame hidden,colback=green!5!white,
  left=.3em, right=.3em, top=-.2em, bottom=.3em,
  beforeafter skip balanced=.4\baselineskip plus 2pt,
  before upper={\parindent4mm\noindent},
colframe=green!55!black]\begin{xca}}{\end{xca}\end{tcolorbox}}

\newenvironment{bonus}
{\begin{tcolorbox}[boxrule=0pt,frame hidden,colback=yellow!15!white,
  left=.3em, right=.3em, top=-.2em, bottom=.3em,
  beforeafter skip balanced=.4\baselineskip plus 2pt,
  before upper={\parindent4mm\noindent},
colframe=yellow!45!black]\begin{bxca}}{\end{bxca}\end{tcolorbox}}

% \newenvironment{example}
% {\begin{tcolorbox}[boxrule=0pt,frame hidden,colback=red!5!white,
%   left=.3em, right=.3em, top=-.2em, bottom=.3em,
%   beforeafter skip balanced=.4\baselineskip plus 2pt,
%   before upper={\parindent4mm\noindent},
% colframe=red!55!black]\begin{exa}}{\end{exa}\end{tcolorbox}}

\numberwithin{figure}{section}
\numberwithin{equation}{section}

\makeindex

\title{\course}
\author{Leandro Vendramin}
\address{Department of Mathematics and Data
Science, Vrije Universiteit Brussel, Pleinlaan 2, 1050 Brussel}
\email{Leandro.Vendramin@vub.be}
\thanks{}
\date{}

\makeatletter
\renewcommand\section{\@startsection{section}{1}%
  \z@{.7\linespacing\@plus\linespacing}{.5\linespacing}%
  {\color{green!30!black}\normalfont\bfseries\centering}}
\renewcommand\subsection{\@startsection{subsection}{2}%
  \normalparindent{.5\linespacing\@plus.7\linespacing}{-.5em}%
  {\color{-red!55!green!50}\normalfont\bfseries}}
\renewcommand\subsubsection{\@startsection{subsubsection}{3}%
  \normalparindent\z@{-.5em}%
  {\color{green!50!blue}\normalfont\itshape}}


\usepackage{fancyhdr}
\pagestyle{fancy}
\fancyhf{}
\fancyfoot[R]{\thepage}
\fancyhead[L]{\course}
\fancyhead[R]{Lecture \thesection}
\setlength{\headheight}{14pt}

\usepackage[english.nosectiondot]{babel}


\usepackage{makecell}
\setcellgapes{3pt}


\renewcommand{\sl}{\mathfrak{sl}}
\newcommand{\cC}{\mathcal{C}}
\renewcommand{\cf}{\operatorname{ClassFun}}
\newcommand{\cD}{\mathcal{D}}
\newcommand{\Sets}{\mathfrak{Sets}}
\newcommand{\Groups}{\mathfrak{Groups}}
\newcommand{\AbelianGroups}{\mathfrak{AbelianGroups}}
\newcommand{\Rings}{\mathfrak{Rings}}
\newcommand{\Vect}{\mathfrak{Vec}}

\newcommand{\Inf}{\operatorname{Inf}}
\newcommand{\Tchar}{\mathbf{1}}

%\newcommand{\Q}{\mathbb{Q}}
%\newcommand{\Z}{\mathbb{Z}}
%\newcommand{\F}{\mathbb{F}}
%\newcommand{\R}{\mathbb{R}}
%\newcommand{\C}{\mathbb{C}}
%\renewcommand{\H}{\mathbb{H}}

%\usepackage{mathptmx}
%\usepackage{newtxtext}

\begin{document}

 \begin{abstract}
 The notes correspond to the master
 course \textbf{Representation Theory of Algebras} of the
 Vrije Universiteit Brussel,
 Faculty of Sciences,
 Department of Mathematics and Data Sciences. \end{abstract}

\maketitle

\setcounter{tocdepth}{1}
\tableofcontents

\TOCpart{Part 1}

\thispagestyle{plain}
\section*{Introduction}

The notes correspond to the master  
course \emph{Representation theory of algebras} of the 
Vrije Universiteit Brussel, 
Faculty of Sciences, 
Department of Mathematics and Data Sciences. The course
is divided into twelve two-hour lectures. 

Most of the material is based on standard 
results of the representation theory of finite groups. 
Basic texts on representation theory are \cite{MR1369573} 
and \cite{MR2270898}. 

The notes include Magma code, which we use to verify examples and offer alternative solutions to certain exercises. Magma \cite{zbMATH01077111} is a powerful software tool designed for working with algebraic structures. There is a free \href{https://magma.maths.usyd.edu.au/calc/}{online} version of Magma available.


Thanks go to Luca Descheemaeker, Wannes Malfait, Silvia Properzi, Lukas Simons.  



This version 
was compiled on \today~at~\currenttime.


 \begin{figure}[b]
     \includegraphics[scale=0.2]{VUB.jpg}
 \end{figure}
%\include{subsections}

% \usepackage{fancyhdr}
% \pagestyle{fancy}
% \fancyhf{}
% \fancyfoot[R]{\thepage}
% \fancyhead[L]{\course}
% \fancyhead[R]{Lecture \thesection}
% \setlength{\headheight}{14pt}

\section{Lecture: Week 1}

\subsection{The Artin--Wedderburn theorem}

We first review the basic definitions concerning
finite-dimensional semisimple algebras. 
Proofs can be found in the
notes to the course \emph{Associative Algebras} (see
Lectures 1, 2 and 3). 

Our base field will be the field $\C$ of complex numbers. 

\index{Algebra}
\index{Algebra!unitary}
A (complex) \emph{algebra} $A$ is a (complex) vector space  
with an associative multiplication $A\times A\to A$ such that
\[
a(\lambda b+\mu c)=\lambda(ab)+\mu(ac),
\quad
(\lambda a+\mu b)c=\lambda(ac)+\mu (bc)
\]
for all $a,b,c\in A$. If $A$ contains 
an element $1_A\in A$ such that $1_Aa=a1_A=a$ for all $a\in A$, then $A$ is 
a unitary algebra. Our algebras will be unitary. 

Our algebras will also be finite-dimensional. 
Clearly, $\C$ is an algebra. Other 
examples of algebras are $\C[X]$ and $M_n(\C)$. 

\index{Module}
\index{Submodule}
A (left) \emph{module} $M$ (over a unitary 
algebra $A$) is an abelian group $M$
together with a map $A\times M\to M$, $(a,m)\mapsto am$, such that
$1_Am=m$ for all $m\in M$ and 
$a(bm)=(ab)m$ and $a(m+m_1)=am+am_1$ for all $a,b\in A$ and $m,m_1\in M$. 
A \emph{submodule} $N$ of $M$ is a subgroup 
$N$ such that $an\in N$ for all $a\in A$ and $n\in N$. 

\begin{exercise}
Let $A$ be a finite-dimensional algebra. If $M$ 
is an $A$-module, then $M$ is a vector space with 
$\lambda m=(\lambda 1_A)m$ for $\lambda\in\C$ and $m\in M$. Moreover, 
$M$ is finitely generated (as an $A$-module) if and only if $M$ is finite-dimensional. 
\end{exercise}

\index{Module!simple}
\index{Module!semisimple}
A module $M$ is said to be \emph{simple} if $M\ne\{0\}$ and $\{0\}$ and $M$ 
are the only submodules of $M$.	
A finite-dimensional module $M$  
is said to be \emph{semisimple} if $M$ is a direct sum of 
finitely many simple submodules. 
Clearly, simple modules are semisimple. Moreover, any finite direct sum of semisimples is semisimple. 


\index{Algebra!semisimple}
A finite-dimensional algebra $A$ is said to be \emph{semisimple} if
every finitely-generated $A$-module is semisimple. 

\begin{theorem}[Artin--Wedderburn]
Let $A$ be a complex finite-dimensional semisimple algebra, say with  
$k$ isomorphism classes of simple modules. Then 
\[
A\simeq M_{n_1}(\C)\times\cdots\times M_{n_k}(\C)
\]
for some $n_1,\dots,n_k\in\Z_{>0}$. \end{theorem}

The unique simple module of the algebra $M_{n_j}(\C)$ 
is the column space $\C^{n_j}$. This means that the simple component of dimension $n_j^2$ has a simple module of dimension $n_j$. 

We also give some basic facts on the Jacobson radical
of finite-dimensional algebras. If $A$ is a finite-dimensional algebra, the \emph{Jacobson radical} is defined as 
\[
J(A)=\bigcap\{M:M\text{ is a maximal left ideal of $A$}\}. 
\]
It turns out that $J(A)$ is an ideal of $A$. If $A$ is
unitary, then Zorn's lemma implies that there a 
maximal left ideal of $A$ and hence $J(A)\ne A$. 

An ideal $I$ of $A$ is said to be \emph{nilpotent}
if $I^m=\{0\}$ for some $m$, that is 
$x_1\cdots x_m=0$ for all $x_1,\dots,x_m\in I$. 
One proves that the Jacobson radical of $A$ 
contains every nilpotent ideal of $A$. An important
fact is that 
\begin{align*}
A\text{ is semisimple }
&\Longleftrightarrow 
J(A)=\{0\}\\
&\Longleftrightarrow 
A\text{ has no non-zero nilpotent ideals}.
\end{align*}

\subsection{Group algebras}

\index{Group algebra}
Let $G$ be a finite group. The (complex) \emph{group
algebra} $\C[G]$ is the $\C$-vector space with
basis $\{g:g\in G\}$ and multiplication
\[
\left(\sum_{g\in G}\lambda_gg\right)\left(\sum_{h\in G}\mu_hh\right)
=\sum_{g,h\in G}\lambda_g\mu_h(gh).
\]

Clearly, $\dim \C[G]=|G|$. Moreover, 
$\C[G]$ is commutative if and only if $G$ is abelian. 

\index{Augmentation ideal}
If $G$ is non-trivial, 
then $\C[G]$ contains proper non-trivial ideals. For example, 
the \emph{augmentation ideal} 
\[
I(G)=\left\{\sum_{g\in G}\lambda_gg\in \C[G]:\sum_{g\in G}\lambda_g=0\right\}
\]
is a non-zero proper ideal of $\C[G]$. 

\begin{exercise}
    Let $G$ be a finite non-trivial group. Prove that
    $\C[G]$ has zero divisors. 
\end{exercise}

For $n\in\Z_{\geq2}$, we write $C_n$ to denote the (multiplicative) cyclic group of order $n$. 

\begin{exercise}
Prove that $\C[C_n]\simeq \C[X]/(X^n-1)$. 
\end{exercise}

\begin{exercise}
    Let $G$ be a finite group. The set
    \[
    \Fun(G,\C)=\{\alpha\colon G\to\C\}
    \]
    is a complex vector space with 
    the operations 
    \[
    (\alpha+\beta)(x)=\alpha(x)+\beta(x),
    \quad
    (\lambda\alpha)(x)=\lambda\alpha(x),
    \]
    for all $\alpha,\beta\in\Fun(G,\C)$, $x\in G$ 
    and $\lambda\in\C$. It is an algebra
    with the \emph{convolution product} 
    \[
    (\alpha*\beta)(x)=\sum_{y\in G}\alpha(xy^{-1})\beta(y).
    \]
    Let 
    \[
    \delta_x(y)=\begin{cases}
            1 & \text{if $x=y$},\\
            0 & \text{otherwise}.
        \end{cases}
    \]
    Prove the following statements:
    \begin{enumerate}
        \item 
        The set $\{\delta_x:x\in G\}$ is a basis
        of $\Fun(G,\C)$. 
        \item The map $\C[G]\to\Fun(G,\C)$, $g\mapsto\delta_g$, 
            extends linearly to an algebra isomorphism. 
    \end{enumerate}
\end{exercise}


\index{Module!semisimple}
Recall that a finite-dimensional module $M$ is semisimple 
if and only if for every submodule $S$ of $M$ there 
is a submodule $T$ of $M$ such that $M=S\oplus T$.    

\begin{theorem}[Maschke]
\index{Maschke's theorem}
    Let $G$ be a finite
    group and $M$ be a finite-dimensional $\C[G]$-module.
    Then $M$ is semisimple. 
\end{theorem}

\begin{proof}
We must show that every submodule $S$ of $M$ admits a complement. 
Since $S$ is a subspace of $M$, there exists a subspace $T_0$ of $M$ 
such that $M=S\oplus T_0$ (as vector spaces). We use 
$T_0$ to construct a submodule $T$ of $M$ that complements $S$. Since $M=S\oplus T_0$, 
every $m\in M$ can be written uniquely as $m=s+t_0$ for some $s\in S$ and $t_0\in T$. 
Let 
\[
p_0\colon M\to S,\quad
p_0(m)=s,
\]
where $m=s+t_0$ with $s\in S$ and $t_0\in T$. 
If $s\in S$, then $p_0(s)=s$. In particular, $p_0^2=p_0$, as 
$p_0(m)\in S$. 

Generally, $p_0$ is not a $\C[G]$-modules homomorphism. 
Let 
\[
p\colon M\to S,\quad
p(m)=\frac{1}{|G|}\sum_{g\in G}g^{-1}\cdot p_0(g\cdot m).
\]

We claim that $p$ is a homomorphism of $\C[G]$-modules. For that purpose, we need to show that 
$p(g\cdot m)=g\cdot p(m)$ for all $g\in G$ and $m\in M$. In fact, 
\[
p(g\cdot m)=\frac{1}{|G|}\sum_{h\in G}h^{-1}\cdot p_0(h\cdot (g\cdot m))
=\frac{1}{|G|}\sum_{h\in G}(gh^{-1})\cdot p_0(h\cdot m)=g\cdot p(m).
\]

We now claim that $p(M)=S$. The inclusion $\subseteq$ is trivial to prove, as $S$ is a submodule of $M$ 
and $p_0(M)\subseteq S$. Conversely, if $s\in S$, then $g\cdot s\in S$, as 
$S$ is a submodule. Thus 
$s=g^{-1}\cdot (g\cdot s)=g^{-1}\cdot p_0(g\cdot s)$ and hence 
\[
s=\frac{1}{|G|}\sum_{g\in G}g^{-1}\cdot (g\cdot s)=\frac{1}{|G|}\sum_{g\in G}g^{-1}\cdot (p_0(g\cdot s))=p(s).
\]
Since $p(m)\in S$ for all $m\in M$, it follows that $p^2(m)=p(m)$, so $p$ is a projector onto $S$. 
Hence $S$ admits a complement in $M$, that is $M=S\oplus\ker(p)$.
\end{proof}

\begin{exercise}
Let $G=\langle g\rangle$ be the cyclic group 
of order four and $\rho_g=\begin{pmatrix}
0&-1\\
1&0\end{pmatrix}$. 
Let $M=\C^{2\times 1}$ as an $\C[G]$-module with 
\[
g\cdot\begin{pmatrix}u\\v\end{pmatrix}
=\begin{pmatrix}-v\\u\end{pmatrix}.
\]
Prove that $M$ is a semisimple non-simple $\C[G]$-module.
\end{exercise}

\begin{exercise}
Let $G=\langle g\rangle$ be the cyclic group 
of order four and $\rho_g=\begin{pmatrix}
0&-1\\
1&0\end{pmatrix}$. 
Let $M=\R^{2\times 1}$ as an $\R[G]$-module with 
\[
g\cdot\begin{pmatrix}u\\v\end{pmatrix}
=\begin{pmatrix}-v\\u\end{pmatrix}.
\]
Prove that $M$ is a simple $\R[G]$-module. 
\end{exercise}

If $G$ is a finite group, 
then $\C[G]$ is semisimple. By Artin--Wedderburn theorem, 
\[
\C[G]\simeq\prod_{i=1}^r M_{n_i}(\C),
\]
where $r$ is the number of isomorphism classes of simple modules of $\C[G]$. Moreover, 
\[
|G|=\dim\C[G]=\sum_{i=1}^r n_i^2,
\]
and the integers 
$n_1,n_2,\dots,n_r$ are
the dimensions of the non-isomorphic simple modules of
the complex group algebra 
$\C[G]$. 
% By convention, 
% we always assume that $n_1=1$. 
% This corresponds, of course, to the \emph{trivial module}. 

\begin{theorem}
    Let $G$ be a finite group. The number of simple 
    modules of $\C[G]$ coincides with the number of conjugacy classes of $G$. 
\end{theorem}

\begin{proof}
    By Artin--Wedderburn theorem, $\C[G]\simeq\prod_{i=1}^rM_{n_i}(\C)$. Thus 
    \[
		Z(\C[G])\simeq\prod_{i=1}^rZ(M_{n_i}(\C))\simeq\C^r.
	\]
	In particular, $\dim Z(\C[G])=r$. If $\alpha=\sum_{g\in
	G}\lambda_gg\in Z(\C[G])$, then $h^{-1}\alpha h=\alpha$ for all $h\in
	G$. Thus 
	\[
		\sum_{g\in G}\lambda_{hgh^{-1}}g=
		\sum_{g\in G}\lambda_g h^{-1}gh=\sum_{g\in G}\lambda_gg
	\]
	and hence $\lambda_{g}=\lambda_{hgh^{-1}}$ for all $g,h\in G$. A basis for 
	$Z(\C[G])$ is given by elements of the form 
	\[
		\sum_{g\in K}g,
	\]
	where $K$ is a conjugacy class of $G$. Therefore $\dim Z(\C[G])$ equals 
	the number of conjugacy classes of $G$.
\end{proof}

If $G$ is a finite group,
then 
\[
\C[G]\simeq \prod_{i=1}^k M_{n_i}(\C),
\]
where $k$ is the number of conjugacy classes of $G$. 
In particular, 
\[
|G|=\dim\C[G]=\sum_{i=1}^k n_i^2.
\]

\begin{exercise}
\label{xca:C4}
    Prove that $\C[C_4]\simeq\C^4$. 
\end{exercise}

For $n\geq1$, let $\Sym_n$ denote the symmetric group in $n$ letters. 

\begin{example}
    The group $\Sym_3$ has three conjugacy classes:
    $\{\id\}$, $\{(12),(13),(23)\}$ and $\{(123),(132)\}$. 
    Since $6=a^2+b^2+c^2$, it follows that 
    $\C[G]\simeq\C\times\C\times M_2(\C)$. 
\end{example}    

% \begin{problem}[Brauer]
% \index{Brauer's problem}
%     Which algebras are group algebras?
% \end{problem}

% This question might be impossible to answer, but it is extremely interesting.

% \begin{example}
%     The algebra $\C^2\times M_2(\C)\times M_3(\C)$ is not complex group algebra, as all groups of order 15 are abelian. 
% \end{example}

%\subsection{Some comments}

There is a multiplicative version of Maschke's theorem. A group $G$ \emph{acts 
by automorphisms} on $A$ if there is a group homomorphism 
$\lambda\colon G\to\Aut(A)$. In this case, a subgroup $B$ of $A$ is said to be 
$G$-invariant if $\lambda_g(B)\subseteq B$ for all $g\in G$. 

\begin{bonus}
\label{xca:Maschke_multiplicative1}
\index{Maschke theorem!multiplicative version}
    Let $K$ be a finite group of order $m$. Assume that 
    $K$ acts by automorphisms on $V=U\times W$, where
    $U$ and $W$ are subgroups of $V$ and $U$ is abelian and $K$-invariant. Prove that 
    if the map $U\to U$, $u\mapsto u^m$, is bijective, 
    there exists a normal $K$-invariant subgroup $N$ of $V$ such that $V=U\times N$. 
\end{bonus}

\begin{bonus}
\label{xca:Maschke_multiplicative2}
    Let $p$ be a prime number and $K$ be a finite
    group with order not divisible by $p$. Let $V$ be
    a $p$-elementary abelian group. Assume that $K$ acts
    by automorphism on $V$. Prove that if $U$ be a $K$-invariant subgroup of $V$, 
    there exists a $K$-invariant subgroup $N$ of $V$ 
    such that $V=U\times N$. 
\end{bonus}


\subsection{Representations}

Unless we state differently, we will always work
with finite groups. All our vector spaces will
be complex vector spaces. 

\begin{definition}
\index{Representation}
    Let $G$ be a finite group. A \emph{representation}
    of $G$ is a group homomorphism $\rho\colon G\to\GL(V)$, where
    $V$ is a finite-dimensional vector space. The \emph{degree} (or dimension) 
    of the representation is the integer $\deg\rho=\dim V$. 
\end{definition}

\index{Matrix representation}
Let $G\to\GL(V)$ be a representation. 
If we fix a basis of $V$, then we obtain
a \emph{matrix representation} of $G$, that is a 
group homomorphism 
\[
\rho\colon G\to\GL(V)\simeq\GL_n(\C),
\quad 
g\mapsto\rho_g,
\]
where
$n=\dim V$. 

\begin{example}
Since $\Sym_3=\langle (12),(123)\rangle$, the map $\rho\colon \Sym_3\to\GL_3(\C)$,
\[
(12)\mapsto\begin{pmatrix}
0 & 1 & 0\\
1 & 0 & 0\\
0 & 0 & 1
\end{pmatrix},\quad
(123)\mapsto\begin{pmatrix}
0 & 0 & 1\\
1 & 0 & 0\\
0 & 1 & 0
\end{pmatrix},
\] 
is a representation of $\Sym_3$. 
\end{example}

\begin{example}
Let $G=\langle g\rangle$ be cyclic of order six. 
The map $\rho\colon G\to\GL_2(\C)$, 
\[
g\mapsto
\begin{pmatrix}
1&1\\
-1&0
\end{pmatrix}
\] 
is a representation of $G$. 
\end{example}

\begin{example}
Let $G=\langle g\rangle$ be cyclic of order four. 
The map $\rho\colon G\to\GL_2(\C)$, 
\[
g\mapsto
\begin{pmatrix}
0&-1\\
1&0
\end{pmatrix}
\] 
is a representation of $G$. 
\end{example}

\begin{example}
  Let $G=\langle a,b:a^2=b^3=(ab)^3=1\rangle$. The map 
  \[
    a\mapsto\begin{pmatrix}
    0 & 1 & -1\\
    1 & 0 & -1\\
    0 & 0 & -1
    \end{pmatrix},
    \quad
    b\mapsto\begin{pmatrix}
      0 & 0 & 1\\
      1 & 0 & 0\\
      0 & 1 & 0
    \end{pmatrix},
  \]
  defines a representation $G\to\GL_3(\C)$. 
\end{example}

\begin{example}
    Let $Q_8=\{-1,1,i,-i,j,-j,k,-k\}$ be the quaternion group. Recall that
    \[
    i^2=j^2=k^2=-1,\quad
    ijk=-1.
    \]
    The group $Q_8$ is generated by $\{i,j\}$ 
    and the map $\rho\colon Q_8\to\GL_2(\C)$, 
    \[
    i\mapsto\begin{pmatrix}
    i&0\\0&-i
    \end{pmatrix},
    \quad
    j\mapsto\begin{pmatrix}
    0&1\\-1&0
    \end{pmatrix},
    \]
    is a representation.
\end{example}

\begin{example}
  Let $G$ be a finite group that acts on a finite set $X$. 
  Let $V=\C X$ the complex vector space with basis $\{x:x\in
  X\}$. The map 
  \[
	\rho\colon G\to\GL(V),\quad
	\rho_g\left(\sum_{x\in X}\lambda_x x\right)
	=\sum_{x\in X}\lambda_x\rho_g(x)
	=\sum_{x\in X}\lambda_{g^{-1}\cdot x}x, 
  \]
  is a representation of degree $|X|$.
\end{example}

\begin{example}
\index{Trivial!representation}
\index{Trivial!module}
    The map $\rho\colon G\to\C^{\times}$, $g\mapsto 1$, 
    is a representation, that is $\C$ is a $\C[G]$-module with
    $g\cdot \lambda=\lambda$ for all $g\in G$ 
    and $\lambda\in\C^{\times}$. This representation 
    is known 
    as the \emph{trivial representation}. 
\end{example}

\begin{example}
    The map $\sgn\colon\Sym_n\to\GL_1(\C)=\C^{\times}$ is a representation of $\Sym_n$.
\end{example}

An important fact is that there exists a bijective
correspondence 
between 
representations of a finite group $G$ 
and 
finite-dimensional modules over $\C[G]$. The correspondence
is given as follows. If $\rho\colon G\to\GL(V)$ is a representation, 
then $V$ is a $\C[G]$-module with
\[
\left(\sum_{g\in G}\lambda_gg\right)\cdot v=\sum_{g\in G}\lambda_g\rho_g(v).
\]
Conversely, if $V$ is a $\C[G]$-module, then
$\rho\colon G\to\GL(V)$, $\rho_g\colon V\to V$, $v\mapsto g\cdot v$, 
is a representation. 

This bijection between representations of groups and modules over group algebras allows us to construct a dictionary between concepts in the language of representations and that of modules. Both languages are useful, so depending on our convenience, we will use one or the other.

\begin{exercise}
    Let $G$ be a finite group and 
    $\rho\colon G\to\GL(V)$ be a representation. Prove that 
    each $\rho_g$ is diagonalizable. 
\end{exercise}

The previous exercise uses properties of the minimal polynomial. We will 
see a different proof later. 

\begin{definition}
\index{Equivalent representations}
Let $G$ be a group and $\phi\colon G\to\GL(V)$ and $\psi\colon G\to\GL(W)$ be representations of $G$. 
We say that $\phi$ and $\psi$ are \emph{equivalent} if 
there exists a linear isomorphism $T\colon V\to W$ such that 
\[
	\psi_g T=T \phi_g
\]
for all $g\in G$. In this case, we write $\phi\simeq\psi$. 
\end{definition}

Note that $\phi\simeq\psi$ if and only if $V$
and $W$ are isomorphic as $\C[G]$-modules.

\begin{example}
  The representation 
  \begin{gather*}
  \phi\colon\Z/n\to\GL_2(\C),\quad
  \phi(m)=
  \begin{pmatrix}
    \cos(2\pi m/n) & -\sin(2\pi m/n)\\
    \sin(2\pi m/n) & \cos(2\pi m/n)
  \end{pmatrix},
  \shortintertext{is equivalent to the representation}
  \psi\colon\Z/n\to\GL_2(\C),
  \quad 
  \psi(m)=\begin{pmatrix}
    e^{2\pi im/n} & 0\\
    0 & e^{-2\pi im/n}
  \end{pmatrix}.
  \end{gather*}
  The equivalence is obtained with the matrix $T=\begin{pmatrix} i & -i\\
    1&1\end{pmatrix}$, as a direct calculation shows that
    $\phi_m T=T\psi_m$ for all $m$.
\end{example}

\begin{exercise}
    Let $\rho\colon G\to\GL(V)$ be a representation. Fix a basis 
    of $V$ and consider the corresponding matrix representation $\phi$ 
    of $\rho$. Prove that $\rho$ and $\phi$ are equivalent. 
\end{exercise}

\begin{definition}
    Let $\phi\colon G\to\GL(V)$ be a representation. A subspace 
    $W\subseteq V$ is said to be \emph{$G$-invariant} if
    $\phi_g(W)\subseteq W$ for all $g\in G$.  
\end{definition}

Let $\rho\colon G\to\GL(V)$ be a representation. 
If $W$ is a $G$-invariant subspace of $V$, 
then the restriction $\rho|_W\colon G\to\GL(W)$
is a representation. In particular, $W$ is a submodule (over $\C[G]$) 
of $V$. 

\begin{definition}
\index{Representation!irreducible}
\index{Module!simple}
    A non-zero representation $\rho\colon G\to\GL(V)$ is 
    said to be \emph{irreducible} if 
    $\{0\}$ and $V$ are the only 
    $G$-invariant subspaces of $V$. 
\end{definition}

Note that a representation $\rho\colon G\to\GL(V)$ is irreducible
if and only if $V$ is simple. 

\begin{example}
    Degree-one representations are irreducible. 
\end{example}

\begin{exercise}
\label{xca:degree-one}
    Let $G$ be a finite group. 
    Prove that there exists a bijective correspondence between 
    degree-one representations of $G$ and 
    degree-one representations of $G/[G,G]$. 
\end{exercise}

\begin{exercise}
\label{xca:inequality}
    Let $G$ be a finite group of order $n$ with $k$ conjugacy classes.
    Let $m=(G:[G,G])$. Prove that $n+3m\geq4k$. 
\end{exercise}

In the following example, we work over the real numbers. 

\begin{example}
Let $G=\langle g\rangle$ be the cyclic group of three elements and 
\[
\rho\colon G\to\GL(\R^3),\quad
\rho_g(x,y,z)=(y,z,x).
\]
%\mapsto\begin{pmatrix}
%  0&1&0\\
%  0&0&1\\
%  1&0&0
%\end{pmatrix}.
%\]
%Thus $g$ acts on $\R^{3}$ by left matrix multiplication,
%\[
%g\cdot (x,y,z)=
%\begin{pmatrix}
%  0&1&0\\
%  0&0&1\\
%  1&0&0
%\end{pmatrix}\begin{pmatrix}
%x\\
%y\\
%z
%\end{pmatrix}.
%\]
The set 
\[
N=\{(x,y,z)\in\R^{3}:x+y+z=0\}
\]
is a $G$-invariant subspace of $\R^3$. 

We claim that $N$ is irreducible. 
If $N$ contains a non-zero $G$-invariant subspace $S$, 
let $(x_0,y_0,z_0)\in S\setminus\{(0,0,0)\}$. Since $S$ is $G$-invariant, 
\[
(y_0,z_0,x_0)=g\cdot (x_0,y_0,z_0)\in S. 
\]
We claim that $\{(x_0,y_0,z_0),(y_0,z_0,x_0)\}$ is linearly independent. If there exists $\lambda\in\R$ 
such that $\lambda(x_0,y_0,z_0)=(y_0,z_0,x_0)$, then $x_0=\lambda^3 x_0$. Since $x_0=0$ implies 
$y_0=z_0=0$, it follows that $\lambda=1$. In particular, $x_0=y_0=z_0$, a contradiction, as $x_0+y_0+z_0=0$. 
Hence $\dim S=2$ and therefore $S=N$. 
\end{example}

What happens in the previous example if we consider complex numbers?

\begin{exercise}
  \label{xca:deg2}
  Let $\phi\colon G\to \GL(V)$, $g\mapsto\phi_g$, be a degree-two representation. Prove that
  $\phi$ is irreducible if and only if there is no common eigenvector for all the $\phi_g$.
\end{exercise}

\begin{example}
\label{exa:S3_deg2}
  Recall that $\Sym_3$ is generated by $(12)$ and $(23)$. The map 
  \[(12)\mapsto\begin{pmatrix}
    -1 & 1\\
    0 & 1
  \end{pmatrix},
  \quad
  (23)\mapsto\begin{pmatrix}
    1 & 0\\
    1 & -1
  \end{pmatrix},
  \]
  defines a representation $\phi$ of $\Sym_3$. 
  Exercise \ref{xca:deg2} shows that $\phi$ is  
  irreducible.
\end{example}
\section{Lecture: Week 2}


We now describe some crucial examples of representations. 



\begin{example}
    Let $\rho\colon G\to\GL(V)$ and 
    $\psi\colon G\to\GL(W)$ be representations. The \emph{direct sum} 
    $\rho\oplus\psi\colon G\to\GL(V\oplus W)$, $g\mapsto (\rho_g,\psi_g)$, 
    is a representation. This is equivalent to say that
    the vector space $V\oplus W$ is a $\C[G]$-module with
    \[
    g\cdot (v,w)=(g\cdot v,g\cdot w),\quad
    g\in G,\;v\in v,\;w\in W.
    \]
\end{example}

Let $V$ be a vector space with basis $\{v_1,\dots,v_k\}$ and 
$W$ be a vector space with basis $\{w_1,\dots,w_l\}$. A 
\emph{tensor product} of $V$ and $W$ is a vector space $X$ with 
together with a bilinear map 
\[
V\times W\to X,
\quad
(v,w)\mapsto v\otimes w,
\]
such that $\{v_i\otimes w_j:1\leq i\leq k,\,1\leq j\leq l\}$ is a  
basis of $X$. The tensor product of $V$ and $W$ is unique up to isomorphism 
and it is denoted by $V\otimes W$. Note that
\[
\dim(V\otimes W)=(\dim V)(\dim W).
\]


\begin{example}
    Let $V$ and $W$ be $\C[G]$-modules. The \emph{tensor product}
    $V\otimes W$
    is a $\C[G]$-module 
    with
    \[
    g\cdot v\otimes w=g\cdot v\otimes g\cdot w,
    \quad
    g\in G,\;v\in V,\;w\in W.
    \]
\end{example}

Let $\rho\colon G\to\GL(V)$ and $\psi\colon G\to\GL(W)$ be representations. 
The \emph{tensor product} of $\rho$ and $\psi$ is the representation of $G$ given by 
\begin{gather*}
	\rho\otimes\psi\colon G\to\GL(V\otimes W),
	\quad 
	g\mapsto (\rho\otimes\psi)_g,
	\shortintertext{where}
	(\rho\otimes\psi)_g(v\otimes w)=\rho_g(v)\otimes \psi_g(w)
\end{gather*}
for $v\in V$ and $w\in W$. 

\begin{exercise}
    Let $G$ be a finite group and $V$ be a $\C[G]$-module. 
    Prove that the dual $V^*$  
is a $\C[G]$-module with
\[
(g\cdot f)(v)=f(g^{-1}v),\quad
f\in V^*,\;v\in V,\;g\in G.
\]
\end{exercise}

\begin{exercise}
    Let $G$ be a finite group and
    $V$ and $W$ be $\C[G]$-modules. Prove that
    the set $\Hom(V,W)$ of complex linear maps $V\to W$ 
     is a $\C[G]$-module
    with
    \[
    (g\cdot f)(v)=gf(g^{-1}v),\quad
    f\in\Hom(V,W),\;v\in V,\;g\in G.
    \]
    If, moreover, $V$ and $W$ are finite-dimensional, then
    \[
    V^{*}\otimes W\simeq\Hom(V,W)
    \]
    as $\C[G]$-modules.
\end{exercise}



\begin{definition}
    \index{Representation!completely reducible}
    A representation $\rho\colon G\to\GL(V)$ is said to be 
    \emph{completely reducible}
    if $\rho$ can be decomposed as
    $\rho=\rho_1\oplus\cdots\oplus \rho_n$ for some irreducible
    representations $\rho_1,\dots,\rho_n$ of $G$. 
\end{definition}

Note that if $\rho\colon G\to\GL(V)$ is completely reducible and 
$\rho=\rho_1\oplus\cdots\oplus \rho_n$ for some irreducible representations 
$\rho_i\colon G\to\GL(V_i)$, $i\in\{1,\dots,n\}$, then 
each $V_i$ is an invariant subspace of $V$ and $V=V_1\oplus \cdots\oplus V_n$. 
Moreover, in some basis of $V$, the matrix  
$\rho_g$ can be written as 
\[
\rho_g=\begin{pmatrix}
(\rho_1)_g &  \\
& (\rho_2)_g  \\
&&\ddots\\
&&&(\rho_n)_g	
\end{pmatrix}.
\]

\begin{definition}
\index{Representation!decomposable}
\index{Representation!indecomposable}
A representation
$\rho\colon G\to\GL(V)$ is \emph{decomposable} if $V$ can be decomposed as $V=S\oplus T$
where $S$ and $T$ are non-zero invariant subspaces of $V$. 
\end{definition}

A representation is 
\emph{indecomposable} if it is not decomposable. 

\begin{exercise}
\label{xca:equivalence}
	Let $\rho\colon G\to\GL(V)$ and $\psi\colon G\to\GL(W)$ be equivalent representations.
	Prove the following facts:
	\begin{enumerate}
		\item If $\rho$ is irreducible, then $\psi$ is irreducible.
		\item If $\rho$ is decomposable, then $\psi$ is decomposable.
		\item If $\rho$ is completely reducible, then $\psi$ is completely reducible. 
	\end{enumerate}	
\end{exercise}


\subsection{Characters}

Fix a finite group $G$ and consider
(matrix) representations of $G$. We use linear algebra to study 
these representations. 

\begin{definition}
	\index{Character}
	Let $\rho\colon G\to\GL(V)$ be a representation. The \emph{character} of $\rho$ 
	is the map $\chi_\rho\colon G\to\C$, $g\mapsto\trace\rho_g$. 	
\end{definition}

If a representation $\rho$ is irreducible, its character is said to be an 
\emph{irreducible character}. The \emph{degree} of a character is the degree of the affording
representation. 

\begin{example}
    We can compute the character of the representation
    \[
    (12)\mapsto\begin{pmatrix}
    -1 & 1\\
    0 & 1
  \end{pmatrix},
  \quad
  (23)\mapsto\begin{pmatrix}
    1 & 0\\
    1 & -1
  \end{pmatrix},
  \]
    of Example \ref{exa:S3_deg2}. Since
    \[
\rho_{(132)}=\rho_{(23)(12)}=\rho_{(23)}\rho_{(12)}
=\begin{pmatrix}
    -1&1\\
-1&0
    \end{pmatrix},
\]
we conclude that $\rho_{(132)}=-1$. Similar calculations show
that
    \[
    \chi_{\id}=2, 
    \quad\chi_{(12)}=\chi_{(13)}=\chi_{(23)}=0,
    \quad 
    \chi_{(123)}=\chi_{(132)}=-1.
    \]
\end{example}

\begin{proposition}
	Let $\rho\colon G\to\GL(V)$ be a representation, $\chi$ be its character and $g\in G$.
	The following statements hold:
	\begin{enumerate}
		\item $\chi(1)=\dim V$. 
		\item $\chi(g)=\chi(hgh^{-1})$ for all $h\in G$.
		\item $\chi(g)$ is the sum of $\chi(1)$ roots of one of order $|g|$. 
		\item $\chi(g^{-1})=\overline{\chi(g)}$. 
		\item $|\chi(g)|\leq\chi(1)$.  
	\end{enumerate} 
\end{proposition}

\begin{proof}
	The first statement is trivial. 	
    
    To prove 2) note that
	\[
	\chi(hgh^{-1})=\trace(\rho_{hgh^{-1}})=\trace(\rho_h\rho_g\rho_h^{-1})=\trace\rho_g=\chi(g).
	\]
    
	Statement 3) follows from the fact that the trace of $\rho_g$ is the sum
	of the eigenvalues of $\rho_g$ and these numbers are roots of the polynomial
	$X^{|g|}-1\in\C[X]$. To prove 4) write $\chi(g)=\lambda_1+\cdots+\lambda_k$, where 
	the $\lambda_j$ are roots of one. Then
	\[
	\overline{\chi(g)}=\sum^k_{j=1}\overline{\lambda_j}
	=\sum_{j=1}^k\lambda_j^{-1}
	=\trace(\rho_g^{-1})
	=\trace(\rho_{g^{-1}})
	=\chi(g^{-1}).
	\] 
    
	Finally, we prove 5). We use 3) to write $\chi(g)$ as the sum of
	$\chi(1)$ roots of one, say $\chi(g)=\lambda_1+\cdots+\lambda_k$ for
	$k=\chi(1)$. Then 
	\[
	|\chi(g)|=|\lambda_1+\cdots+\lambda_k|\leq |\lambda_1|+\cdots+|\lambda_k|
	=\underbrace{1+\cdots+1}_{\text{$k$-times}}=k.\qedhere
	\]
\end{proof}

If two representations are equivalent, their characters are equal.

\begin{definition}
	Let $G$ be a group and 
	$f\colon G\to\C$ be a map. Then $f$ is a \emph{class function} if
	$f(g)=f(hgh^{-1})$ for all $g,h\in G$. 	
\end{definition}

Characters are class functions. If $G$ is a finite group, 
we write 
\[
\cf(G)=\{f\colon G\to\C:f\text{ is a class function}\}.
\]
One proves that $\cf(G)$ is a complex vector space. 

\begin{exercise}
    Let $G$ be a finite group. For a conjugacy class $K$ of $G$
    let 
    \[
    \delta_K\colon G\to\C,
    \quad
    \delta_K(g)=\begin{cases}
        1 & \text{if $g\in K$,}\\
        0 & \text{otherwise}.
        \end{cases}
    \]
    Prove that $\{\delta_K:K\text{ is a conjugacy class of $G$}\}$ is a basis of $\cf(G)$. 
    In particular, $\dim\cf(G)$ is the number of conjugacy classes of $G$. 
\end{exercise}

\begin{proposition}
    If $\rho\colon G\to\GL(V)$ and
    $\psi\colon G\to\GL(W)$ are representations, then
    $\chi_{\rho\oplus\psi}=\chi_\rho+\chi_\psi$.
\end{proposition}

\begin{proof}
  For $g\in G$, it follows that 
  $(\rho\oplus\psi)_g=
  \begin{pmatrix}
    \rho_g & 0\\ 
    0 & \psi_g
  \end{pmatrix}$. 
  Thus  
  \[
    \chi_{\rho\oplus\psi}(g)=\trace((\rho\oplus\phi)_g)=\trace(\rho_g)+\trace(\psi_g)=\chi_\rho(g)+\chi_\psi(g).\qedhere
  \]
\end{proof}

\begin{proposition}
  	If $\rho\colon G\to\GL(V)$ and
    $\psi\colon G\to\GL(W)$ are representations, then
    \[
    \chi_{\rho\otimes\psi}=\chi_\rho\chi_\psi.
    \]
\end{proposition}

\begin{proof}
	For each $g\in G$, the map $\rho_g$ is diagonalizable. Let $\{v_1,\dots,v_n\}$
	be a basis of eigenvectors of $\rho_g$ and let $\lambda_1,\dots,\lambda_n\in\C$ be such that
	$\rho_g(v_i)=\lambda_iv_i$ for all $i\in\{1,\dots,n\}$. Similarly, 
	let $\{w_1,\dots,w_m\}$ be a basis of 
	eigenvectors of $\psi_g$ and $\mu_1,\dots,\mu_m\in\C$ be such that $\psi_g(w_j)=\mu_jw_j$ for all $j\in\{1,\dots,m\}$. Each 
	$v_i\otimes w_j$ is eigenvector of $(\rho\otimes\psi)_g$ with eigenvalue 
	$\lambda_i\mu_j$, as  
	\[
		(\rho\otimes\psi)_g(v_i\otimes w_j)=\rho_gv_i\otimes \psi_gw_j=\lambda_iv_i\otimes \mu_jv_j=(\lambda_i\mu_j)v_i\otimes w_j.
	\]
	Thus  
	$\{v_i\otimes w_j:1\leq i\leq n,1\leq j\leq m\}$ is a basis of eigenvectors and the 
	$\lambda_i\mu_j$ are the eigenvalues of $(\rho\otimes\psi)_g$. It follows that 
	\[
	\chi_{\rho\otimes\psi}(g)
	=\sum_{i,j}\lambda_i\mu_j
	=\left(\sum_i\lambda_i\right)\left(\sum_j\mu_j\right)
	=\chi_\rho(g)\chi_\psi(g).\qedhere 
	\]
\end{proof}

We know that
it is also possible to define the dual $\rho^*\colon G\to\GL(V^*)$  
of a representation
$\rho\colon G\to\GL(V)$ by the formula
\[
(\rho^*_gf)(v)=f(\rho^{-1}_gv),\quad
g\in G,\,f\in V^*\text{ and }v\in V.
\]  
We claim that the character of the dual representation is then 
$\overline{\chi_\rho}$. Let $\{v_1,\dots,v_n\}$ be a basis of $V$
and $\lambda_1,\dots,\lambda_n\in\C$ be such that $\rho_gv_i=\lambda_iv_i$ for all $i\in\{1,\dots,n\}$. If $\{f_1,\dots,f_n\}$ is the dual basis of $\{v_1,\dots,v_n\}$, then 
\[
(\rho^*_gf_i)(v_j)=f_i(\rho_g^{-1}v_j)
=\overline{\lambda_j}f_i(v_j)
=\overline{\lambda_j}\delta_{ij}
\]
and the claim follows. 

Let $G$ be a finite group. If $\chi,\psi\colon G\to\C$ are
characters of $G$ and $\lambda\in\C$, we define 
\[
    (\chi+\psi)(g)=\chi(g)+\psi(g),
    \quad
    (\chi\psi)(g)=\chi(g)\psi(g),
    \quad
    (\lambda\chi)(g)=\lambda\chi(g).
\]
We can then form linear combinations of characters. These functions, of course, are not necesarilly 
characters. 

\begin{theorem}
    Let $G$ be a finite group. Then irreducible characters of $G$
    are linearly independent. 
\end{theorem}

\begin{proof}
    Let $S_1,\dots,S_k$ be a complete set of representatives of 
    classes of 
    simple $\C[G]$-modules. Let 
    $\Irr(G)=\{\chi_1,\dots,\chi_k\}$. 
    By Artin--Wedderburn theorem, there is 
    an algebra isomorphism 
    $f\colon \C[G]\to M_{n_1}(\C)\times\cdots\times M_{n_k}(\C)$, 
    where $\dim S_j=n_j$ for all $j$. Moreover, 
    \[
    M_{n_j}(\C)\simeq \underbrace{S_j\oplus\cdots\oplus S_j}_{n_j-\text{times}}
    \]
    for all $j$. For each $j$ let $e_j=f^{-1}(I_j)$, where
    $I_j$ is the identity matrix of $M_{n_j}(\C)$. We claim that 
    \[
        \chi_i(e_j)=\begin{cases}
            \dim S_i & \text{if $i=j$,}\\
            0 & \text{otherwise}.
            \end{cases}
    \]
    In fact, $\chi_i(g)$ is the trace of the action of $g$ on $S_j$. Since 
    $e_ie_j=0$ if $i\ne j$, it follows that 
    $\chi_i(e_j)=0$ if $i\ne j$. Moreover, $e_j$ acts as the identity on $S_j$, thus
    $\chi_j(e_j)=\dim S_j$. 
    
    Now if $\sum\lambda_i\chi_i=0$ for some $\lambda_1,\dots,\lambda_k\in\C$, then
    \[
    (\dim S_j)\lambda_j=\sum\lambda_i\chi_i(e_j)=0
    \]
    and hence $\lambda_j=0$, as $\dim S_j\ne 0$. 
\end{proof}

\begin{theorem}
    Let $G$ be a finite group and $S_1,\dots,S_k$ be the simple
    $\C[G]$-modules (up to isomorphism). If $V=\oplus_{i=1}^k a_jS_j$, then
    $\chi_V=\sum a_i\chi_i$, where 
    $\chi_i=\chi_{S_i}$ for all $i$. Moreover, if $U$ and $V$ 
    are $\C[G]$-modules, 
    \[
    U\simeq V\Longleftrightarrow \chi_U=\chi_V.
    \]
\end{theorem}

\begin{proof}
    The first part is left as an exercise. 
    
    It is also an exercise to prove that $U\simeq V$ implies $\chi_U=\chi_V$. Let us prove
    the converse. Assume that $\chi_U=\chi_V$. Since $\C[G]$ is semisimple, 
    $U\simeq\oplus_{i=1}^k a_iS_i$ and 
    $V\simeq\oplus_{i=1}^k b_iS_i$ for some integers 
    $a_1,\dots,a_k\geq0$ and $b_1,\dots,b_k\geq0$. Since 
    \[
    0=\chi_U-\chi_V=\sum_{i=1}^k (a_i-b_i)\chi_i
    \]
    and the $\chi_i$ are linearly independent, it follows that
    $a_i=b_i$ for all $i$. Hence $U\simeq V$. 
\end{proof}

\begin{exercise}
    Let $G$ be a finite group and $U$ be a $\C[G]$-module.
    Prove  $\chi_{U^*}=\overline{\chi_U}$. 
\end{exercise}

We will use the following exercise later:

\begin{exercise}
\label{xca:char_Hom}
    Prove that if $G$ is a finite group and 
    $U$ and $V$ are $\C[G]$-modules, then 
    \[
        \chi_{\Hom(U,V)}=\overline{\chi_U}\chi_V.
    \] 
\end{exercise}

For a finite group $G$ we write $\Irr(G)$ to denote
the complete set of isomorphism classes of characters of irreducible representations 
of $G$. 

\begin{exercise}
    Let $G$ be a finite group. Prove that the set
    $\Irr(G)$ is a basis
    of $\cf(G)$. 
\end{exercise}

Let $G$ be a finite group and $U$ be a $\C[G]$-module. 
Let 
\[
U^G=\{u\in U:g\cdot u=u\text{ for all $g\in G$}\}.
\]
Then $U^G$ is a subspace of $U$. The following lemma
is important:

\begin{lemma}
\label{lem:dimU^G}
    $\dim U^G=\frac{1}{|G|}\sum_{x\in G}\chi_U(x)$
\end{lemma}

\begin{proof}
    Let $\rho$ be the representation associated with $U$ and 
    let 
    \[
    \alpha=\frac{1}{|G|}\sum_{x\in G}\rho_x\colon U\to U.
    \]
    
    We claim that $\alpha^2=\alpha$.
    Let $g\in G$. Then 
    \begin{gather*}
    \rho_g(\alpha)=\frac{1}{|G|}\sum_{x\in G}\rho_g\rho_x
    =\frac{1}{|G|}\sum_{x\in G}\rho_{gx}=\alpha.
    \shortintertext{Thus}
    \alpha(\alpha(u))=\frac{1}{|G|}\sum_{x\in G}\rho_x(\alpha(u))=\alpha(u)
    \end{gather*}
    for all $u\in U$. This implies that $\alpha$ has eigenvalues 0 and 1. In fact, 
    if $u\in U$ is an eigenvector of $\alpha$ of eigenvalue $\lambda\in\C$, then 
    \[
    \lambda u=\alpha(u)=\alpha(\alpha(u))=\alpha(\lambda u)=\lambda\alpha(u)=\lambda^2u.
    \]
    Thus $\lambda(\lambda-1)=0$. 
    
    Let $V$ be the eigenspace of eigenvalue 1. 
    We now claim that $V=U^G$. Let us first prove that 
    $V\subseteq U^G$. For that purpose, let 
    $v\in V$ and $g\in G$. Then
    \begin{align*}
    g\cdot v &=\rho_g(v)=\rho_g(\alpha(v))\\
    &=\frac{1}{|G|}\sum_{x\in G}\rho_g\rho_x(v)
    =\frac{1}{|G|}\sum_{y\in G}\rho_y(v)=\alpha(v)=v.
    \end{align*}
    Now we prove that $V\supseteq U^G$. Let $u\in U^G$, so
    $\rho_g(u)=u$ for all $g\in G$. Then
    \[
    \alpha(u)=\frac{1}{|G|}\sum_{x\in G}\rho_x(u)
    =\frac{1}{|G|}\sum_{x\in G}u=u.
    \]
    
    Thus 
    \[
    \dim U^G=\dim V=\trace\alpha
    =\frac{1}{|G|}\sum_{x\in G}\trace\rho_x
    =\frac{1}{|G|}\sum_{x\in G}\chi_U(x).\qedhere
    \]
\end{proof}

One proves that 
the operation 
\[
\langle\chi_U,\chi_V\rangle=\frac{1}{|G|}\sum_{g\in G}\chi_U(g)\overline{\chi_V(g)}
\]
defines an inner product. 

\begin{theorem}
    Let $G$ be a finite group and $U$ and $V$ be $\C[G]$-modules. 
    Then 
    \[
    \langle\chi_U,\chi_V\rangle=\dim\Hom_G(U,V).
    \]
\end{theorem}

\begin{proof}
    We claim that 
    \[
    \Hom_G(U,V)=\Hom(U,V)^G.
    \]
    Let us first prove that
    $\Hom_G(U,V)\subseteq\Hom(U,V)^G$. Let $f\in \Hom_G(U,V)$ and 
    $g\in G$. Then
    \[
    (g\cdot f)(u)=g\cdot f(g^{-1}\cdot u)=g\cdot (g^{-1}\cdot f(u))=f(u)
    \]
    for all $u\in U$. Now we prove that $\Hom_G(U,V)\supseteq\Hom(U,V)^G$.
    Let $f\in\Hom(U,V)^G$. Then $f\colon U\to V$ is a linear such that
    $g\cdot f=f$ for all $g\in G$. Then
    we compute 
    \begin{align*}
    (g\cdot f)(u)=f(u)&\implies 
    g\cdot f(g^{-1}\cdot u)=f(u)\\
    &\implies f(g^{-1}\cdot u)=g^{-1}\cdot f(u)\quad 
    \text{for all $g\in G$ and $u\in U$}
    \end{align*}
    This means that one has 
    \[
    f(g\cdot u)=g\cdot f(u)
    \]
    for all $g\in G$ and $u\in U$. 
    
    Using Exercise \ref{xca:char_Hom}, 
    \begin{align*}
        \dim\Hom_G(U,V) &= \dim\Hom(U,V)^G\\
        &=\frac{1}{|G|}\sum_{g\in G}\chi_{\Hom(U,V)}(g)\\
        &=\frac{1}{|G|}\sum_{g\in G}\overline{\chi_U(g)}\chi_V(g)\\
        &=\langle \chi_V,\chi_U\rangle.
    \end{align*}
    Since $\dim\Hom_G(U,V)\in\R$, one has    
    \[
\langle\chi_U,\chi_V\rangle=\overline{\langle\chi_V,\chi_U\rangle}=\langle\chi_V,\chi_U\rangle
    \]
    and the claim follows. 
\end{proof}

Let $G$ be a finite group and $\Irr(G)=\{\chi_1,\dots,\chi_k\}$. 
Note that $k$ is the number of conjugacy classes of $G$. Let 
$g_1,\dots,g_k$ be representatives of conjugacy classes of $G$. 
The \emph{matrix of characters} of $G$ 
is $X=(X_{ij})$, where 
\[
X_{ij}=\chi_i(g_j)
\]
for $i,j\in\{1,\dots,k\}$. 

\begin{example}
\label{exa:S3}
    Let $G=\Sym_3$. The group $G$ has three conjugacy classes, so
    $|\Irr(G)|=3$. Let $g_1=\id$, $g_2=(12)$ and $g_3=(123)$. We 
    know that $6=n_1^2+n_2^2+n_3^2$. We know two degree-one
    (irreducible) representations of $G$, the trivial one and
    the sign. This implies that $n_1=n_2=1$ and 
    $n_3=2$. 
    The matrix of characters is then
    \bigskip 
    \begin{center}
		\begin{tabular}{|c|ccc|}
			\hline
			& $1$ & $(12)$ & $(123)$ \tabularnewline
			\hline 
			$\chi_{1}$ & $1$ & $1$ & $1$\tabularnewline
			$\chi_{2}$ & $1$ & $-1$ & $1$ \tabularnewline
			$\chi_{3}$ & $2$ & ? & ? \tabularnewline
			\hline
		\end{tabular}
	\end{center}
    \bigskip 
\end{example}

Two entries of the character table of Example~\ref{exa:S3} 
are still unknown. As we further develop the theory of characters, we will discover several tricks that can be used to find these missing entries.

\section{Lecture: Week 3}

\subsection{Schur's orthogonality relations}

We start with a crucial exercise. It is known as Schur's lemma. 

\begin{exercise}
\label{xca:Schur}
\index{Schur!lemma}
    If $G$ is a group and  
    $U$ and $V$ are simple $\C[G]$-modules, then 
    a non-zero module homomorphism $U\to V$ is an isomorphism. 
\end{exercise}

We now discuss a handy application of Schur's lemma. 
Let $G$ be a finite group and $S$ be a simple $\C[G]$-module.
We claim that $\Hom_G(S,S)\simeq\C$. Let 
$f\in\Hom_G(S,S)$ and $\lambda\in\C$ be an eigenvalue of $f$. Then 
$f-\lambda\id\colon S\to S$ is not invertible. By Schur's lemma, 
$f-\lambda\id=0$ and hence $f=\lambda\id$. 

\begin{definition}
    Let $G$ be a group. 
    A representation $\rho\colon G\to\GL(V)$ is said
    to be \emph{faithful} if $\rho$ is injective. 
\end{definition}

\begin{exercise}
    \label{xca:Z(G)cyclic}
    Let $G$ be a finite group 
    that admits a faithful irreducible
    representation. Prove that $Z(G)$ is cyclic. 
\end{exercise}

To solve Exercise~\ref{xca:Z(G)cyclic} one needs to use the following
elementary fact: 
A finite subgroup of $S^1=\{z\in\C:|z|=1\}$ is cyclic. 


\begin{theorem}[Schur]
\index{First orthogonality relation}
\index{Orthogonality relations}
    Let $G$ be a finite group and $\chi,\psi\in\Irr(G)$. Then
    \[
    \langle\chi,\psi\rangle=\begin{cases}
    1 & \text{if $\chi=\psi$,}\\
    0 & \text{otherwise.}
    \end{cases}
    \]
\end{theorem}

\begin{proof}
    Let $S_1,\dots,S_k$ be the simples of $\C[G]$. For each $j$, let
    $\chi_j$ be the irreducible character of $S_j$. Then 
    \[
    \langle\chi_i,\chi_j\rangle=\dim\Hom_G(S_i,S_j)
    =\begin{cases}
    1 & \text{if $S_i\simeq S_j$},\\
    0 & \text{otherwise.}
    \end{cases}
    \]
    But we know that $S_i\simeq S_j$ if and only if $\chi_i=\chi_j$. 
\end{proof}

\label{tab:irr_S3}
\index{Character table!of $\Sym_3$}
With the theorem, one can construct the character table of $\Sym_3$.
For example, this can be done using that $\langle\chi_3,\chi_3\rangle=1$ 
and that $\langle\chi_1,\chi_3\rangle=0$. 
As an exercise, verify that the character table of $\Sym_3$ is given in Table~\ref{tab:S3}.

    \begin{table}[h]
    \label{tab:S3}
    \caption{The character table of $\Sym_3$.}
		\begin{tabular}{|c|ccc|}
			\hline
			& $1$ & $(12)$ & $(123)$ \tabularnewline
			\hline 
			$\chi_{1}$ & $1$ & $1$ & $1$\tabularnewline
			$\chi_{2}$ & $1$ & $-1$ & $1$ \tabularnewline
			$\chi_{3}$ & $2$ & $0$ & $-1$ \tabularnewline
			\hline
		\end{tabular}
	\end{table}
	
\begin{exercise}
    Let $G$ be a finite group. 
    Prove that $\Irr(G)$ is an orthonormal basis of $\cf(G)$. 
\end{exercise}

The previous exercise has some consequences. Let $G$ be a finite group
and assume that $\Irr(G)=\{\chi_1,\dots,\chi_k\}$. If 
$\alpha=\sum a_i\chi_i$, then $\alpha=\sum\langle\alpha,\chi_i\rangle\chi_i$.  

\begin{theorem}
\label{thm:regular}
    Let $G$ be a finite group and $S_1,\dots,S_k$ be the simples of $\C[G]$. 
    Then the left regular $\C[G]$-module decomposes as 
    \[
    \C[G]\simeq\bigoplus_{i=1}^k(\dim S_i)S_i.
    \]
\end{theorem}

\begin{proof}
    Let $n=|G|$. Assume that $G=\{g_1,\dots,g_n\}$.
    Decompose the $\C[G]$-module corresponding 
    to the left regular representation as
    \[
    \C[G]\simeq a_1S_1\oplus\cdots\oplus a_kS_k
    \]
    for some integers $a_1,\dots,a_k\geq0$. Let $L\colon G\to \Sym_G$, $g\mapsto L_g$, 
    where $L_g(g_j)=gg_j$ for all $j$. Since the matrix of $L_g$ in the basis
    $\{g_1,\dots,g_n\}$ is
    \begin{gather*}
    (L_g)_{ij}=\begin{cases}
        1 & \text{if $g_i=gg_j$},\\
        0 & \text{otherwise},
        \end{cases}
        \shortintertext{one obtains that}
    \chi_L(g)=\begin{cases}
    |G| & \text{if $g=1$},\\
    0 & \text{otherwise}.
    \end{cases}
    \end{gather*}
    Moreover, 
    \begin{gather*}
    \chi_L=\sum_{i=1}^ka_i\chi_i=\sum_{i=1}^k\langle\chi_L,\chi_i\rangle\chi_i
    \shortintertext{and}
    a_i=\langle\chi_L,\chi_i\rangle=\frac{1}{|G|}\sum_{g\in G}\chi_L(g)\overline{\chi_i(g)}
    =\frac{1}{|G|}|G|\overline{\chi_i(1)}=\dim S_i.
    \end{gather*}
    Thus $\C[G]\simeq\bigoplus_{i=1}^k(\dim S_i)S_i$. 
\end{proof}

If $G$ is a finite group, let $\Char(G)$
be the set of characters of $G$. 

\begin{exercise}
\label{xca:n_irreducible}
    Let $n\in\{1,2,3\}$. Let $G$ be a finite
    group and $\alpha\in\Char(G)$. Prove that
    $\alpha$ is the sum
    of $n$ irreducible characters if and only if $\langle\alpha,\alpha\rangle=n$.  
\end{exercise}

We now prove Schur's second orthogonality relation. 

\begin{theorem}[Schur]
\index{Second orthogonality relation}
\index{Orthogonality relations}
    Let $G$ be a finite group and $g,h\in G$. 
    Then
    \[
    \sum_{\chi\in\Irr(G)}\chi(g)\overline{\chi(h)}
    =\begin{cases}
    |C_G(g)| & \text{if $g$ and $h$ are conjugate},\\
    0 & \text{otherwise}.
    \end{cases}
    \]
\end{theorem}

\begin{proof}
    Let $g_1,\dots,g_r$ be the representatives of the conjugacy classes of $G$. 
    Assume that $\Irr(G)=\{\chi_1,\dots,\chi_r\}$. For each $k\in\{1,\dots,r\}$, 
    let $c_k=(G:C_G(g_k))$ denote the size of the conjugacy class of $g_k$. Then
    \[
    \langle\chi_i,\chi_j\rangle
    =\frac{1}{|G|}\sum_{g\in G}\chi_i(g)\overline{\chi_j(g)}
    =\frac{1}{|G|}\sum_{k=1}^rc_k\chi_i(g_k)\overline{\chi_j(g_k)}.
    \]
    We write this as $I=\frac{1}{|G|}XDX^*$, where $I$ denotes the identity matrix, 
    $X_{ij}=\chi_i(g_j)$, 
    $X^*=\overline{X}^T$ and 
    \[
    D=\begin{pmatrix}
    c_1\\
    &c_2\\
    &&\ddots\\
    &&&c_r
    \end{pmatrix}.
    \]
    Since, in matrices, $AB=I$ implies $BA=I$, it follows that
    $I=\frac{1}{|G|}X^*XD$. Thus, using that $|G|=c_k|C_G(g_k)|$ 
    holds for all $k$, 
    \[
    (|G|D^{-1})_{ij}=(X^*X)_{ij}=\sum_{k=1}^r\overline{\chi_k(g_i)}\chi_k(g_j)
    =\begin{cases}
    |C_G(g_j)| & \text{if $i=j$},\\
    0 & \text{otherwise}.
    \end{cases}\qedhere
    \]
\end{proof}

\begin{exercise}
\label{xca:determinant_chars}
    Let $G$ be a finite group and $g_1,\dots,g_r$ be the representatives of the 
    conjugacy classes of $G$. Assume that $\Irr(G)=\{\chi_1,\dots,\chi_r\}$. Compute 
    the determinant of the matrix $X=(\chi_j(g_i))_{1\leq i,j\leq r}$. 
\end{exercise}

\begin{theorem}[Solomon]
\index{Solomon theorem}
\label{thm:Solomon}
    Let $G$ be a finite group and $\Irr(G)=\{\chi_1,\dots,\chi_r\}$. 
    If $g_1,\dots,g_r$ are the representatives of the conjugacy classes
    of $G$ and $i\in\{1,\dots,r\}$, then 
    \[
    \sum_{j=1}^r\chi_i(g_j)\in\Z_{\geq0}.
    \]
\end{theorem}

\begin{proof}
    Let $n=|G|$. 
    Assume that $G=\{g_1,g_2,\dots,g_r,g_{r+1},\dots,g_n\}$. 
    Let $V$ be the complex vector space with basis $\{g_1,\dots,g_n\}$. 
    The action of $G$ on $G$ by conjugation induces a group homomorphism 
    $\rho\colon G\to\GL(V)$, $g\mapsto\rho_g$, where
    $\rho_g(h)=ghg^{-1}$. The matrix of $\rho_g$ 
    in the basis $\{g_1,\dots,g_n\}$ is
    \[
    (\rho_g)_{ij}=\begin{cases}
        1 & \text{if $g_jg=gg_i$},\\
        0 & \text{otherwise}.
        \end{cases}
    \]
    Then
    \[
    \chi_{\rho}(g)=\trace\rho_g=\sum_{k=1}^{|G|}(\rho_g)_{kk}
    =|\{k:g_kg=gg_k\}|=|C_G(g)|.
    \]
    Write $\chi_{\rho}=\sum_{i=1}^rm_i\chi_i$ for $m_1,\dots,m_r\geq0$. 
    For each $j$ let $c_j=(G:C_G(g_j))$. Then
    \begin{align*}
    m_i=\langle\chi_{\rho},\chi_i\rangle
    &=\frac{1}{|G|}\sum_{g\in G}\chi_{\rho}(g)\overline{\chi_i(g)}\\
    &=\frac{1}{|G|}\sum_{j=1}^r c_j|C_G(g_j)|\overline{\chi_i(g_j)}
    =\sum_{j=1}^r\overline{\chi_i(g_j)}.\qedhere
    \end{align*}
\end{proof}


\begin{exercise}[Solomon]
\label{xca:Solomon}
    Let $G$ be a finite group and $g_1,\dots,g_r$ be the representatives of the conjugacy 
    classes of $G$. Assume that $\Irr(G)=\{\chi_1,\dots,\chi_r\}$. 
    Prove that 
    \[
        |G|\geq \sum_{i=1}^r\sum_{j=1}^r\chi_i(C_j)\in\Z_{\geq1}. 
    \]
    Moreover, if 
    $|G|=\sum_{i=1}^r\sum_{j=1}^r\chi_i(C_j)$, then 
    $G/Z(G)$ is abelian. 
\end{exercise}

\subsection{Algebraic integers and characters}

\begin{definition}
\index{Algebraic integer}
\label{def:algebraic_integer}
    Let $\alpha\in\C$. We say that $\alpha$ is \emph{algebraic integer}
    if $f(\alpha)=0$ for some monic polynomial $f\in\Z[X]$. 
\end{definition}

Let $\A$ be the set of algebraic integers. Note that $\Z\subseteq\A$. 

\begin{example}
    Every root of one is an algebraic integer.
\end{example}

\begin{proposition}
    $\Q\cap\A=\Z$. 
\end{proposition}

\begin{proof}
    Let $m/n\in\Q$ with $\gcd(m,n)=1$ and $n>0$. If 
    $f(m/n)=0$ for some 
    \[
    f=X^k+a_{k-1}X^{k-1}+\cdots+a_1X+a_0\in\Z[X]
    \]
    of degree $k\geq1$, then
    \[
    0=n^kf(m/n)=m^k+a_{k-1}m^{k-1}n+\cdots+a_1mn^{k-1}+a_0n^k.
    \]
    This implies that 
    \[
        m^k=-n\left(a_{k-1}m^{k-1}+\cdots+a_1mn^{k-2}+a_0n^{k-1}\right)
    \]
    and hence $n$ divides $m^k$. Thus $n=1$ and 
    therefore $m/n\in\Z$.
\end{proof}

\begin{proposition}
    Let $x\in\C$. Then $x\in\A$ if and only if $x$ is an eigenvalue of
    an integer matrix.
\end{proposition}

\begin{proof}
    Let us prove the non-trivial implication. Let 
    \[
    f=X^n+a_{n-1}X^{n-1}+\cdots+a_0\in\Z[X]
    \]
    be such that $f(x)=0$. Then $x$ is an eigenvalue
    of the companion matrix of $f$, that is the matrix
    \[
    C(f)=
    \begin{pmatrix}
    0&0&\cdots &0&-a_{0}\\
    1&0&\cdots &0&-a_{1}\\
    0&1&\cdots &0&-a_{2}\\
    \vdots &\vdots &\ddots &\vdots &\vdots \\
    0&0&\cdots &1&-a_{{n-1}}
    \end{pmatrix}
    \in\Z^{n\times n}.\qedhere 
    \]
\end{proof}

\begin{theorem}
\label{thm:Asubring}
    $\A$ is a subring of $\C$. 
\end{theorem}

\begin{proof}
    Let $\alpha,\beta\in\A$. By the previous proposition, 
    $\alpha$ is an eigenvalue 
    of an integer matrix $A\in\Z^{n\times n}$, say
    $Av=\alpha v$ for some $v\ne0$, 
    $\beta$ is an eigenvalue of an integer matrix 
    $B\in\Z^{m\times m}$, say $Bw=\beta w$ for some $w\ne0$. Then
    \[
    (A\otimes I_{m\times m}+I_{n\times n}\otimes B)(v\otimes w)
    =(\alpha+\beta)(v\otimes w), 
    \]
    where $I_{k\times k}$ denotes the $(k\times k)$ identity 
    matrix, and
    \[
    (A\otimes B)(v\otimes w)=(\alpha\beta)v\otimes w.
    \]
    This implies that 
    $\alpha+\beta\in\A$ and $\alpha\beta\in\A$, again 
    by the previous proposition. 
\end{proof}

\begin{theorem}
\label{thm:A}
    Let $G$ be a finite group. If $\chi\in\Char(G)$ and
    $g\in G$, then $\chi(g)\in\A$. 
\end{theorem}

\begin{proof}
    Let $\varphi$ be a representation of $G$ such that 
    $\chi_\varphi=\chi$. Since $\varphi_g$ is diagonalizable with
    eigenvalues $\lambda_1,\dots,\lambda_k\in\A$ (because
    $G$ is finite and the $\lambda_j$ are roots of one), 
    \[
    \chi(g)=\trace\varphi_g=\sum_{i=1}^k\lambda_i\in\A. \qedhere
    \]
\end{proof}


\chapter{}

\topic{Schur's orthogonality relations}

We start with a very important exercise. It is known as Schur's lemma. 

\begin{exercise}
\label{xca:Schur}
\index{Schur's lemma}
    If $G$ is a group and  
    $U$ and $V$ are simple $\C[G]$-modules, then 
    a non-zero module homomorphism $U\to V$ is an isomorphism. 
\end{exercise}

We now discuss a very useful application of Schur's lemma. 
Let $G$ be a finite group and $S$ be a simple $\C[G]$-module.
We claim that $\Hom_G(S,S)\simeq\C$. In fact, let 
$f\in\Hom_G(S,S)$ and $\lambda\in\C$ be an eigenvector of $f$. The such that 
$f-\lambda\id\colon S\to S$ is not invertible. By Schur's lemma, 
$f-\lambda\id=0$ and hence $f=\lambda\id$. 

\begin{theorem}[Schur]
\index{Schur's first orthogonality relation}
    Let $G$ be a finite group and $\chi,\psi\in\Irr(G)$. Then
    \[
    \langle\chi,\psi\rangle=\begin{cases}
    1 & \text{if $\chi=\psi$,}\\
    0 & \text{otherwise.}
    \end{cases}
    \]
\end{theorem}

\begin{proof}
    Let $S_1,\dots,S_k$ be the simples of $\C[G]$. Then 
    \[
    \langle\chi_i,\chi_j\rangle=\dim\Hom_G(S_i,S_j)
    =\begin{cases}
    1 & \text{if $S_i\simeq S_j$},\\
    0 & \text{otherwise.}
    \end{cases}
    \]
    But we know that $S_i\simeq S_j$ if and only if 
    $\chi$...
\end{proof}

With the theorem one can construct the character table of $\Sym_3$.
For example, this can be done using that $\langle\chi_3,\chi_3\rangle=1$ 
and that $\langle\chi_1,\chi_3\rangle=0$. 
As an exercise, check that the character table of $\Sym_3$ 
is given by
    \begin{center}
		\begin{tabular}{|c|ccc|}
			\hline
			& $1$ & $(12)$ & $(123)$ \tabularnewline
			\hline 
			$\chi_{1}$ & $1$ & $1$ & $1$\tabularnewline
			$\chi_{2}$ & $1$ & $-1$ & $1$ \tabularnewline
			$\chi_{3}$ & $2$ & 0 & -1 \tabularnewline
			\hline
		\end{tabular}
	\end{center}
	
\begin{exercise}
    Let $G$ be a finite group. 
    Prove that $\Irr(G)$ is an orthonormal basis of $\cf(G)$. 
\end{exercise}

The previous exercise has some consequences. Let $G$ be a finite group
and assume that $\Irr(G)=\{\chi_1,\dots,\chi_k\}$. If 
$\alpha=\sum a_i\chi_i$, then $\alpha=\sum\langle\alpha,\chi_i\rangle\chi_i$.  

\begin{theorem}
    Let $G$ be a finite group and $S_1,\dots,S_k$ be the simples of $G$. 
    Then 
    \[
    \C[G]\simeq\bigoplus_{i=1}^k(\dim S_i)S_i.
    \]
\end{theorem}

\begin{proof}
    Assume that $G=\{g_1,\dots,g_n\}$.
    Decompose the $\C[G]$-module corresponding 
    to the left regular representation as
    \[
    \C[G]\simeq a_1S_1\oplus\cdots\oplus a_kS_k
    \]
    for some integers $a_1,\dots,a_k\geq0$. Let $L\colon G\to G$, $g\mapsto L_g$, 
    where $L_g(g_j)=gg_j$ for all $j$. Since the matrix of $L_g$ in the basis
    $\{g_1,\dots,g_n\}$ is
    \begin{gather*}
    (L_g)_{ij}=\begin{cases}
        1 & \text{if $g_i=gg_j$},\\
        0 & \text{otherwise},
        \end{cases}
        \shortintertext{one obtains that}
    \chi_L(g)=\begin{cases}
    |G| & \text{if $g=1$},\\
    0 & \text{otherwise}.
    \end{cases}
    \end{gather*}
    Moreover, 
    \begin{gather*}
    \chi_L=\sum_{i=1}^ka_i\chi_i=\sum_{i=1}^k\langle\chi_L,\chi_i\rangle\chi_i
    \shortintertext{and}
    \langle\chi_L,\chi_i\rangle=\frac{1}{|G|}\sum_{g\in G}\chi_L(g)\overline{\chi_i(g)}
    =\frac{1}{|G|}|G|\overline{\chi_i(1)}=\dim S_i.
    \end{gather*}
    Thus $\C[G]\simeq\bigoplus_{i=1}^k(\dim S_i)S_i$. 
\end{proof}

If $G$ is a finite group, let $\Char(G)$
be the set of characters of $G$. 

\begin{exercise}
    Let $n\in\{1,2,3\}$. Let $G$ be a finite
    group and $\alpha\in\Char(G)$. Prove that
    $\alpha$ is the sum
    of $n$ irreducible characters if and only if $\langle\alpha,\alpha\rangle=n$.  
\end{exercise}
We now prove Schur's second orthogonality relation. 

\begin{theorem}[Schur]
\index{Schur's second orthogonality relation}
    Let $G$ be a finite group and $g,h\in G$. 
    Then
    \[
    \sum_{\chi\in\Irr(G)}\chi(g)\overline{\chi(h)}
    =\begin{cases}
    |C_G(g)| & \text{if $g$ and $h$ are conjugate},\\
    0 & \text{otherwise}.
    \end{cases}
    \]
\end{theorem}

\begin{proof}
    Let $g_1,\dots,g_r$ be the representative of conjugacy classes of $G$. 
    Assume that $\Irr(G)=\{\chi_1,\dots,\chi_r\}$. For each $k\in\{1,\dots,r\}$ 
    let $c_k=(G:C_G(g_k))$ denote the size of the conjugacy class of $g_k$. Then
    \[
    \langle\chi_i,\chi_j\rangle
    =\frac{1}{|G|}\sum_{g\in G}\chi_i(g)\overline{\chi_j(g)}
    =\frac{1}{|G|}\sum_{k=1}^rc_k\chi_i(g_k)\overline{\chi_j(g_k)}.
    \]
    We write this as $I=\frac{1}{|G|}XDX^*$, where $I$ denotes the identity matrix, 
    $X_{ij}=\chi_i(g_j)$, 
    $X^*=\overline{X}^T$ and 
    \[
    D=\begin{pmatrix}
    c_1\\
    &c_2\\
    &&\ddots\\
    &&&c_k
    \end{pmatrix}.
    \]
    Since, in matrices, $AB=I$ implies $BA=I$, it follows that
    $I=\frac{1}{|G|}X^*XD$. Thus, using that $|G|=c_k|C_G(g_k)|$ 
    holds for all $k$, 
    \[
    |G|D^{-1}=X^*X=\sum_{k=1}^r\overline{\chi_k(g_i)}\chi_k(g_j)
    =\begin{cases}
    |C_G(g_j)| & \text{if $i=j$},\\
    0 & \text{otherwise}.
    \end{cases}\qedhere
    \]
\end{proof}

\begin{theorem}[Solomon]
\index{Solomon's theorem}
    Let $G$ be a finite group and $\Irr(G)=\{\chi_1,\dots,\chi_r\}$. 
    If $g_1,\dots,g_r$ are the representatives of conjugacy classes
    of $G$ and $i\in\{1,\dots,r\}$, then 
    \[
    \sum_{j=1}^r\chi_i(g_j)\in\Z_{\geq0}.
    \]
\end{theorem}

\begin{proof}
    Let $V=\C[G]$ be the vector space with basis $\{e_g:g\in G\}$. 
    The action of $G$ on $G$ by conjugation induces a group homomorphism 
    $\rho\colon G\to\GL(V)$, $g\mapsto\rho_g$, where
    $\rho_g(e_h)=e_{ghg^{-1}}$. The matrix of $\rho_g$ 
    in the basis $\{e_g:g\in G\}$ is
    \[
    (\rho_g)_{ij}=\begin{cases}
        1 & \text{if $g_ig=gg_j$},\\
        0 & \text{otherwise}.
        \end{cases}
    \]
    Then
    \[
    \chi_{\rho}(g)=\trace\rho_g=\sum_{k=1}^{|G|}(\rho_g)_{kk}
    =|\{k:g_kg=gg_k\}|=|C_G(g)|.
    \]
    Write $\chi=\sum_{i=1}^rm_i\chi_i$ for $m_1,\dots,m_r\geq0$. 
    For each $j$ let $c_j=(G:C_G(g_j))$. Then
    \begin{align*}
    m_i=\langle\chi_{\rho},\chi_i\rangle
    &=\frac{1}{|G|}\sum_{g\in G}\chi_{\rho}(g)\overline{\chi_i(g)}\\
    &=\frac{1}{|G|}\sum_{j=1}^r c_j|C_G(g_j)|\overline{\chi_i(g_j)}
    =\sum_{j=1}^r\overline{\chi_i(g_j)}.\qedhere
    \end{align*}
\end{proof}

\topic{Algebraic numbers and characters}

\begin{definition}
    Let $\alpha\in\C$. We say that $\alpha$ is \textbf{algebraic}
    if $f(\alpha)=0$ for some monic polynomial $f\in\Z[X]$. 
\end{definition}

Let $\A$ be the set of algebraic numbers.

\begin{proposition}
    $\Q\cap\A=\Z$. 
\end{proposition}

\begin{proof}
    Let $m/n\in\Q$ with $\gcd(m,n)=1$ and $n>0$. If 
    $f(m/n)=0$ for some 
    \[
    f=X^k+a_{k-1}X^{k-1}+\cdots+a_1X+a_0\in\Z[X]
    \]
    of degree $k\geq1$, then
    \[
    0=n^kf(m/n)=m^k+a_{k-1}m^{k-1}n+\cdots+a_1mn^{k-1}+a_0n^k.
    \]
    This implies that 
    \[
        m^k=-n\left(a_{k-1}m^{k-1}+\cdots+a_1mn^{k-2}+a_0n^{k-1}\right)
    \]
    and hence $n$ divides $m^k$. Thus $n\in\{-1,1\}$ and 
    therefore $m/n\in\Z$.
\end{proof}

\begin{proposition}
    Let $x\in\C$. Then $x\in\A$ if and only if $x$ is an eigenvalue of
    an integer matrix.
\end{proposition}

\begin{proof}
    Let us prove the non-trivial implication. Let 
    \[
    f=X^n+a_{n-1}X^{n-1}+\cdots+a_0\in\Z[X]
    \]
    be such that $f(x)=0$. Then $x$ is an eigenvalue
    of the companion matrix of $f$, that is the matrix
    \[
    C(f)=
    \begin{pmatrix}
    0&0&\cdots &0&-a_{0}\\
    1&0&\cdots &0&-a_{1}\\
    0&1&\cdots &0&-a_{2}\\
    \vdots &\vdots &\ddots &\vdots &\vdots \\
    0&0&\cdots &1&-a_{{n-1}}
    \end{pmatrix}
    \in\Z^{n\times n}.\qedhere 
    \]
\end{proof}

\begin{theorem}
\label{thm:Asubring}
    $\A$ is a subring of $\C$. 
\end{theorem}

\begin{proof}
    Let $\alpha,\beta\in\A$. By the previous proposition, 
    $\alpha$ is an eigenvalue 
    of an integer matrix $A\in\Z^{n\times n}$, say
    $Av=\alpha v$, 
    $\beta$ is an eigenvalue of an integer matrix 
    $B\in\Z^{m\times m}$, say $Bw=\beta w$. Then
    \[
    (A\otimes I_{m\times m}+I_{n\times n}\otimes B)(v+w)
    =(\alpha+\beta)(v+w), 
    \]
    where $I_{k\times k}$ denotes the $(k\times k)$ identity 
    matrix, and
    \[
    (A\otimes B)(v\otimes w)=(\alpha\beta)v\otimes w.
    \]
    This implies that 
    $\alpha+\beta\in\A$ and $\alpha\beta\in\A$, again 
    by the previous proposition. 
\end{proof}

\begin{theorem}
\label{thm:A}
    Let $G$ be a finite group. If $\chi\in\Char(G)$ and
    $g\in G$, then $\chi(g)\in\A$. 
\end{theorem}

\begin{proof}
    Let $\varphi$ be a representation of $G$ such that 
    $\chi_\varphi=\chi$. Since $\varphi_g$ is diagonalizable with
    eigenvalues $\lambda_1,\dots,\lambda_k\in\A$ (because
    $G$ is finite and the $\lambda_j$ are roots of one), 
    \[
    \chi(g)=\trace\varphi_g=\sum_{i=1}^k\lambda_i\in\A. \qedhere
    \]
\end{proof}

\begin{theorem}
    Let $G$ be a finite group, $\chi\in\Irr(G)$ and $g\in G$. 
    If $K$ is the conjugacy class of $g$ in $G$, then
    \[
    \frac{\chi(g)}{\chi(1)}|K|\in\A. 
    \]
\end{theorem}

To prove the theorem we need a lemma. 

\begin{lemma}
    Let $x\in\C$. Then $x\in\A$ if and only if 
    there exist $z_1,\dots,z_k\in\C$ not all zero such that 
    $xz_i=\sum_{j=1}^ka_{ij}z_j$ for some $a_{ij}\in\Z$ and 
    all $i\in\{1,\dots,k\}$. 
\end{lemma}

\begin{proof}
    Let us first prove $\implies$. Let $f=X^k+a_{k-1}X^{k-1}+\cdots+a_1X+a_0\in\Z[X]$
    be such that $f(x)=0$. For $i\in\{1,\dots,k\}$ let 
    $z_i=x^{i-1}$. Then 
    $xz_i=x^i=z_{i+1}$ for all $i\in\{1,\dots,k-1\}$. Moreover, 
    $xz_k=x^k=-a_0-a_1x-\cdots-a_{k-1}x^{k-1}$.
    
    We now prove $\impliedby$. Let $A=(a_{ij})\in\Z^{k\times k}$ and 
    $Z$ be the column vector 
    $Z=\begin{pmatrix}z_1\\\vdots\\z_k\end{pmatrix}$. Note that $Z$ is non-zero. 
    Moreover, $AZ=xZ$, as 
    \[
    (AZ)_i=\sum_{j=1}^ka_{ij}z_j=xz_i=(xZ)_i
    \]
    for all $i$. Thus $x$ is an eigenvalue of $A\in\Z^{k\times k}$ and
    hence $x\in\A$. 
\end{proof}

We now prove the theorem. We will use the following notation: if $\chi$ is a character
of a group $G$ 
and $C$ is a conjugacy class of $G$, then 
$\chi(g)=\chi(xgx^{-1})$ for all $x\in G$. We write 
$\chi(C)$ to denote the value $\chi(g)$ for any $g\in C$. 

\begin{proof}[Proof of Theorem \ref{thm:A}]
    Let $\varphi$ be a representation of $G$ with character $\chi$. 
    Let $C_1,\dots,C_r$ be the conjugacy classes of $G$ 
    and for every $i\in\{1,\dots,r\}$ let 
    \[
    T_i=\sum_{x\in C_i}\varphi_x. 
    \]
    
    \begin{claim}
        $T_i=\left(\frac{|C_i|}{\chi(1)}\chi(C_i)\right)\id$. 
    \end{claim}
    
    We proceed in several steps. First we prove that 
    $T_i=\lambda\id$ for some $\lambda\in\C$. 
    We prove that $T_i$ is a morphism of representations:
    \[
    \varphi_gT_i\varphi_g^{-1}=\sum_{x\in C_i}\varphi_g\varphi_x\varphi_g^{-1}
    =\sum_{x\in C_i}\varphi_{gxg^{-1}}=\sum_{y\in C_i}\varphi_y=T_i.
    \]
    Now Schur's lemma implies that $T_i=\lambda\id$ for some
    $\lambda\in\C$. 
    
    We now prove that 
    \[
    \lambda=\frac{|C_i|\chi(C_i)}{\chi(1)}.
    \]
    To prove
    this we compute $\lambda$:
    \[
    \lambda\chi(1)=\trace(\lambda\id)
    =\trace T_i
    =\sum_{x\in C_i}\trace\varphi_x
    =\sum_{x\in C_i}\chi(x)
    =|C_i|\chi(C_i).
    \]
    From this the claim follows. 
    
    Now we claim that 
    \[
    T_iT_j=\sum_{k=1}^r a_{ijk}T_k
    \]
    for some $a_{ijk}\in\Z_{\geq0}$. In fact, 
    \begin{align*}
        T_iT_j &= \sum_{x\in C_i}\sum_{y\in C_j}\varphi_x\varphi_y
        =\sum_{x\in C_i}\sum_{y\in C_j}\varphi_{xy}
        =\sum_{g\in G}a_{ijg}\varphi_g,
    \end{align*}
    where $a_{ijg}$ is the number of elements $g\in G$ 
    that can be written 
    as $g=xy$ for $x\in C_i$ and $y\in C_j$. 
    
    \begin{claim}
        The $a_{ijg}$ depend only on the conjugacy class of $g$.
    \end{claim}
    
    Let $X_g=\{(x,y)\in C_i\times C_j:g=xy\}$. If $h=kgk^{-1}$, the map
    \[
    X_g\to X_h,\quad (x,y)\mapsto (kxk^{-1},kyk^{-1}),
    \]
    is well-defined. It is bijective with inverse
    \[
    X_h\to X_g,\quad
    (a,b)\mapsto (k^{-1}ak,k^{-1}bk).
    \]
    Hence $|X_g|=|X_h|$. 
    
    Now 
    \begin{align*}
        T_iT_j & 
        =\sum_{g\in G}a_{ijg}\varphi_g
        =\sum_{k=1}^r\sum_{g\in C_k}a_{ijg}\varphi_g
        =\sum_{k=1}^ra_{ijg}\sum_{g\in C_k}\varphi_g
        =\sum_{k=1}^ra_{ijk}T_k.
    \end{align*}
    Therefore 
    \begin{equation}
        \label{eq:omega}
    \left(\frac{|C_i|}{\chi(1)}\chi(C_i)\right)
    \left(\frac{|C_j|}{\chi(1)}\chi(C_j)\right)
    =\sum_{k=1}^r a_{ijk}\left(\frac{|C_k|}{\chi(1)}\chi(C_k)\right).
    \end{equation}
    By the previous lemma, $x=\frac{|C_j|}{\chi(1)}\chi(C_j)\in\A$.
\end{proof}

\topic{Frobenius' theorem}
\label{degree}

\begin{theorem}[Frobenius]
\index{Frobenius' theorem}
\label{thm:Frobenius_chi(1)}
    Let $G$ be a finite group and $\chi\in\Irr(G)$. 
    Then $\chi(1)$ divides~$|G|$. 
\end{theorem}

\begin{proof}
    Let $\varphi$ be an irreducible representation with character $\chi$. 
    Since $\langle\chi,\chi\rangle=1$, 
    \[
    \frac{|G|}{\chi(1)}=\frac{|G|}{\chi(1)}\langle\chi,\chi\rangle
    =\sum_{g\in G}\frac{\chi(g)}{\chi(1)}\overline{\chi(g)}.
    \]
    Let $C_1,\dots,C_r$ be the conjugacy classes of $G$. 
    Then 
    \[
        \frac{|G|}{\chi(1)}
        =\sum_{i=1}^r\sum_{g\in C_i}\frac{\chi(g)}{\chi(1)}\overline{\chi(g)}
        =\sum_{i=1}^r\left(\frac{|C_i|}{\chi(1)}\chi(C_i)\right)\overline{\chi(C_i)}\in\A\cap\Q=\Z,
    \]
    as $\overline{\chi(C_i)}\in\A$. This implies that $\chi(1)$ divides $|G|$. 
\end{proof}

The character table gives information of the structure of the group. For example,
with the previous result one can easily prove that
groups of order $p^2$ (where $p$ is a prime number) are abelian. 

\begin{exercise}
    Let $p$ and $q$ be prime numbers such that $p<q$.
    If $q\not\equiv1\bmod p$, then a group of order $pq$ is abelian. 
\end{exercise}

Another application:

\begin{theorem}
    Let $G$ be a finite simple group. 
    Then $\chi(1)\ne2$ for all $\chi\in\Irr(G)$. 
\end{theorem}

\begin{proof}
    Let $\chi\in\Irr(G)$ be such that $\chi(1)=2$. Let $\rho\colon G\to\GL_2(\C)$
    be an irreducible representation of $G$ with character $\chi$. Since 
    $G$ is simple, $\ker\rho=\{1\}$. Since $\chi(1)=2$, 
    $G$ is non-abelian and hence $[G,G]=G$. Since 
    $G$ has $(G:[G,G])=1$ degree-one characters, it follows that
    $G$ has only one degree-one character, the trivial one. The composition
    \[
    \begin{tikzcd}
    	G & {\GL_2(\C)} & {\C^{\times}}
    	\arrow["{\rho }", hook, from=1-1, to=1-2]
    	\arrow["{\det }", from=1-2, to=1-3]
    \end{tikzcd}
    \]
    is a degree-one representation, which means that $\det\rho_g=1$ for all $g\in G$. 
    By Frobenius' theorem, $|G|$ is even (because 
    $2=\chi(1)$ divides $|G|$). Let $x\in G$ be such that $|x|=2$ (Cauchy's theorem). 
    Then $|\rho_x|=2$, as $\rho$ is injective. Since $\rho_x$ is diagonalizable, 
    there exists $C\in\GL_2(\C)$ such that
    \[
    C\rho_xC^{-1}=\begin{pmatrix}
    \lambda&0\\
    0&\mu
    \end{pmatrix}
    \]
    for some $\lambda,\mu\in\{-1,1\}$. Since $1=\det\rho_x=\lambda\mu$ and
    $\rho$ is non-trivial, $\lambda=\mu=-1$. In particular, $C\rho_xC^{-1}$ is central
    and hence $\rho_x$ is central. Since $\rho$ is injective, $x$ is central 
    and thus $Z(G)\ne\{1\}$, a contradiction. 
\end{proof}

\section{Lecture: Week 5}

\subsection{McKay's conjecture}
\label{McKay}

Let $G$ be a finite group and let $p$ be a prime number dividing
$|G|$. Write $\Syl_p(G)$ to denote the (non-empty) set of Sylow 
$p$-subgroups of $G$. Recall that 
the \emph{normalizer} of $P$ is the subgroup
\[
N_G(P)=\{g\in G:gPg^{-1}=P\}.
\]

McKay made the following conjecture for the prime $p=2$ and simple groups 
and later generalized by Alperin in~\cite{MR0404417} and independently
by Isaacs in~\cite{MR332945}. 

\begin{conjecture}[McKay]
\index{McKay conjecture}
\label{conjecture:McKay}
Let $p$ be a prime. If  
$G$ is a finite group and $P\in\Syl_p(G)$, then 
\[
|\{\chi\in\Irr(G): p\nmid \chi(1)\}|
=|\{\psi\in\Irr(N_G(P)): p\nmid\psi(1)\}|.
\]
\end{conjecture}

Isaacs 
proved the conjecture for solvable groups; see~\cite{MR332945,MR3791517}. Using~\cite{MR2336079} and the classification of finite simple groups, Malle and Sp\"ath proved the conjecture for p = 2 in~\cite{MR3549625}.

% \begin{theorem}[Malle--Sp\"ath]
% \label{thm:MalleSpath}
% \index{Malle--Sp\"ath theorem}
% If $G$ is finite and $P\in\Syl_2(G)$,
% then 
% \[
% |\{\chi\in\Irr(G): 2\nmid \chi(1)\}|
% =|\{\psi\in\Irr(N_G(P)): 2\nmid\psi(1)\}|.
% \]
% \end{theorem}

% The proof appears in~\cite{MR3549625} and uses the classification of 
% finite simple groups. It uses a deep result of 
% Isaacs, Malle and Navarro~\cite{MR2336079}. 

In full generality, McKay's conjecture was
proved in 2024. 

\begin{theorem}[Cabanes--Sp\"ath]
\label{thm:CabanesSpath}
\index{Cabanes--Sp\"ath theorem}
If $G$ is finite, $p$ a prime number 
and $P\in\Syl_p(G)$,
then 
\[
|\{\chi\in\Irr(G): p\nmid \chi(1)\}|
=|\{\psi\in\Irr(N_G(P)): p\nmid\psi(1)\}|.
\]
\end{theorem}

We cannot prove Theorem~\ref{thm:CabanesSpath} here. However, 
we can use the computer to prove some particular cases
with the following function: 

\begin{lstlisting}
> McKay := function(G, p)
function> local N, n, m;
function> N := Normalizer(G, SylowSubgroup(G, p));
function> degG := CharacterDegrees(G);
function> degN := CharacterDegrees(N);
function> n := &+[ d[2] : d in degG | d[1] mod p ne 0 ];
function> m := &+[ d[2] : d in degN | d[1] mod p ne 0 ];
function> return n eq m;
function> end function;
\end{lstlisting}

As a concrete example, let us 
verify McKay's conjecture for the Mathieu simple group 
$M_{11}$ of order 7920. 

\begin{lstlisting}
> M11 := sub<Sym(11) | (1,10)(2,8)(3,11)(5,7), (1,4,7,6)(2,11,10,9)>;
> McKay(M11,2);
true
> McKay(M11,3);
true
> McKay(M11,5);
true
> McKay(M11,11);
true
\end{lstlisting}


% \begin{lstlisting}
% gap> McKay := function(G, p)
% > local N, n, m;
% > N := Normalizer(G, SylowSubgroup(G, p));
% > n := Number(Irr(G), x->Degree(x) mod p <> 0);
% > m := Number(Irr(N), x->Degree(x) mod p <> 0);
% > return n = m;
% > end;
% function( G, p ) ... end
% \end{lstlisting}



% \begin{lstlisting}
% gap> M11 := MathieuGroup(11);;
% gap> PrimeDivisors(Order(M11));
% [ 2, 3, 5, 11 ]
% gap> McKay(M11,2);
% true
% gap> McKay(M11,3);
% true
% gap> McKay(M11,5);
% true
% gap> McKay(M11,11);
% true
% \end{lstlisting}

\begin{bonus}
    Verify the McKay's conjecture for 
    all sporadic simple groups. 
\end{bonus}

The following conjecture refines McKay's conjecture. It was
formulated by Isaacs and Navarro:

\begin{conjecture}[Isaacs--Navarro]
\index{Isaacs--Navarro conjecture}
\label{conjecture:IsaacsNavarro}
Let $p$ be a prime and $k\in\Z$. 
If $G$ is a finite group and $P\in\Syl_p(G)$,
then
\begin{align*}
|\{\chi\in\Irr(G)&:p\nmid \chi(1)\text{ and }\chi(1)\equiv\pm k\bmod p\}|\\
&=|\{\psi\in\Irr(N_G(P)):p\nmid\psi(1)\text{ and }\psi(1)\equiv\pm k\bmod p\}|.
\end{align*}
\end{conjecture}

The Isaacs--Navarro conjecture is still open. However, 
it is known to be true for solvable groups, 
sporadic simple groups and 
symmetric groups, see~\cite{MR1935849}. 

% IsaacsNavarro := function(G, k, p)
%     local N, degG, degN, m, n;

%     // Compute the normalizer of a Sylow p-subgroup of G
%     N := Normalizer(G, SylowSubgroup(G, p));

%     // Get the character degrees along with their multiplicities
%     degG := CharacterDegrees(G);
%     degN := CharacterDegrees(N);

%     // Define the valid mod p range [-k, k] mod p
%     validDegrees := { (i mod p) : i in [-k..k] };

%     // Count characters in G with degrees not divisible by p and in the valid range
%     m := &+[ d[2] : d in degG | d[1] mod p ne 0 and (d[1] mod p) in validDegrees ];

%     // Count characters in N with degrees not divisible by p and in the valid range
%     n := &+[ d[2] : d in degN | d[1] mod p ne 0 and (d[1] mod p) in validDegrees ];

%     return n eq m;
% end function;

% \begin{lstlisting}
% gap> IsaacsNavarro := function(G, k, p)
% > local m, n, N;
% > N := Normalizer(G, SylowSubgroup(G, p));
% > m := Number(Filtered(Irr(G), x->Degree(x)\
% > mod p <> 0), x->Degree(x) mod p in [-k,k] mod p);
% > n := Number(Filtered(Irr(N), x->Degree(x)\
% > mod p <> 0), x->Degree(x) mod p in [-k,k] mod p);
% > return n=m;
% > end;
% function( G, k, p ) ... end
% \end{lstlisting}

\begin{bonus}
Verify the Isaacs--Navarro conjecture in some small  groups, such as the Mathieu simple group $M_{11}$. 
\end{bonus}

%It is an exercise to 

\subsection{Commutators}
\label{commutators}

Let $G$ be a finite group with conjugacy classes $C_1,\dots,C_s$. For
$i\in\{1,\dots,s\}$ and $\chi\in\Irr(G)$ let  
\[
\omega_{\chi}(C_i)=\frac{|C_i|\chi(C_i)}{\chi(1)}\in\A.
\]
In the proof of Theorem \ref{thm:B}, Equality \eqref{eq:omega}, 
we obtained
that 
\begin{equation}
\label{eq:again_omega}
\omega_\chi(C_i)\omega_\chi(C_j)=\sum_{k=1}^sa_{ijk}\omega_{\chi}(C_k),
\end{equation}
where $a_{ijk}$ is the number of solutions 
of $xy=z$ with $x\in C_i$, $y\in C_j$ and $z\in C_k$. 

\begin{theorem}[Burnside]
    \index{Burnside!theorem}
    Let $G$ be a finite group with conjugacy classes $C_1,\dots,C_s$. 
    Then
    \[
    a_{ijk}
    =\frac{|C_i||C_j|}{|G|}
    \sum_{\chi\in\Irr(G)}\frac{\chi(C_i)\chi(C_j)\overline{\chi(C_k)}}{\chi(1)}.
    \]
\end{theorem}

\begin{proof}
    By \eqref{eq:again_omega}, 
    \[
    \frac{|C_i||C_j|}{\chi(1)}\chi(C_i)\chi(C_j)
    =\sum_{k=1}^sa_{ijk}|C_k|\chi(C_k).
    \]
    Multiply by $\overline{\chi(C_l)}$ and sum over all
    $\chi\in\Irr(G)$ to obtain 
    \begin{align*}
    |C_i||C_j|\sum_{\chi\in\Irr(G)}\frac{\overline{\chi(C_l)}}{\chi(1)}\chi(C_i)\chi(C_j)
    &=\sum_{\chi\in\Irr(G)}\sum_{k=1}^sa_{ijk}|C_k|\chi(C_k)\overline{\chi(C_l)}\\
    &=\sum_{k=1}^sa_{ijk}|C_k|\sum_{\chi\in\Irr(G)}\chi(C_k)\overline{\chi(C_l)}\\
    &=a_{ijl}|G|,
    \end{align*}
    because 
    \[
    \sum_{\chi\in\Irr(G)}\chi(C_k)\overline{\chi(C_l)}=\begin{cases}
        \frac{|G|}{|C_l|} & \text{if $k=l$},\\
        0 & \text{otherwise}.
        \end{cases}\qedhere
    \]
\end{proof}

\begin{theorem}[Burnside]
\index{Burnside!theorem}
    Let $G$ be a finite group and $g,x\in G$. Then
    $g$ and $[x,y]$ are conjugate for some $y\in G$ if and only if
    \[
    \sum_{\chi\in\Irr(G)}\frac{|\chi(x)|^2\chi(g)}{\chi(1)}>0.
    \]
\end{theorem}

\begin{proof}
    Let $C_1,\dots,C_s$ be the conjugacy classes of $G$. Assume that
    $x\in C_i$ and $g\in C_k$ for some $i$ and $k$. Then
    $C_i^{-1}=\{z^{-1}:z\in C_i\}=C_j$ for some $j$. By Burnside's theorem,
    \[
    a_{ijk}
    =\frac{|C_i|^2}{|G|}\sum_{\chi\in\Irr(G)}\frac{|\chi(C_i)|^2\overline{\chi(C_k)}}{\chi(1)}.
    \]
    We first prove $\impliedby$. Since $a_{ijk}>0$, 
    there exist $u\in C_i$ and $v\in C_j$ such that 
    $g=uv$ (since $zgz^{-1}=u_1v_1$ for some $u_1\in C_i$ and
    $v_1\in C_j$, it follows that 
    $g=(z^{-1}u_1z)(z^{-1}v_1z)$, so take $u=z^{-1}u_1z\in C_i$ and 
    $v=z^{-1}v_1z\in C_j$). If $x$ and $u$ are conjugate, say 
    $u=zxz^{-1}$ for some $z$, then $x^{-1}$ and 
    $v$ are conjugate, as 
    \[
    zxz^{-1}=u\implies zx^{-1}z^{-1}=u^{-1}\in C_i^{-1}=C_j.
    \]
    Let $z_2\in G$ be such that $z_2x^{-1}z_2^{-1}=v$. 
    If $y=z^{-1}z_2$, then $g$ and $[x,y]$ are conjugate, as 
    \[
    g=uv=(zxz^{-1})(z_2x^{-1}z_2^{-1})=(zxyx^{-1}y^{-1})yz_2^{-1}
    =z[x,y]z^{-1}.
    \]
    
    We now prove $\implies$. Let $y\in G$ be such that $g$ and
    $[x,y]$ are conjugate, say $g=z[x,y]z^{-1}$ for some $z\in G$. Let
    $v=yxy^{-1}$. Then 
    $g$ and $xv^{-1}=xyx^{-1}y^{-1}=[x,y]$ are conjugate. In particular, 
    since $g\in C_iC_j$, $a_{ijk}>0$. 
\end{proof}    

\begin{exercise}
\label{xca:commutators}
    Let $G$ be a finite group, $g\in G$ and $\chi\in\Irr(G)$. 
    Prove that 
    \begin{align*}
        &\sum_{h\in G}\chi([h,g])=\frac{|G|}{\chi(1)}|\chi(g)|^2.
    \shortintertext{Prove also that }
        &\chi(g)\chi(h)=\frac{\chi(1)}{|G|}\sum_{z\in G}\chi(zgz^{-1}h)
    \end{align*}
    holds for all $h\in G$. 
\end{exercise}

% con este ejercicio se puede mejor el 
% teorema de Solomon; ver paper de Solomon en PAMS.
% agregar BONUS 

We now prove a theorem of Frobenius that 
uses character tables to recognize commutators. For that purpose, 
let 
\[
\tau(g)=|\{(x,y)\in G\times G:[x,y]=g\}|.
\]

\begin{theorem}[Frobenius]\label{thm:Frobenius_tau(g)}
    \index{Frobenius!theorem on involutions}
    Let $G$ be a finite group. Then
    \[
    \tau(g)=|G|\sum_{\chi\in\Irr(G)}\frac{\chi(g)}{\chi(1)}.
    \]
\end{theorem}

\begin{proof}
    Let $\chi\in\Irr(G)$. Since $\chi$ is irreducible, 
    \[
    1=\langle\chi,\chi\rangle
    =\frac{1}{|G|}\sum_{z\in G}\chi(z)\overline{\chi(z)}
    =\frac{1}{|G|}\sum_{C}|C|\chi(C)\overline{\chi(C)},
    \]
    where the last sum is taken over all conjugacy classes of $G$. 
    Let $C$ be the conjugacy class of $g$. The equation
    $xu^{-1}=g$ with $x\in C$ and $u\in C$ has 
    \[
        \frac{|C||C^{-1}|}{|G|}\sum_{\chi\in\Irr(G)}\frac{\chi(C)\chi(C^{-1})\chi(g^{-1})}{\chi(1)}
    \]
    solutions. If $(x,u)$ is a solution of $xu^{-1}=g$, then
    there are exactly $|C_G(x)|$ elements $y$ such that $yxy^{-1}=u$. In fact, since $x$ and $u$ are conjugate, there exists $y$ such that $yxy^{-1}=u$. And if $u=y_1xy_1^{-1}$ for some $y_1$, then 
    $y_1^{-1}y\in C_G(x)$ which implies that $y_1=y\xi$ for some $\xi\in C_G(x)$. Now 
    $[x,y]=x(yx^{-1}y^{-1})=g$ has 
    \[
    |C|\sum_{\chi}\frac{\chi(C)\chi(C^{-1})\chi(g^{-1})}{\chi(1)}
    \]
    solutions, where the sum is taken over all irreducible characters of $G$. 
    Now we sum over all conjugacy classes $C$ of $G$:
    \begin{align*}
        \sum_{C}\sum_{\chi}|C|\frac{\chi(C)\chi(C^{-1})\chi(g^{-1})}{\chi(1)}
        &=\sum_{\chi}\frac{\chi(g^{-1})}{\chi(1)}\left(\sum_C|C|\chi(C)\chi(C^{-1})\right)\\
        &=|G|\sum_{\chi}\frac{\chi(g^{-1})}{\chi(1)}.
    \end{align*}
    From this, the formula follows. 
\end{proof}

Application:

\begin{corollary}
    Let $G$ be a finite group and $g\in G$. Then $g$ 
    is a commutator if and only if 
    \[
    \sum_{\chi\in\Irr(G)}\frac{\chi(g)}{\chi(1)}\ne0.
    \]
\end{corollary}

\subsection{Ore's conjecture}
\label{Ore}

In 1951 Ore and independently It\^o
proved that every element of any alternating simple group is a commutator. 
Ore also mentioned that ``it is possible that a similar theorem holds for any simple group of finite order, but it seems that at present we do not have the necessary methods to investigate the question". 

\begin{conjecture}[Ore]
\index{Ore conjecture}
\label{conjecture:Ore}
    Let $G$ be a finite simple non-abelian group. 
    Then every element of $G$ is a commutator. 
\end{conjecture}

Ore's conjecture was proved in 2010:

\begin{theorem}[Liebeck--O'Brien--Shalev--Tiep]
\index{Liebeck--O'Brien--Shalev--Tiep theorem}
    Every element of a non-abelian finite simple group is a commutator.     
\end{theorem}

The proof appears in~\cite{MR2654085}. It needs about 70 pages and
uses the classification of finite simple groups (CFSG) and character theory.
See \cite{zbMATH06690771} for more information on Ore's conjecture and its proof. 

Although the proof of Ore's conjecture is too complicated for 
this course, we can use the computer to 
prove the conjecture in some particular cases:

\begin{bonus}
    Verify Ore's conjecture  
    for sporadic simple groups.    
\end{bonus}

% \begin{proposition}
%     Ore's conjecture is true 
%     for sporadic simple groups.
% \end{proposition}

% % IsaacsNavarro := function(G, k, p)
% %     local N, degG, degN, m, n;

% %     // Compute the normalizer of a Sylow p-subgroup of G
% %     N := Normalizer(G, SylowSubgroup(G, p));

% %     // Get the character degrees along with their multiplicities
% %     degG := CharacterDegrees(G);
% %     degN := CharacterDegrees(N);

% %     // Define the valid mod p range [-k, k] mod p
% %     validDegrees := { (i mod p) : i in [-k..k] };

% %     // Count characters in G with degrees not divisible by p and in the valid range
% %     m := &+[ d[2] : d in degG | d[1] mod p ne 0 and (d[1] mod p) in validDegrees ];

% %     // Count characters in N with degrees not divisible by p and in the valid range
% %     n := &+[ d[2] : d in degN | d[1] mod p ne 0 and (d[1] mod p) in validDegrees ];

% %     return n eq m;
% % end function;

% \begin{proof}
%     Let $G$ be a finite simple group. 
%     We know that $g\in G$ is a commutator if and only if 
%     $\sum_{\chi\in\Irr(G)}\frac{\chi(g)}{\chi(1)}\ne 0$. Let us write
%     a computer script to check whether every element in a group 
%     is a commutator. Our
%     function needs the character table of a group and returns 
%     \lstinline{true} if every element of the group is a commutator and
%     \lstinline{false} otherwise. 
% \begin{lstlisting}
% gap> Ore := function(char) 
% > local s,f,k;
% > for k in [1..NrConjugacyClasses(char)] do
% > s := 0;
% > for f in Irr(char) do
% > s := s+f[k]/Degree(f);  
% > od;
% > if s<=0 then
% > return false;
% > fi;
% > od;
% > return true;
% > end;
% function( char ) ... end
% \end{lstlisting}
% Now we check Ore's conjecture for Mathieu simple groups
% and for the Monster group: 
% \begin{lstlisting}
% gap> Ore(CharacterTable("M11"));
% true
% gap> Ore(CharacterTable("M12"));
% true
% gap> Ore(CharacterTable("M22"));
% true
% gap> Ore(CharacterTable("M23"));
% true
% gap> Ore(CharacterTable("M24"));
% true
% gap> Ore(CharacterTable("M"));
% true
% \end{lstlisting}
% It is an exercise to check the conjecture for the other finite sporadic 
% simple groups $McL$, $Ru$, $Ly$, $Suz$, $He$, $HN$, $Th$, $Fi_{22}$, $Fi_{23}$, $Fi_{24}'$, $B$, $M$ 
% \end{proof}

See \cite{MR3821142} for other applications of character theory. 

\subsection{The Cauchy--Frobenius--Burnside theorem}

The result we will now see is often called Burnside’s lemma. Burnside proved this lemma in his book on finite groups, attributing it to Frobenius. However, the formula was already known to Cauchy in 1845. Because of this, the result is sometimes referred to as the lemma that is not Burnside’s; see~\cite{MR562002}.

\begin{theorem}[Cauchy--Frobenius--Burnside]
\label{thm:CFB}
\index{Cauchy--Frobenius--Burnside theorem}
    Let $G$ be a finite group that acts on a finite set $X$. 
    If $m$ is the number of orbits, then 
    \[
    m=\frac{1}{|G|}\sum_{g\in G}|\Fix(g)|,
    \]
    where $\Fix(g)=\{x\in X:g\cdot x=x\}$. 
\end{theorem}

\begin{proof}
    Let $X=\{x_1,\dots,x_n\}$ and $V$ be the complex vector space with basis $\{x_1,\dots,x_n\}$. 
    Let $\rho\colon G\to\GL_n(\C)$, $g\mapsto\rho_g$, be the representation
    \[
    (\rho_g)_{ij}=\begin{cases}
        1 & \text{if $g\cdot x_j=x_i$},\\
        0 & \text{otherwise}.
        \end{cases}
    \]
    In particular, $(\rho_g)_{ii}=1$ if $x_i\in\Fix(g)$ and 
    $(\rho_g)_{ii}=0$ if $x_i \notin \Fix(g)$. Thus
    \[
    \chi_\rho(g)=\trace\rho_g=\sum_{i=1}^n(\rho_g)_{ii}=|\Fix(g)|.
    \]
    
    Recall that 
    \begin{gather*}
        V^G=\{v\in V:g\cdot v=v\text{ for all $g\in G$}\}
    \shortintertext{and that}    
        \dim V^G=\frac{1}{|G|}\sum_{z\in G}\chi_{\rho}(z)=\langle\chi_\rho,\chi_1\rangle
    \end{gather*}
    where $\chi_1$ is the trivial character of $G$ (see Lemma~\ref{lem:dimU^G}). 
    
    We can assume that, after a possible re-enumeration,
    $x_1,\dots,x_m$ are the representatives of the orbits 
    of $G$ on $X$. For $i\in\{1,\dots,m\}$, let
    $v_i=\sum_{x\in G\cdot x_i}x$.
    
    \begin{claim}
        $\{v_1,\dots,v_m\}$ is a basis of $V^G$. 
    \end{claim}
    
    If $g\in G$, then $g\cdot v_i=\sum_{x\in G\cdot x_i}g\cdot x=
    \sum_{y\in G\cdot x_i}y=v_i$. Hence $\{v_1,\dots,v_m\}\subseteq V^G$. Moreover, 
    $\{v_1,\dots,v_m\}$ is linearly independent because the $v_j$ are
    orthogonal and non-zero:
    \[
    \langle v_i,v_j\rangle=\begin{cases}
        |G\cdot x_i| & \text{if $i=j$},\\
        0 & \text{otherwise}.
        \end{cases}
    \]
    We now prove that $V^G=\langle v_1,\dots,v_m\rangle$. Let $v\in V^G$.
    Then $v=\sum_{x\in X}\lambda_xx$ for some coefficients $\lambda_x\in\C$.
    If $g\in G$, then $g\cdot v=v$. Since 
    \[
    \sum_{x\in X}\lambda_xx=v=g\cdot v
    =\sum_{x\in X}\lambda_x(g\cdot x)
    =\sum_{x\in X}\lambda_{g^{-1}\cdot x}x,
    \]
    it follows that $\lambda_x=\lambda_{g^{-1}\cdot x}$ for all $x\in X$ and 
    $g\in G$. This means that if $y,z\in X$ and $g\in G$ is such that
    $g\cdot y=z$, then $\lambda_y=\lambda_z$. Thus 
    \[
    v=\sum_{x\in X}\lambda_xx=\sum_{i=1}^m\lambda_{x_i}\sum_{y\in G\cdot x_i}y
    =\sum_{i=1}^m \lambda_{x_i}v_i.
    \]
    
    Hence 
    \[
    m=\dim V^G=\langle\chi_\rho,\chi_1\rangle=\frac{1}{|G|}\sum_{z\in G}\chi_\rho(z)
    =\frac{1}{|G|}\sum_{z\in G}|\Fix(z)|.\qedhere 
    \]
\end{proof}

\begin{optional}
    
It is possible to give an alternative short proof of the theorem. For example, 
for transitive actions (i.e., $m=1$), we proceed as follows:
\[
\sum_{g\in G}|\Fix(g)|=\sum_{g\in G}\sum_{\substack{x\in X\\g\cdot x=x}}1
=\sum_{x\in X}\sum_{\substack{g\in G\\g\cdot x=x}}1
=\sum_{x\in X}|G_x|=|G_{x_0}||X|=|G|,
\]
where $x_0\in X$ is any fixed element of $X$. I learned about this analytic number theory-style proof in Serre’s paper~\cite{MR1997347}.

\begin{xca}
\label{xca:CFB}
    Use the previous idea to prove Theorem \ref{thm:CFB}. 
\end{xca}

A probabilistic proof of Theorem~\ref{thm:CFB} is presented in~\cite{MR4043987}.
\end{optional}


\index{Orbital}
\index{Rank}
Let $G$ act on a finite set $X$. Then $G$ acts
on $X\times X$ by
\begin{equation}
    \label{eq:orbitals}
    g\cdot (x,y)=(g\cdot x,g\cdot y).
\end{equation}
The orbits of this action are called
the \emph{orbitals} of $G$ on $X$. The \emph{rank} 
of $G$ on $X$ is the number of orbitals. 

\begin{proposition}
    Let $G$ be a group that acts on a finite set $X$.
    The rank of $G$ on $X$~is 
    \[
    \frac{1}{|G|}\sum_{g\in G}|\Fix(g)|^2.
    \]
\end{proposition}

\begin{proof}
    The action \eqref{eq:orbitals} has 
    $\Fix(g)\times\Fix(g)$ as fixed points, as 
    \begin{align*}
        g\cdot (x,y)=(x,y)&\Longleftrightarrow
        (g\cdot x,g\cdot y)=(x,y)\\
        &\Longleftrightarrow g\cdot x=x\text{ and }g\cdot y=y\Longleftrightarrow
        (x,y)\in\Fix(g)\times\Fix(g).
    \end{align*}
    Now the claim follows from Cauchy--Frobenius--Burnside theorem. 
\end{proof}

\begin{definition}
    Let $G$ act on a finite set $X$. 
    We say that $G$ is \emph{2-transitive} on $X$ 
    if given $x,y\in X$ with $x\ne y$ and 
    $x_1,y_1\in X$ with $x_1\ne y_1$ there exists 
    $g\in G$ such that $g\cdot x=x_1$ and $g\cdot y=y_1$. 
\end{definition}

The symmetric group $\Sym_n$ acts 2-transitively on $\{1,\dots,n\}$. 

\begin{proposition}
    If $G$ is 2-transitive on $X$, then the rank of $G$ on $X$ is two. 
\end{proposition}

\begin{proof}
    The set $\Delta=\{(x,x):x\in X\}$ is an orbital. The complement
    $X\times X\setminus\Delta$ is another orbital: if $x,x_1,y,y_1\in X$
    are such that $x\ne y$ 
    and $x_1\ne y_1$, then there exists $g\in G$ such that 
    $g\cdot x=x_1$ and $g\cdot y=y_1$, so $g\cdot (x,y)=(x_1,y_1)$. 
\end{proof}

% \begin{example}
% \label{exa:A5_deg4}
%     The group $G=\Alt_5$ acts transitively on $X=\{1,\dots,5\}$. In the proof 
% of Theorem~\ref{thm:CFB} we have seen that 
% if $\rho\colon G\to\GL_n(\C)$, $g\mapsto\rho_g$, is the representation
%     \[
%     (\rho_g)_{ij}=\begin{cases}
%         1 & \text{if $g\cdot x_j=x_i$},\\
%         0 & \text{otherwise},
%         \end{cases}
%     \]
% then 
%    \[
%     \chi_\rho(g)=\trace\rho_g=\sum_{i=1}^n(\rho_g)_{ii}=|\Fix(g)|.
%     \]
% In particular, 
% $\langle\chi_\rho,\chi_1\rangle$ equals the number of orbits on $X$. Thus 
% $\langle\chi_\rho,\chi_1\rangle=1$ because $G$ acts transitively on $X$. This implies
% that $\chi_\rho-\chi_1$ is a linear combination of irreducible characters of $G$ 
% different from $\chi_1$. 

% \bigskip 
% \begin{center}
%         \begin{tabular}{|c|c|c|c|c|c|}
%         \hline  
%          $\chi_\rho$ & $5$ & $1$ & $2$ & $0$ & $0$\\
%          \hline 
%          $\chi_\rho-\chi_1$ & $4$ & $0$ & $1$ & $-1$ & $-1$\\
%          \hline 
% \end{tabular}
% \end{center}
% \bigskip 

% Since $\langle\chi_\rho-\chi_1,\chi_\rho-\chi_1\rangle=1$, it follows that 
% $\chi_\rho-\chi_1\in\Irr(G)$. 
% \end{example}
\section{Lecture: Week 6}

The Cauchy--Frobenius--Burnside theorem is helpful to
find characters. 

\begin{proposition}
\label{pro:2transitive}
    Let $G$ be 2-transitive on $X$ with character $\chi(g)=|\Fix(g)|$.
    Then $\chi-\chi_1$ is an irreducible character. 
\end{proposition}

\begin{proof}
    By assumption, $G$ is 2-transitive on $X$. In particular, $G$ is transitive on $X$. Let $\Irr(G)=\{\chi_1,\dots,\chi_k\}$, where
    $\chi_1$ is the trivial character. Since $\chi_1$ is irreducible, $\langle\chi_1,\chi_1\rangle=1$. 
    By the Cauchy--Frobenius--Burnside theorem, the rank of $G$ on $X$ is  
    \begin{gather*}
        2=\frac{1}{|G|}\sum_{g\in G}|\Fix(g)|^2=\langle \chi,\chi\rangle.
   \end{gather*}
   Moreover, again by the Cauchy--Frobenius--Burnside theorem, 
   \[
   \frac{1}{|G|}\sum_{g\in G}\chi(g)=1,
   \]
   since the action of $G$ on $X$ is, in particular, 
   transitive. 
   Thus 
   \begin{align*}
       \langle \chi-\chi_1,\chi-\chi_1\rangle
       &=\langle\chi,\chi\rangle-\langle\chi,\chi_1\rangle-\langle\chi_1,\chi\rangle+\langle\chi_1,\chi_1\rangle
       =2-\frac{2}{|G|}\sum_{g\in G}\chi(g)+1
       =1.
   \end{align*}
   Now write $\chi-\chi_1=\sum_{i=1}^ka_i\chi_i$ for
   some integers $a_1,\dots,a_k\in\Z$. Since 
   $a_1=\langle\chi,\chi_1\rangle$, it follows that 
   \[
   1=\left\langle\sum_{i=1}a_i\chi_i,\sum_{j=1}a_j\chi_j\right\rangle
   =\sum_{i=2}^ka_i^2. 
   \]
   Since $\chi$ is a character, $\chi-\chi_1$
   is an integer linear combination of the irreducible characters of $G$. Then there exists a unique $i\in\{2,\dots,k\}$ such that $a_i\in\{-1,1\}$ and $a_j=0$ for all $j\ne i$. 
   Hence $\chi-\chi_1=\pm\chi_i$. Since 
   $(\chi-\chi_1)(1)=|X|-1\geq 0$, it follows that $\chi-\chi_1=\chi_i$.
\end{proof}

\begin{example}
    The symmetric group $\Sym_n$ is 2-transitive on $\{1,\dots,n\}$. The
    alternating group $\Alt_n$ is 2-transitive on $\{1,\dots,n\}$ if 
    $n\geq4$. These groups then have an irreducible character $\chi$ 
    given by $\chi(g)=|\Fix(g)|-1$. 
\end{example}

%For the group $\Alt_5$
%the previous example produces the 
%following irreducible character:
% \bigskip 
% \begin{center}
%         \begin{tabular}{|c|c|c|c|c|c|}
%         \hline  
%          %$\chi_\rho$ & $5$ & $1$ & $2$ & $0$ & $0$\\
%          %\hline 
%          $\chi_\rho-\chi_1$ & $4$ & $0$ & $1$ & $-1$ & $-1$\\
%          \hline 
% \end{tabular}
% \end{center}
% \bigskip 
%
\begin{example}
    Let $p$ be a prime number and let $q=p^{m}$. Let $V$ 
    be the vector space of dimension $2$ 
    over the finite field of $q$ elements. 
    The group $G=\GL_2(q)$ acts 2-transitively on the set $X$ of
    one-dimensional subspaces of $V$. In fact, 
    if $\langle v\rangle\ne\langle v_1\rangle$ and $\langle w\rangle\ne\langle w_1\rangle$, 
    then $\{v,v_1\}$ and $\{w,w_1\}$ are bases of $V$. 
    The matrix $g$ that corresponds to the linear map 
    $v\mapsto w$, $v_1\mapsto w_1$, is invertible. Thus $g\in\GL_2(q)$. 
    The previous proposition produces the irreducible character
    $\chi(g)=|\Fix(g)|-1$. 
\end{example}

\begin{example}
    In how many ways can we color (in black and white) the vertices of a square? 
    We will count colorings up to symmetric. This means that, for example, 
    the colorings 
    \begin{equation}
    \label{eq:orbita}
    \begin{tikzpicture}
        \node[shape=circle,draw=black,fill=black] (A) at (1,0){};
        \node[shape=circle,draw=black,fill=black] (B) at (1,1){};
        \node[shape=circle,draw=black] (C) at (0,1){}; 
        \node[shape=circle,draw=black] (D) at (0,0){};
        \path [-](A) edge node[left]{} (B);
        \path [-](B) edge node[left]{} (C);
        \path [-](C) edge node[left]{} (D);
        \path [-](D) edge node[left]{} (A);
    \end{tikzpicture}
    \qquad
    \begin{tikzpicture}
        \node[shape=circle,draw=black] (A) at (1,0) {};
        \node[shape=circle,draw=black,fill=black] (B) at (1,1) {};
        \node[shape=circle,draw=black,fill=black] (C) at (0,1) {};
        \node[shape=circle,draw=black] (D) at (0,0) {};
        \path [-](A) edge node[left]{} (B);
        \path [-](B) edge node[left]{} (C);
        \path [-](C) edge node[left]{} (D);
        \path [-](D) edge node[left]{} (A);
    \end{tikzpicture}
    \qquad
    \begin{tikzpicture}
        \node[shape=circle,draw=black] (A) at (1,0) {};
        \node[shape=circle,draw=black] (B) at (1,1) {};
        \node[shape=circle,draw=black,fill=black] (C) at (0,1) {};
        \node[shape=circle,draw=black,fill=black] (D) at (0,0) {};
        \path [-](A) edge node[left]{} (B);
        \path [-](B) edge node[left]{} (C);
        \path [-](C) edge node[left]{} (D);
        \path [-](D) edge node[left]{} (A);
    \end{tikzpicture}
    \qquad
    \begin{tikzpicture}
        \node[shape=circle,draw=black,fill=black] (A) at (1,0) {};
        \node[shape=circle,draw=black] (B) at (1,1) {};
        \node[shape=circle,draw=black] (C) at (0,1) {};
        \node[shape=circle,draw=black,fill=black] (D) at (0,0) {};
        \path [-](A) edge node[left]{} (B);
        \path [-](B) edge node[left]{} (C);
        \path [-](C) edge node[left]{} (D);
        \path [-](D) edge node[left]{} (A);
    \end{tikzpicture}
\end{equation}
will be considered equivalent. Let $G=\langle g\rangle$ the cyclic 
group of order four. Let $X$ be 
the set of colorings of the square. Then 
$|X|=16$. 

Let $g$ act on $X$ by anti-clockwise rotations  
of 90\textdegree. All the colorings of~\eqref{eq:orbita} belong to the same orbit. 
Another orbit of $X$ is
\[
\begin{tikzpicture}
    \node[shape=circle,draw=black,fill=black] (A) at (1,0) {};
    \node[shape=circle,draw=black] (B) at (1,1) {};
    \node[shape=circle,draw=black,fill=black] (C) at (0,1) {};
    \node[shape=circle,draw=black] (D) at (0,0) {};
    \path [-](A) edge node[left]{} (B);
    \path [-](B) edge node[left]{} (C);
    \path [-](C) edge node[left]{} (D);
    \path [-](D) edge node[left]{} (A);
\end{tikzpicture}
\qquad
\begin{tikzpicture}
    \node[shape=circle,draw=black] (A) at (1,0) {};
    \node[shape=circle,draw=black,fill=black] (B) at (1,1) {};
    \node[shape=circle,draw=black] (C) at (0,1) {};
    \node[shape=circle,draw=black,fill=black] (D) at (0,0) {};
    \path [-](A) edge node[left]{} (B);
    \path [-](B) edge node[left]{} (C);
    \path [-](C) edge node[left]{} (D);
    \path [-](D) edge node[left]{} (A);
\end{tikzpicture}
\]

Cauchy--Frobenius--Burnside theorem states that
there are  
\[
\frac{1}{|G|}\sum_{x\in G}|\Fix(x)|
\]
orbits. 

For each $x\in G=\{1,g,g^2,g^3\}$ we compute $\Fix(x)$. The identity fixes 
the 16 elements of $X$, both 
$g$ and  $g^3$ fix only two elements of $X$ and 
$g^2$ fixes four elements of $X$. For example, 
the elements of $X$ fixed by $g^2$ are 
\[
\begin{tikzpicture}
    \node[shape=circle,draw=black] (A) at (1,0){};
    \node[shape=circle,draw=black] (B) at (1,1){};
    \node[shape=circle,draw=black] (C) at (0,1){}; 
    \node[shape=circle,draw=black] (D) at (0,0){};
    \path [-](A) edge node[left]{} (B);
    \path [-](B) edge node[left]{} (C);
    \path [-](C) edge node[left]{} (D);
    \path [-](D) edge node[left]{} (A);
\end{tikzpicture}
\qquad
\begin{tikzpicture}
    \node[shape=circle,draw=black,fill=black] (A) at (1,0) {};
    \node[shape=circle,draw=black,fill=black] (B) at (1,1) {};
    \node[shape=circle,draw=black,fill=black] (C) at (0,1) {};
    \node[shape=circle,draw=black,fill=black] (D) at (0,0) {};
    \path [-](A) edge node[left]{} (B);
    \path [-](B) edge node[left]{} (C);
    \path [-](C) edge node[left]{} (D);
    \path [-](D) edge node[left]{} (A);
\end{tikzpicture}
\qquad
\begin{tikzpicture}
    \node[shape=circle,draw=black,fill=black] (A) at (1,0) {};
    \node[shape=circle,draw=black] (B) at (1,1) {};
    \node[shape=circle,draw=black,fill=black] (C) at (0,1) {};
    \node[shape=circle,draw=black] (D) at (0,0) {};
    \path [-](A) edge node[left]{} (B);
    \path [-](B) edge node[left]{} (C);
    \path [-](C) edge node[left]{} (D);
    \path [-](D) edge node[left]{} (A);
\end{tikzpicture}
\qquad
\begin{tikzpicture}
    \node[shape=circle,draw=black] (A) at (1,0) {};
    \node[shape=circle,draw=black,fill=black] (B) at (1,1) {};
    \node[shape=circle,draw=black] (C) at (0,1) {};
    \node[shape=circle,draw=black,fill=black] (D) at (0,0) {};
    \path [-](A) edge node[left]{} (B);
    \path [-](B) edge node[left]{} (C);
    \path [-](C) edge node[left]{} (D);
    \path [-](D) edge node[left]{} (A);
\end{tikzpicture}
\]
Thus $X$ is the union of  
\[
\frac{1}{|G|}\sum_{x\in G}|\Fix(x)|=\frac{1}{4}(16+2+4+2)=6
\]
orbits. 
\end{example}

\begin{bonus}
    In how many ways (up to symmetry) can you
    arrange eight non-attacking rooks on a chessboard? Symmetries 
    are given by the dihedral group $\D_4$ of eight elements.
\end{bonus}

There are 5282 ways (up to symmetry) to arrange 
eight non-attacking rooks on a chessboard. 

\subsection{Commuting probability}

\index{Group commutativity}
For a finite group $G$, let $\cp(G)$ be the probability 
that two random elements of $G$ commute. This number
is also known as the \emph{commutativity} of $G$. 
As an application of Cauchy--Frobenius--Burnside theorem, we
prove that 
$\cp(G)=k/|G|$, where $k$ is the number of conjugacy classes
of $G$. Let 
\[
C=\{(x,y)\in G\times G:xy=yx\}.
\]
We claim that  
    \[
    \cp(G)=\frac{|C|}{|G|^2}=\frac{k}{|G|}.
    \]

Let $G$ act on $G$ by conjugation. 
    By Cauchy--Frobenius--Burnside theorem, 
    \[
    k=\frac{1}{|G|}\sum_{g\in G}|\Fix(g)|=\frac{1}{|G|}\sum_{g\in G}|C_G(g)|=\frac{|C|}{|G|},
    \]
    as $\Fix(g)=\{x\in G:gxg^{-1}=x\}=C_G(g)$ and $\sum_{g\in G}|C_G(g)|=|C|$. 
Alternatively, using Theorem \ref{thm:Frobenius_tau(g)} with $g = 1$,
\[
    \cp(G) = \frac{\tau(1)}{|G|^2} = \frac{1}{|G|}\sum_{\chi \in \Irr(G)} 1 = \frac{k}{|G|},
\]
as $k=|\Irr(G)|$. 
\begin{theorem}
\index{Theorem!5/8}
\label{thm:5/8}
    If $G$ is a non-abelian finite group, then $\cp(G)\leq5/8$.
\end{theorem}

\begin{proof}
%
%    We now claim that $k/|G|\leq 5/8$ if $G$ is non-abelian.
    Let $y_1,\dots,y_m$ the representatives of conjugacy classes of $G$ 
    of size $\geq2$. By the class equation, 
    \[
    |G|=|Z(G)|+\sum_{i=1}^m(G:C_G(y_i))\geq |Z(G)|+2m.
    \]
    Thus $m\leq(1/2)(|G|-|Z(G)|)$ and hence 
    \[
    k=|Z(G)|+m\leq |Z(G)|+\frac12(|G|-|Z(G)|)=\frac12(|Z(G)|+|G|).
    \]
    Since $G$ is non-abelian, $G/Z(G)$ is not cyclic. In particular, 
    $(G:Z(G))\geq4$. Therefore
    \[
    k\leq\frac12(|Z(G)|+|G|)\leq\frac12\left(\frac14+1\right)|G|,
    \]
    that is $k/|G|\leq 5/8$. 
\end{proof}

\begin{exercise}\
\begin{enumerate}
    \item Prove that $\cp(Q_8)=5/8$. 
    \item Prove that $\cp(\Alt_5)=1/12$. 
\end{enumerate}
\end{exercise}

\begin{exercise}
\label{xca:least_p}
    Let $G$ be a finite non-abelian group and $p$ be the smallest prime number
    dividing $|G|$. Prove that $\cp(G)\leq (p^2+p-1)/p^3$. 
    %Moreover, 
    %the equality holds if and only if $(G:Z(G))=p^2$. 
\end{exercise}

\begin{bonus}
\label{xca:cp(G)}
    Let $G$ be a finite group and $H$ be a subgroup of $G$.
    \begin{enumerate}
        \item $\cp(G)\leq\cp(H)$.
        \item If $H$ is normal in $G$, then $\cp(G)\leq\cp(G/H)\cp(H)$.
    \end{enumerate}
\end{bonus}

\index{Cauchy--Schwarz inequality}
For the next proposition, which provides a lower bound for the commuting probability, we will use
the \emph{Cauchy–Schwarz inequality}:
\[
x_1,\dots,x_n\in\R\implies
\sum x_i^2\geq\frac{1}{n}(\sum x_i)^2.
\]

%Degrees of irreducible characters give a lower bound. 


\begin{proposition}
If $G$ is a finite group, then
\[
\cp(G)\geq\left(\frac{\sum_{\chi\in\Irr(G)}\chi(1)}{|G|}\right)^2.
\]
\end{proposition}

\begin{proof}
    Let $k$ be the number of conjugacy classes of $G$.
    By the Cauchy--Schwarz inequality, 
    \begin{align*}
        \left(\sum_{\chi\in\Irr(G)}\chi(1)\right)^2
        &\leq\left(\sum_{\chi\in\Irr(G)}\chi(1)^2\right)\left(\sum_{\chi\in\Irr(G)}1\right)
        %=\left(\sum_{\chi\in\Irr(G)}\chi(1)^2\right)k
        =|G|k.
    \end{align*}
    From this, the claim follows.
\end{proof}

Using basic facts about irreducible characters, we derive a generalization of Theorem \ref{thm:5/8}.

\begin{theorem}
\label{thm:[GG]}
    Let $G$ be a finite group. Then
    \[
    %|[G,G]|\leq 3/(4\cp(G)-1).
        \cp(G)\leq\frac14\left(1+\frac3{|[G,G]|}\right).
    \]
\end{theorem}

\begin{proof}
    For $n\in\Z_{>0}$, let $\rho_n$ be the number
    of irreducible characters of degree $n$. Then 
    the number of conjugacy classes of $G$ is $k=\sum_{i\geq1}\rho_i$
    and $|G|=\sum_{i\geq1}i^2\rho_i$. 
    It follows that 
    \begin{align*}
    |G|-\rho_1 &=\sum_{i\geq 2}i^2\rho_i\geq 4\sum_{i\geq2}\rho_i
    =4(k-\rho_1)=4(|G|\cp(G)-\rho_1).
    \end{align*}
    Since $\rho_1=(G:[G,G])$, 
    \[
    \cp(G)\leq \frac14+\frac34\frac{\rho_1}{|G|}=\frac14+\frac3{4|[G,G]|}.\qedhere 
    \]
\end{proof}

\begin{exercise}
    \label{xca:5/8}
    Use Theorem \ref{thm:[GG]} to prove Theorem \ref{thm:5/8}.
\end{exercise}

Theorem \ref{thm:[GG]} 
can also be used to
prove similar statements. 

\begin{exercise}
    \label{xca:cp_NS}
    Let $G$ be a finite group. Prove the following statements:
    \begin{enumerate}
        \item If $\cp(G)>1/2$, then $G$ is nilpotent.
        \item If $\cp(G)>21/80$, then $G$ is solvable. 
    \end{enumerate}
\end{exercise}

In the following exercise, we will discuss the notion 
of isoclinic groups. We first need
a preliminary result:

\begin{exercise}
\index{Commutator map}
\label{xca:commutator_map}
    Let $G$ be a group. Prove that the commutator map
    \[
    c_G\colon G/Z(G)\times G/Z(G)\to [G,G],
    \quad
    c_G(xZ(G),yZ(G))=[x,y],
    \]
    is well-defined. 
\end{exercise}

The idea is that two groups are said to be isoclinic 
if their commutator functions are somewhat equal. 

\begin{exercise}
\index{Isoclinism}
\label{xca:isoclinism}
    Let $G$ and $H$ be groups. 
    A pair $(\sigma,\tau)$ of maps is an \emph{isoclinism}
    between $G$ and $H$ if 
    $\sigma\colon G/Z(G)\to H/Z(H)$ and  
    $\tau\colon [G,G]\to [H,H]$ are group isomorphisms and 
    the diagram
    \begin{equation}
    \label{eq:isoclinism}
    \begin{tikzcd}
	{G/Z(G)\times G/Z(G)} & {H/Z(H)\times H/Z(H)} \\
	{[G,G]} & {[H,H]}
	\arrow["{\sigma\times\sigma }", from=1-1, to=1-2]
	\arrow["{c_H}", from=1-2, to=2-2]
	\arrow["{c_G}"', from=1-1, to=2-1]
	\arrow["\tau"', from=2-1, to=2-2]
    \end{tikzcd}
    \end{equation} 
    commutes. We write $G\sim H$ when there exists 
    an isoclinism between $G$ and $H$. 
    
    Prove the following statements:
    \begin{enumerate}
        \item If $G\simeq H$, then $G\sim H$.
        \item If $G$ and $H$ are finite groups such that $G\sim H$, then $\cp(G)=\cp(H)$. 
    \end{enumerate}
\end{exercise}

\begin{exercise}
\label{xca:isoclinism_simple}
    Let $S$ be a non-abelian simple group and
    $G$ be a group such that $G\sim S$. Prove that 
    $G\simeq S\times A$ for some abelian group $A$.
\end{exercise}

\begin{exercise}
\label{xca:isoclinism_factorization}
    Let $H$ be a subgroup of $G$. If $G=HZ(G)$, then $G\sim H$. 
    Conversely, if $G\sim H$ and $H$ is finite, then 
    $G=HZ(G)$. 
\end{exercise}

The following theorem appeared in 1970 as a problem in 
volume 13 of the \textit{Canadian Math. Bulletin}. The solution
appeared in 1973. 
Iv\'an Sadosfchi Costa found in 2018 
the proof we present here. 

\begin{theorem}[Dixon]
    \index{Dixon theorem}
    \label{thm:Dixon}
    The commuting probability of every finite 
    non-abelian simple group is at most $1/12$. 
   %If $G$ is a finite non-abelian simple group, then $\cp(G)\leq1/12$.
\end{theorem}

\begin{proof}[Sketch of the proof]
Let $G$ be a finite non-abelian simple group. We claim that 
$\cp(G)\leq1/12$. 
We assume that $\cp(G)>1/12$. Since
$G$ is a non-abelian simple group, 
the identity of $G$ is the only central element of $G$. 

Let us assume first that there is a conjugacy class of $G$ of size $m$, where
$m$ is such that $1<m\leq 12$. Then $G$ is a transitive subgroup of $\Sym_m$.
For these groups, the problem is easy: we show that there are no non-abelian simple groups
that act transitively on sets of size $m\in\{2,\dots,12\}$ with commuting
probability $>1/12$. To do this, we list these transitive groups and their commuting
probabilities and verify that all commuting probabilities are $\leq
1/12$. This is left as an exercise. 
% \begin{lstlisting}
% gap> l := AllTransitiveGroups(NrMovedPoints, [2..12], \\
% > IsAbelian, false, IsSimple, true);;
% [ A5, L(6) = PSL(2,5) = A_5(6), A6, 
%   L(7) = L(3,2), A7, L(8)=PSL(2,7), A8, 
%   L(9)=PSL(2,8), A9, A_5(10), L(10)=PSL(2,9), 
%   A10, L(11)=PSL(2,11)(11), M(11), A11, A_5(12), 
%   L(2,11), M_11(12), M(12), A12 ]
% gap> List(l, CommutingProbability);           
% [ 1/12, 1/12, 7/360, 1/28, 1/280, 1/28, 1/1440, 
%   1/56, 1/10080, 1/12, 7/360, 1/75600, 2/165, 
%   1/792, 31/19958400, 1/12, 2/165, 1/792, 1/6336, 
%   43/239500800 ]
% gap> ForAny(l, x->CommutingProbability(x)>1/12);
% false
% \end{lstlisting}

Now assume that all non-trivial conjugacy classes of $G$ have at least 13 elements. Let $k$ be the number of conjugacy classes of $G$. 
Then the class equation implies that
\begin{align*}
	|G|&\geq 1+(k-1)13=13k-12.
\end{align*}
Since $\cp(G)=k/|G|>1/12$, $k>|G|/12$. Thus 
\[
|G|>\frac{13}{12}|G|-12
\]
and therefore $|G|<144$. Thus one needs to check what happens with groups
of order $<144$. 
But we know that the only non-abelian simple group of size
$<144$ is the alternating simple group $\Alt_5$. 
% \begin{lstlisting}
% gap> AllGroups(Size, [2..143], \\
% > IsAbelian, false, \\
% > IsSimple, true);
% [ Alt( [ 1 .. 5 ] ) ]
% \end{lstlisting}    
This completes the proof. 
\end{proof}

\begin{bonus}
    Provide the details of the proof of Theorem~\ref{thm:Dixon}.
\end{bonus}

The alternating group $\Alt_5$ is important in this setting:

\begin{theorem}[Guralnick--Robinson]
    \index{Guralnick--Robinson theorem}
    If $G$ is a finite non-solvable group such that $\cp(G)>3/40$, then
    $G\simeq\Alt_5\times T$ for some abelian group 
    $T$ and $\cp(G)=1/12$. 
\end{theorem}

The proof appears in~\cite{MR2228209}.

Results on probability of commuting elements generalize in other directions. 
In~\cite{MR230809,MR276325,MR313378,MR369512}, 
Thompson proved the following result:

\begin{theorem}[Thompson]
\index{Thompson theorem}
    If $G$ is a finite group such that 
    every pair of elements of $G$ generate
    a solvable group, then $G$ is solvable. 
\end{theorem}

The proof uses the classification of finite simple groups (CFSG). A simpler
proof independent of the CFSG appears in~\cite{MR1346207}.

There is a probabilistic version of Thompson's theorem:

\begin{theorem}[Guralnick--Wilson]
    \index{Guralnick--Wilson theorem}
    Let $G$ be a finite group.
    \begin{enumerate}
        \item If the probability that two random elements of $G$ 
        generate a solvable group is $>11/30$, then $G$ is solvable. 
        \item If the probability that two random elements of $G$ 
        generate a nilpotent group is $>1/2$, then $G$ is nilpotent.
        \item If the probability that two random elements of $G$ 
        generate a group of odd order is $>11/30$, then $G$ has odd order.
    \end{enumerate}
\end{theorem}

The proof uses the CFSG and appears in~\cite{MR1770615}.

\subsection{Jordan's theorem and applications}

We now follow~\cite{MR1997347} to present other applications. 

\begin{theorem}[Jordan]
\index{Jordan theorem}
    Let $G$ be a non-trivial finite group. If $G$ acts transitively 
    on a finite set $X$ and $|X|>1$, then there exists 
    $g\in G$ with no fixed points.
\end{theorem}

\begin{proof}
    The Cauchy--Frobenius--Burnside theorem implies that
    \[
    1=\frac{1}{|G|}\sum_{g\in G}|\Fix(g)|=\frac{1}{|G|}\left(|X|+\sum_{g\ne 1}|\Fix(g)|\right).
    \]
    If every $g\in G\setminus\{1\}$ contains at least one fixed-point, then
    \[
    1=\frac{1}{|G|}\left(|X|+\sum_{g\ne 1}|\Fix(g)|\right)\geq \frac{1}{|G|}(|X|+|G|-1)=1+\frac{|X|-1}{|G|}
    \]
    and thus $|X|\leq1$, a contradiction. 
\end{proof}

\begin{corollary}
\label{cor:Jordan}
    Let $G$ be a finite group and $H$ be a proper subgroup of $G$. 
    Then $G\ne\bigcup_{g\in G}gHg^{-1}$.
\end{corollary}

\begin{proof}
    The group $G$ acts transitively by left multiplication on $X=G/H$. The stabilizer
    of $xH$ is 
    \[
    G_{xH}=\{g\in G:gxH=xH\}=xHx^{-1}.
    \]
    Since $H\ne G$, it follows that $|X|=|G/H|>1$. Jordan's theorem now implies
    that there exists $g\in G$ with no fixed-points, that is 
    there is an element $g\in G$ such that $g\not\in\bigcup_{x\in G}xHx^{-1}$. 
\end{proof}

Let $G$ be a finite group. We say that the conjugacy classes $C$ and $D$ 
\emph{commute} if there exist 
$c\in C$ and $d\in D$ such that $[c,d]=1$. 
Note that $C$ and $D$ commute if and only if for all $c\in C$ there exists $d\in D$ 
such that $[c,d]=1$. 

\begin{corollary}[Wildon]
\index{Wildon theorem}
    Let $G$ be a finite group and $C$ be a conjugacy class of $G$. 
    Then $|C|=1$ if and only if $C$ commutes 
    with every conjugacy class of $G$.
\end{corollary}
    
\begin{proof}
    We prove $\impliedby$. 
    Assume that $C$ commutes with every conjugacy class of $G$. 
    Let $c\in C$ and $H=C_G(c)$. Then $H\cap D\ne\emptyset$ for every conjugacy class
    $D$. We claim that $G=\bigcup_{g\in G}gHg^{-1}$. In fact, let $x\in G$. Then
    $x\in D$ 
    for some conjugacy class $D$. 
    Let 
    $h\in H\cap D$. There exists $y\in G$ such that $h=yxy^{-1}$, that is
    $x=y^{-1}hy\in \bigcup_{g\in G}gHg^{-1}$. By Corollary~\ref{cor:Jordan},  
    $H=G$. Thus $c$ is central and hence $C=\{c\}$. 
    
    We now prove $\implies$. If $C=\{c\}$, then $c\in Z(G)$ and $C$ commute with every 
    conjugacy class of~$G$. 
\end{proof}

There is a theorem similar to Jordan’s.

\begin{theorem}[Fein--Kantor--Schacher]
    \index{Fein--Kantor--Schacher theorem}
    \label{thm:FKS}
    Let $G$ be a non-trivial finite group. If $G$ acts transitively
    on a finite set $X$ and $|X|>1$, then
    there exist a prime number $p$ and an element $g\in G$ with no fixed-points
    with order a power of $p$.
\end{theorem}

The proof appears in~\cite{MR636194} and depends on the CFSG. It is unknown whether a proof of Theorem~\ref{thm:FKS} without relying on the CFSG exists.

\subsection{Derangements: The Cameron--Cohen theorem}

\index{Derangements}
Let $G$ be a finite group that acts faithfully and transitively 
on a finite set $X$, say 
$G\leq\Sym_n$, where $X=\{1,2,\dots,n\}$. Let 
$G_0$ be the set of elements $g\in G$ with no fixed-points, 
that is $g(x)\ne x$ for all $x\in X$. 
Such permutations are known as \emph{derangements}. 

\begin{example}
Let $G=\Sym_3$. Then $G_0=\{(123),(132)\}$.
\end{example}

\begin{example}
Let $G=\Sym_4$. Then 
    \[
    G_0=\{(12)(34),(13)(24),(14)(23),(1234),(1243),(1324),(1342),(1423),(1432)\}.
    \]
\end{example}

We want to estimate the number of derangements. For this purpose, let $c_0 = |G_0| / |G|$.

% The number of derangements of a set of size $n$ is known as the \emph{subfactorial} $!n$ of $n$. Some values of the number
% of derangements are
% \[
% 1,2,9,44,265,1854,14833\cdots 
% \]

% How to estimate the number of deragements in an arbitrary 
% permutation group? 

\begin{theorem}[Cameron--Cohen]
    \index{Cameron--Cohen!theorem}
    \label{thm:CameronCohen}
    If $G$ is a subgroup of $\Sym_n$ that acts transitively on 
    $\{1,\dots,n\}$, then $c_0\geq\frac{1}{n}$.
\end{theorem}

\begin{proof}
    Let $X=\{1,\dots,n\}$. By definition, the rank of $G$ is the number
    of orbitals of $G$ on $X$. It follows that the rank is $\geq2$, as
    $X\times X$ decomposes as 
    \[
    X\times X=\Delta\cup\left((X\times X)\setminus\Delta\right)
    \]
    Let $\chi(g)=|\Fix(g)|$ and $G_0=\{g\in G:\chi(g)=0\}$. If $g\not\in G_0$, then $1\leq\chi(g)\leq n$. Since  
    $(\chi(g)-1)(\chi(g)-n)\leq 0$,
    \[
    \frac{1}{|G|}\sum_{g\in G\setminus G_0}(\chi(g)-1)(\chi(g)-n)\leq 0.
    \]
    On the one hand, 
    \begin{align*}
    \frac{1}{|G|}\sum_{g\in G}(\chi(g)&-1)(\chi(g)-n)\\
    &=\frac{1}{|G|}\left\{\sum_{g\in G_0}+\sum_{g\in G\setminus G_0}\right\}(\chi(g)-1)(\chi(g)-n)\\
    &=\frac{1}{|G|}\sum_{g\in G_0}(\chi(g)^2-(n+1)\chi(g)+n)
    +\underbrace{\frac{1}{|G|}\sum_{g\in G\setminus G_0}(\chi(g)-1)(\chi(g)-n)}_{\leq0}\\
    &\leq n\frac{|G_0|}{|G|}=nc_0.
    \end{align*}
    On the other hand, since the rank of $G$ is $\geq2$, 
    \begin{align*}
        \frac{1}{|G|}\sum_{g\in G}(\chi(g)-1)(\chi(g)-n)
        &=\frac{1}{|G|}\sum_{g\in G}(\chi(g)^2-(n+1)\chi(g)+n)\\
        &\geq 2-\frac{n+1}{|G|}\sum_{g\in G}\chi(g)+n
        %\leq 
        %\frac{1}{|G|}\sum_{g\in G}(\chi(g)-1)(\chi(g)-n)\leq nc_0.
    \end{align*}
    Since $G$ is transitive on $X$, the Cauchy--Frobenius--Burnside theorem implies that
    \[
    \sum_{g\in G}\chi(g)=|G|.
    \]
    Thus $2-(n+1)+n\leq nc_0$ and hence
    $1/n\leq c_0$. 
\end{proof}

The Cameron--Cohen theorem contains another claim: If
$n$ is not the power of a prime number, then 
$c_0>1/n$ (see~Theorem~\ref{thm:CameronCohen>1/n}). The proof uses Frobenius' theorem. 

With the CFSG the bound in the 
Cameron--Cohen theorem can be improved:

\begin{theorem}[Guralnick--Wan]
    \index{Guralnick--Wan theorem}
    Let $G$ be a finite transitive group of degree $n\geq2$. If $n$ 
    is not a power of a prime number and 
    $G\ne\Sym_n$ for $n\in\{2,4,5\}$, then $c_0\geq 2/n$.
\end{theorem}

The proof appears in~\cite{MR1484879} and uses
the classification of finite 2-transitive groups, 
which depends on the CFSG. 




\section{Lecture: Week 7}

\subsection{The Brauer--Fowler theorem}

\index{Symmetric}
\index{Antisymmetric}
Let $\rho\colon G\to\GL(V)$ 
be a representation with character $\chi$. The $\C[G]$-module $V\otimes V$ 
has character $\chi^2$. Let 
$\{v_1,\dots,v_n\}$ be a basis of $V$ and 
\[
T\colon V\otimes V\to V\otimes V,\quad
v_i\otimes v_j\mapsto v_j\otimes v_i.
\]
It is an exercise to check that 
\[
T(v\otimes w)=w\otimes v
\]
for all 
$v,w\in V$. (Thus 
$T$ does not depend on the basis $\{v_1,\dots,v_n\}$.) Note that
$T$ is a homomorphism of $\C[G]$-modules, as
\[
T(g\cdot (v\otimes w))=T((g\cdot v)\otimes (g\cdot w))=(g\cdot w)\otimes (g\cdot v)=g\cdot T(v\otimes w)
\]
for all $g\in G$ y $v,w\in V$. 
In particular, the \emph{symmetric part} 
\begin{gather*}
S(V\otimes V)=\{x\in V\otimes V:T(x)=x\}
\shortintertext{and the \emph{antisymmetric} part}
A(V\otimes V)=\{x\in V\otimes V:T(x)=-x\}
\end{gather*}
of $V\otimes V$ are both  
$\C[G]$-submodules of $V\otimes V$. 
The terminology is motivated by the following fact:
\[
V\otimes V=S(V\otimes V)\oplus A(V\otimes V).
\]
In fact, 
$S(V\otimes V)\cap A(V\otimes V)=\{0\}$, as   
$x\in S(V\otimes V)\cap A(V\otimes V)$ implies
$x=T(x)$ and $x=-T(x)$. Hence $x=0$. Moreover, 
$V\otimes V=S(V\otimes V)+ A(V\otimes V)$, as every $x\in V\otimes V$ can be written 
as 
\[
x=\frac12(x+T(x))+\frac12(x-T(x))
\]
with $\frac12(x+T(x))\in S(V\otimes V)$ and $\frac12(x-T(x))\in A(V\otimes V)$. 

We claim that 
\[
\{v_i\otimes v_j+v_j\otimes v_i:1\leq i\leq j\leq n\}
\]
is
a basis of $S(V\otimes V)$, 
and that  
\[
\{v_i\otimes v_j-v_j\otimes v_i:1\leq i<j\leq n\}
\]
is a basis of $A(V\otimes V)$. Since both sets are linearly independent, 
\[
\dim S(V\otimes V)\geq n(n+1)/2\text{ and }
\dim A(V\otimes V)\geq n(n-1)/2.
\]
Moreover, 
\[
n^2=\dim (V\otimes V)=\dim S(V\otimes V)+\dim A(V\otimes V),
\]
so it follows that
$\dim S(V\otimes V)=n(n+1)/2$ and $\dim A(V\otimes V)=n(n-1)/2$. 

\begin{proposition}
\label{pro:SandA}
    Let $G$ be a finite group and
    $V$ be a finite-dimensional 
    $\C[G]$-module with character $\chi$. If $S(V\otimes V)$ 
    has character $\chi_S$ and $A(V\otimes V)$ has character
    $\chi_A$, then 
    \begin{align*}
        &\chi_S(g)=\frac12(\chi^2(g)+\chi(g^2)) && \text{and} &&
        \chi_A(g)=\frac12(\chi^2(g)-\chi(g^2)).
    \end{align*}
\end{proposition}

\begin{proof}
    Let $g\in G$ and $\rho\colon G\to\GL(V)$ be the representation
    associated with $V$, that is $\rho(g)(v)=\rho_g(v)=g\cdot v$. 
    Since $\rho_g$ is diagonalizable, let $\{e_1,\dots,e_n\}$ 
    be a basis of eigenvectors of $\rho_g$, say
    $g\cdot e_i=\lambda_ie_i$ with $\lambda_i\in\C$ for all $i\in\{1,\dots,n\}$. In particular, $\chi(g)=\sum_{i=1}^n\lambda_i$. 
    
    Since $\{e_i\otimes e_j-e_j\otimes e_i:1\leq i<j\leq n\}$ is a basis of
    $A(V\otimes V)$ and 
    \[
    g\cdot (e_i\otimes e_j-e_j\otimes e_i)=\lambda_i\lambda_j(e_i\otimes e_j-e_j\otimes e_i),
    \]
    it follows that
    \[
    \chi_A(g)=\sum_{1\leq i<j\leq n}\lambda_i\lambda_j.
    \]
    On the other hand,
    $g^2\cdot e_i=\lambda_i^2e_i$ for all $i$,
    $\chi(g^2)=\sum_{i=1}^n\lambda_i^2$. Thus 
    \[
    \chi^2(g)=\chi(g)^2=\sum_{i=1}^n\sum_{j=1}^n\lambda_i\lambda_j=2\sum_{1\leq i<j\leq n}\lambda_i\lambda_j+\sum_{i=1}^n\lambda_i^2=2\chi_A(g)+\chi(g^2).
    \]
    Since $V\otimes V=S(V\otimes V)\oplus A(V\otimes V)$, it follows that  
    $\chi^2(g)=\chi_S(g)+\chi_A(g)$, that is 
    $\chi_S(g)=\frac12(\chi^2(g)+\chi(g^2))$.
\end{proof}

\index{Involution}
An \emph{involution} of a group is an element $x\ne 1$ such that $x^2=1$. 
It is possible to use the character table to count the number
of involutions.

\begin{proposition}
\label{pro:involutions}
    If $G$ is a finite group with $t$ involutions, then
    \[
        1+t=\sum_{\chi\in\Irr(G)}\langle\chi_S-\chi_A,\chi_1\rangle\chi(1),
    \]
    where $\chi_1$ is 
    the trivial character of $G$.
\end{proposition}

\begin{proof}
    Assume that $\Irr(G)=\{\chi_1,\dots,\chi_k\}$.  
    For $x\in G$ let 
    \[
    \theta(x)=|\{y\in G:y^2=x\}|.
    \]
    Since $\theta$ is a class function, 
    $\theta$ is a linear combination of the $\chi_j$'s, say 
    \[
    \theta=\sum_{\chi\in\Irr(G)}\langle\theta,\chi\rangle\chi.
    \]
    For every $\chi\in\Irr(G)$ we compute: 
    \begin{align*}
        \langle\chi_S-\chi_A,\chi_1\rangle 
        &=\frac{1}{|G|}\sum_{g\in G}\chi(g^2)\\
        &=\frac{1}{|G|}\sum_{x\in G}\sum_{\substack{g\in G\\g^2=x}}\chi(g^2)
        =\frac{1}{|G|}\sum_{x\in G}\theta(x)\chi(x)=\langle\theta,\chi\rangle.
    \end{align*}
    Thus $\theta=\sum_{\chi\in\Irr(G)}\langle\chi_S-\chi_A,\chi_1\rangle\chi$. Now
    the claim follows after evaluating this expression in 
    $x=1$. 
\end{proof}

\begin{example}
    We know that $\Sym_3$ has three involutions, namely $(12)$, $(23)$ and $(13)$. Thus $t=3$. 
    Let us use Proposition~\ref{pro:involutions} to verify this. 
    We have already computed the  character table
    of $\Sym_3$: 
\bigskip 
    \begin{center}
		\begin{tabular}{|c|ccc|}
			\hline
			& $1$ & $(12)$ & $(123)$ \tabularnewline
			\hline 
			$\chi_{1}$ & $1$ & $1$ & $1$\tabularnewline
			$\chi_{2}$ & $1$ & $-1$ & $1$ \tabularnewline
			$\chi_{3}$ & $2$ & $0$ & $-1$ \tabularnewline
			\hline
		\end{tabular}
	\end{center}
\bigskip 
A direct calculation shows that 
\[
(\chi_1)_S=(\chi_2)_S=\chi_1\quad\text{and}
\quad (\chi_1)_A=(\chi_2)_A=0.
\]
Moreover, the values of $(\chi_3)_S$ and $(\chi_3)_A$ 
are
given by the following table: 
\bigskip 
    \begin{center}
		\begin{tabular}{|c|ccc|}
			\hline
			& $1$ & $(12)$ & $(123)$ \tabularnewline
			\hline 
			$(\chi_{3})_S$ & $3$ & $1$ & $0$ \tabularnewline
			$(\chi_{3})_A$ & $1$ & $-1$ & $1$ \tabularnewline
			\hline
		\end{tabular}
	\end{center}
\bigskip 
Let $t$ be the number of elements 
of order two of $\Sym_3$. 
Since 
\begin{align*}
\langle\chi_S-\chi_A,\chi_1\rangle=1
\end{align*}
for all $\chi\in\{\chi_1,\chi_2\}$ and
\[
\langle (\chi_3)_S-(\chi_3)_A,\chi_1\rangle
=\frac{1}{6}(12+6-2)=\frac16(2+6-2)=1,
\]
Proposition~\ref{pro:involutions} yields 
\begin{align*} 
1+t&=\langle (\chi_1)_S-(\chi_1)_A,\chi_1\rangle\chi_1(1)
+\langle (\chi_2)_S-(\chi_2)_A,\chi_1\rangle\chi_2(1)
+\langle (\chi_3)_S-(\chi_3)_A,\chi_1\rangle\chi_3(1)\\
&=1+1+2.
\end{align*}
Thus $t=3$. 
\end{example}

Before proving the Brauer--Fowler theorem, we
need a lemma. 

\begin{lemma}
    Let $G$ be a finite group with $k$ conjugacy classes. 
    If $t$ is the number of involutions of $G$, then
    $t^2\leq (k-1)(|G|-1)$. 
\end{lemma}

\begin{proof}
    Assume that $\Irr(G)=\{\chi_1,\dots,\chi_k\}$, where $\chi_1$ is the
    trivial character of $G$. 
    If $\chi\in\Irr(G)$, then 
    \[
        \langle\chi^2,\chi_1\rangle=\frac{1}{|G|}\sum_{g\in G}\chi(g)\chi(g)=\langle\chi,\overline{\chi}\rangle=\begin{cases}
        1 & \text{if $\chi=\overline{\chi}$},\\
        0 & \text{otherwise}.
        \end{cases}
    \]
    Since $\chi^2=\chi_S+\chi_A$, if $\langle\chi^2,\chi_1\rangle=1$, then
    the trivial character is an irreducible component either of $\chi_S$ or $\chi_A$, but not both. 
    Thus
    \[
    \langle\chi_S-\chi_A,\chi_1\rangle\in\{-1,1,0\}.
    \]
    
    We claim that 
    $t\leq\sum_{i=2}^k\chi_i(1)$. In fact, since 
    $|\langle\chi_S-\chi_A,\chi_1\rangle|\leq 1$, 
    \begin{align*}
        1+t=\theta(1)
        &=\left|\sum_{\chi\in\Irr(G)}\langle\chi_S-\chi_A,\chi_1\rangle\chi(1)\right|\\
        &\leq\sum_{\chi\in\Irr(G)}|\langle\chi_S-\chi_A,\chi_1\rangle|\chi(1)
        \leq\sum_{\chi\in\Irr(G)}\chi(1).
    \end{align*}
    It follows that $t\leq\sum_{i=2}^k\chi_i(1)$. 
    By the Cauchy--Schwarz inequality, 
    \[
        t^2\leq\left(\sum_{i=2}^k\chi_i(1)\right)^2
        \leq(k-1)\sum_{i=2}^k\chi_i(1)^2=(k-1)(|G|-1).\qedhere
    \]
\end{proof}

Now we prove the Brauer--Fowler theorem. 

\begin{theorem}[Brauer--Fowler]
    \index{Brauer--Fowler theorem}
    Let $G$ be a finite simple group and $x$ be an involution of $G$. If $|C_G(x)|=n$, then $|G|\leq (n^2)!$	
\end{theorem}

\begin{proof}
    If $G$ is abelian, the claim is trivial. Let $G$ be a finite non-abelian simple group.
    We first assume the existence of a proper subgroup $H$ of $G$ 
    such that 
    \[
    (G:H)\leq n^2.
    \]
    Let $G$ act on $G/H$ 
    by left multiplication, and let 
    $\rho\colon G\to\Sym_{n^2}$ be the corresponding
    group homomorphism. Since $G$ is simple, either 
    $\ker\rho=\{1\}$ or $\ker\rho=G$. If $\ker\rho=G$, then
    $\rho(g)(yH)=yH$ for all $g\in G$ and $y\in G$. 
    Hence $H=G$, a contradiction. Therefore $\rho$ is injective
    and hence $G$ is isomorphic to a subgroup of $\Sym_{n^2}$. 
    In particular, $|G|$ divides $(n^2)!$. 

    Let $m=(|G|-1)/t$, where $t$ is the number of involutions of $G$. 
    Since $|C_G(x)|=n$, the group $G$ has at least $|G|/n$ involutions (because
    the conjugacy class of $x$ has size $|G|/n$ and all its elements are involutions), 
    that is $t\geq |G|/n$. Hence 
    \[
    m=(|G|-1)/t<n.
    \]
    It is enough to show that
    $G$ contains a subgroup of index $\leq m^2$. 

    Let $C_1,\dots,C_k$ be the conjugacy classes of $G$, where $C_1=\{1\}$. 
    Since $G$ is simple and non-abelian, $|C_i|>1$ 
    for all $i\in\{2,\dots,k\}$. By the previous lemma, 
    \[
    t^2\leq(k-1)(|G|-1)\implies |G|-1=\frac{mt^2}{t}\leq\frac{(k-1)(|G|-1)^2}{t^2}=(k-1)m^2.
    \]
    If $|C_i|>m^2$ for all $i\in\{2,\dots,k\}$, then
    \[
    |G|-1=\sum_{i=2}^k|C_i|>(k-1)m^2,
    \]
    a contradiction. Thus there exists a non-trivial conjugacy class
    $C$ of $G$ such that $|C|\leq m^2$. If $g\in C$, then
    $C_G(g)$ is a proper subgroup of $G$ of index $|C|\leq m^2$.
\end{proof}

The bound of the Brauer--Fowler theorem is not essential.
What matters is the following consequence:

\begin{corollary}
    Let $n\geq 1$ be an integer. There are at most finitely many 
    finite simple groups with an involution with a centralizer of order $n$.
\end{corollary}

As an exercise, a simple application: 

\begin{exercise}
    If $G$ is a finite simple group and $x$ is an involution with
    centralizer of order two, then  
    $G\simeq\Z/2$. 
\end{exercise}

\subsection{The character table of $\Sym_5$}
\index{Character table!of $\Sym_5$}
Let $G=\Sym_5$. The conjugacy classes 
of $G$ are given in the following table:

\bigskip 
\begin{center}
    \begin{tabular}{c|ccccccc}
        Representative & $\id$ & $(12)$ & $(123)$ & $(12)(34)$ & $(1234)$ & $(123)(45)$  & $(12345)$ \\
        \hline 
        Size & $1$ & $10$ & $20$ & $15$ & $30$ & $20$ & $24$ \\
    \end{tabular}
\end{center}
\bigskip 

Thus there are seven irreducible characters. The trivial character $\chi_1$ and the sign $\chi_2$ are degree-one (hence irreducible) 
characters. 

\bigskip 
\begin{center}
    \begin{tabular}{|c|ccccccc|}
        \hline 
        & $\id$ & $(12)$ & $(123)$ & $(12)(34)$ & $(1234)$ & $(123)(45)$  & $(12345)$ \\
        \hline 
        $\chi_1$ & $1$ & $1$ & $1$ & $1$ & $1$ & $1$ & $1$ \\
        $\sgn$ & $1$ & $-1$ & $1$ & $1$ & $-1$ & $-1$ & $1$ \\
        \hline 
    \end{tabular}
\end{center}
\bigskip 

Since 
$[G,G]=\Alt_5$ and $|G/[G,G]|=2$, it follows
from Exercise~\ref{xca:degree-one} that $\chi_1$ and $\sgn$ 
are the only degree-one characters. 

Since $G$ acts 2-transitively on $\{1,\dots,5\}$, Proposition~\ref{pro:2transitive} implies that 
$\varsigma(g)=|\Fix(g)|-1$ is an irreducible character. 
A direct
calculation yields the values of $\varsigma$: 
\bigskip 
\begin{center}
    \begin{tabular}{|c|ccccccc|}
        \hline 
        & $\id$ & $(12)$ & $(123)$ & $(12)(34)$ & $(1234)$ & $(123)(45)$  & $(12345)$ \\
        \hline 
        $\varsigma$ & $4$ & $2$ & $1$ & $0$ & $0$ & $-1$ & $-1$ \\
        \hline 
    \end{tabular}
\end{center}
\bigskip 

The values of the product 
$\sgn\varsigma$ are easily computed: 
\bigskip 
\begin{center}
    \begin{tabular}{|c|ccccccc|}
        \hline 
        & $\id$ & $(12)$ & $(123)$ & $(12)(34)$ & $(1234)$ & $(123)(45)$  & $(12345)$ \\
        \hline 
        $\sgn\varsigma$ & $4$ & $-2$ & $1$ & $0$ & $0$ & $1$ & $-1$ \\
        \hline 
    \end{tabular}
\end{center}
\bigskip 

Since 
\begin{align*}
\langle\sgn\varsigma,\sgn\varsigma\rangle&=
\frac{1}{120}(4^2+10(-2)^2+20+15\cdot 0+30\cdot 0+20+24)\\
&=\frac{1}{120}(16+40+20+20+24)=1,
\end{align*}
it follows that $\sgn\varsigma\in\Irr(G)$. 

We now consider the characters 
\[
\psi(g)=\frac12(\varsigma^2(g)+\varsigma(g^2))\quad\text{and}\quad  
\eta(g)=\frac12(\varsigma^2(g)-\varsigma(g^2)),
\]
where $\varsigma^2(g)=\varsigma(g)\varsigma(g)=\varsigma(g)^2$ (see Proposition~\ref{pro:SandA}). 
A straightforward 
calculation shows that 
\bigskip 
\begin{center}
    \begin{tabular}{|c|ccccccc|}
        \hline 
        & $\id$ & $(12)$ & $(123)$ & $(12)(34)$ & $(1234)$ & $(123)(45)$  & $(12345)$ \\
        \hline 
        $\psi$ & $10$ & $4$ & $1$ & $2$ & $0$ & $1$ & $0$ \\
        $\eta$ & $6$ & $0$ & $0$ & $-2$ & $0$ & $0$ & $1$ \\
        \hline 
    \end{tabular}
\end{center}
\bigskip 

Since 
\[
\langle\eta,\eta\rangle
=\frac{1}{120}(6^2+15(-2)^2+24)=1,
\]
it follows that $\eta\in\Irr(G)$. On the other hand,
\[
\langle\psi,\psi\rangle
=\frac{1}{120}(10^2+10\cdot16+20+15\cdot 4+20)=3. 
\]
Thus $\psi$ is the sum of three irreducible characters (see Exercise~\ref{xca:n_irreducible}). Since 
\begin{align*}
\langle\psi,\chi_1\rangle&=\frac{1}{120}(10+10\cdot 4+20+15\cdot 2+20)=1,\\
\langle\psi,\varsigma\rangle&=\frac{1}{120}(10\cdot 4+10\cdot 4\cdot 2+20-20)=1,
\end{align*}
it follows that 
$\psi=\chi_1+\varsigma+\chi$ for some $\chi\in\Irr(G)$. Thus
we can compute $\chi$:
\bigskip 
\begin{center}
    \begin{tabular}{|c|ccccccc|}
        \hline 
        & $\id$ & $(12)$ & $(123)$ & $(12)(34)$ & $(1234)$ & $(123)(45)$  & $(12345)$ \\
        \hline 
        $\chi$ & $5$ & $1$ & $-1$ & $1$ & $-1$ & $1$ & $0$ \\
        \hline 
    \end{tabular}
\end{center}
\bigskip 
We are missing one irreducible character. Let $n$ be 
the degree of this character. Since 
$120=1+1+16+16+36+25+n^2$, it follows that 
$n=5$. Since we need a degree-five
irreducible character, we can try with
$\xi=\sgn\chi$:
\bigskip 
\begin{center}
    \begin{tabular}{|c|ccccccc|}
        \hline 
        & $\id$ & $(12)$ & $(123)$ & $(12)(34)$ & $(1234)$ & $(123)(45)$  & $(12345)$ \\
        \hline 
        $\xi$ & $5$ & $-1$ & $-1$ & $1$ & $1$ & $-1$ & $0$ \\
        \hline 
    \end{tabular}
\end{center}
\bigskip 

Since 
\[
\langle\xi,\xi\rangle=\frac{1}{120}(25+10(-1)^2+20(-1)^2+15+30+20(-1)^2)
=1,
\]
it follows that $\xi\in\Irr(G)$. We have found the character table of $G$. 

\begin{table}[h]
    \caption{The character table of $\Sym_5$.}
    \begin{tabular}{|c|ccccccc|}
        \hline 
        & $1$ & $10$ & $20$ & $15$ & $30$ & $20$ & $24$ \\
        & $\id$ & $(12)$ & $(123)$ & $(12)(34)$ & $(1234)$ & $(123)(45)$  & $(12345)$ \\
        \hline 
        $\chi_1$ & $1$ & $1$ & $1$ & $1$ & $1$ & $1$ & $1$ \\
        $\sgn$ & $1$ & $-1$ & $1$ & $1$ & $-1$ & $-1$ & $1$ \\
        $\varsigma$ & $4$ & $2$ & $1$ & $0$ & $0$ & $-1$ & $-1$ \\
        $\sgn\varsigma$ & $4$ & $-2$ & $1$ & $0$ & $0$ & $1$ & $-1$ \\
        $\eta$ & $6$ & $0$ & $0$ & $-2$ & $0$ & $0$ & $1$ \\
        $\chi$ & $5$ & $1$ & $-1$ & $1$ & $-1$ & $1$ & $0$ \\
        $\xi$ & $5$ & $-1$ & $-1$ & $1$ & $1$ & $-1$ & $0$ \\
        \hline 
    \end{tabular}
    \end{table}

% g=id,(12),(123),(12)(34),(1234),(123)(45),(12345)
% chi_3^2(g) : 16,4,1,0,0,1,1
% chi_3(g^2) : 4,4,1,4,0,1,-1


\subsection{(optional) An elementary proof of the Brauer--Fowler theorem}

We need to find a subgroup of index $\leq 2n^2$. 
Let $X$ be the conjugacy class of $x$. For $g\in G$ let
\[
J(g)=\{z\in X:zgz^{-1}=g^{-1}\}.
\]
We claim that $|J(g)|\leq|C_G(g)|$. The map $J(g)\to C_G(g)$, $z\mapsto gz$, 
is well-defined,~as 
\[
(gz)g(gz)^{-1}=g(zgz^{-1})g^{-1}=g^{-1}\in C_G(g).
\]
It is injective, as $gz=gz_1$ implies $z=z_1$.

Let $J=\{(g,z)\in G\times X:zgz^{-1}=g^{-1}\}$.  
Since $X\times X\to J$, $(y,z)\mapsto (yz,z)$, 
is well-defined (since $z(yz)z^{-1}=zy=(yz)^{-1}$) and
it is trivially injective, 
\[
|X|^2\leq |J|=\sum_{(g,z)\in J}1\leq\sum_{g\in G}|J(g)|
\leq\sum_{g\in G}|C_G(g)|=k|G|,
\]
where $k$ is the number of conjugacy classes of $G$, 
as $(g,z)\in J$ if and only if $z\in J(g)$. Thus $|G|\leq kn^2$, as
\[
\left(\frac{|G|}{|C_G(x)|}\right)^2=|X|^2=\frac{|G|^2}{n^2}\leq k|G|.
\]

\begin{claim}
    There exists a non-trivial conjugacy class with $\leq 2n^2$ elements.
\end{claim}

Assume that the claim is not true. Let
$C_1,\dots,C_k$ be the conjugacy classes of $G$, where 
$C_1=\{1\}$ and $|C_i|>2n^2$ for all $i\in\{2,\dots,k\}$. Then
\[
|G|=1+\sum_{i=2}^k|C_i|>1+\sum_{i=2}^k2n^2=1+(k-1)2n^2\geq |G|,
\]
a contradiction. 

\begin{claim}
    There exists a subgroup $H$ of $G$ such that
    $(G:H)\leq 2n^2$.
\end{claim}

Let $C$ be a conjugacy class of $G$ such that 
$|C|\leq 2n^2$. Let $g\in C$.  
Then $H=C_G(g)$ is a subgroup of $G$ such that
$(G:H)\leq 2n^2$. 



\subsection{Frobenius's reciprocity}

We now present a very quick version of Frobenius'
reciprocity theorem. We first 
define the restriction of class functions. 

\begin{definition}
    Let $G$ be a finite group and $f\colon G\to\C$ be
    a map. For a subgroup $H$ of $G$, the \emph{restriction}
    of $f$ to $H$ is the map 
    $\Res_H^G=f|_H\colon H\to\C$, $h\mapsto f(h)$. 
\end{definition}

\begin{exercise}
\label{xca:restriction}
    Let $G$ be a finite group. Prove that
    the map 
    \[
    \Res_H^G\colon\cf(G)\to\cf(H),\quad  f\mapsto\Res_H^G(f),
    \]
    is a well-defined linear map. 
\end{exercise}

One important property is the following: 

\begin{exercise}
\label{xca:Res}
Let $G$ be a finite group, $H$ a subgroup of $G$ and $\chi \in \Char(G)$. Prove that $\Res_H^G(\chi) \in \Char(H)$.
\end{exercise}

We now define induction. Let $G$ be a finite group
and $H$ be a subgroup of $G$. If $f\colon H\to\C$ is a map, 
then 
\[
f^0(x)=\begin{cases}
    f(x) & \text{if $x\in H$},\\
    0 & \text{otherwise}.
    \end{cases}
\]
It is an exercise to prove that
the map $f\mapsto f^0$ is linear. 

\begin{definition}
    Let $G$ be a finite group and $H$ be a subgroup of $G$. Let
    $f\colon H\to\C$ be
    a map. The \emph{induction}
    of $f$ to $G$ is the map 
    \begin{align*}
      g\mapsto\Ind_H^Gf(g)=\frac{1}{|H|}\sum_{x\in G}f^0(x^{-1}gx).
    \end{align*}
\end{definition}

\begin{exercise}
\label{xca:induction}
    Let $G$ be a finite group. Prove that
    the map 
    \[
    \Ind_H^G\colon\cf(H)\to\cf(G),\quad  f\mapsto\Ind_H^G(f),
    \]
    is a well-defined linear map. 
\end{exercise}

Before proving that the induction of a character is a character, we  mention the following crucial property:

\begin{theorem}[Frobenius' reciprocity]
\index{Frobenius' reciprocity theorem}
    Let $G$ be a finite group and $H$ be a subgroup of $G$. 
    If $a\in\cf(H)$ and $b\in\cf(G)$, then
    \[
    \langle\Ind_H^Ga,b\rangle=\langle a,\Res_H^Gb\rangle
    \quad\text{and}\quad
    \langle\Res_H^Ga,b\rangle=\langle a,\Ind_H^Gb\rangle.
    \]
\end{theorem}

\begin{proof}
    We only need to prove the first equality. We compute 
    \begin{equation}
    \label{eq:reciprocity}
    \begin{aligned}
        \langle\Ind_H^Ga,b\rangle 
        &= \frac{1}{|G|}\sum_{x\in G}\Ind_H^Ga(x)\overline{b(x)}
        = \frac{1}{|G|}\frac{1}{|H|}\sum_{x,y\in G}a^0(y^{-1}xy)\overline{b(x)}.
    \end{aligned}
    \end{equation}
    Setting $h=y^{-1}xy$, 
    % Since 
    % \[
    % a^0(y^{-1}xy)\ne 0\Longrightarrow
    % y^{-1}xy\in H\Longleftrightarrow x\in yHy^{-1},
    % \]
    % setting $h=y^{-1}xy$ 
    we can write \eqref{eq:reciprocity} as 
    \begin{align*}
        \langle\Ind_H^Ga,b\rangle
        &=\frac{1}{|G|}\frac{1}{|H|}\sum_{y\in G}\sum_{h\in H}a(h)\overline{b(yhy^{-1})}\\
        &=\frac{1}{|G|}\frac{1}{|H|}\sum_{y\in G}\sum_{h\in H}a(h)\overline{b(h)}\\
        &=\frac{1}{|G|}\sum_{y\in G}\langle a,\Res_H^Gb\rangle.\qedhere 
    \end{align*}
\end{proof}

\begin{corollary}
\label{cor:reciprocity}
    Let $G$ be a finite group and $H$ be a subgroup of $G$. 
    Let $\chi\in\Char(H)$ be such that 
    $\chi(1)=n$. Then 
    $\Ind_H^G\chi\in\Char(G)$ and 
    has degree $n(G:H)$. 
\end{corollary}

\begin{proof}
    It is enough to show that
    each $m_\psi=\langle\Ind_H^G\chi,\psi\rangle\in\Z_{\geq0}$ 
    for all $\psi\in\Irr(G)$. Let $\psi\in\Irr(G)$. By 
    Frobenius' reciprocity theorem, 
    \[
    m_\psi=\langle\Ind_H^G\chi,\psi\rangle
    =\langle\chi,\Res_H^G\psi\rangle\in\Z_{\geq0}
    \]
    because both $\chi$ and $\Res_H^G\psi$ are
    characters of $H$. In fact, let $\Irr(H)=\{\theta_1,\dots,\theta_k\}$. Since $\chi\in\cf(H)$ and 
    $\Res_H^G\in\cf(H)$, there are 
    non-negative integers $a_1,\dots,a_k$ and $b_1,\dots,b_k$ such that
    $\chi=\sum_{i=1}^ka_i\theta_i$ and 
    $\Res_H^G\psi=\sum_{j=1}^kb_j\theta_j$. Then
    \[
\langle\chi,\Res_H^G\psi\rangle=\sum_{i=1}^k\sum_{j=1}^ka_ib_j\langle\theta_i,\theta_j\rangle=\sum_{i=1}^ka_ib_i\in\Z_{\geq0}.
    \]
    Therefore   
    \[
    \Ind_H^G\chi=\sum_{\psi\in\Irr(G)}m_{\psi}\psi\in\Char(G). 
    \]
    In particular, 
    \[
    \left(\Ind_H^G\chi\right)(1)=\frac{1}{|H|}\sum_{x\in G}\chi^0(1)=\frac{1}{|H|}|G|\chi(1)=\chi(1)(G:H).\qedhere 
    \]
\end{proof}

\begin{exercise}
    Let $G$ be a finite group and $\chi_1$ be
    the trivial character. If $H=\{1\}$, compute
    $\Ind_H^G\chi_1$. 
\end{exercise}

\begin{exercise}
    Let $G=\Sym_3$ and $H=\langle (12)\rangle$. 
    Let $\varphi=\sgn|_H$ be the restriction sign homomorphism 
    to the subgroup $H$. Compute 
    $\Ind_H^G\varphi$. 
\end{exercise}

There are some useful properties that are easy to show. 

\begin{exercise}
    Let $G$ be a finite group, $H$ be a subgroup of $G$, 
    $a\in\cf(H)$ and $b\in\cf(G)$. Prove that 
    \[
    \Ind_H^G((\Res_H^Gb)a)=b(\Ind_H^Ga).
    \]
\end{exercise}

\begin{exercise}[Transitivity of induction]
    Let $G$ be a finite group, $H\subseteq K$ be 
    subgroups of $G$ and $a\in\cf(H)$. 
    Prove that 
    \[
    \Ind_K^G\Ind_H^Ka=\Ind_H^Ga.
    \]
\end{exercise}

\begin{exercise}
\label{xca:useful_Ind}
    Let $G$ be a finite group, $H$ be a subgroup
    of $G$ and $t_1,\dots,t_k$ be a transversal
    of $H$ in $G$. Prove that
    \[
    (\Ind_H^G\alpha)(g)=\sum_{i=1}^k\alpha^0(t_i^{-1}gt_i)
    \]
    for all $\alpha\in\cf(H)$.
\end{exercise}

\begin{exercise}
\label{xca:induction_G/H}
    Let $G$ be a finite group, $H$ be a normal
    subgroup and $\chi$ be the trivial
    character of $H$. Prove that
    $\Ind_H^G\chi$ is the character of the
    representation induced by the action of $G$ on $G/H$ by left multiplication. 
\end{exercise}

% 8.1.4 de Steinberg
% Seccion 8.2 define la inducción de representaciones
% 8.2.1 Induce la trivial del trivial a todo el grupo o obtiene la regular
% 8.2.2 Induce la representación por permutaciones
% Define la matriz y mira el dihedral de orden 2n (yo tengo un caso particular)
% Hace el ejemplo de Q8
% En el teorema 8.2.5 prueba la inducción da una representación


\section{Lecture: Week 8}

\subsection{The correspondence theorem}

Let $N$ be a normal subgroup of $G$ 
and 
\[
\pi\colon G\to G/N,\quad 
g\mapsto gN,
\]
be the canonical map. 
If $\rho\colon G/N\to\GL(V)$, $gN\mapsto\rho_{gN}$,  
is a representation of $G/N$ with 
character
$\chi$, the map  
\[
\Inf_{G/N}^G\rho\colon G\to\GL(V),\quad g\mapsto \rho_{gN},
\]
is a representation of $G$. This representation $\Inf_{G/N}^G\rho$ of $G$ 
is called the \emph{inflation} of the
representation $\rho$. It follows that the character $\Inf_{G/N}^G\chi$ of  
$\Inf_{G/N}^G\rho$ is 
\[
(\Inf_{G/N}^G\chi)(g)=\trace((\Inf_{G/N}^G\rho)_g)=\trace\rho_{gN}=\chi(gN).
\]
In particular, $\chi(1)=(\Inf_{G/N}^G\chi)(1)$. The character $\Inf_{G/N}^G\chi$ is 
also called the \emph{lifting} to $G$ of the character 
$\chi$ of $G/N$. 

\begin{definition}
    \index{Kernel!of a character}
    \index{Kernel!of a representation}
    Let $\chi\in\Char(G)$. The \emph{kernel} of $\chi$ is 
    the subset 
    \[
    \ker\chi=\{g\in G:\chi(g)=\chi(1)\}.
    \]
\end{definition}

\begin{proposition}
Let $\rho\colon G\to\GL_n(\C)$ be a representation with character $\chi$. Then 
$\ker\chi=\ker\rho$. In particular, $\ker\chi$ is a normal subgroup of $G$. 
\end{proposition}

\begin{proof}
Note that $\ker\rho\subseteq\ker\chi$, as $\rho_g=\id$ implies 
$\chi(g)=\trace(\rho_g)=n=\chi(1)$. 

We now prove that  
$\ker\chi\subseteq\ker\rho$. If $g\in G$ is such that $\chi(g)=\chi(1)$, since 
$\rho_g$ is diagonalizable, there exist eigenvalues $\lambda_1,\dots,\lambda_n\in\C$ such that
\[
n=\chi(1)=\chi(g)=\sum_{i=1}^n\lambda_i.
\]
Since each $\lambda_i$ is a root of one,  
$\lambda_1=\cdots=\lambda_n=1$. Hence $\rho_g=\id$. 
\end{proof}

\begin{theorem}[Correspondence theorem]
\index{Correspondence theorem!for characters}
\label{thm:correspondence}
Let $N$ be a normal subgroup of a finite group $G$. There exists
a bijective correspondence 
\[
\Char(G/N) \longleftrightarrow \{\psi\in\Char(G): 
N\subseteq\ker\psi\}
\]
that maps irreducible characters to irreducible characters.
\end{theorem}

\begin{proof}
For $\chi\in\Char(G/N)$, let $\psi=\Inf_{G/N}^G\chi$. Let $n\in N$. 
Then 
\[
\psi(n)=\chi(nN)=\chi(N)=\psi(1)
\]
and thus $N\subseteq\ker\psi$. Thus we have constructed a well-defined
map 
\[
\sigma\colon \Char(G/N) \to \{\psi\in\Char(G): 
N\subseteq\ker\psi\},\quad \chi\mapsto\Inf_{G/N}^G\chi.
\]

If $\psi\in\Char(G)$ is such that $N\subseteq\ker\psi$, let $\rho\colon G\to\GL(V)$, $g\mapsto\rho(g)$, be a representation
with character $\psi$. 
Let $\widetilde{\rho}\colon G/N\to\GL(V)$, $gN\mapsto \rho(g)$. We claim that $\widetilde{\rho}$
is well-defined: 
\[
gN=hN\Longleftrightarrow h^{-1}g\in N\Longrightarrow\rho(h^{-1}g)=\id\Longleftrightarrow \rho(h)=\rho(g).
\]
Moreover, $\widetilde{\rho}$ is a representation, as 
\[
\widetilde{\rho}((gN)(hN))=\widetilde{\rho}(ghN)=\rho(gh)=\rho(g)\rho(h)=\widetilde{\rho}(gN)\widetilde{\rho}(hN).
\]
If $\widetilde{\psi}$ is the character of $\widetilde{\rho}$, then 
$\widetilde{\psi}(gN)=\psi(g)$. Thus we have constructed a well-defined map
\[
\tau\colon \{\psi\in\Char(G): 
N\subseteq\ker\psi\}\to\Char(G/N),\quad \psi\mapsto\widetilde{\psi}.
\]

If $\chi\in\Char(G/N)$ and $g\in G$, then a direct calculation shows 
that $\tau(\sigma(\chi))(gN)=\chi(gN)$. Similarly, if $\psi\in\Char(G)$ is such that $N\subseteq\ker\psi$ and $g\in G$, then 
\[
\sigma(\tau(\psi))(g)=\sigma(\widetilde{\psi})(g)=(\Inf_{G/N}^G\widetilde{\psi})(g)=\widetilde{\psi}(gN)=\psi(g).
\]
% \[
% \tau(\sigma(\chi))(gN)=\tau(\Inf_{G/N}^G\chi)(gN)=\widetilde{\Inf_{G/N}^G\chi}(gN)
% =(\Inf_{G/N}^G\chi)(g)=\chi(gN). 
% \]



We now prove that $\chi$ is irreducible if and only if 
$\widetilde{\chi}$ is irreducible. If $U$ is a subspace of $V$, then 
\begin{align*}
\text{$U$ is $G$-invariant}
&\Longleftrightarrow \rho(g)(U)\subseteq U\text{ for all $g\in G$}\\
&\Longleftrightarrow \widetilde{\rho}(gN)(U)\subseteq U\text{ for all $g\in G$}.
\shortintertext{Thus}
\chi\text{ is irreducible }&\Longleftrightarrow
\rho\text{ is irreducible }\\
&\Longleftrightarrow\widetilde{\rho}\text{ is irreducible }\Longleftrightarrow
\widetilde{\chi}\text{ is irreducible }\qedhere.
\end{align*}
\end{proof}

\begin{example}
    Let $G=\Sym_4$ and $N=\{\id,(12)(34),(13)(24),(14)(23)\}$. We know that $N$ is normal in $G$ 
    and that $G/N=\langle a,b\rangle\simeq\Sym_3$, where 
    $a=(123)N$ and $b=(12)N$. 
    The character table of $G/N$ is 
    \begin{center}
		\begin{tabular}{|c|rrr|}
			\hline
			%& $1$ & $3$ & $2$\tabularnewline
			& $N$ & $(12)N$ & $(123)N$ \tabularnewline
			\hline 
			$\widetilde{\chi}_{1}$ & $1$ & $1$ & $1$\tabularnewline
			$\widetilde{\chi}_{2}$ & $1$ & $-1$ & $1$ \tabularnewline
			$\widetilde{\chi}_{3}$ & $2$ & $0$ & $-1$ \tabularnewline
			\hline
		\end{tabular}
	\end{center}
    For each $i\in\{1,2,3\}$ we compute the lifting $\chi_i$ to $G$ of the character  
    $\widetilde{\chi}_i$ of $G/N$. 
    Since $(12)(34)\in N$ and $(13)(1234)=(12)(34)\in N$, 
    \begin{align*}
        \chi( (12)(34) )=\widetilde{\chi}(N),\quad
        \chi( (1234) )=\widetilde{\chi}((13)N)=\widetilde{\chi}((12)N).
    \end{align*}
    Since the characters $\widetilde{\chi_i}$ are irreducibles, 
    the liftings $\chi_i$ are also irreducibles. With this process
    we obtain the following irreducible characters of $G$:
    	\begin{center}
		\begin{tabular}{|c|rrrrr|}
			\hline
			& $1$ & $(12)$ & $(123)$ & $(12)(34)$ & $(1234)$ \tabularnewline
			\hline 
			$\chi_{1}$ & $1$ & $1$ & $1$ & 1 & 1\tabularnewline
			$\chi_{2}$ & $1$ & $-1$ & $1$ & 1 & -1 \tabularnewline
			$\chi_{3}$ & $2$ & $0$ & $-1$ & 2 & 0\tabularnewline
			\hline
		\end{tabular}
	\end{center}
\end{example}

Theorem~\ref{thm:correspondence} has some unexpected applications. For example, the following exercise is elementary but tricky. A simpler solution uses the second orthogonality relation and Theorem~\ref{thm:correspondence}.

\begin{exercise}
\label{xca:centralizer}
    Let $G$ be a finite group, $g\in G$ and $N$ be a normal subgroup of $G$. 
    Prove that $|C_{G/N}(gN)|\leq|C_G(g)|$. 
\end{exercise}



The character table of a group can be used to find the lattice 
of normal subgroups. In particular, the character table detects simple groups. 

\begin{lemma}
    Let $G$ be a finite group and 
    let $g,h\in G$. Then $g$ and $h$ 
    are conjugate if and only if 
    $\chi(g)=\chi(h)$ for all
    $\chi\in\Char(G)$. 
\end{lemma}

\begin{proof}
    If $g$ and $h$ are conjugate, then $\chi(g)=\chi(h)$, as characters are class functions
    of $G$.
    Conversely, if $\chi(g)=\chi(h)$ for all $\chi\in\Char(G)$, then 
    $f(g)=f(h)$ for all class function $f$ of $G$, 
    as characters $G$ generate the space of class functions of $G$. In particular, 
    $\delta(g)=\delta(h)$, where
    \[
    \delta(x)=\begin{cases}
    1 & \text{if $x$ and $g$ are conjugate},\\
    0 & \text{otherwise}.
    \end{cases}
    \]
    This implies that $g$ and $h$ are conjugate.
\end{proof}

As a consequence, we get that 
\begin{equation}
\label{eq:kernels}
\bigcap_{\chi\in\Irr(G)}\ker\chi=\{1\}.
\end{equation}
Indeed, if $g\in\ker\chi$ for all $\chi\in\Irr(G)$, then $g=1$ since 
the lemma implies that $g$ and $1$ are conjugate
because 
$\chi(g)=\chi(1)$ for all $\chi\in\Irr(G)$.

\begin{proposition}
\label{pro:normal}
    Let $G$ be a finite group. 
    If $N$ is a normal subgroup of $G$, 
    then there exist characters
    $\chi_1,\dots,\chi_k\in\Irr(G)$ 
    such that
    \[
    N=\bigcap_{i=1}^k\ker\chi_i.
    \]
\end{proposition}

\begin{proof}
    Apply the previous remark to the group $G/N$ to obtain that 
    \[
    \bigcap_{\widetilde{\chi}\in\Irr(G/N)}\ker\widetilde{\chi}=\{N\}.
    \]
    Assume that $\Irr(G/N)=\{\widetilde{\chi}_1,\dots,\widetilde{\chi}_k\}$. 
    We lift the irreducible characters of $G/N$ to $G$ 
    to obtain (some) irreducible characters $\chi_1,\dots,\chi_k$ 
    of $G$ such that 
    \[
    N\subseteq\ker\chi_1\cap\cdots\cap\ker\chi_k.
    \]
    If $g\in\ker\chi_i$ for all $i\in\{1,\dots,k\}$, then 
    \[
    \widetilde{\chi}_i(N)=\chi_i(1)=\chi_i(g)=\widetilde{\chi}_i(gN)
    \]
    for all $i\in\{1,\dots,k\}$. This implies that
    \[
    gN\in\bigcap_{i=1}^k\ker\widetilde{\chi}_i=\{N\},
    \]
    that is $g\in N$. 
\end{proof}

\index{Group!simple}
Recall that a non-trivial group is \emph{simple} if it contains no non-trivial normal 
proper subgroups. Examples of simple groups are cyclic groups of prime order
and the alternating groups $\Alt_n$ for $n\geq5$. 
As a corollary of Proposition \ref{pro:normal}, 
we can use the character table to detect simple groups.

\begin{proposition}
    Let $G$ be a finite group. Then $G$ is not simple if and only if 
    there exists a non-trivial irreducible character $\chi$ such that
    $\chi(g)=\chi(1)$ 
    for some $g\in G\setminus\{1\}$. 
\end{proposition}

\begin{proof}
    If $G$ is not simple, there exists a normal subgroup $N$ of $G$ such that
    $N\ne G$ and $N\ne\{1\}$. 
    By Proposition \ref{pro:normal}, there exist characters 
    $\chi_1,\dots,\chi_k\in\Irr(G)$
    such that 
    $N=\ker\chi_1\cap\cdots\cap\ker\chi_k$.
    In particular, there exists a non-trivial character
    $\chi_i$ such that $\ker\chi_i\ne\{1\}$. Thus 
    there exists $g\in G\setminus\{1\}$ such that
    $\chi_i(g)=\chi_i(1)$. 
    
    Assume now that there exists a non-trivial irreducible character $\chi$ 
    such that $\chi(g)=\chi(1)$ for some $g\in G\setminus\{1\}$. In particular, $g\in\ker\chi$ 
    and hence $\ker\chi\ne\{1\}$. Since $\chi$ is non-trivial, $\ker\chi\ne G$. 
    Thus $\ker\chi$ is a proper non-trivial normal subgroup of $G$.
\end{proof}

\begin{example}
\index{Mathieu group $M_9$}
    If there exists a group $G$ with
    a character table 
    of the form
    \begin{center}
		\begin{tabular}{|c|rrrrrr|}
			\hline
			$\chi_{1}$ & 1 & 1 & 1 & 1 & 1 & 1\tabularnewline
			$\chi_{2}$ & 1 & 1 & 1 & -1 & 1 & -1 \tabularnewline
			$\chi_{3}$ & 1 & 1 & 1 & 1 & -1 & -1\tabularnewline
		    $\chi_{4}$ & 1 & 1 & 1 & -1 & -1 & 1\tabularnewline
			$\chi_{5}$ & 2 & -2 & 2 & 0 & 0 & 0\tabularnewline
			$\chi_{6}$ & 8 & 0 & -1 & 0 & 0 & 0\tabularnewline
			\hline
		\end{tabular}
	\end{center}
	then $G$ cannot be simple. Note that such a group $G$ would have order $\sum_{i=1}^6\chi_i(1)^2=72$. 
	Mathieu's group $M_{9}$ has precisely this character table! 
\end{example}

\begin{example}
    Let $\alpha=\frac{1}{2}(-1+\sqrt{7}i)$. 
    If there exists a group $G$ with a character table
    of the form
    \begin{center}
		\begin{tabular}{|c|rrrrrr|}
			\hline
			$\chi_{1}$ & 1 & 1 & 1 & 1 & 1 & 1\tabularnewline
			$\chi_{2}$ & 7 & -1 & -1 & 1 & 0 & 0 \tabularnewline
			$\chi_{3}$ & 8 & 0 & 0 & -1 & 1 & 1\tabularnewline
		    $\chi_{4}$ & 3 & -1 & 1 & 0 & $\alpha$ & $\overline{\alpha}$ \tabularnewline
			$\chi_{5}$ & 3 & -1 & 1 & 0 & $\overline{\alpha}$ & $\alpha$\tabularnewline
			$\chi_{6}$ & 6 & 2 & 0 & 0 & 0 & 0\tabularnewline
			\hline
		\end{tabular}
	\end{center}    
	then $G$ is simple. Note that such a group $G$ would have order 
	$\sum_{i=1}^6\chi_i(1)^2=168$. 
	The group  
	\[
	\PSL_2(7)=\SL_2(7)/Z(\SL_2(7))
	\]
	is a simple group that has precisely this character table!  
\end{example}

\subsection{Frobenius' groups}
\label{Frobenius}

If $p$ is a prime number, then
the units $(\Z/p)^{\times}$ 
of $\Z/p$ form a multiplicative group. Moreover, 
$(\Z/p)^{\times}$ 
is cyclic of order $p-1$. 

Let 
\[
G=\left\{\begin{pmatrix}
x & y\\
0 & 1
\end{pmatrix}
:x\in(\Z/p)^\times,\,y\in\Z/p\right\}.
\]
Then $G$ is a group with the usual matrix multiplication
and $|G|=p(p-1)$. 
Let $p$ and $q$ be prime numbers such that $q$ divides $p-1$, 
$z\in\Z$ be an element of multiplicative order $q$ modulo $p$ 
and 
\[
a=\begin{pmatrix}
1&1\\
0&1
\end{pmatrix},
\quad
b=\begin{pmatrix}
z&1\\
0&1
\end{pmatrix},
\quad
H=\langle a,b\rangle.
\]
A direct calculation shows that 
\begin{equation}
\label{eq:pq}
a^p=b^q=\begin{pmatrix}
1&0\\
0&1
\end{pmatrix},
\quad
bab^{-1}=\begin{pmatrix}
1&z\\
0&1
\end{pmatrix}
=a^z.
\end{equation}
Every element of $H$ is of the form $a^ib^j$ for $i\in\{0,\dots,p-1\}$ and  $j\in\{0,\dots,q-1\}$. 
Thus $|H|=pq$. Using~\eqref{eq:pq} we can compute 
the multiplication table of $G$. 

\begin{exercise}
    Let $p$ and $q$ be prime numbers such that $q$ divides $p-1$. Let
    $u,v\in\Z$ be elements of order $q$ modulo $p$. 
    Prove that 
    \[
    \langle a,b:a^p=b^q=1,bab^{-1}=a^u\rangle
    \simeq \langle a,b:a^p=b^q=1,bab^{-1}=a^v\rangle.
    \]
\end{exercise}

The group   
\[
\langle a,b:a^p=b^q=1,bab^{-1}=a^u\rangle,
\]
where $u\in\Z$ has order $q$ modulo $p$, 
is a particular case of a  
\emph{Frobenius group}. 

\begin{exercise}
\label{xca:Frobenius_pq}
    Let $p$ and $q$ be prime numbers such that $p>q$. Let  
    $G$ be a group of order $pq$. Then either $G$ is abelian or
    $q$ divides $p-1$ and 
    \[
    G\simeq \langle a,b:a^p=b^q=1,bab^{-1}=a^u\rangle
    \]
    for some $u\in\Z$ of order
    $q$ modulo $p$. 
\end{exercise}


Using Exercise~\ref{xca:Frobenius_pq}, we can prove, for example, that every group of order $15$ is abelian. 
% We can also show that, up to isomorphism, $\Z/20$ and $F_{5,4}$ are the only groups of order 20.

\begin{definition}
  \index{Frobenius!complement}
  \index{Frobenius!kernel}
  \index{Frobenius!group}
  We say that a finite group $G$ is a 
  \emph{Frobenius group} if $G$ 
  has a non-trivial proper subgroup $H$ such that $H\cap
  xHx^{-1}=\{1\}$ for all $x\in G\setminus H$. In this case, the subgroup 
  $H$ is called a \emph{Frobenius complement}.
\end{definition}

\index{Malnormal subgroup}
A subgroup $H$ such that $gHg^{-1}\cap H=\{1\}$ for all 
$g\not\in H$ is called a \emph{malnormal} subgroup. 
Note that if $H$ is malnormal, then $N_G(H)=H$. 

\begin{exercise}
\label{xca:malnormal}
    Let $G$ be a group and $H$ be a subgroup of $G$. Prove that the following statements are equivalent: 
    \begin{enumerate}
        \item $H$ is malnormal. 
        \item The action of $H$ 
            on $G/H\setminus\{H\}$ by left multiplication is free. 
        \item Any $g\in G\setminus\{1\}$ has zero
            or one fixed point on $G/H$. 
    \end{enumerate}
\end{exercise}

For any group $G$, the subgroups $\{1\}$ and $G$ are malnormal in $G$. Moreover, they are the only subgroups of $G$ that are both normal and malnormal

\begin{exercise}
    Let $G$ be a group. Prove the following statements:
    \begin{enumerate}

        \item If $H$ is malnormal in $G$, then
        $gHg^{-1}$ is malnormal in $G$ for all $g\in G$. 
        \item If $H$ is malnormal in $G$ and 
        $K$ is malnormal in $H$, then $K$ is malnormal
        in $G$. 
        \item The intersection of malnormal
        subgroups is malnormal.
        \item If $H$ is malnormal in $G$ and 
        $S$ is a subgroup of $G$, then 
        $H\cap S$ is malnormal in $S$. 
        
    \end{enumerate}
\end{exercise}

\begin{example}
    Let 
    \[
    G=\left\{\begin{pmatrix}a&b\\0&1\end{pmatrix}:a\in\R^{\times},b\in\R\right\}\quad\text{and}\quad 
    H=\left\{\begin{pmatrix}a&0\\0&1\end{pmatrix}:a\in\R^{\times}\right\}\subseteq G. 
    \]
    Let $g=\begin{pmatrix}x&y\\0&1\end{pmatrix}\in G\setminus H$. Then $y\ne 0$. Since  
    \[
    g\begin{pmatrix}a&0\\0&1\end{pmatrix}g^{-1}
    =\begin{pmatrix}a&-ay+y\\0&1\end{pmatrix},
    \]
    it follows that the subgroup $H$ 
    is malnormal in $G$. 
\end{example}

\begin{exercise}
\label{xca:malnormal_center}
    Let $G$ be a group and $H$ be a non-trivial
    subgroup of $G$. Prove that if $Z(G)\ne\{1\}$, then
    $H$ is not malnormal in $G$. 
\end{exercise}

\begin{bonus}
\label{xca:malnormal_no2torsion}
    Let $G$ be a group with no 2-torsion 
    that contains a normal infinite cyclic group. Prove 
    that $G$ cannot contain a non-trivial proper malnormal subgroup. 
\end{bonus}


\begin{example}
    Let $G$ be a finite group 
    and $P\in\Syl_p(G)$ be such that $|P|=p$ and $N_G(P)=P$. Then $G$ is a Frobenius group
    with complement $P$. 
\end{example}

The previous example shows that 
$\Alt_4$ is a Frobenius group
with complement $\langle(123)\rangle$. Another situation
where the example applies is the dihedral
group 
\[
\D_{2n+1}=\langle r,s:r^{2n+1}=s^2=1,srs=r^{-1}\rangle
\]
of order $2(2n+1)$. It follows that
$\D_{2n+1}$ is a Frobenius
group with complement $\langle s\rangle$. 

\begin{theorem}[Frobenius]
  \label{thm:Frobenius}
  \index{Frobenius!theorem}
  Let $G$ be a Frobenius group with complement $H$. Then
  \[
	N=\left( G\setminus\bigcup_{x\in G}xHx^{-1}\right)\cup\{1\}
  \]
  is a normal subgroup of $G$.
\end{theorem}

\begin{proof}
    Let $1_H$ and $1_G$ be the trivial characters of $H$ and $G$, respectively.  
  For each $\chi\in\Irr(H)$, $\chi\ne1_H$, let 
  $\alpha=\chi-\chi(1)1_H\in\cf(H)$, where $1_H$ denotes the trivial character of $H$. 

  We claim that $\Res_H^G\Ind_H^G\alpha=\alpha$.
  First, $\Ind_H^G\alpha(1)=\alpha(1)=0$. If $h\in H\setminus\{1\}$, then 
  \[
    \Ind_H^G\alpha(h)=\frac{1}{|H|}\sum_{\substack{x\in G\\x^{-1}hx\in H}}\alpha(x^{-1}hx)
    =\frac{1}{|H|}\sum_{x\in H}\alpha(h)=\alpha(h),
  \]
  since, if $x\not\in H$, then $x^{-1}hx\in H$ implies that 
  $h\in H\cap xHx^{-1}=\{1\}$.

  By Frobenius' reciprocity and the definition of $\alpha$, 
  \begin{equation}
    \label{eq:<a,a>=1+chi2}
    \langle\Ind_H^G\alpha,\Ind_H^G\alpha\rangle
    =\langle\alpha,\Res_H^G\Ind_H^G\alpha\rangle=\langle\alpha,\alpha\rangle
    =1+\chi(1)^2.
  \end{equation}
  Again, by Frobenius' reciprocity, 
  \[
  \langle\Ind_H^G\alpha,1_G\rangle
  =\langle\alpha,\Res_H^G1_G\rangle
  =\langle\alpha,1_H\rangle
  =\langle\chi-\chi(1)1_H,1_H\rangle
  =-\chi(1),
  \]
  where $1_G$ is the trivial character of $G$. If we write 
  \[
  \Ind_H^G\alpha=\sum_{\eta\in\Irr(G)}\langle\Ind_H^G\alpha,\eta\rangle\eta
  =\langle\Ind_H^G\alpha,1_G\rangle1_G+\underbrace{\sum_{\substack{1_G\ne\eta\\\eta\in\Irr(G)}}\langle\Ind_H^G\alpha,\eta\rangle\eta}_{\phi},
  \]
  then $\Ind_H^G\alpha=-\chi(1)1_G+\phi$, where $\phi$ is a linear combination of non-trivial 
  irreducible characters of $G$. We compute 
  \[
  1+\chi(1)^2=\langle\Ind_H^G\alpha,\Ind_H^G\alpha\rangle
  =\langle\phi-\chi(1)1_G,\phi-\chi(1)1_G\rangle
  =\langle\phi,\phi\rangle+\chi(1)^2
  \]
  and hence $\langle\phi,\phi\rangle=1$. 
  
  \begin{claim}
  If $\eta\in\Irr(G)$ is such that $\eta\ne 1_G$, then 
  $\langle\Ind_H^G\alpha,\eta\rangle\in\Z$. 
  \end{claim}
  
  By Frobenius' reciprocity, $\langle\Ind_H^G\alpha,\eta\rangle=\langle\alpha,\Res_H^G\eta\rangle$. 
  If we decompose $\Res_H^G\eta$ into irreducibles of $H$, say 
  \[
  \Res_H^G\eta=m_11_H+m_2\chi+m_3\theta_3+\cdots+m_t\theta_t
  \]
  for some $m_1,m_2,\dots,m_t\geq0$, 
  then, since 
  \begin{align*}
  \langle\alpha,1_H\rangle=\langle\chi-\chi(1)1_H,1_H\rangle=-\chi(1),
  &&\langle\alpha,\chi\rangle=\langle\chi-\chi(1)1_H,\chi\rangle=1,
  \end{align*}
  and 
  \[
  \langle\alpha,\theta_j\rangle=\langle\chi-\chi(1)1_H,\theta_j\rangle=0
  \]
  for all $j\in\{3,\dots,t\}$, we conclude that 
  \[
  \langle\Ind_H^G\alpha,\eta\rangle=-m_1\chi(1)+m_2\in\Z.
  \]
  
  \begin{claim}
  $\phi\in\Irr(G)$.
  \end{claim}
  
  Since $\langle\Ind_H^G\alpha,\eta\rangle\in\Z$ for all $\eta\in\Irr(G)$ such that 
  $\eta\ne 1_G$ and 
  \[
  1=\langle\phi,\phi\rangle
  =\sum_{\substack{\eta,\theta\in\Irr(G)\\\eta,\theta\ne1_G}}\langle\Ind_H^G\alpha,\eta\rangle\langle\Ind_H^G\alpha,\theta\rangle\langle\eta,\theta\rangle
  =\sum_{\substack{\eta\ne 1_G\\\eta\in\Irr(G)}}\langle\Ind_H^G\alpha,\eta\rangle^2,
  \]
  there is a unique $\eta\in\Irr(G)$ such that 
  $\langle\Ind_H^G\alpha,\eta\rangle^2=1$ and all the other products are zero, 
  that is 
  $\phi=\pm\eta$ for some $\eta\in\Irr(G)$. Since 
  \[
  \chi-\chi(1)1_H=\alpha=\Res_H^G\Ind_H^G\alpha=\Res_H^G(\phi-\chi(1)1_G)=\Res_H^G\phi-\chi(1)1_H,
  \]
  it follows that $\phi(1)=\Res_H^G\phi(1)=\chi(1)\in\Z_{\geq1}$. Thus $\phi\in\Irr(G)$. 

  \medskip
  We have proved that if $\chi\in\Irr(H)$ is such that $\chi\ne 1_H$, then 
  there exists $\phi_\chi\in\Irr(G)$ such that $\Res_H^G(\phi_\chi)=\chi$. 
  
  \medskip
  We prove that $N$ is equal to 
  \[
	M=\bigcap_{\substack{\chi\in\Irr(H)\\\chi\ne1_H}}\ker\phi_{\chi}.
  \]

  We first prove that $N\subseteq M$. 
  Let $n\in N\setminus\{1\}$ and $\chi\in\Irr(H)\setminus\{1_H\}$. Since $n$ 
  does not belong to a conjugate of 
  $H$, 
  \[
	\Ind_H^G\alpha(n)=\frac{1}{|H|}\sum_{\substack{x\in G\\x^{-1}nx\in H}}\alpha(x^{-1}nx)=0, 
  \]
  as $n\in N$ implies that the set $\{x\in G:x^{-1}nx\in H\}$ is empty. Since 
  \[
  0=\Ind_H^G\alpha(n)
  =\phi_{\chi}(n)-\chi(1)=\phi_{\chi}(n)-\phi_{\chi}(1),
  \]
  we conclude that $n\in\ker\phi_{\chi}$. 
  
  We now prove that $M\subseteq N$. 
  Let $h\in M\cap H$ and $\chi\in\Irr(H)\setminus\{1_H\}$. Then 
  \[
    \phi_{\chi}(h)-\chi(1)=\Ind_H^G\alpha(h)=\alpha(h)=\chi(h)-\chi(1),
  \]
  and $h\in\ker\chi$, as
  \[
    \chi(h)=\phi_{\chi}(h)=\phi_{\chi}(1)=\chi(1).
  \]
  Therefore 
  \[
  h\in\bigcap_{\chi\in\Irr(H)}\ker\chi=\{1\}.
  \]
  By~\eqref{eq:kernels}, the kernels
  of irreducible characters have trivial intersection. 
  We now prove that $M\cap
  xHx^{-1}=\{1\}$ for all $x\in G$. Let $x\in G$ and $m\in M\cap xHx^{-1}$. Since 
  $m=xhx^{-1}$ for some $h\in H$, $x^{-1}mx\in H\cap M=\{1\}$.  This implies that 
  $m=1$.
\end{proof}

There is no proof of Frobenius’ theorem that is  independent of character theory. Purely group-theoretic proofs exist in cases where the Frobenius complement has even order or is solvable; see
\cite[Remark 16.2]{MR1645304}. The Feit--Thompson theorem (which relies heavily on character theory and is significantly more difficult than Frobenius’ theorem) implies that these two cases cover all possibilities. 

In 2013, Terence Tao discovered an \href{https://terrytao.wordpress.com/2013/05/24/a-fourier-analytic-proof-of-frobeniuss-theorem/}{alternative 
Fourier-analytic
proof} of Frobenius’ theorem, though it resembles the original character-theoretic approach.

\begin{optional}
 
\begin{definition}
  \index{Frobenius!kernel}
  Let $G$ be a Frobenius group. The normal subgroup 
  $N$ of Frobenius' theorem is called the \emph{Frobenius kernel}. 
\end{definition}

\begin{corollary}
  Let $G$ be a Frobenius group with complement $H$. 
  Then there exists a normal subgroup $N$ of $G$ 
  such that 
  $G=HN$ and $H\cap N=\{1\}$.
\end{corollary}

\begin{proof}
  Frobenius' theorem yields the subgroup $N$. Since 
  $H\cap gHg^{-1}=\{1\}$ for all $g\in G\setminus H$, 
  it follows that 
  $N_G(H)=H$. It follows that $H$
  has $(G:H)$ conjugates. 
  Let 
  \[
  N=\left( G\setminus\bigcup_{x\in G}xHx^{-1}\right)\cup\{1\}.
  \]
  Then  
  $|N|=|G|-(G:H)(|H|-1)=(G:H)$.
  Since, moreover, $N\cap H=\{1\}$, we conclude that
  \[
  |HN|=|N||H|/|H\cap N|=|N||H|=|G|.
  \]
  Therefore $G=NH$.
\end{proof}

In his doctoral thesis Thompson proved the following result, conjectured
by Frobenius. 

\begin{theorem}[Thompson]
\index{Thompson theorem}
    Let $G$ be a Frobenius group. If $N$ is the Frobenius kernel, then $N$ 
    is nilpotent.
\end{theorem}

See~\cite[Theorem 6.24]{MR2426855} for the proof.


\end{optional}
\section{Lecture: Week 9}
 
\begin{definition}
  \index{Frobenius!kernel}
  Let $G$ be a Frobenius group. The normal subgroup 
  $N$ of Frobenius' theorem is called the \emph{Frobenius kernel}. 
\end{definition}

\begin{proposition}
\label{pro:Frobenius_groups}
  Let $G$ be a Frobenius group with complement $H$. 
  Then there exists a normal subgroup $N$ of $G$ 
  such that 
  $G=HN$ and $H\cap N=\{1\}$.
\end{proposition}

\begin{proof}
  Frobenius' theorem yields the subgroup $N$. Since 
  $H\cap gHg^{-1}=\{1\}$ for all $g\in G\setminus H$, 
  it follows that 
  $N_G(H)=H$. It follows that $H$
  has $(G:H)$ conjugates. 
  Let 
  \[
  N=\left( G\setminus\bigcup_{x\in G}xHx^{-1}\right)\cup\{1\}.
  \]
  Then  
  $|N|=|G|-(G:H)(|H|-1)=(G:H)$.
  Since, moreover, $N\cap H=\{1\}$, we conclude that
  \[
  |HN|=|N||H|/|H\cap N|=|N||H|=|G|.
  \]
  Therefore $G=NH$.
\end{proof}

\begin{optional}
In his doctoral thesis Thompson proved the following result, conjectured
by Frobenius. 

\begin{theorem}[Thompson]
\index{Thompson theorem}
    Let $G$ be a Frobenius group. If $N$ is the Frobenius kernel, then $N$ 
    is nilpotent.
\end{theorem}

See~\cite[Theorem 6.24]{MR2426855} for the proof.
\end{optional}

\subsection{The Cameron--Cohen theorem (again)}

In this section, we use Frobenius’ theorem to strengthen the Cameron--Cohen theorem on derangements (Theorem~\ref{thm:CameronCohen}). To do so, we first require an alternative version of Frobenius’ theorem.

% Luca: 
% I feel like we should split corollary 9.4 into different cases. The case when G acts strictly transitive because then there exists no x \in X such that G_x is non-trivial (which is needed to obtain a Frobenius group), but in this case the set N formed by the identity and the derangements of G is completely G which is normal. The other case is the one you do in the proof, but you should mention that N is a normal subgroup not just a subgroup. Same for example 9.5 N is moreover a normal subgroup (which is the point of your example I think…)


\begin{corollary}[Frobenius]
    \label{cor:Frobenius_combinatorio}
    \index{Frobenius!theorem}
    Let $G$ be a group acting transitively on a finite set $X$. 
    Assume that each $g\in G\setminus\{1\}$ fixes 
    at most one element of 
     $X$. The set $N$ formed by the identity and the derangements 
     of $G$ is a normal subgroup of $G$.
\end{corollary}

\begin{proof}
  Let $x\in X$ and $H=G_x$. We claim that 
  if $g\in G\setminus H$, then $H\cap
  gHg^{-1}=\{1\}$. If $h\in H\cap gHg^{-1}$, then
  $h\cdot x=x$ and $(g^{-1}hg)\cdot
  x=x$. Since $g\cdot x\ne x$, $h$ fixes two elements of
  $X$. Thus 
  $h=1$, as every non-trivial element fixes at most one element of $X$. 

  By Theorem~\ref{thm:Frobenius}, 
  \[
    N=\left(G\setminus\bigcup_{g\in G}gHg^{-1}\right)\cup\{1\}
  \]
  is a subgroup of $G$. Let us compute the elements of $N$. If 
  $h\in\bigcup_{g\in G}gHg^{-1}$, then there exists  $g\in G$ such that $g^{-1}hg\in H$,
  that is $(g^{-1}hg)\cdot x=x$; equivalently, 
  $h\in G_{g\cdot x}$. Therefore, the 
  non-identity elements of $N$ are the elements of $G$
  moving every element of $X$.
\end{proof}

\begin{example}
  Let $F$ be a finite field and $G$ be the group of maps 
  $f\colon F\to F$ of the form 
  $f(x)=ax+b$, $a,b\in F$ with $a\ne0$. The group $G$ acts on 
  $F$ and every 
  $f\ne\id$ fixes at most one element of $F$, as 
  \[
	x=f(x)=ax+b\implies a\ne 1\text{ and } x=b/(1-a).
  \]
  In this case, $N=\{f:f(x)=x+b\,,b\in F\}$ 
  is a subgroup of $G$.
\end{example}

\begin{exercise}
    Prove that Theorem~\ref{thm:Frobenius} can be obtained from
    Corollary~\ref{cor:Frobenius_combinatorio}.
\end{exercise}

\subsection{Derangements: The Cameron--Cohen theorem}

\index{Derangements}
Let $G$ be a finite group that acts faithfully and transitively 
on a finite set $X$, say 
$G\leq\Sym_n$, where $X=\{1,2,\dots,n\}$. Let 
$G_0$ be the set of elements $g\in G$ with no fixed-points, 
that is $g(x)\ne x$ for all $x\in X$. 
Such permutations are known as \emph{derangements}. 

\begin{example}
Let $G=\Sym_3$. Then $G_0=\{(123),(132)\}$.
\end{example}

\begin{example}
Let $G=\Sym_4$. Then 
    \[
    G_0=\{(12)(34),(13)(24),(14)(23),(1234),(1243),(1324),(1342),(1423),(1432)\}.
    \]
\end{example}

We want to estimate the number of derangements. For this purpose, let $c_0 = |G_0| / |G|$.

% The number of derangements of a set of size $n$ is known as the \emph{subfactorial} $!n$ of $n$. Some values of the number
% of derangements are
% \[
% 1,2,9,44,265,1854,14833\cdots 
% \]

% How to estimate the number of deragements in an arbitrary 
% permutation group? 

\begin{theorem}[Cameron--Cohen]
    \index{Cameron--Cohen!theorem}
    \label{thm:CameronCohen}
    If $G$ is a subgroup of $\Sym_n$ that acts transitively on 
    $\{1,\dots,n\}$, then $c_0\geq\frac{1}{n}$. Moreover, 
    if $n$ is not the power of a prime number, then
    $c_0>\frac{1}{n}$. 
\end{theorem}

\begin{proof}
    Let $X=\{1,\dots,n\}$. By definition, the rank of $G$ is the number
    of orbitals of $G$ on $X$. It follows that the rank is $\geq2$, as
    $X\times X$ decomposes as 
    \[
    X\times X=\Delta\cup\left((X\times X)\setminus\Delta\right)
    \]
    Let $\chi(g)=|\Fix(g)|$ and $G_0=\{g\in G:\chi(g)=0\}$. If $g\not\in G_0$, then $1\leq\chi(g)\leq n$. Since  
    $(\chi(g)-1)(\chi(g)-n)\leq 0$,
    \[
    \frac{1}{|G|}\sum_{g\in G\setminus G_0}(\chi(g)-1)(\chi(g)-n)\leq 0.
    \]
    On the one hand, 
    \begin{align*}
    \frac{1}{|G|}\sum_{g\in G}(\chi(g)&-1)(\chi(g)-n)\\
    &=\frac{1}{|G|}\left\{\sum_{g\in G_0}+\sum_{g\in G\setminus G_0}\right\}(\chi(g)-1)(\chi(g)-n)\\
    &=\frac{1}{|G|}\sum_{g\in G_0}(\chi(g)^2-(n+1)\chi(g)+n)
    +\underbrace{\frac{1}{|G|}\sum_{g\in G\setminus G_0}(\chi(g)-1)(\chi(g)-n)}_{\leq0}\\
    &\leq n\frac{|G_0|}{|G|}=nc_0.
    \end{align*}
    On the other hand, since the rank of $G$ is $\geq2$, 
    \begin{align*}
        \frac{1}{|G|}\sum_{g\in G}(\chi(g)-1)(\chi(g)-n)
        &=\frac{1}{|G|}\sum_{g\in G}(\chi(g)^2-(n+1)\chi(g)+n)\\
        &\geq 2-\frac{n+1}{|G|}\sum_{g\in G}\chi(g)+n
        %\leq 
        %\frac{1}{|G|}\sum_{g\in G}(\chi(g)-1)(\chi(g)-n)\leq nc_0.
    \end{align*}
    Since $G$ is transitive on $X$, the Cauchy--Frobenius--Burnside theorem implies that
    \[
    \sum_{g\in G}\chi(g)=|G|.
    \]
    Thus $2-(n+1)+n\leq nc_0$ and hence
    $1/n\leq c_0$. 

    Assume now that 
    $c_0=1/n$. Then
    \[
    \frac{1}{|G|}\sum_{g\in G}(\chi(g)^2-(n+1)\chi(g)+n)=1
    \]
    and hence $\frac{1}{|G|}\sum_{g\in G}\chi(g)^2=2$. Moreover, 
    since 
    \[
    \frac{1}{|G|}\sum_{g\in G_0}(\chi(g)-1)(\chi(g)-n)
    +\frac{1}{|G|}\sum_{g\in G\setminus G_0}(\chi(g)-1)(\chi(g)-n)=1,
    \]
    it follows that 
    \[
    \sum_{g\in G\setminus G_0}(\chi(g)-1)(\chi(g)-n)=0.
    \]
    Hence $(\chi(g)-1)(\chi(g)-n)=0$
    for all $g\in G\setminus G_0$. 
    
    By Corollary~\ref{cor:Frobenius_combinatorio}, 
    the subset $N=G_0\cup\{\id\}$ is a normal subgroup of $G$. Moreover, $G=N\rtimes H$ for some 
    subgroup $H$ of $G$ of order $n$. Since 
    $n=|H|=|N|-1$, $H$ acts freely and transitively 
    on $N\setminus\{1\}$. 

    We claim that $N$ is a $p$-group for some prime number $p$. Let $n,m\in N\setminus\{1\}$. Since $H$ is transitive on $N\setminus\{1\}$, 
    there exists $h\in H$ such that $h\cdot n=m$. Then
    \[
    |n|=|h\cdot n|=|m|,
    \]
    since for each $h\in H$, the map 
    $x\mapsto h\cdot x$ is an automorphism of $N$. Thus every two elements of $N\setminus\{1\}$ have 
    the same order. Let $p$ be a prime divisor 
    of $|N|$. By Cauchy's theorem, there exists 
    $n\in N$ such that $|n|=p$. Since all non-trivial
    elements of $N$ have the same order, 
    $N$ is a $p$-group. Therefore 
    $n=|N|$ is a power of a prime.
\end{proof}

% The Cameron--Cohen theorem contains another claim: If
% $n$ is not the power of a prime number, then 
% $c_0>1/n$ (see~Theorem~\ref{thm:CameronCohen>1/n}). The proof uses Frobenius' theorem. 

\begin{optional}
    
In some cases, the bound in the 
Cameron--Cohen theorem can be improved:

\begin{theorem}[Guralnick--Wan]
    \index{Guralnick--Wan theorem}
    Let $G$ be a finite transitive group of degree $n\geq2$. If $n$ 
    is not a power of a prime number and 
    $G\ne\Sym_n$ for $n\in\{2,4,5\}$, then $c_0\geq 2/n$.
\end{theorem}

The proof appears in~\cite{MR1484879} and uses
the classification of finite 2-transitive groups, 
which depends on the CFSG. 
\end{optional}





%Using Frobenius’ theorem (Corollary~\ref{cor:Frobenius_combinatorio}), we can present a refinement of the Cameron–Cohen theorem.

% Wielandt 8.5.4
% 8.5.6 para ver algo de grupos de permutaciones
% 7.1 para ejemplo H(q)
% 10.5.6 (Thompson) N es nilpotente, se usa 10.5.4 

% \begin{theorem}[Cameron--Cohen]
% \index{Cameron--Cohen!theorem}
% \label{thm:CameronCohen>1/n}
%     Let $G\leq\Sym_n$ be a transitive subgroup. 
%     If $n$ is not the power of a prime number, then
%     $c_0>\frac{1}{n}$. 
% \end{theorem}

% \begin{proof}
%     Let us go back to the proof
%     of Theorem~\ref{thm:CameronCohen}. Assume that 
%     $c_0=1/n$. Then
%     \[
%     \frac{1}{|G|}\sum_{g\in G}(\chi(g)^2-(n+1)\chi(g)+n)=1
%     \]
%     and hence $\frac{1}{|G|}\sum_{g\in G}\chi(g)^2=2$. Moreover, 
%     since 
%     \[
%     \frac{1}{|G|}\sum_{g\in G_0}(\chi(g)-1)(\chi(g)-n)
%     +\frac{1}{|G|}\sum_{g\in G\setminus G_0}(\chi(g)-1)(\chi(g)-n)=1,
%     \]
%     it follows that 
%     \[
%     \sum_{g\in G\setminus G_0}(\chi(g)-1)(\chi(g)-n)=0.
%     \]
%     Hence $(\chi(g)-1)(\chi(g)-n)=0$
%     for all $g\in G\setminus G_0$. 
    
%     By Corollary~\ref{cor:Frobenius_combinatorio}, 
%     the subset $N=G_0\cup\{\id\}$ is a normal subgroup of $G$. Moreover, $G=N\rtimes H$ for some 
%     subgroup $H$ of $G$ of order $n$. Since 
%     $n=|H|=|N|-1$, $H$ acts freely and transitively 
%     on $N\setminus\{1\}$. 

%     We claim that $N$ is a $p$-group for some prime number $p$. Let $n,m\in N\setminus\{1\}$. Since $H$ is transitive on $N\setminus\{1\}$, 
%     there exists $h\in H$ such that $h\cdot n=m$. Then
%     \[
%     |n|=|h\cdot n|=|m|,
%     \]
%     since for each $h\in H$, the map 
%     $x\mapsto h\cdot x$ is an automorphism of $N$. Thus every two elements of $N\setminus\{1\}$ have 
%     the same order. Let $p$ be a prime divisor 
%     of $|N|$. By Cauchy's theorem, there exists 
%     $n\in N$ such that $|n|=p$. Since all non-trivial
%     elements of $N$ have the same order, 
%     $N$ is a $p$-group. Therefore 
%     $n=|N|$ is a power of a prime.
% \end{proof}


\begin{exercise}
\label{xca:Frobenius_size20}
Let $G$ be the group of matrices 
of the form $\begin{pmatrix}a&b\\0&1\end{pmatrix}$ where $a,b\in\Z/5$ and $a\ne 0$. Then $|G|=20$. Let 
\[
    h=\begin{pmatrix}
        2\\
        &1
    \end{pmatrix},\quad 
    k=\begin{pmatrix}
        1&1\\
        &1
    \end{pmatrix}.
\]
A direct calculation shows that 
$h^4=1$, $k^5=1$ and $hkh^{-1}=k^2$. Let $H=\langle h\rangle$ 
and $K=\langle k\rangle$. Prove the following statements: 

\begin{enumerate}
    \item Prove that $G=K\rtimes H$.
    \item Find the conjugacy classes of $G$: 
\begin{center}
        \begin{tabular}{cccccc}
             Size & $1$ & $4$ & $5$ & $5$ & $5$\\
             \hline 
             Representative & $1$ & $k$ & $h$ & $h^2$ & $h^3$\\
        \end{tabular}
\end{center}
\item Prove that $G/K$ is cyclic of order four. 
\item Prove that $[G,G]=K$. 
\item Use Theorem~\ref{thm:correspondence} on $G/K$ 
    to find the degree-one characters of $G$. 
\item Let $\chi\in\Irr(K)$ be such that $\chi(k)=\exp(2\pi i/5)$. Prove that 
$\Ind_K^G\chi\in\Irr(G)$. 
\end{enumerate}
\end{exercise}

The previous exercise demonstrates that the character table of the 
Frobenius group of order $20$ corresponds to that of Table~\ref{tab:F5,4}.

\index{Character table!of $F_{5,4}$}
\begin{table}[ht]
\caption{Character table of the Frobenius group $F_{5,4}$ of order $20$.}
\label{tab:F5,4}
        \begin{tabular}{|c|ccccc|}
             \hline
             & $1$ & $k$ & $h$ & $h^2$ & $h^3$\\
             \hline
             $\chi_1$ & $1$ & $1$ & $1$ & $1$ & $1$\\
             $\chi_2$ & $1$ & $1$ & $i$ & $-1$ & $-i$\\
             $\chi_3$ & $1$ & $1$ & $-1$ & $1$ & $-1$\\
             $\chi_4$ & $1$ & $1$ & $-i$ & $-1$ & $i$\\
             $\chi_5$ & $4$ & $-1$ & $0$ & $0$ & $0$\\
             \hline
        \end{tabular}
    \end{table} 

\subsection{Burnside's theorem on real characters}

For $n\geq1$ let $\{e_1,\dots,e_n\}$ be the standard basis of $\C^n$.  
The \emph{natural representation} of $\Sym_n$ is 
$\rho\colon\Sym_n\to\GL_n(\C)$, $\sigma\mapsto\rho_{\sigma}$, 
where $\rho_\sigma(e_j)=e_{\sigma(j)}$ for all $j\in\{1,\dots,n\}$. 
The matrix of $\rho_\sigma$ in the standard basis is  
\begin{equation}
    \label{eq:Sn_natural}
    (\rho_\sigma)_{ij}=\begin{cases}
      1 & \text{if $i=\sigma(j)$},\\
      0 & \text{otherwise}.
    \end{cases}
\end{equation}

\begin{lemma}
	\label{lem:permutaciones}
	For $n\geq1$ let $\rho\colon\Sym_n\to\GL_n(C)$ be the natural 
	representation of the symmetric group. 
	If $A\in\C^{n\times n}$ and $\sigma\in\Sym_n$, then
	\[
		A_{ij}=(\rho_{\sigma}A)_{\sigma(i)j}=(A\rho_{\sigma})_{i\sigma^{-1}(j)}
	\]
    for all $i,j\in\{1,\dots,n\}$.
\end{lemma}

\begin{proof}
	With~\eqref{eq:Sn_natural} we compute:
	\[
		(A\rho_{\sigma})_{ij}=\sum_{k=1}^n A_{ik}(\rho_{\sigma})_{kj}=A_{i\sigma(j)},
		\quad
		(\rho_\sigma A)_{ij}=\sum_{k=1}^n (\rho_\sigma)_{ik}A_{kj}=A_{\sigma^{-1}(i)j}.\qedhere
	\]
\end{proof}

\begin{definition}
  \index{Real!character}
  Let $G$ be a finite group. A character $\chi$ of $G$ is said to be
  \emph{real} if
  $\chi=\overline{\chi}$, that is $\chi(g)\in\R$ for all $g\in G$. 
\end{definition}

\begin{exercise}
	\label{xca:chi_irreducible}
	Let $G$ be a finite group. If $\chi\in\Irr(G)$, then 
	$\overline{\chi}$ is irreducible.
\end{exercise}

\begin{definition}
  \index{Real!conjugacy class}
  Let $G$ be a group. A conjugacy class $C$ of $G$ is said to be
  \emph{real} if for every $g\in C$ one has $g^{-1}\in C$. 
\end{definition}

We use the following notation: if $G$ is a group and $C=\{xgx^{-1}:x\in G\}$ is a conjugacy class of  
$G$, then $C^{-1}=\{xg^{-1}x^{-1}:x\in G\}$.  

\begin{theorem}[Burnside]
    \index{Burnside!theorem on real characters}
    Let $G$ be a finite group. The number of real conjugacy classes 
    equals the number of real irreducible characters. 
\end{theorem}

\begin{proof}
  Let $C_1,\dots,C_r$ be the conjugacy classes of $G$ and  
  let $\chi_1,\dots,\chi_r$ be the irreducible characters of $G$. 
  Let $\alpha,\beta\in\Sym_r$ be such that $\overline{\chi_i}=\chi_{\alpha(i)}$ and 
  $C_i^{-1}=C_{\beta(i)}$ for all $i\in\{1,\dots,r\}$. Note that $\chi_i$
  is real if and only if $\alpha(i)=i$ and that $C_i$ is real if and only if 
  $\beta(i)=i$. The number $n$ of fixed points of $\alpha$ is equal to the number
  of real irreducible characters of $G$, and the number $m$ of fixed points of $\beta$ is equal
  to the number of real classes. 
  Let $\rho\colon\Sym_r\to\GL_r(\C)$ be the natural representation of $\Sym_r$, with character $\chi_\rho$.
  Then $\chi_\rho(\alpha)=n$ and $\chi_\rho(\beta)=m$. 
  
  We claim that 
  $\trace\rho_\alpha=\trace\rho_\beta$. Let $X=(\chi_i(C_j))\in\GL_r(\C)$ be the character matrix of $G$. 
  Then 
  \[
	\rho_\alpha X=\overline{X}=X\rho_\beta.
  \]
  For example, using Lemma~\ref{lem:permutaciones}, 
  \[
  \overline{X_{ij}}=\overline{\chi_i(C_j)}
  =\chi_i(C_j^{-1})=\chi_i(C_{\beta(j)}=X_{i\beta(j)}=(X\rho_\beta)_{ij}.
  \]
  
  Since $X$ is invertible, $\rho_{\alpha}=X\rho_{\beta}X^{-1}$. Thus 
  \[
    n=\chi_{\rho}(\alpha)=\trace\rho_{\alpha}=\trace\rho_{\beta}=\chi_{\rho}(\beta)=m.\qedhere
  \]
\end{proof}

\begin{corollary}
  \label{corollary:|G|impar}
  Let $G$ be a finite group. Then $|G|$ is odd if and only if
  the only real $\chi\in\Irr(G)$ is the trivial character. 
\end{corollary}

\begin{proof}
    If $|G|$ is even, there exists 
    $g\in G$ of order two (Cauchy's theorem). The conjugacy class of $g$ 
    is real. 

    Conversely, assume that $G$ has a non-trivial 
    real conjugacy class $C$. Let $g\in C$. We claim that 
    $G$ has an element of even order. Let $h\in G$ be such that
    $hgh^{-1}=g^{-1}$. Then $h^2\in C_G(g)$, as $h^2gh^{-2}=g$. 
    If $h\in\langle h^2\rangle\subseteq C_G(g)$, then $g$ has 
    even order, as $g^{-1}=g$. If $h\not\in\langle h^2\rangle$, then 
    $h^2$ does not generate $\langle h\rangle$. Hence $h$ has even order, 
    as $|h|\ne|h^2|=|h|/\gcd(|h|,2)$, so $\gcd(|h|,2)\ne 1$.  
\end{proof}

\begin{theorem}[Burnside]
  \index{Burnside!theorem}
  \label{thm:Burnside_mod16}
  Let $G$ be a finite group of odd order 
  with $r$ conjugacy classes. Then
  $r\equiv|G|\bmod{16}$.
\end{theorem}

\begin{proof}
  Since $|G|$ is odd, every non-trivial $\chi\in\Irr(G)$ is not real by
  the previous corollary. The irreducible characters 
  of $G$ are  
  \[
    \chi_1,\chi_2,\overline{\chi_2},\dots,\chi_k,\overline{\chi_k},
    \quad
    r=1+2(k-1),
  \]
  where $\chi_1$ denotes the trivial character. 
  For every $j\in\{2,\dots,k\}$ let $d_j=\chi_j(1)$. 
  Since each $d_j$ divides 
  $|G|$ by Frobenius' theorem and  $|G|$ is odd, 
  every $d_j$ is an odd number, 
  say $d_j=1+2m_j$. Thus  
  \begin{align*}
    |G|&=1+\sum_{j=2}^k 2d_j^2=1+\sum_{j=2}^k2(2m_j+1)^2\\
    &=1+\sum_{j=2}^k2(4m_j^2+4m_j+1)
    =1+2(k-1)+8\sum_{j=2}^km_j(m_j+1).
  \end{align*}
  Hence $|G|\equiv r\bmod{16}$, 
  as $r=1+2k$ and every $m_j(m_j+1)$ is even. 
\end{proof}

As an immediate consequence of Theorem~\ref{thm:Burnside_mod16}, we conclude that every group of order 15 is abelian.

% \subsection{The character table of $\PSL_2(7)$}
% See James and Liebeck, chapter 27

% Let 
% \[
% G=\PSL_2(7)=\SL_2(7)/Z(\SL_2(7)).
% \]
% Let $K=Z(\SL_2(7))$. 
% Then $K=\{I,-I\}$, where $I$ denotes the $2\times 2$ identity
% matrix with coefficients in the field of seven elements. 
% $|G|=168$. We compute the
% character table of $G$. 

% \begin{exercise}
%     The group $G$ has six conjugacy classes with representatives
%     $g_1K,\dots,g_6K$, where 
%     \begin{align*}
%     &g_1=\begin{pmatrix}
%          1&0\\
%          0&1
%          \end{pmatrix},
%     &&g_2=\begin{pmatrix}
%          0&1\\
%          -1&0
%          \end{pmatrix},
%     &&g_3=\begin{pmatrix}
%          2&-2\\
%          2&2
%          \end{pmatrix},\\
%     &g_4=\begin{pmatrix}
%          2&0\\
%          0&4
%          \end{pmatrix},
%     &&g_5=\begin{pmatrix}
%          1&1\\
%          0&1
%          \end{pmatrix},
%     &&g_6=\begin{pmatrix}
%          1&-1\\
%          0&1
%          \end{pmatrix}.
%     \end{align*}
%     Moreover, prove that size of each conjugacy class is given in Table~\ref{tab:cc:PSL(2,7)}
% \end{exercise}

% For example, for computing the conjugacy class
% of $g_4$, we note that if 
% \[
% \begin{pmatrix}
%     a&b\\
%     c&d
% \end{pmatrix}g_4=\pm g_4\begin{pmatrix}
%     a&b\\
%     c&d
% \end{pmatrix},
% \]
% then $b=c=0$. Thus 
% \[
% C_G(g_4)=\left\{\pm\begin{pmatrix}
%     1&0\\
%     0&1
% \end{pmatrix},\pm\begin{pmatrix}
%     2&0\\
%     0&4
% \end{pmatrix},\pm\begin{pmatrix}
%     4&0\\
%     0&2
% \end{pmatrix}\right\}.
% \]
% Thus $|C_G(g_4)|=6$ and 
% hence the conjugacy class of $g_4$ has 42 elements. 

% \begin{table}[ht]
%     \caption{Conjugacy classes of $\PSL_2(7)$.}
%     \begin{tabular}{|c|c|c|}
%         \hline 
%         Representative & Order & Size  \\
%         \hline 
%          $g_1K$ & 1 & 1 \\
%          $g_2K$ & 2 & 21 \\
%          $g_3K$ & 4 & 42\\
%          $g_4K$ & 3 & 56 \\
%          $g_5K$ & 7 & 24 \\
%          $g_6K$ & 7 & 24 \\
%          \hline 
%     \end{tabular}
%     \label{tab:cc:PSL(2,7)}.
% \end{table}

% \begin{exercise}
%     Use the information in Table~\ref{tab:cc:PSL(2,7)}
%     to prove that $\PSL_2(7)$ is a simple group. 
% \end{exercise}

% \begin{proposition}
%     Let $\chi\in\Irr(G)$. Then $\chi(g_jK)\in\Z$ for all 
%     $j\in\{1,2,3,4\}$ and $\chi(g_5K)\not\in\R$. 
% \end{proposition}

% \begin{proof}
%     Let $j\in\{1,2,3,4\}$. Since 
%     $g_jK$ is conjugate to some $g_j^kK$
%     when they both have the same order...   
    
%     Since $g_6K=g_5^{-1}K$, it follows
%     that $g_5K$ is not conjugate to its inverse. Thus 
%     the class of $g_5K$ is not real and hence 
%     $\chi(g_5K)\not\in\R$. 
% \end{proof}

% \begin{proposition}
%     Let 
%     \[
%     T=\left\{\begin{pmatrix}
%         a&b\\
%         0&a^{-1}
%         \end{pmatrix}
%         :a\in\Z/7\setminus\{0\},
%         b\in\Z/7\right\}.
%     \]
    
% \end{proposition}
\section{Lecture: Week 10}

\subsection{Clifford theory}

We begin with a routine exercise. 

%\begin{exercise}
%$    Let $G$ be a finite group and $\chi\in\Char(G)$. 
%$    Let $f\colon G\to H$ be an isomorphism of 
%$    groups. Then 
%$    \[
%$    (f\cdot )(f(g))
%$    
%\end{exercise}

\begin{exercise}
\label{xca:conjugate_chars1}
Let $G$ be a finite group and $N$ be a normal subgroup
of $G$. Prove that $G$ acts on $\Irr(N)$ via 
\[
(g\cdot\theta)(n)=\theta(g^{-1}ng),\quad 
g\in G,\theta\in\Irr(N),n\in N.
\]
\end{exercise}

\begin{exercise}
\label{xca:conjugate_chars2}
Let $G$ be a finite group and $N$ be a normal subgroup of $G$. 
Let $\chi\in\cf(G)$, $\theta\in\cf(N)$ and $g\in G$. Prove that
\[
\langle\Res_N^G\chi,g\cdot\theta\rangle=\langle\Res_N^G\chi,\theta\rangle.
\]
% for all $\chi\in\cf(G)$.
%     \begin{enumerate}
%         \item $\langle g\cdot\alpha,g\cdot\beta\rangle=\langle\alpha,\beta\rangle$.
%         \item $\langle\Res_N^G\chi,g\cdot\alpha\rangle=\langle\Res_N^G\chi,\alpha\rangle$ for all $\chi\in\cf(G)$. 
%     \end{enumerate}
\end{exercise}

\index{Irreducible constituent}
Recall that every character $\chi$ of a finite group is uniquely 
a sum of irreducible characters. These are called
the \emph{irreducible constituents} of $\chi$. The set 
of irreducible constituents of $\chi$ is the set  
\[
\{\eta\in\Irr(G):\langle\chi,\eta\rangle>0\}.
\]

\begin{theorem}[Clifford]
\label{thm:Clifford}
\index{Clifford!theorem}
    Let $G$ be a finite group and $N$ be a normal
    subgroup of $G$. Let $\chi\in\Irr(G)$ and $\theta\in\Irr(N)$ be 
    an irreducible constituent of $\Res_N^G\chi$. 
    Then 
    \[
    \Res_N^G\chi = e(\theta_1+\cdots+\theta_t),
    \]
    where $\theta=\theta_1,\dots,\theta_t$ are the conjugates 
    of $\theta$ in $G$, 
    and $e$ is a positive integer. In particular, all the constituents of $\Res_N^G\chi$ have the same degree. 
\end{theorem}

\begin{proof}
    Let $G\cdot\theta=\{\theta_1,\dots,\theta_t\}$ be the 
    orbit of $\theta$. For each $i\in\{1,\dots,t\}$, 
    \[
    \langle\Res_N^G\chi,\theta_i\rangle 
    =\langle g_i\cdot \Res_H^G\chi,g_i\cdot\theta\rangle
    =\langle\Res_H^G\chi,\theta\rangle>0
    \]
    since by assumption $\theta$ is an irreducible constituent of $\Res_N^G\chi$. 
    Let $e=\langle\Res_N^G\chi,\theta\rangle$. Then 
    \[
    \Res_N^G\chi=e(\theta_1+\cdots+\theta_t)+\eta
    \]
    for some $\eta=0$ or $\eta\in\Char(N)$. Since 
    \[
    e=\langle\Res_H^G\chi,\theta\rangle
    =\langle e(\theta_1+\cdots+\theta_t)+\eta,\theta\rangle
    =e\sum_{i=1}^t\langle\theta_i,\theta\rangle+\langle\eta,\theta\rangle
    =e+\langle\eta,\theta\rangle,
    \]
    it follows that $\langle\eta,\theta\rangle=0$. By Frobenius' reciprocity, 
    $\langle\chi,\Ind_N^G\theta\rangle
    =\langle\Res_N^G\chi,\theta\rangle=e$.
    Thus 
    \[
    \Ind_N^G\theta=e\chi+\lambda 
    \]
    for some $\lambda=0$ or $\lambda\in\Char(G)$. Since 
    \[
    e=\langle\chi,\Ind_N^G\theta\rangle 
    =\langle\chi, e\chi+\lambda\rangle
    =e\langle\chi,\chi\rangle+\langle\chi,\lambda\rangle 
    =e+\langle\chi,\lambda\rangle,
    \]
    it follows that $\langle\chi,\lambda\rangle=0$.

    \begin{claim}
        $\Res_N^G\Ind_N^G\theta=t\frac{1}{|N|}\sum_{i=1}^t\theta_i$.
        %=\frac{1}{|N|}\sum_{x\in G}(x\cdot\theta)$.
    \end{claim}

    Let $n\in N$. For $i\in\{1,\dots,t\}$ let 
    $x_i\in G$ be such that $x_i\cdot\theta=\theta_i$. Then 
    \begin{align*}
        (\Ind_N^G\theta)(n) &= \frac{1}{|N|}\sum_{x\in G}\theta^0(x^{-1}nx)\\
        &=\frac{1}{|N|}\sum_{x\in G}(x\cdot\theta)(n)\\
        &=\frac{1}{|N|}\sum_{1=1}^t t(x_i\cdot\theta)(n)\\
        &=\frac{t}{|N|}\sum_{1=1}^t \theta_i(n), 
    \end{align*}
    where we have used that $n\in N$ and $N$ is normal in $G$ (because $x^{-1}nx\in N$ if and only if
    $n\in xNx^{-1}=N$). 

    \bigskip 
    Therefore 
    \begin{align*}
        \frac{t}{|N|}(\theta_1+\cdots+\theta_t)
        &=\Res_N^G\Ind_N^G\theta\\
        &=\Res_N^G(e\chi+\lambda)\\
        &=e\Res_N^G\chi+\Res_N^G\lambda\\
        &=e^2(\theta_1+\cdots+\theta_t)+e\eta+\Res_N^G\lambda.
    \end{align*}
    Taking inner product against $\eta$,  
    \[
    \frac{t}{|N|}\sum_{i=1}^t\langle\theta_i,\eta\rangle 
    =e^2\sum_{i=1}^t\langle\theta_i,\eta\rangle+e\langle\eta,\eta\rangle+\langle\Res_N^G\lambda,\eta\rangle. 
    \]
    Since $\langle\theta_i,\eta\rangle=0$ for all $i\in\{1,\dots,t\}$, 
    \begin{equation}
    \label{eq:eta}
    0=e\langle\eta,\eta\rangle+\langle\Res_N^G\lambda,\eta\rangle.
    \end{equation}
    We know that $e>0$. Moreover, since $\eta\in\Char(N)$ and 
    $\Res_N^G\lambda\in\Char(N)$, each term of the right hand side of~\eqref{eq:eta} is non-negative, that is 
    $\langle\eta,\eta\rangle\geq0$ and 
    $\langle\Res_N^G\lambda,\eta\rangle\geq0$. Therefore 
    $\langle\eta,\eta\rangle=0$ and hence $\eta=0$. 
\end{proof}

% \begin{proof}
%     We claim that 
%     \[
%     \Res_N^G\Ind_N^G\theta=\frac{1}{|N|}\sum_{g\in G}(g\cdot\theta).
%     \]
%     To prove this formula, let $n\in N$. Then 
%     \[
%     \left(\Ind_N^G\theta\right)(n)
%     =\frac{1}{|N|}\sum_{g\in G}\theta^0(g^{-1}ng)
%     =\frac{1}{|N|}\sum_{g\in G}\theta(g^{-1}ng)
%     =\frac{1}{|N|}\sum_{g\in G}(g\cdot\theta)(n).
%     \]
%     By Frobenius' reciprocity, 
%     $\langle\Ind_N^G\theta,\chi\rangle=\langle\theta,\Res_N^G\chi\rangle>0$. 
%     Thus $\chi$ is an irreducible constituent of $\Ind_N^G\theta$. 
    
%     % Moreover, $g\cdot\chi=\chi$ 
%     % for all $g\in G$, since $\chi\in\Irr(G)\subseteq \cf(G)$. 
%     Now 
%     let $\varphi\in\Irr(N)\setminus\{\theta_1,\dots,\theta_t\}$. Then
%     \[
%     \langle \Ind_N^G\theta, \Ind_N^G\varphi\rangle
%     =\langle \Res_N^G\Ind_N^G\theta,\varphi\rangle
%     =\sum_{g\in G}\langle g\cdot\theta,\varphi\rangle=0,
%     \]
%     since each $g\cdot\theta=\theta_j$ for some $j\in\{1,\dots,t\}$. 
%     It follows that $\langle\Res_N^G\chi,\varphi\rangle=0$. 
%     Thus all irreducible constituents of $\Res_N^G\chi$ belong
%     to $\{\theta_1,\dots,\theta_t\}$, that is
%     \[
% \Res_N^G\chi=\sum_{i=1}^t\langle\Res_N^G\chi,\theta_i\rangle\theta_i.
%     \]
%     Moreover, for each $i\in\{1,\dots,t\}$, 
%     there exists $g_i\in G$ such that
%     $\theta_i=g_i\cdot\theta$. Thus, using Exercise~\ref{xca:conjugate_chars2}, 
%     \[
%     \langle\Res_N^G\chi,\theta_i\rangle=\langle\Res_N^G\chi,g_i\cdot\theta\rangle=\langle\Res_N^G\chi,\theta\rangle=e.\]
%     From this the theorem follows. 
% \end{proof}

\index{Ramification index}
The integer $e$ in Theorem~\ref{thm:Clifford} is known as the \emph{ramification index} of $\chi$ on $N$. In general, the number $e$ is not easy to control. 

\begin{exercise}
\label{xca:Clifford_divisibility}
    Let $G$ be a finite group and $N$ be a normal subgroup of $G$. Let $\chi\in\Irr(G)$ and $\theta$ 
    be an irreducible constituent of $\Res_N^G\chi$. 
    Prove that $\theta(1)$ divides $\chi(1)$. 
\end{exercise}

\index{Inertia subgroup}
Let $G$ be a group and $\theta\in\Irr(G)$. The set 
\[
I_G(\theta)=\{g\in G:g\cdot\theta=\theta\}
\]
is a subgroup of $G$ and is called \emph{inertia subgroup} of $\theta$ in $G$. Note that the inertia
subgroup is the stabilizer of the action of $G$ 
on characters by conjugation (see 
of Exercise~\ref{xca:conjugate_chars1}). In particular, 
$\theta$ has 
$(G:I_G(\theta))$ conjugates. 

\begin{theorem}[Clifford correspondence]
\label{thm:Clifford_correspondence}
\index{Clifford!correspondence theorem}
    Let $G$ be a finite group and $N$ be a normal subgroup of $G$. Let $\theta\in\Irr(N)$ and $I=I_G(\theta)$.  Then 
    the map 
    \[
    \{\psi\in\Irr(I):\langle\Res_N^I\psi,\theta\rangle>0\}\to 
    \{\chi\in\Irr(G):\langle\Res_N^G\chi,\theta\rangle>0\},\quad 
    \psi\mapsto\Ind_I^G\psi,
    \]
    is bijective. Moreover, if $\psi$ is a constituent of $\Res_N^I\theta$, then 
    $\langle\Res_N^I\psi,\theta\rangle=\langle\Res_N^G\chi,\theta\rangle$. 
\end{theorem}

\begin{proof}
    There are several things to prove. 

    \begin{claim}
        The map is $\psi\mapsto\Ind_I^G\psi$ well-defined. 
    \end{claim}

    Let $\psi\in\Irr(I)$ be such that $e=\langle\Res_N^I\psi,\theta\rangle>0$ and 
    let $\chi\in\Irr(G)$ be a constituent of $\Ind_I^G\psi$. By Frobenius' reciprocity, 
    \[
    \langle\psi,\Res_I^G\chi\rangle=\langle\Ind_I^G\psi,\chi\rangle>0.
    \]
    Thus $\psi$ is a constituent of $\Res_I^G\chi$, that is 
    \[
    \Res_I^G\chi=\psi+\lambda 
    \]
    for some $\lambda=0$ or $\lambda\in\Char(I)$. Thus
    \[
    \Res_N^G\chi=\Res_N^I\Res_I^G\chi=\Res_N^I(\psi+\lambda)
    =\Res_N^I\psi+\Res_N^I\lambda, 
    \]
    that is 
    $\Res_N^I\psi$ is a constituent of $\Res_N^I\Res_I^G\chi$. 
    Moreover, 
    \[
    \chi(1)\leq(\Ind_I^G\psi)(1)=(G:I)\psi(1).
    \]
    Let $f=\langle\Res_N^G\chi,\theta\rangle$. Then  
    \[
    f=\langle\Res_N^G\chi,\theta\rangle=\langle\Res_N^I\Res_I^G\chi,\theta\rangle
    \geq\langle\Res_N^I\psi,\theta\rangle=e>0.
    \]
    Since $\Res_N^G\chi=f(\theta_1+\cdots+\theta_t)$, where $G\cdot\theta=\{\theta_1,\dots,\theta_t\}$ is 
    the orbit of $\theta$ under the action of $G$ and $t=(G:I)$,  
    \[
    ft\theta(1)=\chi(1)=(\Res_N^G\chi)(1)\leq(\Ind_I^G\psi)(1)=t\psi(1)=et\theta(1)\leq ft\theta(1),
    \]
    where the last equality follows since 
    $\Res_N^I\psi=e\theta$ by Clifford's theorem. Therefore $e=f$ and 
    $\Ind_I^G\psi=\chi$. 

    \begin{claim}
        The map is $\psi\mapsto\Ind_I^G\psi$ is injective. 
    \end{claim}

    Let $\psi_1\in\Irr(I)$ and $\psi_2\in\Irr(I)$ 
    be such that 
    $\langle\Res_N^I\psi_i,\theta\rangle>0$ for all $i\in\{1,2\}$ and 
    $\chi=\Ind_I^G\psi_1=\Ind_I^G\psi_2$. In the first claim, we proved that $\chi\in\Irr(G)$. 
    We want to prove that $\psi_1=\psi_2$. 
    
    Suppose $\psi_1\ne\psi_2$. 
    We know that $\psi_1$ and $\psi_2$ from the first claim that
    are constituents
    of $\Res_I^G\chi$, that is 
    \[
    \Res_I^G\chi=\psi_1+\psi_2+\xi 
    \]
    for some map $\xi\colon I\to\C$. (The map $\xi$ is either zero
    or a character of $I$.) 
    Then both $\Res_N^I\psi_1$ and 
    $\Res_N^I\psi_2$ are constituents 
    of 
    $\Res_N^G\chi$, as  
    \begin{align*}
        \Res_N^G\chi&=\Res_N^I\Res_I^G\chi\\
        &=\Res_N^I(\psi_1+\psi_2+\xi)\\
        &=\Res_N^I\psi_1+\Res_N^I\psi_2+\Res_N^I\xi.       
    \end{align*}

    Moreover, 
    \begin{align*}
        \langle\Res_N^G\chi,\theta\rangle &= \langle\Res_N^I\psi_1+\Res_N^I\psi_2+\Res_N^I\xi,\theta\rangle\\
        &=\langle\Res_N^I\psi_1,\theta\rangle+\langle\Res_N^I\psi_2,\theta\rangle+\langle\Res_N^I\xi,\theta\rangle\\
&\geq\langle\Res_N^I\psi_1,\theta\rangle+\langle\Res_N^I\psi_2,\theta\rangle\\
        &=\langle\Res_N^G\chi,\theta\rangle+\langle\Res_N^G\chi,\theta\rangle,
    \end{align*}
    where the last equality holds because we proved in the previous claim that 
    \[
        \langle\Res_N^I\psi_i,\theta\rangle=\langle\Res_N^G\chi,\theta\rangle
    \]
    for all $i\in\{1,2\}$. 
    This implies that $\langle\Res_N^G\chi,\theta\rangle=0$, a contradiction. 

    \begin{claim}
        The map is $\psi\mapsto\Ind_I^G\psi$ is surjective. 
    \end{claim}

    Let $\chi\in\Irr(G)$ be such that 
    $e=\langle\Res_N^G\chi,\theta\rangle>0$. Since 
    \[
    \Res_I^G\chi=\sum_{\psi\in\Irr(I)}\langle\Res_I^G,\psi\rangle\psi,
    \]
    it follows that 
    \[
    \Res_N^G\chi=\Res_N^I\Res_I^G\chi
    =\sum_{\psi\in\Irr(I)}\langle\Res_I^G\chi,\psi\rangle\Res_N^I\psi.
    \]
    Since 
    \[
    e=\langle\Res_N^G\chi,\theta\rangle=
    \sum_{\psi\in\Irr(I)}\langle\Res_I^G\chi,\psi\rangle\langle\Res_N^I\psi,\theta\rangle
    \]
    is a positive number, there exists some $\psi\in\Irr(I)$ 
    such that $\langle\Res_I^G\chi,\psi\rangle\langle\Res_N^I\psi,\theta\rangle>0$. In particular, $\langle\Res_N^I\psi,\theta\rangle$ and 
    \[
    \langle\chi,\Ind_I^G\psi\rangle=\langle\Res_I^G\chi,\psi\rangle>0
    \]
    Hence $\chi=\Ind_I^G\psi$. 
\end{proof}

\subsection{It\^o's theorem}

We now present a result that is stronger than Schur’s Theorem~\ref{thm:Schur_chi(1)}.
To that end, we introduce some exercises on basic properties of the center of characters.

\begin{definition}
    \index{Center of a character}
    Let $G$ be a finite group and $\chi\in\Char(G)$. 
    The \emph{center} of $\chi$ is
    \[
    Z(\chi)=\{g\in G:|\chi(g)|=\chi(1)\}.
    \]
\end{definition}

\begin{exercise}
\label{xca:center}
    Let $G$ be a finite group and $\rho\colon G\to\GL_n(\C)$ be a representation with character 
    $\chi$. Prove the following statements: 
    \begin{enumerate}
        \item $Z(\chi)=\{g\in G:\rho_g\text{ is a scalar matrix}\}$. 
       % $=\lambda I\text{ for some $\lambda\in\C\}$}\}$.
        \item $Z(\chi)$ is a normal subgroup of $G$. 
        \item $Z(\chi)/\ker\chi$ is cyclic.
    \end{enumerate}
\end{exercise}

% Recall that an $n\times n$ matrix $A$ is a \emph{scalar matrix} if 
% $A=\lambda I$ for some $\lambda\in\C$, where $I$ is the $n\times n$ identity matrix. 

\begin{exercise}
\label{xca:center_quotient}
    Let $G$ be a finite group and $\chi\in\Irr(G)$. 
    Prove that 
    \[
    Z(\chi)/\ker\chi=Z(G/\ker\chi).
    \]
\end{exercise}

\begin{exercise}
\label{xca:center_ofG}
    Let $G$ be a finite group. Prove that
    \[
    Z(G)=\bigcap\{Z(\chi):\chi\in\Irr(G)\}.
    \]
\end{exercise}

The previous exercise shows that the center of a finite group can be determined
from its character table. It follows that the character table detects
nilpotency. To do this, one computes $Z(G)$ from the character table of $G$, then
the character table of $G/Z(G)$, and by iterating this process, one obtains the
upper central series of the group $G$. 

\begin{lemma}
\label{lem:Ito}
    Let $G$ be a finite group and 
    $\chi\in\Irr(G)$. Then $\chi(1)$ divides $(G:Z(\chi))$. 
\end{lemma}

\begin{proof}
    Let $Q=G/\ker\chi$. 
    By Theorem~\ref{thm:correspondence}, $\chi$ corresponds to 
    $\eta\in\Char(Q)$. 
    By Schur's theorem~\ref{thm:Schur_chi(1)}, 
    $\chi(1)=\eta(1)$ divides $(Q:Z(Q))$. By Exercise~\ref{xca:center_quotient}, 
    $(Q:Z(Q))=(G:Z(\chi))$.    
\end{proof}

\begin{theorem}[It\^o]
\index{It\^o theorem}
\label{thm:Ito}
Let $G$ be a finite group and $\chi\in\Irr(G)$. Then 
$\chi(1)$ divides $(G:A)$ for all normal abelian subgroup $A$ of $G$.  
\end{theorem}

\begin{proof}
    Let $A$ be a normal abelian subgroup of $G$ and 
    $\theta\in\Irr(A)$ be an irreducible constituent of $\Res_A^G\chi$, that 
    is $\langle\Res_A^G\chi,\theta\rangle>0$. Let $I=I_G(\theta)$. 
    By Clifford correspondence (Theorem~\ref{thm:Clifford_correspondence}), 
    $\chi=\Ind_I^G(\psi)$ for some $\psi\in\Irr(I)$ such that 
    $\langle\Res_A^I\psi,\theta\rangle>0$. By Clifford's theorem, since 
    $I$ acts trivially on $\theta$, 
    $\Res_A^I\psi=e\theta$, where $e=\langle\Res_A^I\psi,\theta\rangle>0$. Since $A$ is abelian and
    $\theta\in\Irr(A)$, $\theta(1)=1$. 
    
    We claim that $A\subseteq Z(\psi)$. In fact, if $a\in A$, then 
    \[
    |\psi(a)|=|\Res_A^I\psi(a)|=|e\theta(a)|=e|\theta(a)|=e1=e=\psi(1).
    \]
    By Lagrange's theorem, $|A|$ divides $|Z(\psi)|$. Thus $(I:Z(\psi))$ divides $(I:A)$. 

    By Lemma~\ref{lem:Ito}, 
    $e\theta(1)=\psi(1)$ divides $(I:Z(\psi))$. Then 
    $\psi(1)$ divides $(I:A)$. Now 
    \[
    \chi(1)=(\Ind_I^G\psi)(1)=(G:I)\psi(1)
    \]
    divides $(G:I)(I:A)=(G:A)$.
\end{proof}

\begin{bonus}
    \label{xca:Reynolds}
	Prove that Itô’s theorem remains valid under the assumption that $A$ is subnormal in $G$.
\end{bonus}

%The proof of Theorem~\ref{thm:Ito} is no more difficult than that of Schur's Theorem~\ref{thm:Schur_chi(1)}. For a proof, %see \cite[\S8.1]{MR0450380}.


\section{Lecture: Week 11}

\subsection{Solvable groups and Burnside's theorem}

\index{Derived series}
For a group $G$ let 
$G^{(0)}=G$ and 
$G^{(i+1)}=[G^{(i)},G^{(i)}]$ for $i\geq0$.
The \emph{derived series} of $G$ is the sequence
\[
G=G^{(0)}\supseteq G^{(1)}\supseteq G^{(2)}\supseteq\cdots
\]
Each $G^{(i)}$ is a characteristic subgroup of $G$. We say that 
$G$ is \emph{solvable} if $G^{(n)}=\{1\}$ for some $n$.  

\begin{example}
	Abelian groups are solvable. 
\end{example}

\begin{example}
	The group $\SL_2(3)$ is solvable. 
	Let us see what the computer says:
\begin{lstlisting}
> G := SL(2,3);;
> IsSolvable(G);
true
> [GroupName(x) : x in DerivedSeries(G)];
[ SL(2,3), Q8, C2, C1 ]
\end{lstlisting}
\end{example}

\begin{example}
	Non-abelian simple groups cannot be solvable. 
\end{example}

For $n\geq5$, the group $\Alt_n$ is not solvable.

\begin{exercise}
	\label{xca:solvable}
	Let $G$ be a group. Prove the following statements:
	\begin{enumerate}
		\item A subgroup $H$ of $G$ is solvable, when $G$ is solvable.
		\item Let $K$ be a normal subgroup of $G$. 
		    Then $G$ is solvable if and only if $K$ and $G/K$ are solvable.
	\end{enumerate}
\end{exercise}

For $n\geq5$, the group $\Sym_5$ is not solvable. 

\begin{exercise}
\label{xca:pgroups_solvable}
	Let $p$ be a prime number. Prove that 
	finite $p$-groups are solvable.
\end{exercise}

Exercises~\ref{xca:solvable} and~\ref{xca:pgroups_solvable} may be omitted if the reader is already familiar with solvable groups.

\begin{theorem}[Burnside]
	\index{Burnside's theorem}
	\label{thm:Burnside_auxiliar}
	Let $G$ be a finite group. If $\phi\colon G\to\GL_n(\C)$ is a representation
	with character $\chi$ and $C$ is a conjugacy class of $G$ such that 
	$\gcd(|C|,n)=1$, then for every $g\in C$ either 
	$\chi(g)=0$ or $\phi_g$ is a scalar matrix. 
\end{theorem}

We need a lemma.

\begin{lemma}
	\label{lem:4Burnside}
	Let $\epsilon_1,\dots,\epsilon_n$ be roots of one such that 
	$(\epsilon_1+\cdots+\epsilon_n)/n\in\A$. Then either 
	$\epsilon_1=\cdots=\epsilon_n$ or 
	$\epsilon_1+\cdots+\epsilon_n=0$.
\end{lemma}

\begin{proof}
	Let $\alpha=(\epsilon_1+\cdots+\epsilon_n)/n$.
	If the $\epsilon_j$s are not all equal, then $\|\alpha\|<1$. Moreover, 
	$\|\beta\|<1$ for every algebraic conjugate $\beta$ of $\alpha$. Since the product 
	of the algebraic conjugates of $\alpha$ is an integer of absolute value 
	$<1$, it follows that it is zero. 
\end{proof}

Now we prove the theorem.

\begin{proof}[Proof of Theorem \ref{thm:Burnside_auxiliar}]
	Let $\epsilon_1,\dots,\epsilon_n$ be the eigenvalues of $\phi_g$. By assumption, 
	$\gcd(|C|,n)=1$, there exist $a,b\in\Z$ such that $a|C|+bn=1$. Since 
	$|C|\chi(g)/n\in\A$, after multiplying by $\chi(g)/n$ we obtain that  
	\[
		a|C|\frac{\chi(g)}{n}+b\chi(g)=\frac{\chi(g)}{n}=\frac{1}{n}(\epsilon_1+\cdots+\epsilon_n)\in\A.
	\]
	The previous lemma implies that there are two cases to consider: 
	either $\epsilon_1=\cdots=\epsilon_n$ or $\epsilon_1+\cdots+\epsilon_n=0$. In the first
	case, since $\phi_g$ is diagonalizable, $\phi_g$ is a scalar matrix. 
	In the second case, $\chi(g)=0$.
\end{proof}

\begin{theorem}[Burnside]
	\index{Burnside's theorem}
    \label{thm:pq_notsimple}
	Let $p$ be a prime number. If $G$ is a finite group and 
	$C$ is a conjugacy class of $G$ with $p^k>1$ elements, then $G$ 
	is not simple.
\end{theorem}

\begin{proof}
	Let $g\in C\setminus\{1\}$. Column orthogonality implies that 
	\begin{equation}
	\label{eq:Burnside}
	\begin{aligned}
		0&=\sum_{\chi\in\Irr(G)}\chi(1)\chi(g)\\
		&=\sum_{p\mid\chi(1)}\chi(1)\chi(g)+\sum_{p\nmid\chi(1):\chi\ne\chi_1}\chi(1)\chi(g)+1,
	\end{aligned}
	\end{equation}
	where the one corresponds to the trivial representation of
	$G$.
	
	Look at this equation modulo $p$. If $\chi(g)=0$ for all
	$\chi\in\Irr(G)$
	such that $\chi\ne\chi_1$ and $p\nmid\chi(1)$, then
	\[
	-\frac{1}{p}=\sum\frac{\chi(1)}{p}\chi(g)\in\A\cap\Q=\Z,
	\]
	where the sum is taken over all non-trivial irreducibles
	of $G$ of degree divisible by $p$, a contradiction. Hence 
	there exists an irreducible non-trivial representation 
	$\phi$ with character $\chi$ such that $p$ does not divide
	$\chi(1)$ and $\chi(g)\ne0$. By the previous theorem, 
	$\phi_g$ is a scalar matrix. If $\phi$ is faithful, then 
	$g$ is a non-trivial central element, a contradiction since 
    $|C|>1$. If $\phi$ is not faithful, then 
    $G$ is not simple (because 
	$\ker\phi$ is a non-trivial proper normal subgroup of $G$). 
\end{proof}

\begin{theorem}[Burnside]
  \index{Burnside's theorem}
  \label{thm:pq}
  Let $p$ and $q$ be prime numbers. If $G$ has order $p^aq^b$, then $G$ is solvable.
\end{theorem}

\begin{proof}
	If $G$ is abelian, then it is solvable.
	Suppose now $G$ is non-abelian.
	Let us assume that the theorem is not true. Let $G$ be a group
	of minimal order $p^aq^b$
	that is not solvable. Since $|G|$ is minimal, $G$ is a non-abelian simple group.
	By the previous theorem, 
	$G$ has no conjugacy classes of size $p^k$ nor 
	conjugacy classes of size $q^l$ with $k,l\geq1$. The size
	of every conjugacy class of $G$ is one or divisible by $pq$. 
	Note that, since $G$ is a non-abelian simple group,
	the center of $G$ is trivial.
	Thus there is only one conjugacy class of size one.
	By the class
	equation,
	\[
		|G|=1+\sum_{C:|C|>1}|C|\equiv 1 \bmod pq,
	\]
	where the sum is taken over all conjugacy classes 
	with more than one element, a contradiction.
\end{proof}

\subsection{Some generalizations of Burnside's theorem}

If the reader does not know what nilpotent groups are, this section can be safely omitted.

\begin{theorem}[Kegel--Wielandt]
    \index{Kegel--Wielandt's theorem}
    \label{thm:KegelWielandt}
    If $G$ is a finite group and there are nilpotent subgroups 
    $A$ and $B$ of $G$ such that 
    $G=AB$, then $G$ is solvable.
\end{theorem}

See~\cite[Theorem 2.4.3]{MR1211633} for the proof.

\begin{exercise}
    Prove that Theorem~\ref{thm:KegelWielandt} 
    implies Theorem~\ref{thm:pq}.
\end{exercise}

Another generalization of Burnside's theorem
is based on \emph{word maps}. A word map
of a group $G$ is a map 
\[
G^k\to G,\quad 
(x_1,\dots,x_k)\mapsto w(x_1,\dots,x_k)
\]
for some word $w(x_1,\dots,x_k)$ of the free group $F_k$ of rank $k$. 
Some word maps are surjective in certain families of groups. For example, 
Ore's conjecture is precisely the surjectivity of the word map
$(x,y)\mapsto [x,y]=xyx^{-1}y^{-1}$ in every finite non-abelian simple 
group. 

\begin{theorem}[Guralnick--Liebeck--O'Brien--Shalev--Tiep]
    Let $a,b\geq0$, $p$ and $q$ be prime numbers and $N=p^aq^b$. The map 
    $(x,y)\mapsto x^Ny^N$ is surjective in every finite simple group. 
\end{theorem}

The proof appears in~\cite{MR3827208}. 

The theorem implies Burnside's theorem. Let $G$ be a group of order
$N=p^aq^b$. Assume that $G$ is not solvable. 
Fix a composition series of $G$. There is a non-abelian factor $S$ 
of order that divides $N$. Since 
$S$ is simple non-abelian and $s^N=1$, it follows that the word map
$(x,y)\mapsto x^Ny^N$ has trivial image in $S$, a contradiction 
to the theorem. 

\subsection{The Feit--Thompson theorem}

\begin{theorem}[Feit--Thompson]
    \index{Feit--Thompson!theorem}
    Groups of odd order are solvable. 
\end{theorem}

The proof of Feit--Thompson theorem is extremely hard. 
It occupies a full volume of the 
\emph{Pacific Journal of Mathematics}~\cite{MR166261}. 
A formal verification of the proof 
(based on the computer software Coq) 
was announced in~\cite{MR3111271}.  

Back in the day it was believed that if a certain divisibility 
conjecture is true, 
the proof of Feit--Thompson theorem 
could be simplified. 

\begin{conjecture}[Feit--Thompson]
\index{Feit--Thompson!conjecture}
    There are no prime numbers $p$ and $q$ such that
    $\frac {p^{q}-1}{p-1}$ divides $\frac{q^{p} - 1}{q - 1}$. 
\end{conjecture}

The conjecture remains open. However, now we know that 
proving the conjecture will not simplify further
the proof of Feit--Thompson theorem. 

In 2012 Le proved that the conjecture is true for $q=3$, see 
\cite{MR2900154}. 


In~\cite{MR297686} 
Stephens proved that a certain stronger version of the conjecture 
does not hold, as the integers 
$\frac {p^{q}-1}{p-1}$ and $\frac{q^{p} - 1}{q - 1}$ 
could have common factors. In fact, if $p=17$ and $q=3313$, 
then 
\[
\gcd\left(\frac {p^{q}-1}{p-1},\frac{q^{p} - 1}{q - 1}\right)=112643.
\]
Nowadays we can check this easily in almost every desktop computer:
% \begin{lstlisting}
% gap> Gcd((17^3313-1)/16,(3313^17-1)/3312);
% 112643
% \end{lstlisting}
\begin{lstlisting}
> p := 17; 
> q := 3313;
> bool, a := IsCoercible(Integers(), (p^q-1)/(p-1));
> bool, b := IsCoercible(Integers(), (q^p-1)/(q-1));
> Gcd(a,b);
112643    
\end{lstlisting}
No other counterexamples have been found of Stephen’s stronger version of the conjecture.



\subsection{The character table of \texorpdfstring{$\Alt_5$}{A5}}

\index{Character table!of $\Alt_5$}
Let $G=\Alt_5$. 
The group $G$ is a non-abelian simple group of order 60. It has five conjugacy classes, namely

\bigskip 
\begin{center}
    \begin{tabular}{c|ccccc}
         Representative & $\id$  & $(12)(34)$ & $(123)$  & $(12345)$ & $(12354)$\\
         \hline 
         Size & $1$ & $15$ & $20$ & $12$ & $12$ \\
    \end{tabular}
\end{center}
\bigskip 

One can easily get the conjugacy classes of 
$\Alt_5$ with Magma:
\begin{lstlisting}
> A5 := Alt(5);
> ConjugacyClasses(A5);
Conjugacy Classes of group A5
-----------------------------
[1]     Order 1       Length 1
        Id(A5)

[2]     Order 2       Length 15
        (1, 2)(3, 4)

[3]     Order 3       Length 20
        (1, 2, 3)

[4]     Order 5       Length 12
        (1, 2, 3, 4, 5)

[5]     Order 5       Length 12
        (1, 3, 4, 5, 2)    
\end{lstlisting}

Let us see how to obtain all conjugacy classes
of $\Alt_5$ without computers. Let $\sigma\in\Alt_5$ and $C$ be its
conjugacy class in $\Sym_5$. Thus $|C|=(\Sym_5:C_{\Sym_5}(\sigma))$. There are two cases to consider

Assume first that $C_{\Sym_5}(\sigma)\not\subseteq\Alt_5$. Since $\Alt_5$ is a maximal subgroup of $\Sym_5$, it follows that 
$\Alt_5C_{\Sym_5}(\sigma)=\Sym_5$. Using the isomorphism theorems, 
\[
\Sym_5/\Alt_5=\Alt_5C_{\Sym_5(\sigma)}/\Alt_5
\simeq C_{\Sym_5}(\sigma)/(C_{\Sym_5}(\sigma)\cap\Alt_5)
=C_{\Sym_5}(\sigma)/C_{\Alt_5}(\sigma).
\]
Hence 
\[
(\Alt_5:C_{\Alt_5}(\sigma))=\frac{(\Sym_5:C_{\Alt_5}(\sigma))}{(\Sym_5:\Alt_5)}
=\frac{(\Sym_5:C_{\Alt_5}(\sigma))}{(C_{\Sym_5}(\sigma):C_{\Alt_5}(\sigma))}
=(\Sym_5:C_{\Sym_5}(\sigma))=|C|.
\]
Therefore $C$ is the class of $\sigma$ in $\Alt_5$. 

Assume now that $C_{\Sym_5}(\sigma)\subseteq\Alt_5$. Then 
$C_{\Alt_5}(\sigma)=C_{\Sym_5}(\sigma)\cap\Alt_5=C_{\Sym_5}(\sigma)$
and therefore 
\[
(\Alt_5:C_{\Alt_5}(\sigma))=(\Alt_5:C_{\Sym_5}(\sigma))
=\frac12(\Sym_5:C_{\Sym_5}(\sigma))=\frac12|C|.
\]
Thus $C$ splits into two conjugacy classes of $\Alt_5$ of equal size. 

The identity permutation is central. The even permutations 
$(12)(34)$ and $(123)$ both commutes with some odd permutation in $\Sym_5$ (e.g. 
$[(12)(34),(34)]=[(123),(45)]=\id$). Thus these classes do not split
in $\Alt_5$. There are twenty-four 5-cycles in $\Sym_5$. Since $24$ does not
divide $|\Alt_5|=60$, it follows that the class of 5-cycles
splits in $\Alt_5$. As representatives of these classes
we can take $(12345)$ and $(12354)$. 

Since $\Alt_5$ has five conjugacy classes, $|\Irr(G)|=5$. Assume that 
\[
\Irr(G)=\{\chi_1,\chi_2,\chi_3,\chi_4,\chi_5\}, 
\]
where $\chi_1$ is the trivial character. 

Let $H=\Alt_4$. We compute $\Ind_H^G\chi_1$. By Corollary~\ref{cor:reciprocity}, 
\[
\left(\Ind_H^G\chi_1\right)(\id) = 5.
\]
And a direct calculation shows
\begin{align*}
    &\left(\Ind_H^G\chi_1\right)((12)(34)) = 1,\\
    &\left(\Ind_H^G\chi_1\right)((123)) = 2,\\
    &\left(\Ind_H^G\chi_1\right)((12345)) = 0\\ 
    &\left(\Ind_H^G\chi_1\right)((12354)) = 0.
\end{align*}

Now, using Frobenius' reciprocity and the fact that 
$\Res_H^G\chi_1$ is the trivial character of $H$, 
\begin{align*}
    \langle\Ind_H^G\chi_1,\chi_1\rangle = \langle\chi_1,\Res_H^G\chi_1\rangle=1.
\end{align*}

Let $\chi_2=\Ind_H^G\chi_1-\chi_1$. Since 
\[
\langle\Ind_H^G\chi_1-\chi_1,\Ind_H^G\chi_1-\chi_1\rangle=1, 
\]
it follows that $\chi_2\in\Irr(G)$. 

\begin{exercise}
\label{xca:A5_chi2}
    Use Proposition~\ref{pro:2transitive} to derive (once again) the values of $\chi_2$.
\end{exercise}

So far we have the 
following table: 

\bigskip 
\begin{center}
        \begin{tabular}{|c|ccccc|}
        \hline  
         & $\id$ & $(12)(34)$ & $(123)$ & $(12345)$ & $(12354)$\\
        \hline 
        $\chi_1$ & $1$ & $1$ & $1$ & $1$ & $1$\\
        $\chi_2$ & $4$ & $0$ & $1$ & $-1$ & $-1$\\
        $\chi_3$ & $n_3$ & $\cdot$ & $\cdot$ & $\cdot$& $\cdot$\\
        $\chi_4$ & $n_4$ & $\cdot$ & $\cdot$ & $\cdot$& $\cdot$\\
        $\chi_5$ & $n_5$ & $\cdot$ & $\cdot$ & $\cdot$& $\cdot$\\
        \hline 
    \end{tabular}
\end{center}
\bigskip 

As $G$ is simple non-abelian, 
$|G/[G,G]|=1$. It follows that
$\chi_1$ is the only linear character of $G$. Moreover, 
$\chi_j(1)\geq3$ by Theorem~\ref{thm:simple}. Since 
\[
60=1+16+n_3^2+n_4^2+n_5^2
\]
and each $n_j$ divides $|G|=60$ 
(see Theorem \ref{thm:Frobenius_chi(1)}), it follows that 
$n_j\in\{3,4,5,6\}$. If some $n_j=6$, say without
loss of generality $n_3=6$, then 
\[
7=43-36=n_2^2+n_3^2, 
\]
a contradiction. Thus $n_j\in\{3,4,5\}$ for 
all $j\in\{3,4,5\}$. Without loss of generality, 
we may assume that $n_3=n_4=3$ and $n_5=5$. 

\bigskip 
\begin{center}
        \begin{tabular}{|c|ccccc|}
        \hline  
         & $\id$ & $(12)(34)$ & $(123)$ & $(12345)$ & $(12354)$\\
        \hline 
        $\chi_1$ & $1$ & $1$ & $1$ & $1$ & $1$\\
        $\chi_2$ & $4$ & $0$ & $1$ & $-1$ & $-1$\\
        $\chi_3$ & $3$ & $\cdot$ & $\cdot$ & $\cdot$& $\cdot$\\
        $\chi_4$ & $3$ & $\cdot$ & $\cdot$ & $\cdot$& $\cdot$\\
        $\chi_5$ & $5$ & $\cdot$ & $\cdot$ & $\cdot$& $\cdot$\\
        \hline 
    \end{tabular}
\end{center}
\bigskip 

The group $\Alt_5$ acts on the set $Y$ of subsets 
of $\{1,2,\dots,5\}$ of two elements, namely
\[
g\cdot \{a,b\}=\{g\cdot a,g\cdot b\}.
\]
Note that $|Y|=\binom{5}{2}=10$. Moreover, 
this action is transitive. Let us compute 
the character $\psi$ of the corresponding 
$\C\Alt_5$-module and the difference 
$\psi-\chi_1$ (We know $\psi$ counts
fixed points.)

\bigskip 
\begin{center}
        \begin{tabular}{|c|ccccc|}
        \hline  
         & $\id$ & $(12)(34)$ & $(123)$ & $(12345)$ & $(12354)$\\
        \hline 
        $\psi$ & $10$ & $2$ & $1$ & $0$ & $0$\\
        $\psi-\chi_1$ & $9$ & $1$ & $0$ & $-1$ & $-1$\\
        \hline 
    \end{tabular}
\end{center}
\bigskip 

The identity, of course, fixes all the ten elements
of $Y$. The permutation 
$(12)(34)$ fixed two two-elements subsets, namely
$\{1,2\}$ and $\{3,4\}$. The permutation 
$(123)$ fixes only one two-elements subset, namely
$\{4,5\}$. Finally, $(12345)$ and 
$(12354)$ fix no two-element subsets. 

Now we compute 
\[
\langle \psi-\chi_1,\psi-\chi_1\rangle=2
\]
and hence $\psi-\chi_1$ is the sum of two irreducible
characters (see Exercise~\ref{xca:n_irreducible}). Since
\[
\langle \psi-\chi_1,\chi_2\rangle=1,
\]
it follows that $\psi-\chi_1-\chi_2\Irr(G)$. Let 
$\chi_5=\psi-\chi_1-\chi_2$. Then

\bigskip 
\begin{center}
        \begin{tabular}{|c|ccccc|}
        \hline  
         & $\id$ & $(12)(34)$ & $(123)$ & $(12345)$ & $(12354)$\\
        \hline 
        $\chi_1$ & $1$ & $1$ & $1$ & $1$ & $1$\\
        $\chi_2$ & $4$ & $0$ & $1$ & $-1$ & $-1$\\
        $\chi_3$ & $3$ & $\cdot$ & $\cdot$ & $\cdot$& $\cdot$\\
        $\chi_4$ & $3$ & $\cdot$ & $\cdot$ & $\cdot$& $\cdot$\\
        $\chi_5$ & $5$ & $1$ & $-1$ & $0$& $0$\\
        \hline 
    \end{tabular}
\end{center}
\bigskip 

Let $K=\langle(12345)\rangle$ and 
$\eta\in\Irr(K)$ be such that $\eta((12345))=\zeta$, where
$\zeta=\exp(2\pi i/5)$ is a primitive $5$-th root of one. We can then compute 
$\Ind_K^G$. 

\bigskip 
\begin{center}
        \begin{tabular}{|c|ccccc|}
        \hline  
         & $\id$ & $(12)(34)$ & $(123)$ & $(12345)$ & $(12354)$\\
         \hline 
         $\Ind_K^G\psi$ & $12$ & $0$ & $0$ & $\zeta^2+\zeta^3$ & $\zeta+\zeta^4$\\
         \hline 
\end{tabular}
\end{center}
\bigskip 

Since 
$\langle\Ind_K^G\psi,\chi_2\rangle=1=\langle\Ind_H^G,\chi_5\rangle$,
it follows that 
\bigskip 
\begin{center}
        \begin{tabular}{|c|ccccc|}
        \hline  
         & $\id$ & $(12)(34)$ & $(123)$ & $(12345)$ & $(12354)$\\
         \hline 
         $\Ind_K^G\psi-\chi_2-\chi_5$ & $3$ & $-1$ & $0$ & $-\zeta-\zeta^4$ & $-\zeta^2-\zeta^3$\\
         \hline 
\end{tabular}
\end{center}
\bigskip 
Let $\chi_3=\Ind_K^G\psi-\chi_2-\chi_5$. Then $\chi_3\in\Irr(G)$, because it is
not the sum of three copies of the trivial character. 

\bigskip 
\begin{center}
        \begin{tabular}{|c|ccccc|}
        \hline  
         & $\id$ & $(12)(34)$ & $(123)$ & $(12345)$ & $(12354)$\\
        \hline 
        $\chi_1$ & $1$ & $1$ & $1$ & $1$ & $1$\\
        $\chi_2$ & $4$ & $0$ & $1$ & $-1$ & $-1$\\
        $\chi_3$ & $3$ & $-1$ & $0$ & $-\zeta-\zeta^4$ & $-\zeta^2-\zeta^3$\\
        $\chi_4$ & $3$ & $\cdot$ & $\cdot$ & $\cdot$& $\cdot$\\
        $\chi_5$ & $5$ & $1$ & $-1$ & $0$& $0$\\
        \hline 
    \end{tabular}
\end{center}
\bigskip 

\begin{exercise}
    Use the orthogonality relations
    to compute the missing row of the character table
    of $\Alt_5$. 
\end{exercise}

The previous exercise finishes the calculation
of the character table of $\Alt_5$.

\bigskip 
\begin{center}
        \begin{tabular}{|c|ccccc|}
        \hline  
        & $1$ & $15$ & $20$ & $12$ & $12$ \\
         & $\id$ & $(12)(34)$ & $(123)$ & $(12345)$ & $(12354)$\\
        \hline 
        $\chi_1$ & $1$ & $1$ & $1$ & $1$ & $1$\\
        $\chi_2$ & $4$ & $0$ & $1$ & $-1$ & $-1$\\
        $\chi_3$ & $3$ & $-1$ & $0$ & $-\zeta-\zeta^4$ & $-\zeta^2-\zeta^3$\\
        $\chi_4$ & $3$ &  $-1$ & $0$ & $-\zeta^2-\zeta^3$ & $-\zeta-\zeta^4$ \\
        $\chi_5$ & $5$ & $1$ & $-1$ & $0$& $0$\\
        \hline 
    \end{tabular}
\end{center}
\bigskip 

One last observation: 
Since $\zeta=\exp(2\pi i/5)$, it follows
that 
\[
-\zeta-\zeta^4=\frac{1-\sqrt{5}}{2},
\quad 
-\zeta^2-\zeta^3=\frac{1+\sqrt{5}}{2}.
\]

% Let us see what Magma says:

% \begin{lstlisting}
% > CharacterTable(Alt(5));


% Character Table
% ---------------


% ---------------------------
% Class |   1  2  3    4    5
% Size  |   1 15 20   12   12
% Order |   1  2  3    5    5
% ---------------------------
% p  =  2   1  1  3    5    4
% p  =  3   1  2  1    5    4
% p  =  5   1  2  3    1    1
% ---------------------------
% X.1   +   1  1  1    1    1
% X.2   +   3 -1  0   Z1 Z1#2
% X.3   +   3 -1  0 Z1#2   Z1
% X.4   +   4  0  1   -1   -1
% X.5   +   5  1 -1    0    0


% Explanation of Character Value Symbols
% --------------------------------------

% # denotes algebraic conjugation, that is,
% #k indicates replacing the root of unity w by w^k

% Z1     = (CyclotomicField(5: Sparse := true)) ! [ RationalField() | 0, 0, -1, -1 ]    
% \end{lstlisting}
\section{Lecture: Week 12}

\subsection{Kronecker's theorem}

We begin with a classical theorem of Kronecker on algebraic integers. Recall 
that $\alpha\in\C$ is an \emph{algebraic integer} if there is a monic
polynomial $f\in\Z[X]$ such that $f(\alpha)=0$ (see Definition~\ref{def:algebraic_integer}). Let $\A$ 
be the set of algebraic integers. 

\begin{exercise}
    \label{xca:irreducible}
    Let $\alpha\in\A$. Prove that there exists a monic polynomial $f\in\Z[X]$, 
    irreducible $f\in\Q[X]$ such that $f(\alpha)=0$. 
\end{exercise}

The polynomial of Exercise~\ref{xca:irreducible} is called the \emph{minimal polynomial} of $\alpha$. 

\begin{exercise}
\label{xca:distinct}
    Let $\alpha\in\A$. Prove that the roots of the
    minimal polynomial of $\alpha$ are pairwise distinct. 
\end{exercise}

The \emph{conjugates} of $\alpha$ are the roots of the minimal polynomial of $\alpha$. 

Recall that for an $n\times n$ matrix $A=(a_{ij})$, its \emph{norm} (more precisely, \emph{infinity-norm}) 
is defined 
as the maximum absolute row sum of the matrix, that is 
\[
\|A\|=\max_{1\leq i\leq n}\sum_{j=1}^n|a_{ij}|.
\]

For $A,B\in\C^{n\times n}$ and $\lambda\in\C$, 
the following properties hold: 
\begin{enumerate}
    \item $\|A\|\geq0$.
    \item $\|A\|=0$ if and only if $A$ is the $n\times n$ zero matrix. 
    \item $\lambda\|A\|=|\lambda|\|A\|$. 
    \item $\|A+B\|\leq\|A\|+\|B\|$. 
    \item $\|AB\|\leq\|A\|\|B\|$. 
\end{enumerate}

For our purposes, the choice of norm is not important at all (and any other norm could have been chosen). Nevertheless, we provide an example. Let  
\[
A=\begin{pmatrix}
    1 & 2 & -3\\
    0 &-5 &-7\\
    11 & 2 &-3
\end{pmatrix}.
\]
Then $\|A\|=\max\{6,12,16\}=16$. 

\begin{theorem}[Kronecker]
\index{Kronecker theorem}
\label{thm:Kronecker}
Let $\alpha\in\A$. Assume that all the conjugates of $\alpha$ 
have absolute value at most one. Then either $\alpha=0$ or $\alpha$ is a root of one. 
\end{theorem}

\begin{proof}
    Assume that $\alpha\ne 0$. 
    Let $f\in\Z[X]$ be the minimal polynomial of $\alpha$, say 
    \[
    f=X^n+a_{n-1}X^{n-1}+\cdots+a_1X+a_0
    \]
    for integers $a_0,a_1,\dots,a_{n-1}\in\Z$. Then $f(0)\ne0$ because $f$ is irreducible in $\Q[X]$ (see Exercise~\ref{xca:irreducible}).  
    Let 
    \[
    F=\begin{pmatrix}
        0 & 0 & \cdots & 0 & -a_0\\
        1 & 0 & \cdots & 0 & -a_1\\
        0 & 1 & \cdots & 0 & -a_2\\
        \vdots & \vdots & \ddots & \vdots & \vdots\\
        0 & 0 & \cdots & 1 & -a_{n-1}
    \end{pmatrix}\in\Z^{n\times n}
    \]
    be the \emph{companion matrix} of $f$. The characteristic polynomial and the minimal polynomial of 
    the matrix $F$ are equal to $f$. Moreover, the roots of $f$ are the eigenvalues of $F$. Since 
    all the roots of $f$ are distinct, all the eigenvalues of $F$ are different. Thus 
    $F$ is diagonalizable, so there exists $P\in\GL_n(\C)$ such that $F=PDP^{-1}$, where
    $D$ is the $n\times n$ diagonal matrix with diagonal 
    entries $\alpha=\alpha_1,\alpha_2,\dots,\alpha_n$, 
    the roots of $f$ (i.e., the conjugates of $\alpha$), 
    so all with absolute value at most one. Thus 
    $\|D\|\leq 1$. 
    Since $0\not\in\{\alpha_1,\dots,\alpha_n\}$, the matrix
    $F$ is invertible. Moreover, 
    \[
    F^k=(PDP^{-1})^k=PD^kP^{-1}
    \]
    for all $k\geq1$. Note that the set 
    $X=\{F^k:k\geq 1\}\subseteq M_n(\Z)$ is bounded in $M_n(\C)$, 
    as 
    \[
    \|F^k\|=\|PD^kP^{-1}\|\leq 
    \|P\|\|D\|^k\| P^{-1}\|
    \leq \underbrace{\|P\|\| P^{-1}\|}_{\text{This is independent of $k$}}.
    \]
    Thus $X$ is finite. In particular, there are integers $i<j$ such that 
    $F^i=F^j$. Since $F$ is invertible, $F^{j-i}$ is the $n\times n$ 
    identity matrix. Since $\alpha$ is an eigenvalue of $F$, it follows
    that $\alpha^{j-i}=1$.  
    % Let $\{e_1,\dots,e_n\}$ be the standard basis of $\C^{n\times1}$. 
    % A direct calculation shows that $Fe_{j}=e_{j+1}$ for all $j\in\{2,\dots,n-1\}$ and 
    % \begin{align*}
    %     Fe_n&=-a_0e_1-a_1e_2-\cdots-a_{n-1}e_n.
    % \end{align*} 
\end{proof}

The proof of the theorem presented here goes back to Greiter~\cite{MR514044}. 
Kronecker’s original proof is somewhat similar, relying on 
Vieta’s formulas and estimates involving binomial coefficients; see~\cite{MR1834706}.


\subsection{Solvable groups and Burnside's theorem}

\index{Derived series}
For a group $G$ let 
$G^{(0)}=G$ and 
$G^{(i+1)}=[G^{(i)},G^{(i)}]$ for $i\geq0$.
The \emph{derived series} of $G$ is the sequence
\[
G=G^{(0)}\supseteq G^{(1)}\supseteq G^{(2)}\supseteq\cdots
\]
Each $G^{(i)}$ is a characteristic subgroup of $G$. We say that 
$G$ is \emph{solvable} if $G^{(n)}=\{1\}$ for some $n$.  

\begin{example}
	Abelian groups are solvable. 
\end{example}

\begin{example}
	The group $\SL_2(3)$ is solvable. 
	Let us see what the computer says:
\begin{lstlisting}
> G := SL(2,3);;
> IsSolvable(G);
true
> [GroupName(x) : x in DerivedSeries(G)];
[ SL(2,3), Q8, C2, C1 ]
\end{lstlisting}
\end{example}

\begin{example}
	Non-abelian simple groups cannot be solvable. 
\end{example}

For $n\geq5$, the group $\Alt_n$ is not solvable.

\begin{exercise}
	\label{xca:solvable}
	Let $G$ be a group. Prove the following statements:
	\begin{enumerate}
		\item A subgroup $H$ of $G$ is solvable, when $G$ is solvable.
		\item Let $K$ be a normal subgroup of $G$. 
		    Then $G$ is solvable if and only if $K$ and $G/K$ are solvable.
	\end{enumerate}
\end{exercise}

For $n\geq5$, the group $\Sym_5$ is not solvable. 

\begin{exercise}
\label{xca:pgroups_solvable}
	Let $p$ be a prime number. Prove that 
	finite $p$-groups are solvable.
\end{exercise}

Exercises~\ref{xca:solvable} and~\ref{xca:pgroups_solvable} may be omitted if the reader is already familiar with solvable groups.

\begin{theorem}[Burnside]
	\index{Burnside!theorem}
	\label{thm:Burnside_auxiliar}
	Let $G$ be a finite group. If $\phi\colon G\to\GL_n(\C)$ is a representation
	with character $\chi$ and $C$ is a conjugacy class of $G$ such that 
	$\gcd(|C|,n)=1$, then for every $g\in C$ either 
	$\chi(g)=0$ or $\phi_g$ is a scalar matrix. 
\end{theorem}

% We need a lemma.

% \begin{lemma}
% 	\label{lem:4Burnside}
% 	Let $\epsilon_1,\dots,\epsilon_n$ be roots of one such that 
% 	$(\epsilon_1+\cdots+\epsilon_n)/n\in\A$. Then either 
% 	$\epsilon_1=\cdots=\epsilon_n$ or 
% 	$\epsilon_1+\cdots+\epsilon_n=0$.
% \end{lemma}

% \begin{proof}
% 	Let $\alpha=(\epsilon_1+\cdots+\epsilon_n)/n$.
% 	If the $\epsilon_j$s are not all equal, then $\|\alpha\|<1$. Moreover, 
% 	$\|\beta\|<1$ for every algebraic conjugate $\beta$ of $\alpha$. Since the product 
% 	of the algebraic conjugates of $\alpha$ is an integer of absolute value 
% 	$<1$, it follows that it is zero. 
% \end{proof}

%Now we prove the theorem.

 \begin{proof}
% [Proof of Theorem \ref{thm:Burnside_auxiliar}]
	Let $\epsilon_1,\dots,\epsilon_n$ be the eigenvalues of $\phi_g$. Then 
    $\epsilon_1,\dots,\epsilon_n$ are roots of one. 
    By assumption, 
	$\gcd(|C|,n)=1$, there exist $a,b\in\Z$ such that $a|C|+bn=1$. Since 
	$|C|\chi(g)/n\in\A$, after multiplying by $\chi(g)/n$ we obtain that  
	\[
		a|C|\frac{\chi(g)}{n}+b\chi(g)=\frac{\chi(g)}{n}=\frac{1}{n}(\epsilon_1+\cdots+\epsilon_n)\in\A.
	\]
    Let $\alpha_1=\chi(g)/n\in\A$ and $\alpha_2,\dots,\alpha_n$ be its conjugates. Since $|\alpha_1|\leq 1$ 
    and $\alpha_2,\dots,\alpha_n$ are conjugates of $\alpha_1$, it follows that  
    $|\alpha_j|\leq 1$ for all $j\in\{1,\dots,n\}$. By Kronecker's theorem, 
    either $\alpha_1=0$ or $\alpha_1$ is a root of one. If $\alpha_1=0$, then $\chi(g)=0$. If 
    $\alpha_1$ is a root of one, then 
    \[
    1=|\alpha_1|=\frac{|\chi(g)|}{n}=\frac1{n}.
    \]
    Thus $|\chi(g)|=n=\chi(1)$. This means that $g\in\Z(\chi)$. By Exercise~\ref{xca:center}, 
    $\phi_g$ is a scalar matrix. 
	% The previous lemma implies that there are two cases to consider: 
	% either $\epsilon_1=\cdots=\epsilon_n$ or $\epsilon_1+\cdots+\epsilon_n=0$. In the first
	% case, since $\phi_g$ is diagonalizable, $\phi_g$ is a scalar matrix. 
	% In the second case, $\chi(g)=0$.
\end{proof}

\begin{theorem}[Burnside]
	\index{Burnside!theorem}
    \label{thm:pq_notsimple}
	Let $p$ be a prime number. If $G$ is a finite group and 
	$C$ is a conjugacy class of $G$ with $p^k>1$ elements, then $G$ 
	is not simple.
\end{theorem}

\begin{proof}
	Let $g\in C\setminus\{1\}$. Column orthogonality implies that 
	\begin{equation}
	\label{eq:Burnside}
	\begin{aligned}
		0&=\sum_{\chi\in\Irr(G)}\chi(1)\chi(g)\\
		&=\sum_{p\mid\chi(1)}\chi(1)\chi(g)+\sum_{p\nmid\chi(1):\chi\ne\chi_1}\chi(1)\chi(g)+1,
	\end{aligned}
	\end{equation}
	where the one corresponds to the trivial representation of
	$G$.
	
	Look at this equation modulo $p$. If $\chi(g)=0$ for all
	$\chi\in\Irr(G)$
	such that $\chi\ne\chi_1$ and $p\nmid\chi(1)$, then
	\[
	-\frac{1}{p}=\sum\frac{\chi(1)}{p}\chi(g)\in\A\cap\Q=\Z,
	\]
	where the sum is taken over all non-trivial irreducibles
	of $G$ of degree divisible by $p$, a contradiction. Hence 
	there exists an irreducible non-trivial representation 
	$\phi$ with character $\chi$ such that $p$ does not divide
	$\chi(1)$ and $\chi(g)\ne0$. By the previous theorem, 
	$\phi_g$ is a scalar matrix. If $\phi$ is faithful, then 
	$g$ is a non-trivial central element, a contradiction since 
    $|C|>1$. If $\phi$ is not faithful, then 
    $G$ is not simple (because 
	$\ker\phi$ is a non-trivial proper normal subgroup of $G$). 
\end{proof}

\begin{theorem}[Burnside]
  \index{Burnside!$p^aq^b$-theorem}
  \label{thm:pq}
  Let $p$ and $q$ be prime numbers. If $G$ has order $p^aq^b$, then $G$ is solvable.
\end{theorem}

\begin{proof}
	If $G$ is abelian, then it is solvable.
	Suppose now $G$ is non-abelian.
	Let us assume that the theorem is not true. Let $G$ be a group
	of minimal order $p^aq^b$
	that is not solvable. Since $|G|$ is minimal, $G$ is a non-abelian simple group.
	By the previous theorem, 
	$G$ has no conjugacy classes of size $p^k$ nor 
	conjugacy classes of size $q^l$ with $k,l\geq1$. The size
	of every conjugacy class of $G$ is one or divisible by $pq$. 
	Note that, since $G$ is a non-abelian simple group,
	the center of $G$ is trivial.
	Thus there is only one conjugacy class of size one.
	By the class
	equation,
	\[
		|G|=1+\sum_{C:|C|>1}|C|\equiv 1 \bmod pq,
	\]
	where the sum is taken over all conjugacy classes of $G$ 
	with more than one element, a contradiction.
\end{proof}

%\subsection{Some generalizations of Burnside's theorem}

There are several interesting theorems that generalize Burnside’s theorem. 
We briefly mention two of them.

\begin{optional}
\begin{theorem}[Kegel--Wielandt]
    \index{Kegel--Wielandt theorem}
    \label{thm:KegelWielandt}
    If $G$ is a finite group and there are nilpotent subgroups 
    $A$ and $B$ of $G$ such that 
    $G=AB$, then $G$ is solvable.
\end{theorem}

See~\cite[Theorem 2.4.3]{MR1211633} for the proof.

\begin{xca}
     Prove that Theorem~\ref{thm:KegelWielandt} 
     implies Theorem~\ref{thm:pq}.
\end{xca}
\end{optional}

There is an interesting variation on the Kegel--Wielandt theorem, but with an extra condition on the
order of the finite group. 

\begin{optional}
\begin{theorem}[Syskin]
    \index{Syskin theorem}
    Let $G$ be a finite group of order coprime with three. If 
    $G=AB$ for solvable subgroups $A$ and $B$, then $G$ is 
    solvable. 
\end{theorem}

For the proof, see~\cite{MR537379}. 
\end{optional}

\begin{optional}
\index{Word maps}
Another generalization of Burnside's theorem
is based on \emph{word maps}. A word map
of a group $G$ is a map 
\[
G^k\to G,\quad 
(x_1,\dots,x_k)\mapsto w(x_1,\dots,x_k)
\]
for some word $w(x_1,\dots,x_k)$ of the free group $F_k$ of rank $k$. 
Some word maps are surjective in certain families of groups. For example, 
Ore's conjecture is precisely the surjectivity of the word map
$(x,y)\mapsto [x,y]=xyx^{-1}y^{-1}$ in every finite non-abelian simple 
group. 

\begin{theorem}[Guralnick--Liebeck--O'Brien--Shalev--Tiep]
\index{Guralnick--Liebeck--O'Brien--Shalev--Tiep theorem}
    Let $a,b\geq0$, $p$ and $q$ be prime numbers and $N=p^aq^b$. The map 
    $(x,y)\mapsto x^Ny^N$ is surjective in every finite simple group. 
\end{theorem}

The proof appears in~\cite{MR3827208}. 

The theorem implies Burnside's theorem. Let $G$ be a group of order
$N=p^aq^b$. Assume that $G$ is not solvable. 
Fix a composition series of $G$. There is a non-abelian factor $S$ 
of order that divides $N$. Since 
$S$ is simple non-abelian and $s^N=1$, it follows that the word map
$(x,y)\mapsto x^Ny^N$ has trivial image in $S$, a contradiction 
to the theorem. 
\end{optional}


\subsection{The Feit--Thompson theorem}

\begin{theorem}[Feit--Thompson]
    \index{Feit--Thompson!theorem}
    Groups of odd order are solvable. 
\end{theorem}

The proof of Feit--Thompson theorem is extremely hard. 
It occupies a full volume of the 
\emph{Pacific Journal of Mathematics}~\cite{MR166261}. 
A formal verification of the proof 
(based on the computer software Coq) 
was announced in~\cite{MR3111271}.  

\begin{optional}
    
Back in the day it was believed that if a certain divisibility 
conjecture is true, 
the proof of Feit--Thompson theorem 
could be simplified. 

\begin{conj}[Feit--Thompson]
\index{Feit--Thompson!conjecture}
    There are no prime numbers $p$ and $q$ such that
    $\frac {p^{q}-1}{p-1}$ divides $\frac{q^{p} - 1}{q - 1}$. 
\end{conj}

The conjecture remains open. However, now we know that 
proving the conjecture will not simplify further
the proof of Feit--Thompson theorem. 

In 2012 Le proved that the conjecture is true for $q=3$, see 
\cite{MR2900154}. 


In~\cite{MR297686} 
Stephens proved that a certain stronger version of the conjecture 
does not hold, as the integers 
$\frac {p^{q}-1}{p-1}$ and $\frac{q^{p} - 1}{q - 1}$ 
could have common factors. In fact, if $p=17$ and $q=3313$, 
then 
\[
\gcd\left(\frac {p^{q}-1}{p-1},\frac{q^{p} - 1}{q - 1}\right)=112643.
\]

Nowadays, this can be easily checked on almost any desktop computer. 
As far as I know, no other counterexamples to Stephen’s stronger version of the conjecture have been found.

% \begin{lstlisting}
% gap> Gcd((17^3313-1)/16,(3313^17-1)/3312);
% 112643
% \end{lstlisting}
% \begin{lstlisting}
% > p := 17; 
% > q := 3313;
% > bool, a := IsCoercible(Integers(), (p^q-1)/(p-1));
% > bool, b := IsCoercible(Integers(), (q^p-1)/(q-1));
% > Gcd(a,b);
% 112643    
% \end{lstlisting}
\end{optional}





\TOCpart{Part 2}

\pagestyle{fancy}
\fancyhf{}
\fancyfoot[R]{\thepage}
\fancyhead[L]{\course}
\fancyhead[R]{Final projects}
\setlength{\headheight}{14pt}
\newpage

\section{Project: Irreducible characters of dihedral groups}
\index{Character table!of dihedral groups}
Let $n\geq3$. 
Recall that the \emph{dihedral group} of order $2n$ 
is the group
\[
\D_{n}=\langle r,s:r^n=s^2=1,srs=r^{-1}\rangle. 
\]
Every element of $\D_{n}$ is of the form
$s^ir^j$ for some $i\in\{0,1\}$ and $j\in\{0,\dots,n-1\}$. 

Our goal is the construct the character table of $\D_{n}$. 

\begin{proposition}
\label{pro:classes_dihedral}
    Let $n\geq3$. If $n$ is odd, then 
    \[
    \{1\},
    \{r,r^{-1}\},
    \{r^2,r^{-2}\},
    \cdots,\{r^{(n-1)/2},r^{(1-n)/2}\},
    \{s,sr,sr^2,\dots,sr^{n-1}\}
    \]
    are the conjugacy classes of $\D_n$. If $n$ is even, then
    \begin{align*}
        \{1\},
    \{r,r^{-1}\},
    \{r^2,r^{-2}\},
    \cdots,&\{r^{n/2-1},r^{1-n/2}\},\\
    &\{r^{n/2}\},
    \{s,sr^2,sr^4,\dots,sr^{n-2}\},
    \{sr,sr^3,\dots,sr^{n-1}\}
    \end{align*}
    are the conjugacy classes of $\D_n$.
\end{proposition}

\begin{proof}
    Recall that $sr^j=r^{-j}s$ for all $j$. Let $g=s^ir^j\in\D_n$ and 
    $x=s^kr^l\in\D_n$. Let us compute $xgx^{-1}$. We split the
    proof into several steps. 

    Assume first that $i=0$, that is $g=r^j$. Then
    \[
    xgx^{-1}=(s^kr^l)r^j(r^{-l}s^{-k})=s^kr^js^{-k}
    =\begin{cases}
        r^j & \text{if $k=0$,}\\
        r^{-j} & \text{if $k=1$.}
    \end{cases}
    \]
    Hence the conjugacy class of $g=r^j$ is $\{r^j,r^{-j}\}$. 

    Now assume that $i=1$, that is $g=sr^j$. Since $k\in\{0,1\}$, a direct
    calculation using the fact that $r^ls=sr^{-l}$ yields 
    \[
    xgx^{-1}=\begin{cases} 
        sr^{-2l+j} & \text{if $k=0$,}\\
        sr^{2l-j} & \text{if $k=1$.}
        \end{cases}
    \]
    Hence the conjugacy class of $g=sr^j$ is $\{sr^{2l-j},sr^{-2l+j}:0\leq l\leq n-1\}$. 

    Assume that $n$ is odd. We have determined the conjugacy classes 
    \[
    \{1\},\{b,b^{-1}\},\{b^2,b^{-2}\},\dots,\{b^{(n-1)/2},b^{(1-n)/2}\}
    \]
    which together cover all the elements of the subgroup 
    $\langle b\rangle=\{1,b,b^2,\dots,b^{n-1}\}$. Since $n$ is odd, for every 
    integer 
    $m$ there exists an integer $x$ such that $2x\equiv m\bmod n$. Thus 
    the conjugacy class of $s$ is $\{s,sr,sr^2,\dots,sr^{n-1}\}$. These classes together cover 
    all the elements of $\D_n$. 
    
    Now assume that $n$ is even. We have determined the conjugacy classes 
    \[
    \{1\},\{b,b^{-1}\},\{b^2,b^{-2}\},\dots,\{b^{n/2-1},b^{1-n/2}\},\{b^{n/2}\}
    \]
    which together cover all the elements of the subgroup 
    $\langle b\rangle=\{1,b,b^2,\dots,b^{n-1}\}$. The class of $s$ is $\{s,sr^2,sr^4,\dots,sr^{n-2}\}$ and 
    the class of $sr$ is $\{sr,sr^3,\dots,sr^{n-1}\}$. These classes together cover 
    all the elements of $\D_n$. 
\end{proof}

The previous proposition gives the number of conjugacy classes of the dihedral group $\D_n$, namely
\[
\frac{2n+9+(-1)^n3}{4}=\begin{cases}
    \frac{n+6}{2} & \text{if $n$ is even},\\
    \frac{n+3}{2} & \text{if $n$ is odd}.\\
\end{cases}
\]
This number is precisely the number of irreducible representations of $\D_n$. 

\begin{exercise}
\label{xca:center_dihedral}
    Compute $Z(\D_n)$. 
\end{exercise}

\begin{exercise}
\label{xca:cp_dihedral}
    Prove that $\lim_{n\to\infty}\cp(\D_{n})=1/4$.
\end{exercise}

To determine the number of degree-one representations of our group, 
we need the index of the commutator subgroup.

\begin{exercise}
\label{xca:commutator_dihedral}
    Prove that $[\D_n,\D_n]=\langle r^2\rangle$. Moreover, 
    \[
    (G:[G,G])=\begin{cases}
        2 & \text{if $n$ is odd.}\\
        4 & \text{if $n$ is even.}
    \end{cases}
    \]
\end{exercise}

\subsection{$n$ odd}

By Proposition~\ref{pro:classes_dihedral}, the representatives of the conjugacy classes of $\D_n$ are 
$1,r,r^2,\dots,r^{(n-1)/2},s$. 
By Exercise~\ref{xca:commutator_dihedral}, there are two degree-one characters, namely the trivial character 
and the character $\eta$ such that $r\mapsto 1$ and $s\mapsto -1$. 

\bigskip 
\begin{center}
    \begin{tabular}{|c|cccccc|}
         \hline 
         & $1$ & $r$ & $r^2$ & $\cdots$  & $r^{(n-1)/2}$ & $s$\\
         \hline 
         trivial & $1$ & $1$ & $1$ & $\cdots$ & $1$ & $1$\\
         $\eta$ & $1$ & $1$ & $1$ & $\cdots$ & $1$ & $-1$\\
         \hline 
    \end{tabular}
\end{center}
\bigskip 

Assume that $n=2k-1$. 
We need $\frac{n-1}{2}=k-1$ additional irreducible characters.
For $m\in\{1,\dots,k-1\}$, let $\omega_m=e^{2\pi im/k}$ and 
\[
\rho_m\colon\D_n\to\GL_2(\C),\quad 
r\mapsto\begin{pmatrix}\omega_m&0\\0&\omega_m^{-1}\end{pmatrix},\quad 
s\mapsto\begin{pmatrix}0&1\\1&0\end{pmatrix}.
\]

\begin{exercise}
\label{xca:rho_m}
    Prove that each $\rho_m$ is a group homomorphism.
\end{exercise}

A direct calculation produces the values of the 
character $\chi_m$ of $\rho_m$.

\bigskip 
\begin{center}
    \begin{tabular}{|c|cccccc|}
         \hline 
         & $1$ & $r$ & $r^2$ & $\cdots$  & $r^{(n-1)/2}$ & $s$\\
         \hline 
         $\chi_m$ & $1$ & $\omega_m+\omega_m^{-1}$ & $\omega_m^2+\omega_m^{-2}$ & $\cdots$ & $\omega_m^{(n-1)/2}+\omega_m^{(1-n)/2}$ & 0\\
         \hline 
    \end{tabular}
\end{center}
\bigskip 

\begin{exercise}
\label{xca:chim_alldifferent}
    Let $i,j\in\{1,\dots,k-1\}$. Prove that $\chi_i\ne\chi_j$ whenever $i\ne j$.
\end{exercise}

\begin{exercise}
\label{xca:chim_irreducible}
    Prove that each $\chi_m$ is irreducible. 
\end{exercise}

It remains only to note that we have constructed 
$\frac{n+3}{2}$ irreducible characters of $\D_n$, 
so the character table of $\D_n$ for odd $n$ 
is complete!

\subsection{$n$ even}

In this case, by Exercise~\ref{xca:commutator_dihedral}, there are four 
degree-one representations. These are the group homomorphisms defined as follows: For $j\in\{1,2,3,4\}$, let $\eta_j\colon\D_n\to\C$ be given by 
\begin{align*}
    \eta_1(r)&=1, & \eta_2(r)&=1, &  \eta_3(r)&=-1, & \eta_4(r)&=-1,\\
    \eta_1(s)&=1, & \eta_2(s)&=-1, & \eta_3(s)&=1, &  \eta_4(s)&=-1.\\
\end{align*}

Of course, $\eta_1$ is the trivial character of $\D_n$. 
By a direct calculation, we compute the values of the other characters: 
\bigskip 
\begin{center}
    \begin{tabular}{|c|ccccccc|}
         \hline 
         & $1$ & $r$ & $r^2$ & $\cdots$  & $r^{n/2}$ & $s$ & $sr$\\
         \hline 
         $\eta_1$ & $1$ & $1$ & $1$ & $\cdots$ & $1$ & $1$ & $1$\\
         $\eta_2$ & $1$ & $1$ & $1$ & $\cdots$ & $1$ & $-1$ & $-1$ \\
         $\eta_3$ & $1$ & $-1$ & $1$ & $\cdots$ & $(-1)^{n/2}$ & $1$ & $(-1)^{n/2}$ \\
         $\eta_4$ & $1$ & $-1$ & $1$ & $\cdots$ & $(-1)^{n/2}$ & $-1$ & $(-1)^{n/2+1}$\\
         \hline 
    \end{tabular}
\end{center}
\bigskip 

Assume now that $n=2k$. 
We need $\frac{n-1}{2}=k-1$ additional irreducible characters. For 
$m\in\{1,\dots,k-1\}$, let $\omega_m=e^{2\pi im/k}$ and 
\[
\rho_m\colon\D_n\to\GL_2(\C),\quad 
r\mapsto\begin{pmatrix}\omega_m&0\\0&\omega_m^{-1}\end{pmatrix},\quad 
s\mapsto\begin{pmatrix}0&1\\1&0\end{pmatrix}.
\]

Each $\rho_m$ is a group homomorphism (see Exercise~\ref{xca:rho_m}). 
A direct calculation produces the values of the 
character $\chi_m$ of $\rho_m$.

\bigskip 
\begin{center}
    \begin{tabular}{|c|ccccccc|}
         \hline 
         & $1$ & $r$ & $r^2$ & $\cdots$  & $r^{n/2}$ & $s$ & $sr$\\
         \hline 
         $\chi_m$ & $1$ & $\omega_m+\omega_m^{-1}$ & $\omega_m^2+\omega_m^{-2}$ & $\cdots$ & $\omega_m^{n/2}+\omega_m^{-n/2}$ & $0$ & $0$\\
         \hline 
    \end{tabular}
\end{center}
\bigskip 

In the same way that we constructed the character table when $n$ is odd, 
we now need to verify that we have constructed 
$\frac{n+6}{2}$ irreducible characters of $\D_n$. 

\begin{exercise}
\label{xca:n_even}
    Prove that we have constructed $\frac{n+6}{2}$ irreducible characters of $\D_n$. 
\end{exercise}
\section{Project: Hurwitz' theorem}

\index{Fibonacci identity}
\index{Euler identity}
\index{Hamilton identity}
We know that $x^2y^2=(xy)^2$ holds for all $x,y\in\R$. Fibonacci
found the identity
\begin{equation}
\label{eq:2squares}
	(x_1^2+x_2^2)(y_1^2+y_2^2)=(x_1y_1-x_2y_2)^2+(x_1y_2-x_2y_1)^2.
\end{equation}
Euler and Hamilton independently found 
a similar identity:
\[
	(x_1^2+x_2^2+x_3^2+x_4^2)(y_1^2+y_2^2+y_3^2+y_4^2)=z_1^2+z_2^2+z_3^2+z_4^2,
\]
where
\begin{equation}
\label{eq:Hamilton}
\begin{aligned}
	 z_1&=x_1y_1-x_2y_2-x_3y_3-x_4y_4,\\
	 z_2&=x_1y_2+x_2y_1+x_3y_4-x_4y_3,\\
	 z_3&=x_1y_3-x_2y_4+x_3y_1+x_4y_2,\\ 
	 z_4&=x_1y_4+x_2y_3-x_3y_2+x_4y_1.
\end{aligned}
\end{equation}
Cayley found a similar identity for sums of eight squares. 
Are there other identities of this type? Hurwitz
proved that this is not the case. 

The question can be reformulated as follows. For which $n$ does there 
exist a bilinear map $\R^n\times\R^n\to\R^n$, 
$(x,y)\mapsto xy$, such that
\[
\|xy\|=\|x\|\|y\|
\]
for all $x,y\in\R^n$? Here, of course, we use the 
standard notation
\[
\|(x_1,\dots,x_n)\|=\sqrt{x_1^2+\cdots+x_n^2}.
\]

 \begin{lemma}
 \label{lem:hurwitz_group}
 	Let $n>2$ be an even number. If 
 	there exists a group $G$ with generators
 	$\epsilon,x_1,\dots,x_{n-1}$ and relations 
 	\[
 		x_1^2=\cdots=x_{n-1}^2=\epsilon\ne1,\quad
 		\epsilon^2=1,\quad
 		[x_i,x_j]=\epsilon\quad\text{if}\quad i\ne j,
 	\]
 	then the following statements hold:
 	\begin{enumerate}
 		\item $|G|=2^n$.
 		\item $[G,G]=\{1,\epsilon\}$. In particular, $G$ 
 		    has exactly $2^{n-1}$ degree-one representations. 
 		\item If $g\not\in Z(G)$, then the conjugacy class of $g$ is $\{g,\epsilon g\}$.
 		\item $Z(G)=\{1,\epsilon,x_1\cdots x_{n-1},\epsilon x_1\cdots x_{n-1}\}$. 
 		\item $G$ has $2^{n-1}+2$ conjugacy classes.
 		\item $G$ has two irreducible representations of degree $2^{\frac{n-2}{2}}>1$. 
 	\end{enumerate}
 \end{lemma}

 \begin{proof}
     Let us prove 1) and 2). Note that $\epsilon\in Z(G)$, as
     $\epsilon=x_i^2$ for all 
 	$i\in\{1,\dots,n-1\}$. Since $n-1>2$, $[x_1,x_2]=\epsilon$. Hence 
 	$\epsilon\in [G,G]$. Moreover, $G/\langle\epsilon\rangle$ is abelian. Thus 
 	$[G,G]=\langle \epsilon\rangle$. Since $G/[G,G]$ is elementary 
 	abelian of order 
 	$2^{n-1}$, it follows that 
 	$|G|=2^n$. 

 	We now prove 3). Let $g\in G\setminus Z(G)$ and 
 	$x\in G$ be such that $[x,g]\ne 1$. Then $[x,g]=\epsilon$ and 
 	$xgx^{-1}=\epsilon g$. 

 	To prove 4) let $g\in G$. Write
 	\[
 		g=\epsilon^{a_0}x_1^{a_1}\cdots x_{n-1}^{a_{n-1}},
 	\]
 	where $a_j\in\{0,1\}$ for all $j\in\{1,\dots,n-1\}$. 
 	If $g\in Z(G)$, then $gx_i=x_ig$ for all $i$. Hence 
 	$g\in Z(G)$ if and only if 
 	\[
 		\epsilon^{a_0}x_1^{a_1}\cdots x_{n-1}^{a_{n-1}}=x_i(\epsilon^{a_0}x_1^{a_1}\cdots x_{n-1}^{a_{n-1}})x_i^{-1}.
 	\]
 	Since $x_ix_j^{a_j}x_i=\epsilon^{a_j}x_j^{a_j}$ 
 	whenever $i\ne j$ and $\epsilon\in Z(G)$, the element $g$ is 
 	central if and only if 
 	\[
 		\sum_{\substack{j=1\\j\ne i}}^{n-1}a_j\equiv 0\bmod 2
 	\]
 	for all $i\in\{1,\dots,n-1\}$. In particular, 
 	\[
 	\sum_{j\ne i}a_j\equiv \sum_{j\ne k}a_j
 	\]
 	for all $k\ne i$. Therefore $a_i\equiv a_k\bmod 2$ for all 
 	$i,k\in\{1,\dots,n-1\}$. Thus $a_1=\cdots=a_{n-1}$ and  
 	$Z(G)=\{1,x_1\cdots x_{n-1},\epsilon,\epsilon x_1\cdots
 	x_{n-1}\}$. 
	
     To prove 5) we use the class equation:
     \[
 		2^n=|G|=|Z(G)|+\sum_{i=1}^N2=4+2N. 
 	\]
 	It follows that $G$ has $N+4=2^{n-1}+2$ conjugacy classes.
	
 	Finally we prove 6). 
 	Since $G$ 
 	has exactly $2^{n-1}$ degree-one representations (because 
 	$|G/[G,G]|=2^{n-1}$) and 
 	has $2^{n-1}+2$ conjugacy classes, 
 	it follows from 
 	\[
 		2^n=|G|=\underbrace{1+\cdots+1}_{2^{n-1}}+f_1^2+f_2^2=2^{n-1}+f_1^2+f_2^2,
 	\]
 	that $G$ has two irreducible representations
 	of degrees $f_1=f_2=2^{\frac{n-2}{2}}>1$. 
 \end{proof}

 \begin{example}
 	The formulas~\eqref{eq:Hamilton} give a representation for the
 	group $G$ of the previous lemma. Write each $z_i$ as 
 	\[
    z_i=\sum_{k=1}^4a_{1k}(x_1,\dots,x_4)y_k.
    \]
    Let $A$ be a matrix
 	such that 
 	$A_{ij}=a_{ij}(x_1,\dots,x_4)$, that is 
 	\[
 		A=\begin{pmatrix}
 			x_1 & -x_2 & -x_3 & -x_4\\
 			x_2 & x_1 & -x_4 & x_3\\
 			x_3 & x_4 & x_1 & -x_2\\
 			x_4 & -x_3 & x_2 & x_1
 		\end{pmatrix}.
 	\]
 	The matrix $A$ can be written as $A=A_1x_1+A_2x_2+A_3x_3+A_4x_4$, where
 	\begin{align*}
 		&A_1=\begin{psmallmatrix}
 		1\\
 		&1\\
 		&&1\\
 		&&&1\\
 		\end{psmallmatrix},
 		&&
 		A_2=\begin{psmallmatrix}
 			& -1\\
 			1 \\
 			&&&-1\\
 			&&1
 		\end{psmallmatrix},
 		&&
 		A_3=\begin{psmallmatrix}
 			&& -1 \\
 			&&&1 & \\
 			1\\
 			&-1
 		  \end{psmallmatrix},
 		  &&
 		  A_4=\begin{psmallmatrix}
 			&&&-1\\
 			&&-1\\
 			&1\\
 			1
 		\end{psmallmatrix}.
 	\end{align*}
 	For $i\in\{1,\dots,4\}$, let $B_i=A_4^TA_i$. Then
 	$B_i=-B_i^T$ and  $B_i^2=-I$ 
 	for all $i\in\{1,2,3\}$. Moreover, $B_iB_j=-B_jB_i$ for all $i,j\in\{1,2,3\}$ and
 	$i\ne j$.  
 	The group generated by $\{B_1,B_2,B_3\}$ has $2^3$ element, all of them
 	of the form
 	\[
 		\pm B_1^{k_1}B_2^{k_2}B_3^{k_3}
 	\]
 	for $k_j\in\{0,1\}$. This group provides an example 
    of the group $G$ of Lemma~\ref{lem:hurwitz_group}. 
  %   The map 
 	% \[
 	% 	x_1\mapsto B_1,\quad
 	% 	x_2\mapsto B_2,\quad
 	% 	x_3\mapsto B_3 
 	% \]
 	% extends to a group isomomorphism 
  %   \[
    
  %   \]
 \end{example}

\begin{theorem}[Hurwitz]
	\index{Hurwitz theorem}
    \label{thm:Hurwitz}
	If there is an identity of the form 
	\begin{equation}
		\label{eq:Hurwitz}
		(x_1^2+\cdots+x_n^2)(y_1^2+\cdots+y_n^2)=z_1^2+\cdots+z_n^2,
	\end{equation}
	where the $x_j$'s and the $y_j$'s are real numbers and
	each $z_k$ is a bilinear function in the $x_j$'s and the $y_j$'s, then 
	$n\in\{1,2,4,8\}$.
\end{theorem}

\begin{proof}
    Without loss of generality, we may assume that $n>2$.  For 
	$i\in\{1,\dots,n\}$ let  
	\[
		z_i=\sum_{k=1}^n a_{ik}(x_1,\dots,x_n)y_k,
	\]
	where the $a_{ik}$'s are linear functions. Then 
	\[
		z_i^2=\sum_{k,l=1}^na_{ik}(x_1,\dots,x_n)a_{il}(x_1,\dots,x_n)y_ky_l
	\]
	for all $i\in\{1,\dots,n\}$. Using these expressions for each $z_i$
	in~\eqref{eq:Hurwitz} and comparing coefficients, 
	\begin{equation}
		\label{eq:delta}
		\sum_{i=1}^n a_{ik}(x_1,\dots,x_n)a_{il}(x_1,\dots,x_n)=\delta_{k,l}(x_1^2+\cdots+x_n^2),
	\end{equation}
	where $\delta_{k,l}$ is the usual Kronecker's map. Let 
	$A$ be the $n\times n$ matrix given by 
	\[
	A_{ij}=a_{ij}(x_1,\dots,x_n).
	\]
	Then 
	\begin{equation}
		\label{eq:AAT}
		AA^T=(x_1^2+\cdots+x_n^2)I,
	\end{equation}
	where $I$ denotes the $n\times n$ identity matrix, 
	as 
	\[
		(AA^T)_{kl}=\sum_{i=1}^na_{ki}(x_1,\dots,x_n)a_{li}(x_1,\dots,x_n)=\delta_{kl}(x_1^2+\cdots+x_n^2)
	\]
	by~\eqref{eq:delta}. Since each $a_{ki}(x_1,\dots,x_n)$ is a linear function, 
	there exist $\alpha_{ij1},\dots,a_{ijn}\in\C$ such that 
	\[
		a_{ij}(x_1,\dots,x_n)=\alpha_{ij1}x_1+\cdots+\alpha_{ijn}x_n.
	\]
	Write 
	\[
		A=A_1x_1+\cdots+A_nx_n,
	\]
	where each $A_k$ is the matrix $(A_k)_{ij}=\alpha_{ijk}$. 
	The formula~\eqref{eq:AAT} becomes
	\[
		\sum_{i=1}^n\sum_{j=1}^nA_iA_j^Tx_ix_j=(x_1^2+\cdots+x_n^2)I.
	\]
	Thus 
	\begin{equation}
		\label{eq:condiciones}
		A_iA_j^T+A_jA_i^T=0\quad i\ne j,\quad
		A_iA_i^T=I.
	\end{equation}
	We need $n$ complex square matrices of size $n\times n$
	satisfying~\eqref{eq:condiciones}. For $i\in\{1,\dots,n\}$ let  
	$B_i=A_n^TA_i$. Then~\eqref{eq:condiciones} turn into  
	\[
		B_iB_j^T+B_jB_i^T=0\quad i\ne j,\quad
		B_iB_i^T=I,\quad
		B_n=I.
	\]
	Set $j=n$ in the first family of equations to obtain $B_i=-B_i^T$ for all 
	$i\in\{1,\dots,n-1\}$. It follows that 
	\begin{equation}
	\label{eq:representation}
	\begin{aligned}
	    &B_i^2=-I && \text{for all $i\in\{1,\dots,n-1\}$},\\
	    &[B_i,B_j]=-I && \text{for all $i,j\in\{1,\dots,n-1\}$.}
	\end{aligned}
	\end{equation}
    
    \begin{claim}
        $n$ is even. 
    \end{claim}
    
	Computing the determinant of 
	$B_iB_j=-B_jB_i$ we obtain that 
	\[
	1=\det(B_iB_j)=(-1)^n\det(B_jB_i)=(-1)^n.
	\]
	Hence $n$ is even. 

	\begin{claim}
	    The group 
	    $G$ of the lemma admits a faithful
	    representation $\rho\colon G\to\GL_n(\C)$. 
	\end{claim}
	
	By \eqref{eq:representation}, there is a well-defined 
	injective group homomorphism $\rho$ such that 
	$x_i\mapsto B_i$ for all $i\in\{1,\dots,n-1\}$ and 
	$\epsilon\mapsto -I$. 
	
	\begin{claim}
	    $2^{\frac{n-2}{2}}$ divides $n$.
	\end{claim}
	
	Since $\epsilon\in[G,G]$ by Lemma~\ref{lem:hurwitz_group}, 
	every one-dimensional representation satisfies $\epsilon\mapsto 1$.
	This implies that $\rho$ cannot have degree-one sub representations. 
	In fact, 
	if $W=\langle w\rangle$ is $G$-invariant subspace of $\C^n$, 
	then $\psi=\rho|_W\colon G\to\GL(W)\simeq\C^\times$ 
	is a representation. In particular, 
	\[
	-w=-Iw=\psi_{\epsilon}(w)=\psi_{[x_i,x_j]}(w)
	=\psi_{x_i}\psi_{x_j}\psi_{{x_i}}^{-1}\psi_{{x_j}}^{-1}(w)=w, 
	\]
	a contradiction. 
	
	This means that the $\C[G]$-module $\C^n$ 
	decomposes as $\C^n\simeq aS\oplus bT$,
	where $a$ and $b$ are integers and 
	$S$ and $T$ are simple $\C[G]$-modules of dimension
	$2^{\frac{n-2}{2}}$. In particular, 
	\[
	n=\dim V=\dim(aS\oplus bT)=(a+b)2^{\frac{n-2}{2}}.
	\]
	
	To finish the proof of the theorem, write $n=2^ab$ 
	for $a\geq1$ and $b$ an odd integer. 
	Since $\frac{n-2}{2}$ divides $n$, 
	\[
	2^{\frac{n}{2}-1}=2^{\frac{n-2}{2}}\leq n=2^ab. 
	\]
	Thus $\frac{n}{2}-1\leq a$ and hence $2^a\leq n\leq 2(a+1)$. 
	It follows that $n\in\{4,8\}$.  
\end{proof}

We now present an application, see
\cite{MR1534187} for more information. 

\begin{theorem}
	Let $V$ be a real vector space (with an inner product) 
	of dimension $n\geq3$. If there exists a bilinear function 
	$V\times V\to\R$, $(v,w)\mapsto v\times
	w$, such that $v\times w$ is orthogonal both 
	to $v$ and $w$ and 
	\[
		\|v\times w\|^2=\|v\|^2\|w\|^2-\langle v,w\rangle^2,
	\]
	where $\|v\|^2=\langle v,v\rangle$, then $n\in\{3,7\}$. 
\end{theorem}

\begin{proof}
	Let $W=V\oplus\R$ with the inner product  
	\[
		\langle (v_1,r_1),(v_2,r_2)\rangle = \langle v_1,v_2\rangle+r_1r_2.
	\]
	Note that
	\begin{align*}
		\langle v_1\times &v_2+r_1v_2+r_2v_1,v_1\times v_2+r_1v_2+r_2v_1\rangle\\
		&=\|v_1\times v_2\|^2+r_1^2\|v_2\|^2+2r_1r_2\langle v_1,v_2\rangle+r_2^2\|v_1\|^2.
	\end{align*}
	Thus  
	\begin{align*}
		(\|v_1\|^2+r_1^2)&(\|v_2\|^2+r_2)\\
		&= \|v_1\|^2\|v_2\|^2+r_2^2\|v_1\|^2+r_1^2\|v_2\|^2+r_1^2r_2^2\\
		&=\|v_1\times v_2+r_1v_1+r_2v_2\|^2-2r_1r_2\langle v_1,v_2\rangle+\langle v_1,v_2\rangle^2+r_1^2r_2^2\\
		&=\|v_1\times v_2+r_1v_1+r_2v_2\|^2+(\langle v_1,v_2\rangle-r_1r_2)^2\\
		&=z_1^2+\cdots+z_{n+1}^2,
	\end{align*}
	where the $z_k$'s are bilinear functions in $(v_1,r_1)$ and $(v_2,r_2)$. 
	By Hurwitz's theorem, we conclude that 
	$n+1\in\{4,8\}$. Hence $n\in\{3,7\}$.
\end{proof}

In the theorem, if $\dim V=3$, we obtain the usual cross product. 
If $\dim V=7$, let 
\[
	W=\{(v,k,w):v,w\in V,k\in\R\}
\]
with the inner product 
\[
	\langle (v_1,k_1,w_1),(v_2,k_2,w_2)\rangle = \langle v_1,v_2\rangle+k_1k_2+\langle w_1,w_2\rangle.
\]
It is an exercise to show that 
\begin{multline*}
	(v_1,k_1,w_1)\times (v_2,k_2,w_2)\\
	=(k_1w_2-k_2w_1+v_1\times v_2-w_1\times w_2,
	\\-\langle v_1,w_2\rangle+\langle v_2,w_1\rangle, 
	k_2v_1-k_1v_2-v_1\times w_2-w_1\times v_2)
\end{multline*}
satisfies the properties of the theorem. 


%\section{Project: Gallagher's theorem}

\begin{lemma}
    \label{lem:gallagher1}
    Let $G$ be a finite group and 
    $N$ be a normal subgroup of $G$. Let $\psi\in\Irr(N)$ be such that
    $I_G(\psi)=G$. Then
    \[
    \Ind_N^G\psi=\sum_{\chi\in\Irr(G)}e_\chi\chi,
    \]
    where $e_\chi=\langle\Res_N^G,\chi\rangle$.
    In particular, 
    \[
    \sum_{\chi\in\Irr(G)}e_\chi^2=(G:N).
    \]
\end{lemma}

\begin{proof}
    Write $\Ind_N^G\psi$ as 
    \[
    \Ind_N^G\psi=\sum_{\chi\in\Irr(G)}e_\chi\chi,
    \]
    where by Frobenius' reciprocity $e_\chi=\langle\chi,\Ind_N^G\psi\rangle=\langle\Res_N^G\chi,\psi\rangle$.
    On the one hand, by Clifford's theorem, $\Res_N^G\chi=e_\chi\psi$ (because $I_G(\psi)=G$). In particular, 
    \[
    \chi(1)=(\Res_N^G\chi)(1)=e_\chi\psi(1).
    \]
    On the other hand, 
    \[
    (G:N)\psi(1)=(\Ind_N^G\psi)(1)=\sum_{\chi\in\Irr(G)}\langle\chi,\Ind_N^G\psi\rangle\chi(1)
    =\sum_{\chi\in\Irr(G)}e_\chi\chi(1)
    =\sum_{\chi\in\Irr(G)}e_\chi^2\psi(1).
    \]
    Since $\psi(1)\ne0$, it follows that  $(G:N)=\sum_{\chi\in\Irr(G)}e_\chi^2$. 
\end{proof}

\begin{lemma}
    \label{lem:Gallagher_reg}
    Let $G$ be a finite group and $N$ be a normal subgroup of $G$. Let... Then 
    \[
    \chi_L\chi=\Ind_N^G\Res_N^G\chi,
    \]    
    where $\chi_L$ is the character of the regular representation of $G/N$. 
\end{lemma}

\begin{proof}
    On the one hand, in the proof of Theorem~\ref{thm:regular} we have seen a formula to compute
    the values of the character of the regular representation of a group. In our case, 
    \[
    \chi_L(g)=\begin{cases}
        (G:N) & \text{if $g\in N$},\\
        0 & \text{otherwise}.
    \end{cases}
    \]
    On the other hand, 
    \begin{align*}
        (\Ind_N^G\Res_N^G)(g) &= \frac{1}{|N|}\sum_{x\in G}(\Res_N^G\chi)^0(x^{-1}gx).
    \end{align*}
    Note that $x^{-1}gx\in N$ if and only if $g\in xNx^{-1}=N$. Thus if $g\not\in N$, then
    $(\Ind_N^G\Res_N^G)(g)=0$ and this coincides with $\chi_L(g)\chi(g)$, since $\chi_L(g)=0$. Suppose 
    then that $g\in N$. Then 
    \[
    (\Ind_N^G\Res_N^G)(g)=\frac{1}{|N|}\sum_{x\in G}\chi(g)=(G:N)\chi(g)=\chi_L(g)\chi(g).\qedhere 
    \]
\end{proof}

\begin{theorem}[Gallagher]
\label{thm:Gallagher}
\index{Gallagher theorem}
    Let $G$ be a finite group and $N$ be a normal subgroup of $G$. 
    Let $\chi\in\Irr(G)$ be such that $\psi=\Res_N^G\chi\in\Irr(N)$. Then
    \[
    \Ind_N^G\psi=\sum_{\lambda\in\Irr(G/N)}\lambda(1)(\Inf_{G/N}^G\lambda)\chi,
    \]
    where the characters $(\Inf_{G/N}^G\lambda)\chi$ are pairwise different and irreducible. 
\end{theorem}

\begin{proof}
    Since $\psi=\Res_N^G\chi$, Lemma~\ref{lem:Gallagher_reg} implies that 
    \[
    \Ind_N^G\psi=\Ind_N^G\Res_N^G\chi=(\Inf_{G/N}^G\xi)\chi,
    \]
    where $\xi$ is the character of the regular representation of $G/N$. In the proof
    of Theorem~\ref{thm:regular} we have seen that
    \[
    \xi=\sum_{\lambda\in\Irr(G/N)}\lambda(1)\lambda. 
    \]
    By Lemma~\ref{lem:gallagher1}, 
    \begin{align*}
        (G:N)&=\sum_{\chi\in\Irr(G)}e_\chi^2
        =\langle\Ind_N^G\psi,\Ind_N^G\psi\rangle\\
        &=\sum_{\lambda,\mu\in\Irr(G/N)}\lambda(1)\mu(1)\langle\Inf_{G/N}^G\lambda,\Inf_{G/N}^G\mu\rangle
        \geq\sum_{\lambda\in\Irr(G/N)}\lambda(1)^2=(G:N).        
    \end{align*}
    It follows that 
    \[
    \langle\Inf_{G/N}^G\lambda,\Inf_{G/N}^G\mu\rangle=\begin{cases}
        1 & \text{if $\lambda=\mu$,}\\
        0 & \text{otherwise.}
    \end{cases}
    \]
    In particular, the characters...
\end{proof}

\begin{exercise}
    Let $G$ be a finite group and $N$ be a normal subgroup of $G$. Let $\chi\in\Irr(G)$ 
    be such that $\theta=\Res_N^G\chi\in\Irr(N)$. Prove that the map 
    $\psi\mapsto(\Inf_{G/N}^G\psi)\chi$ yields 
    a bijective 
    correspondence 
    \[
    \Irr(G/N)\leftrightarrow\{\eta\in\Irr(G):\langle\Res_N^G\eta,\theta\rangle>0\}.
    \]
\end{exercise} % not finished 
\section{Project: Induced representations}


% \index{Restriction}
% Let $G$ be a finite group, $H$ a subgroup of $G$, and 
% $V$ be a $\C[G]$-module. By restricting the action of $G$ on $V$ to $H$, we obtain that $V$ is a $\C[H]$-module. This module is denoted $\Res_H^GV$ and 
% is called the \emph{restriction} of $V$ to $H$. 

\begin{definition}
    \index{Bimodule}
    Let $R$ and $S$ be rings. An abelian group $M$ is called 
    a \emph{$(R,S)$-bimodule} if $M$ is a left $R$-module, 
    $M$ is a right $S$-module, and 
    \[
    r\cdot (m\cdot s)=(r\cdot m)\cdot s
    \]
    holds for all $r\in R$, $s\in S$ and $m\in M$. 
\end{definition}

Note that every left $R$-module is an $(R,\Z)$-bimodule. Similarly, every right $S$-module is an $(\Z,S)$-bimodule. Every 
ring $R$ is an $(R,R)$-bimodule. 

\begin{example}
If $M$ is an $(R,S)$-bimodule and $N$ is a left 
$R$-module, then the set 
$\Hom_R(M,N)$ of left $R$-module homomorphisms $M\to N$ is a left 
$S$-module with 
\[
(s\cdot \varphi)(m)=\varphi(m\cdot s),\quad s\in S,\,\varphi\in\Hom_R(M,N),\,m\in M.
\]
\end{example}

\index{Map!balanced}
\index{Balanced Map}
Let $M$ be an $(R,S)$-bimodule, $N$ be an $S$-module and $U$ be a $R$-module. We say that a map  
$f\colon M\times N\to U$ 
is \emph{balanced} if 
\begin{align*}
    &f(m_1+m_2,n)=f(m_1,n)+f(m_2,n),\\
    &f(m,n_1+n_2)=f(m,n_1)+f(m,n_2),\\
    &f(m\cdot s,n)=f(m,s\cdot n),\\
    &f(r\cdot m,n)=r\cdot f(m,n)
\end{align*}
for all $m,m_1,m_2\in M$, $n,n_1,n_2\in N$, $r\in R$ and $s\in S$. 

\begin{example}
If $M$ is an $R$-module, the map $f\colon R\times M\to M$, $(r,m)\mapsto r\cdot m$, is balanced.  
\end{example}

\index{Tensor product!of bimodules}
Let $M$ be an $(R,S)$-bimodule, $N$ be an $S$-module and $U$ be an $R$-module. 
A \emph{tensor product} $M\otimes_S N$ is an $R$-module with a balanced map
$\eta\colon M\times N\to M\otimes_S N$ satisfying the following universal property:
\begin{quote}
If $f\colon M\times N\to U$ is a balanced map, then 
there exists a unique $R$-module homomorphism $\alpha\colon M\otimes_S N\to U$ such that $f=\alpha\circ\eta$. 
\end{quote}

Notation: $m\otimes n=\eta(m,n)$ for $m\in M$ and $n\in N$.
The tensor product of bimodules exists and one can show it is unique up to isomorphism. More precisely,  $M\otimes_S N$
is the $R$-module generated by 
the set $\{m\otimes n:m\in M,\,n\in N\}$, where the elements $m\otimes n$ satisfy the following properties: 
\begin{align}
    &(m+m_1)\otimes n=m\otimes n+m_1\otimes n &&\text{$m,m_1\in M$, $n\in N$},\\
    &m\otimes(n+n_1)=m\otimes n+m\otimes n_1 &&\text{$m\in M$, $n,n_1\in N$},\\
    &(ms)\otimes n=m\otimes (sn) &&\text{$m\in M$, $n\in N$, $s\in S$},\\
    &(rm)\otimes n=r(m\otimes n) &&\text{$m\in M$, $n\in N$, $r\in R$}.
\end{align}

An arbitrary element of $M\otimes_S N$ is a finite sum of the form 
$\sum_{i=1}^k m_i\otimes n_i$,
where $m_1,\dots,m_k\in M$ and $n_1,\dots,n_k\in N$, and not necessarily an element of the form 
$m\otimes n$. 

\begin{example}
$M\simeq R\otimes_R M$ as $R$-modules. Since the map $R\times M\to M$, $(r,m)\mapsto r\cdot m$, is balanced, it induces an isomomorphism $R\otimes_R M\to M$, $r\otimes m\mapsto r\cdot m$ with inverse $M\to R\otimes_R M$, $m\mapsto 1\otimes m$. 
\end{example}

\begin{example}
If $M_1,\dots,M_k$ are $(R,S)$-bimodules and $N$ is an $S$-module, then
\[
(M_1\oplus\cdots\oplus M_k)\otimes_S N\simeq (M_1\otimes_S N)\oplus\cdots\oplus (M_k\otimes_S N).
\]
\end{example}

Some exercises:

\begin{exercise}
    Prove that  $M\otimes_RN\simeq N\otimes_{R^{\op}}M$.
\end{exercise}

\begin{exercise}
    Prove that  $\Z/n\otimes_{\Z}\Q=\{0\}$.
\end{exercise}

\begin{exercise}
    Let $M$ be an $(R,S)$-bimodule and $N$ be an $(S,T)$-bimodule. 
    Prove that  $M\otimes_SN$ is an $(R,T)$-bimodule 
    with $r(m\otimes n)t=(rm)\otimes (nt)$, 
    where $m\in M$, $n\in N$, $r\in R$, $t\in T$.
\end{exercise}

\begin{exercise}
    Prove that  $(M\otimes_R N)\otimes_RT\simeq M\otimes_R (N\otimes_RT)$.
\end{exercise}

\begin{exercise}
    State and prove the associativity of tensor product of bimodules. 
\end{exercise}

% Atiyah-Mac Donald
% https://math.stackexchange.com/questions/2586211/associativity-of-tensor-products

If $G$ is a finite group, $H$ is a subgroup of $G$
and $V$ is a $\C[H]$-module, then  
$\C[G]$ is a $(\C[G],\C[H])$-bimodule.

\begin{definition}
\index{Module!induced}
Let $G$ be a finite group and  
$H$ be a subgroup of $G$. 
If $V$ is a $\C[H]$-module of $G$, 
we define the \emph{induced} $\C[G]$-module of $V$ 
as 
\[
\Ind_H^GV=\C[G]\otimes_{\C[H]}V.
\]
\end{definition}

% \index{Transversal}
% Si $H$ es un subgrupo de $G$, un \textbf{transversal} (a izquierda) 
% de $H$ en $G$ es un subconjunto $T$ de $G$ que contiene exactamente un elemento de cada coclase (a izquierda) 
% de $H$ en $G$. 

\begin{example}
Let $G=\Sym_3$ and $H=\{\id,(12)\}$. Then 
$T=\{\id,(123),(23)\}$ is a transversal of $H$ in $G$. We can decompose $G$ as 
\[
G=\{\id,(12)\}\cup \{(123),(13)\}\cup\{(132),(23)\}=\bigcup_{t\in T}tH.
\]
Each $g\in G$ can be written uniquely as $g=th$ for some $t\in T$ and $h\in H$. We can define define a linear transformation 
$\varphi\colon \C[G]\to \C[H]\oplus \C[H]\oplus \C[H]=|T|\C[H]$, such that for each $g=th$ returns $h$ in the position corresponding to $t\in T$, namely 
\begin{align*}
\id&\mapsto (\id,0,0), && (12)\mapsto ((12),0,0), && (123)\mapsto (0,\id,0),\\
(23)&\mapsto (0,0,\id), && (13)\mapsto (0,(12),0), && (132)\mapsto (0,0,(12)).
\end{align*}
For example, 
\[
\varphi( 5(12)-3(123)+7\id )=(7\id+5(12),-3\id,0).
\]
Note that $\varphi$ is an isomorphism of right $\C[H]$-modules. 
\end{example}

The previous example is helpful 
to understand the following 
result:

\begin{proposition}
Let $G$ be a finite group and 
$H$ be a subgroup of $G$. If $V$ is a $\C[H]$-module, then  
\[
    \Ind_H^G(V)=\bigoplus_{t\in T}t\otimes V,
\]
where $T$ is a transversal of $H$ in $G$ and $t\otimes V=\{t\otimes v:v\in V\}$. In particular, 
\[
\dim\Ind_H^GV=(G:H)\dim V.
\]
\end{proposition}

\begin{proof}
Decompose $G$ into $H$-cosets with the transversal 
$T$, that is 
\[
G=\bigcup_{t\in T}tH.
\]
Each $g\in G$ can be written uniquely as $g=th$ for some $t\in T$ and $h\in H$. As we did in the previous example, this 
produces an isomorphism 
$\varphi\colon \C[G]\to |T|\C[H]$ of right $\C[H]$-modules, where $\varphi(g)$ is $h$ in the summand corresponding to $t\in T$
and is zero in the rest of the summands. Hence 
\[
\Ind_H^GV=\C[G]\otimes_{\C[H]}V\simeq (|T|\C[H])\otimes_{\C[H]}V\simeq |T|(\C[H]\otimes_{\C[H]}V)\simeq |T|V
\]
as $\C[H]$-modules. In particular, $\dim\Ind_H^GV=|T|\dim V$. 

 Write $g=th$ with $t\in T$ and $h\in H$. Then $g\otimes v=(th)\otimes v=t\otimes h\cdot v\in t\otimes V$. 
Hence $\C[G]\otimes_{\C[H]}V\subseteq \oplus_{t\in T}t\otimes V$. The other inclusion is trivial. By definition, 
the sum over $t\in T$ of the $t\otimes V$'s is direct.
\end{proof}

\begin{theorem}[Frobenius' reciprocity]
\index{Frobenius' reciprocity}
Let $G$ be a finite group and $H$ be a subgroup of $G$. 
If $U$ is a $\C[G]$-module and $V$ is a $\C[H]$-module, then
\[
\Hom_{\C[H]}(V,\Res_H^GU)\simeq \Hom_{\C[G]}(\Ind_H^GV,U)
\]
as vector spaces
\end{theorem}

\begin{proof}
For $\varphi\in\Hom_{\C[H]}(V,\Res_H^GU)$, let 
\[
f_{\varphi}\colon \C[G]\times V\to U,
\quad
(g,v)\mapsto g\cdot\varphi(v).
\]
We claim that $f_{\varphi}$ is balanced. A direct calculation shows that 
\begin{align*}
    &f_{\varphi}(g+g_1,v)=f_{\varphi}(g,v)+f_{\varphi}(g_1,v),&&
    f_{\varphi}(g,v+w)=f_{\varphi}(g,v)+f_{\varphi}(g,w).
\end{align*}
Since $\varphi$ is a $\C[H]$-module homomorphism,
\begin{align*}
    &f_{\varphi}(gh,v)=(gh)\cdot\varphi(v)
    =g\cdot (h\cdot \varphi(v))
    =g\cdot (h\cdot\varphi(v))
    =g\cdot \varphi(h\cdot v)=f_{\varphi}(g,h\cdot v)
\end{align*}
for all $g\in G$, $h\in H$ and $v\in V$. Moreover, 
\begin{align*}
    &f_{\varphi}(gg_1,v)=(gg_1)\cdot\varphi(v)=g\cdot(g_1\cdot\varphi(v))=g\cdot f_{\varphi}(g_1,v)
\end{align*}
for all $g,g_1\in G$ y $v\in V$. 

For every $\varphi\in\Hom_{\C[H]}(V,\Res_H^GU)$, there exists 
$\Gamma(\varphi)\in\Hom_{\C[G]}(\Ind_H^GV,U)$ such that 
$\Gamma(\varphi)(g\otimes v)=g\cdot\varphi(v)$. 
We have defined a map  
\[
\Gamma\colon \Hom_{\C[H]}(V,\Res_H^GU)\to\Hom_{\C[G]}(\Ind_H^GV,U),
\quad
\varphi\mapsto\Gamma(\varphi).
\]

The map $\Gamma$ is linear and injective, something that is quite easy to verify. 

The map is surjective: if $\theta\in\Hom_{\C[H]}(\Ind_H^GV,U)$, then
the map $\varphi(v)=\theta(1\otimes v)$ is such that $\varphi\in\Hom_{\C[H]}(V,\Res_H^GU)$ and satisfies 
\[
\Gamma(\varphi)(g\otimes v)=g\cdot\varphi(v)=g\cdot\theta(1\otimes v)=\theta(g\otimes v).\qedhere
\]
\end{proof}

% Supongamos ahora que $K=\C$. 

Let $H$ be a subgroup of $G$. If $U$ is a $\C[G]$-module with character $\chi$, the character $\Res_H^GU$ will be denoted by $\chi|_H$. Then $\chi|_H(1)=\chi(1)$. Note that 
\begin{align*}
\langle \phi,\chi|_H\rangle
&=\dim\Hom_{\C[H]}(V,\Res_H^GU)
=\dim\Hom_{\C[G]}(\Ind_H^GV,U)
=\langle\phi^G,\chi\rangle,
\end{align*}


\begin{definition}
If $\Irr(G)=\{\chi_1,\dots,\chi_k\}$ and $\Irr(H)=\{\phi_1,\dots,\phi_l\}$, we define the \emph{induction-restriction} matrix as $(c_{ij})\in\C^{l\times k}$, where 
\[
c_{ij}=\langle \phi_i^G,\chi_j\rangle=\langle\phi_i,\chi_j|_H\rangle.
\]
\end{definition}

The $i$-th row of the induction-restriction matrix gives the multiplicity of the character $\chi_j$ in the decomposition of $\phi_i^G$. The $j$-th column is the multiplicity of the 
character $\phi_i$ in the decomposition of $\chi_j|H$.

\begin{example}
Let $G=\Sym_3$. 
The character table of $G$ is 
	\begin{center}
		\begin{tabular}{|c|rrr|}
			\hline
			& $1$ & $3$ & $2$\tabularnewline
			& $1$ & $(12)$ & $(123)$ \tabularnewline
			\hline 
			$\chi_{1}$ & $1$ & $1$ & $1$\tabularnewline
			$\chi_{2}$ & $1$ & $-1$ & $1$ \tabularnewline
			$\chi_{3}$ & $2$ & $0$ & $-1$ \tabularnewline
			\hline
		\end{tabular}
	\end{center}
The character table of the subgroup $H=\{\id,(12)\}$ is  
\begin{center}
\begin{tabular}{|c|rr|}
\hline 
& $1$ & $1$ \tabularnewline
& $\id$ & $(12)$ \tabularnewline
\hline 
$\phi_{1}$ & $1$ & $1$ \tabularnewline
$\phi_{2}$ & $1$ & $-1$\tabularnewline
\hline
\end{tabular}
\end{center}
By inspection, we see that
$\chi_1|_H=\phi_1$, $\chi_2|_H=\phi_2$ and 
$\chi_3|_H=\phi_1+\phi_2$. 
The induction-restriction matrix is 
\[
\begin{pmatrix}
1 & 0 & 1\\
0 & 1 & 1
\end{pmatrix}.
\]
Moreover, $\phi_1^G=\chi_1+\chi_3$ and  $\phi_2^G=\chi_2+\chi_3$. 
\end{example}

Let us see (again) how to compute 
induced characters. 

\begin{proposition}
Let $H$ be a subgroup of $G$ and $V$ a $\C[H]$-module with character $\chi$. If  
$T$ is a transversal of $H$ in $G$, then  
\[
\chi^G(g)=\sum_{\substack{t\in T\\t^{-1}gt\in H}}\chi(t^{-1}gt)
\]
for all $g\in G$. 
\end{proposition}

\begin{proof}
    We know that $\Ind_H^GV=\oplus_{t\in T}t\otimes V$. 
    Assume that $T=\{t_1,\dots,t_m\}$ 
    and let $\{v_1,\dots,v_n\}$ be a basis of $V$. 
    Then $\{t_i\otimes v_k:1\leq i\leq m,\,1\leq k\leq n\}$ is a basis of $\Ind_H^GV$. The action of 
    $g$ on $\Ind_H^GV$ is given by 
    \[
    \rho^G(g)=\begin{cases}
    \rho(t_j^{-1}gt_i) & \text{if $t_j^{-1}gt_i\in H$},\\
    0 & \text{otherwise}.
    \end{cases}
    \]
    In fact, if $gt_i=t_jh$ for $h\in H$ and certain $i,j$, then  
    \[
    g\cdot (t_i\otimes v_k)=gt_i\otimes v_k=t_jh\otimes v_k=t_j\otimes h\cdot v_k. 
    \]
    Moreover, $gt_i=t_jh$ if and only if $t_j^{-1}gt_i=h\in H$. We conclude that 
    $g$ acts like $t^{-1}gt$ on $V$ if $t^{-1}gt\in H$ and as zero otherwise.  
\end{proof}

\begin{corollary}
\label{cor:induccion}
    Let $H$ be a subgroup of $G$ 
    and $V$ a $\C[H]$-module with character $\chi$.
    If $g\in G$, then 
    \[
    \chi^G(g)=\frac{1}{|H|}\sum_{\substack{x\in G\\x^{-1}gx\in H}}\chi(x^{-1}gx).
    \]
\end{corollary}

\begin{proof}
    Let $T$ be a transversal of $H$ in $G$. If $x\in G$, write $x=th$ for some $t\in T$ and $h\in H$. 
    Como $x^{-1}gx=h^{-1}(t^{-1}gt)h$, it follows that $x^{-1}gx\in H\Longleftrightarrow t^{-1}gt\in H$. Moreover,   
    $\chi(x^{-1}gx)=\chi(t^{-1}gt)$, as $\chi$ is a class function. This implies that there are $|H|$ elements $x\in G$ 
    such that $x^{-1}gx\in H$. For those $x$, $\chi(x^{-1}gx)=\chi(t^{-1}gt)$. 
\end{proof} 
\section{Project: A theorem of Herstein}

We present 
a theorem of~\cite{MR93542}. The proof uses Frobenius' theorem (Theorem~\ref{thm:Frobenius}). Recall that 
a proper subgroup $M$ of $G$ is said to be \emph{maximal} 
if $M\subseteq H$ for some subgroup $H$ of $G$ 
implies that $M=H$ or $H=G$. 

\begin{theorem}[Herstein]
\label{thm:Herstein}
\index{Herstein theorem}
    Let $G$ be a finite group and $A$ be an abelian
    subgroup of $G$. If $A$ is maximal, then
    $G$ is solvable. 
\end{theorem}

We start with a lemma.

\begin{lemma}
    Let $G$ be a finite group and $p$ be a prime divisor of $|G|$. Let $A$ be an abelian
    subgroup of $\Aut(G)$ such that if $\phi\in A$ and 
    $\phi(x)=x$, then $x=1$. Then there 
    exists a unique $P\in\Syl_p(G)$
    such that $\phi(P)=P$ for all $\phi\in A$. 
\end{lemma}

\begin{proof}
    Let $\phi\in A$. We claim that 
    $G=\{x^{-1}\phi(x):x\in G\}$. Let us consider the map 
    $x\mapsto x^{-1}\phi(x)$. This map is injective, as 
    \[
    x^{-1}\phi(x)=y^{-1}\phi(y)\implies 
    yx^{-1}=\phi(yx^{-1})\implies yx^{-1}=1\implies x=y.
    \]
    Thus $|G|\leq |\{x^{-1}\phi(x):x\in G\}|\leq |G|$ and hence
    $G=\{x^{-1}\phi(x):x\in G\}$. 

    We claim that there exists $P\in\Syl_p(G)$ such that $\phi(P)=P$. In fact, 
    let $Q\in\Syl_p(G)$. Then $\phi(Q)\in\Syl_p(G)$. Thus 
    there exists $y\in G$ such that $yQy^{-1}=\phi(Q)$. Let $x\in G$ 
    be such that $y^{-1}=x^{-1}\phi(x)$. Then
    \begin{align*}
    \phi(xQx^{-1})&=\phi(x)\phi(Q)\phi(x)^{-1}\\
    &=\phi(x)yQy^{-1}\phi(x)^{-1}\\
    &=\phi(x)\phi(x)^{-1}xQx^{-1}\phi(x)\phi(x)^{-1}\\
    &=xQx^{-1}.      
    \end{align*}
    Thus $P=xQx^{-1}\in\Syl_p(G)$ is such that $\phi(P)=P$. 

    We claim that $\phi(N_G(P))=N_G(P)$. If $y\in \phi(N_G(P))$, then 
    $y=\phi(x)$ for some $x\in N_G(P)$. Then 
    \[
    yPy^{-1}=\sigma(x)P\sigma(x)^{-1}=\sigma(x)\sigma(P)\sigma(x)^{-1}=\sigma(xPx^{-1})=\sigma(P)=P.
    \]
    Thus $y\in N_G(P)$. Conversely, if $x\in N_G(P)$, then $x=\sigma(y)$ for some $y\in G$. Since 
    \[
    P=xPx^{-1}=\sigma(y)P\sigma(y)^{-1}=\sigma(yPy_{-1}),
    \]
    $yPy^{-1}=\sigma^{-1}(P)=P$ and hence $y\in N_G(P)$. 
    
    We now claim that $P$ is the only Sylow $p$-subgroup of $G$ such that $\phi(P)=P$. Suppose that 
    $P_1\in\Syl_p(G)$ is such that $\phi(P_1)=P_1$. 
    Since $P$ and $P_1$ are conjugate, 
    $P_1=xPx^{-1}$ for some $x\in G$. Since
    \[
    xPx^{-1}=P_1=\phi(P_1)=\phi(xPx^{-1})=\phi(x)\phi(P)\phi(x)^{-1}
    =\phi(x)P\phi(x)^{-1},
    \]
    it follows that 
    $x^{-1}\phi(x)\in N_G(P)$. Note that 
    the restriction $\phi|_{N_G(P)}$ of $\phi$ to 
    the subgroup $N_G(P)$ is an isomorphism that 
    only fixes the identity element. 
    \textcolor{red}{There exists $y\in N_G(P)$} such that $x^{-1}\phi(x)=y^{-1}\phi(y)$. Then $x=y\in N_G(P)$ and therefore
    $P_1=P$. 

    Let $\psi\in A$. Since $A$ is abelian, 
    \[
    \psi(P)=\psi(\phi(P))=(\psi\phi)(P)=(\phi\psi)(P)=\phi(\psi(P)).
    \]
    Thus $\psi(P)\in\Syl_p(G)$ is fixed by $\phi$. By the previous
    claim, $P=\psi(P)$. 
\end{proof}

Now we are ready to prove the theorem. 

\begin{proof}[Proof of Theorem~\ref{thm:Herstein}]
    We proceed by induction on $|G|$. There are two cases
    to consider. 

    Assume first that $N_G(A)\ne A$. Then $A\subseteq N_G(A)\subseteq G$. Since $A$ is a maximal subgroup, $N_G(A)=G$. Thus $A$ 
    is normal in $G$. By the correspondence theorem, $G/A$ has no non-trivial proper subgroups. Thus $G/A$ is cyclic (of prime
    order). In particular, $G$ is solvable as both 
    $G/A$ and $A$ are solvable. 

    Assume now that $N_G(A)=A$. Let $x\not\in A$ and $B=xAx^{-1}\cap A$. If $B\ne\{1\}$, let $b\in B$ be such that $b\ne 1$. 
    Since $A$ is abelian, $A\subseteq C_G(b)$. Moreover, since
    $b\in xAx^{-1}$, $A\ne xAx^{-1}\subseteq C_G(b)$. 
    Hence $C_G(b)\ne A$. By the maximality of $A$, $C_G(b)=G$. In particular, $b\in Z(G)$. The subgroup $\langle b\rangle$ 
    is central (so normal in $G$). Let $\pi\colon G\to G/\langle b\rangle$ be the canonical map. By the correspondence 
    theorem, $\pi(A)$ is a maximal subgroup of $G/\langle b\rangle$ and 
    $\pi(A)$ is abelian. Since $|G/\langle b\rangle|<|G|$, the inductive hypothesis implies that $G/\langle b\rangle$ is solvable. Thus $G$ is solvable and both 
    $G/\langle b\rangle$ and $\langle b\rangle$ are solvable. 

    If $B=\{1\}$, then $A$ is a Frobenius group. Let $T$ be 
    the complement of $A$ in $G$. Then $T$ is a normal subgroup 
    of $G$. In particular, $aTa^{-1}=T$ for all $a\in A$. 
    Each $a\in A$ induces an automorphism 
    $\gamma_a\colon T\to T$, $t\mapsto ata^{-1}$. We claim that 
    if $1\ne a\in A$ and $t\in T$ are such that 
    $\gamma_a(t)=t$, then $t=1$. In fact, since $t\not\in A$, 
    \[
    ata^{-1}=\gamma_a(t)=t\implies 
    a=t^{-1}at\in A\cap t^{-1}At=\{1\}.
    \]
    By the lemma, there exists $P\in\Syl_p(T)$ 
    such that $\gamma_a(P)=P$ for all $a\in A$. Thus 
    $A\subseteq N_G(P)$. Moreover, $P\subseteq N_G(P)$. But $P\not\subseteq A$, as $P\subseteq T$ and $T\cap A=\{1\}$. Thus 
    $N_G(P)=G$ since $A$ is a maximal subgroup of $G$. Hence 
    $P$ is normal in $G$ and $AP$ is a subgroup of $G$. 
    Since $A\subsetneq AP\subseteq G$, the maximality of $A$ 
    implies that $AP=G$. If $t\in T$, write
    $t=ax$ for some $a\in A$ and $x\in P\subseteq T$. Thus 
    $a=tx^{-1}\in A\cap T=\{1\}$, that is $t=x\in P$. We have proved
    that $T=P$. Since both $G/T\simeq A$ and $A$ are solvable, it follows that $G$ is solvable. 
\end{proof}

The following result goes back to
Kaplansky and 
Herstein.

\begin{corollary}
    \index{Kaplansky--Herstein theorem}
    Let $G=ABA$ be a finite group where $A$ is an abelian subgroup and $B$ is a cyclic subgroup of prime order. Then $G$ is solvable. 
\end{corollary}

\begin{proof}
    By Herstein's theorem, it is enough to show
    that $A$ is a maximal subgroup of $G$. Assume that 
    $B=\langle b\rangle$. 
    Suppose that $A$ is not maximal. Then there
    exists a proper subgroup $M$ of $G$ such that 
    $A\subsetneq M$. Let $x\in M\setminus A$. Then 
    $x=a_1b^ka_2$ for some $a_1,a_2\in A$ and $k\in\Z$ such that $b^k\ne 1$. Since $A\subseteq M$, 
    \[
    b^k=a_1^{-1}xa_2^{-1}\in M.
    \]
    Since $B$ is cyclic of prime order, $b\in M$. Thus 
    $A\cup B\subseteq M$ and hence $ABA\subseteq M$, a contradiction. 
\end{proof} % this goes to non-commutative algebra
\section{Project: Wall's theorem}

We now present a character-theoretic proof of a theorem of Wall \cite{MR125156}, 
which bounds the number of maximal subgroups of a finite solvable group. 

\begin{exercise}
    \label{xca:number_maximals}
    Let $G$ be a finite group and $M$ be a maximal subgroup of $G$.
    Prove that $M$ has exactly $(G:M)$ conjugates. 
\end{exercise}

\begin{lemma}
\label{lem:multiplicity-free}
    Let $G$ be a finite solvable group and $M$ be a maximal subgroup of $G$. If $M$ is core-free, then 
    every irreducible constituent of $\Ind_M^G\Tchar_M$ has multiplicity one. 
\end{lemma}

\begin{proof}
    Let $N$ be a minimal normal subgroup of $G$. Then 
    $G=NM$ and $N\cap M=\{1\}$ (see 
    By Lemma~\ref{lem:Ore_step2}). In particular, 
    $|N|=(G:M)$. 

    We claim that 
    $\Res_N^G\Ind_M^G\Tchar_M$ is the regular character of $N$. For 
    $n\in N$, 
    \[
    (\Ind_M^G\Tchar_M)(n)=\frac{1}{|M|}\sum_{x\in G}\Tchar_M^0(x^{-1}nx)
    =\begin{cases}
        (G:M) & \text{if $n=1$,}\\
        0 & \text{otherwise.}
    \end{cases}
    \]
    By Theorem~\ref{thm:regular}, 
    $\Res_N^G\Ind_M^G\Tchar_M$ is the regular character of $N$ and hence 
    \[
    \Res_N^G\Ind_M^G\Tchar_M=\sum_{\lambda\in\Irr(N)}\lambda(1)\lambda.
    \]
    Since $N$ is abelian, $\lambda(1)=1$ for all $\lambda\in\Irr(N)$. Now assume
    that some irreducible constituent $\psi\in\Irr(G)$ of $\Ind_M^G\Tchar_M$ has
    multiplicity $m\geq2$, say $\Ind_M^G\Tchar_M=m\psi+\xi$, where $\langle\psi,\xi\rangle=0$. 
    Then 
    \[
    \sum_{\lambda\in\Irr(N)}\lambda=\Res_N^G\Ind_M^G\Tchar_M=m\Res_N^G\psi+\Res_N^G\xi,
    \]
    a contradiction. 
\end{proof}

\begin{lemma}
\label{lem:kernel}
    Let $G$ be a finite solvable group and $M$ be a maximal subgroup of $G$. If $M$ is core-free, then 
    \[
    \ker\Ind_M^G\Tchar_M=\bigcap\{\ker\chi:\chi\in\Irr(G)\text{ such that }\langle\Ind_M^G\Tchar_M,\chi\rangle\ne0\}.
    \]
    Thus $\ker\Ind_M^G\Tchar_M$ is the intersection of the kernels of the irreducible constituents 
    of $\Ind_M^G\Tchar_M$.
\end{lemma}

\begin{proof}
    By Lemma~\ref{lem:multiplicity-free}, every irreducible constituent of 
    $\Ind_M^G\Tchar_M$ appears with multiplicity one, that is 
    \[
    \Ind_M^G\Tchar_M=\sum_{j=1}^k\chi_j
    \]
    for some subset $\{\chi_1,\dots,\chi_k\}\subseteq\Irr(G)$. If $g\in\ker\chi_1\cap\cdots\cap\ker\chi_k$, then
    $\chi_j(g)=\chi_j(1)$ for all $j\in\{1,\dots,k\}$. Thus 
    \[
    \sum_{j=1}^k\chi_j(g)=(\Ind_M^G\Tchar_M)(g)=(\Ind_M^G\Tchar_M)(1)=\sum_{j=1}^k\chi_j(1). 
    \]
    Conversely, let $g\in\ker\Ind_M^G\Tchar_M$. 
    Then 
    \[
    \sum_{j=1}^k\chi_j(g)=(\Ind_M^G\Tchar_M)(g)=(\Ind_M^G\Tchar_M)(1)=\sum_{j=1}^k\chi_j(1).
    \]
    Assume that there exists 
    $i\in\{1,\dots,k\}$ such that 
    $g\not\in\ker\chi_i$. Then 
    $|\chi_i(g)|<\chi_i(1)$ and hence 
    \[
    \sum_{j=1}^k\chi_j(1)=\left|\sum_{j=1}^k\chi(g)\right|\leq
    \sum_{j=1}^k|\chi_j(g)|<\sum_{j=1}^k\chi_j(1),
    \]
    a contradiction. 
\end{proof}

\begin{lemma}
    Let $G$ be a finite solvable group and $M$ be a  
    maximal subgroup of $G$. If $M$ is not normal in $G$, 
    then $\Tchar_G$ is the only degree-one constituent of $\Ind_M^G\Tchar_M$. 
\end{lemma}

\begin{proof}
    Let $\psi\in\Irr(G)$ be a degree-one constituent of $\Ind_M^G\Tchar_M$. 
    By Frobenius' reciprocity, $0\ne\langle\Ind_M^G\Tchar_M,\psi\rangle=\langle\Tchar_M,\Res_M^G\psi\rangle$.
    Since $\Res_M^G\psi$ is a degree-one character, it follows the irreducibility that 
    $\Tchar_M=\Res_M^G\psi$. In particular, 
    \[
    M=\ker\Tchar_M=\ker(\Res_M^G\psi)=M\cap\ker\psi\subseteq\ker\psi. 
    \]
    
    Assume now that $\psi\ne\Tchar_G$. Then 
    \[
    M\subseteq \ker\psi=\Core_GM\subsetneq M,
    \]
    a contradiction. Hence $\psi=\Tchar_G$. 
\end{proof}

\begin{theorem}[Wall]
\index{Wall theorem}
\label{thm:Wall}
Let $G$ be a finite solvable group. Then the number of maximal subgroups of $G$ is at most $|G|-1$. 
\end{theorem}

\begin{proof}
    For a maximal subgroup $M$ of $G$, let 
    $C(M)$ be the set of non-trivial constituents
    of $\Ind_M^G\Tchar_M$, that is
    \[
    C(M)=\{\chi\in\Irr(G):\chi\ne\Tchar_G\text{ and }\langle\eta,\Ind_M^G\Tchar_M\rangle\ne0\}.
    \]

    Let $M_1$ and $M_2$ be maximal subgroups of $G$. 
    We claim that $C(M_1)\cap C(M_2)=\emptyset$ if $M_1$ and $M_2$ are
    not conjugate. 
    
    Assume that $M_1$ and $M_2$ are not conjugate. By  
    Ore's theorem~\ref{thm:Ore}, $G=M_1M_2$. In particular, 
    there is only one double $(M_1,M_2)$-coset, with representative $1$. Thus
    \[
    \Res_{M_1}^G\Ind_{M_2}^G\Tchar_{M_2}=\Ind_{M_1\cap M_2}^{M_1}\Res_{M_1\cap M_2}^{M_2}\Tchar_{M_2}
    =\Ind_{M_1\cap M_2}^{M_1}\Tchar_{M_1\cap M_2}.
    \]
    Using Frobenius' reciprocity (Theorem~\ref{thm:reciprocity}) and 
    Mackey's theorem~\ref{thm:Mackey}, 
    \begin{align*}
    \langle\Ind_{M_1}^G\Tchar_{M_1},\Ind_{M_2}^G\Tchar_{M_2}\rangle
    &=\langle \Tchar_{M_1},\Res_{M_1}^G\Ind_{M_2}^G\Tchar_{M_2}\rangle\\
    &=\langle \Tchar_{M_1},\Ind_{M_1\cap M_2}^{M_1}\Tchar_{M_1\cap M_2}\rangle\\
    &=\langle \Res_{M_1\cap M_2}^{M_1}\Tchar_{M_1},\Tchar_{M_1\cap M_2}\rangle\\
    &=\langle \Tchar_{M_1\cap M_2},\Tchar_{M_1\cap M_2}\rangle\\
    &=1.
    \end{align*}
    For $i\in\{1,2\}$, by Frobenius' reciprocity, 
    \[
    \langle\Ind_{M_i}^G\Tchar_{M_i},\Tchar_G\rangle
    =\langle\Tchar_{M_i},\Res_{M_i}^G\Tchar_G\rangle=\langle \Tchar_{M_i},\Tchar_{M_i}\rangle=1.
    \]
    Thus $\Tchar_{G}$ is an irreducible constituent of both 
    $\Ind_{M_1}^G\Tchar_{M_1}$ and $\Ind_{M_2}^G\Tchar_{M_2}$. Since 
    \[
    \langle\Ind_{M_1}^G\Tchar_{M_1},\Ind_{M_2}^G\Tchar_{M_2}\rangle=1,
    \]
    the set of non-trivial 
    irreducible
    constituents of $\Ind_{M_1}^G\Tchar_{M_1}$ and $\Ind_{M_2}^G\Tchar_{M_2}$ are disjoint, that~is  
    \[
    \Ind_{M_1}^G\Tchar_{M_1}=\Tchar_G+\eta_1,
    \quad 
    \Ind_{M_2}^G\Tchar_{M_2}=\Tchar_G+\eta_2,
    \quad 
    \langle\eta_1,\eta_2\rangle=0.
    \]
    Hence $C(M_1)\cap C(M_2)=\emptyset$. 

    
   Let $X$ be the set of maximal subgroups of $G$ that are normal in $G$, and 
   let $Y$ be the set of representatives of non-normal 
   maximal subgroups of $G$. By Exercise~\ref{xca:number_maximals}, 
   the number of 
   maximal subgroups of $G$ is 
   \[
   m=|X|+\sum_{M\in Y}(G:M).
   \]

   For every maximal subgroup $M$ of $G$ such that $M$ is normal in $G$, 
   we know that $\Ind_M^G\Tchar_M$ decomposes 
   as $\Ind_M^G\Tchar_M=\Tchar_G+\eta_1+\cdots+\eta_k$ for some  
   $\eta_1,\dots,\eta_k\in\Irr(G)\setminus\{\Tchar_G\}$ such that 
   $\eta_j(1)=1$ for all $j\in\{1,\dots,k\}$. Then
   \[
   (G:M)-1=\eta_1(1)+\cdots+\eta_k(1)=\sum_{i=1}^k\eta_i(1)^2
   \]
   since $\eta_j(1)=1$ for all $j$. Since $p$ is the smallest prime divisor of $|G|$, 
   it follows that 
   \[
   \sum_{\eta\in C(M)}\eta(1)^2
   =(G:M)-1\geq p-1.
   \]

   For every maximal subgroup $M$ of $G$ such that $M$ is not normal in $G$, 
   $\Ind_M^G\Tchar_M$ decomposes 
   as $\Ind_M^G\Tchar_M=\Tchar_G+\xi_1+\cdots+\xi_l$ for some distinct characters 
   $\xi_1,\dots,\xi_k\in\Irr(G)$ 
   such that $\xi_j(1)\geq p$ for all $j\in\{1,\dots,k\}$. 
   Then 
   \[
   \sum_{\xi\in C(M)}\xi(1)^2
   \geq p\sum_{\xi\in C(M)}\xi(1)
   =p((G:M)-1)
   \geq (p-1)(G:M).
   \]

   Now 
   \begin{align*}
    |G|-1 &=\sum_{\Tchar_G\ne\chi\in\Irr(G)}\chi(1)^2
    \geq\sum_{M\in X}\sum_{\eta\in C(X)}\eta(1)^2
    +\sum_{M\in Y}\sum_{\xi\in C(X)}\xi(1)^2\\
    &\geq (p-1)|X|+(p-1)\sum_{M\in Y}(G:M)\\
    &=(p-1)\left(|X|+\sum_{M\in Y}(G:M)\right)=(p-1)m.\qedhere 
   \end{align*}
\end{proof}

\begin{exercise}
    Prove that a finite solvable group has exactly $|G|-1$ maximal subgroups
    if and only if it is an elementary abelian $2$-group. 
\end{exercise}

\begin{example}
    Let $G=\Sym_3$. 
    Recall from Table~\ref{tab:S3} that $\Irr(G)=\{\Tchar_G,\sgn,\chi\}$, where $\sgn$ is the sign representation
    and $\chi\colon G\to\C^{\times}$ is given by
    \[
    \chi(g)=\begin{cases}
        2 & \text{if $g=\id$},\\
        0 & \text{if $g\in\{(12),(13),(23)\}$},\\
        -1 & \text{if $g\in\{(123),(132)\}.$}
    \end{cases}
    \]

    The group $G$ has two conjugacy classes of maximal subgroups, namely 
    \[
    \{\langle (123)\rangle\}\text{ and }\{\langle(12)\rangle,\langle(23)\rangle,\langle(13)\rangle\}.
    \]

    As the group $G$ is rather small, this can be easily verified 
    by a direct calculation. In any case, here is the Magma code:
    \begin{lstlisting}
> S3 := Sym(3);
> max := MaximalSubgroups(S3);
> max;
Conjugacy classes of subgroups
------------------------------

[1]     Order 2            Length 3
        Permutation group M acting on a set of cardinality 3
        Order = 2
            (2, 3)
[2]     Order 3            Length 1
        Permutation group N acting on a set of cardinality 3
        Order = 3
            (1, 2, 3)
    \end{lstlisting}

    Let $M=\langle (23)\rangle$ and $N=\langle (123)\rangle$. Thus $X=\{N\}$ and 
    $Y=\{M\}$. 
    
    Let us compute $C(N)$. For that purpose, let $t_1=1$ 
    and $t_2=(12)$ be a transversal of $N$ in $G$. Then 
    \[
    (\Ind_{N}^G\Tchar_N)(g)=\Tchar_N^0(g)+\Tchar_N^0((12)g(12))=\begin{cases}
        2 & \text{if $g=\id$},\\
        0 & \text{if $g\in\{(12),(23),(13)\}$},\\
        2 & \text{otherwise.}
    \end{cases}
    \]
    Thus $\Ind_{N}^G\Tchar_N=\Tchar_G+\sgn$ and $C(N)=\{\sgn\}$. Here is the Magma code: 
\begin{lstlisting}
> N := max[2]`subgroup;
> g := Character(TrivialRepresentation(N));
> ind_N := Induction(g, S3);
> ind_N;
( 2, 0, 2 )
\end{lstlisting}    

To decompose our induced character, with Magma we proceed as follows:
\begin{lstlisting}
> T := CharacterTable(S3);
> Decomposition(T, ind_N);
[
    1,
    1,
    0
]
( 0, 0, 0 )
> InnerProduct(T[3], ind_N);
0
> InnerProduct(T[2], ind_N);
1
> InnerProduct(T[1], ind_N);
1
\end{lstlisting}


    Let us now compute $C(M)$. A direct calculation shows that 
    \[
    (\Ind_{M}^G\Tchar_{M})(g)=\begin{cases}
        3 & \text{if $g=\id$},\\
        1 & \text{if $g\in\{(12),(23),(13)\}$},\\
        0 & \text{otherwise.}
    \end{cases}
    \]
    Thus $\Ind_{M}\Tchar_{M}=\Tchar_G+\chi$ and $C(M)=\{\chi\}$. 
    We leave it as an exercise to verify these calculations, either by 
    hand, with Magma, or perhaps both.
\end{example}

\begin{example}
    Let $G=\Alt_4$. There are two conjugacy classes of maximal subgroups of $G$ with representatives
    are $M=\langle (234)\rangle$ and $N=\{\id,(12)(34),(13)(24),(14)(23)\}$. Then 
    $X=\{N\}$ and $Y=\{M\}$. 

    In Exercise~\ref{xca:A4}, we asked for the construction of the character table of \( \Alt_4 \). The completed table is shown in Table~\ref{tab:A4}.

    \index{Character table!of $\Alt_4$}
    \begin{table}[h]
        \centering\makegapedcells
        \caption{The Character table of $\Alt_4$.}
        \label{tab:A4}
        \begin{tabular}{|c|cccc|}
             \hline
             & $\id$ & $(12)(34)$ & $(123)$ & $(132)$\\
             \hline
             $\chi_1$ & $1$ & $1$ & $1$ & $1$\\
             $\chi_2$ & $1$ & $1$ & $\frac{-1+\sqrt{-3}}{2}$ & $\frac{-1-\sqrt{-3}}{2}$\\
             $\chi_3$ & $1$ & $1$ & $\frac{-1-\sqrt{-3}}{2}$ & $\frac{-1+\sqrt{-3}}{2}$\\
             $\chi_4$ & $3$ & $-1$ & $0$ & $0$\\
             \hline
        \end{tabular}
    \end{table}

    
    A direct calculation shows that
    $\Ind_N^G\Tchar_N=\Tchar_G+\chi_2+\chi_3$ and 
    $\Ind_M^G\Tchar_M=\Tchar_G+\chi_4$. 
\end{example}
% > A4 := Alt(4);
% > max := MaximalSubgroups(A4);
% > #max;
% 2
% > max;
% Conjugacy classes of subgroups
% ------------------------------

% [1]     Order 3            Length 4
%         Permutation group acting on a set of cardinality 4
%         Order = 3
%             (2, 3, 4)
% [2]     Order 4            Length 1
%         Permutation group acting on a set of cardinality 4
%         Order = 4 = 2^2
%             (1, 3)(2, 4)
%             (1, 2)(3, 4)
% > M := max[1]`subgroup;
% > f := Character(TrivialRepresentation(M));
% > f;
% ( 1, 1, 1 )
% > ind_M := Induction(f, A4);
% > ind_M;
% ( 4, 0, 1, 1 )
% > T := CharacterTable(A4);
% > Decomposition(T, ind_M);
% [
%     1,
%     0,
%     0,
%     1
% ]
% ( 0, 0, 0, 0 )
% > T[4];
% ( 3, -1, 0, 0 )
% > T[1];
% ( 1, 1, 1, 1 )
The ideas used here to prove Theorem~\ref{thm:Wall} can be applied 
to obtain the following variant of Wall’s theorem, established  
by Cook, Wiegold, and Williamson in \cite{MR896628}.

\begin{theorem}[Cook--Wiegold--Williamson]
    \label{thm:CookWiegold-Williamson}
    \index{Cook--Wiegold--Williamson theorem}
    Let $G$ be a finite solvable group and $p$ the smallest prime divisor of $|G|$. 
    Then the number of maximal subgroups of $G$ is at most $\frac{|G|-1}{p-1}$. Equality 
    holds if and only if $G$ is an elementary $p$-group.  
\end{theorem}

\begin{bonus}
    Prove Theorem~\ref{thm:CookWiegold-Williamson}.    
\end{bonus}

\begin{exercise}
    \label{xca:kernel_ind}
    Let $H$ be a subgroup of a finite group $G$ and 
    $\chi\in\Char(H)$. Prove that 
    $\ker\Ind_H^G\chi=\bigcap_{x\in G}x(\ker\chi)x^{-1}$. 
\end{exercise}

\begin{sol}{xca:kernel_ind}
    Let $g\in\ker\Ind_H^G\chi$. Then \[
    \sum_{x\in G}\chi^0(x^{-1}gx)=\sum_{x\in G}\chi(1).
    \]
    Then $\chi^0(x^{-1}gx)=\chi(1)$ for all $x\in G$. (Otherwise, $|\chi^0(y^{-1}gy)|<\chi(1)$ for some
    $y\in G$ and hence 
    \[
    \sum_{x\in G}\chi(1)=\left|\sum_{x\in G}|\chi^0(x^{-1}gx)\right|
    \leq\sum_{x\in G}|\chi^0(x^{-1}gx)|<\sum_{x\in G}\chi(1),
    \]
    a contradiction.) In particular, $x^{-1}gx\in\ker\chi\subseteq H$ for all $x\in G$, that 
    is $g\in x(\ker\chi)x^{-1}$ for all $x\in G$. 
    
    Conversely, let $g\in\bigcap_{x\in G}x(\ker\chi)x^{-1}$. Then $g\in x(\ker\chi)x^{-1}$ for all $x\in G$. 
    This implies that $x^{-1}gx\in\ker\chi\subseteq H$ for all $x\in G$. Thus 
    \[
    \chi^0(x^{-1}gx)=\chi(x^{-1}gx)=\chi(1)
    \]
    for all $x\in G$. Summing over all $x\in G$ and dividing by $|H|$, 
    \[
    (\Ind_H^G\chi)(g)=\frac{1}{|H|}\sum_{x\in G}\chi^0(x^{-1}gx)=(G:H)\chi(1)=(\Ind_H^G\chi)(1).
    \]
\end{sol}

\begin{exercise}
\label{xca:kernel_constituent}
    Let $M$ be a maximal subgroup of a finite group $G$, and $\Tchar_G\ne\chi\in\Irr(G)$ be
    a constituent
    of $\Ind_M^G\Tchar_M$. Prove that $\ker\chi=\Core_GM$. 
\end{exercise}

\begin{sol}{xca:kernel_constituent}
    We first prove that $\ker\chi\subseteq\Core_GM$. If not, $G=(\ker\chi)M$ because $M$ is a maximal subgroup of $G$.
    
    We claim that $\Res_M^G\chi\in\Irr(M)$. Let $\rho\colon G\to\GL(V)$ be a representation with character $\chi$.  
    Every $g\in G$ can be written as $g=xm$ for $x\in\ker\chi$ and $m\in M$. 
    Thus $\rho_g=\rho_x\rho_m=\rho_m$ and a subspace $W$ of $V$ such that $\rho|_M(W)\subseteq W$ 
    will also be such that $\rho(W)\subseteq W$. 

    By Frobenius' reciprocity, 
    \[
    0\ne\langle\chi,\Ind_M^G\Tchar_M\rangle=\langle\Res_M^G\chi,\Tchar_M\rangle.
    \]
    Hence $\Res_M^G\chi=\Tchar_M$. This implies that $M\subseteq\ker\chi$ and hence 
    $M=(\ker\chi)M=G$, a contradiction.  Therefore $\ker\chi\subseteq M$ and 
    hence $\ker\chi\subseteq\Core_GM$. 

    For the other inclusion, use Exercise~\ref{xca:kernel_ind}.
\end{sol}

 
\section{Project: Fourier analysis on groups}

\subsection{Fourier transform on abelian groups}

\begin{definition}
	\index{Convolution}
	Let $G$ be a finite group and $\alpha,\beta\in L(G)$. The \emph{convolution} of $\alpha$ and 
    $\beta$ is the function 
	\[
	\alpha*\beta\colon G\to\C,\quad
	(\alpha*\beta)(x)=\sum_{y\in G}\alpha(xy^{-1})\beta(y).
	\]
\end{definition}

\begin{exercise}
	\label{xca:delta}
	Let $G$ be a finite group. For $x\in G$, let 
    \[
	\delta_x\colon G\to\C,\quad
	\delta_x(y)=\begin{cases}
		1 & \text{if $x=y$},\\
		0 & \text{otherwise}.
	\end{cases}
	\]
	Prove that $\delta_{xy}=\delta_x*\delta_y$.
\end{exercise}

A direct calculation shows that $L(G)$ is a 
commutative ring with the operations
	\[
	(\alpha+\beta)(x)=\alpha(x)+\beta(x),\quad
	(\alpha*\beta)(x)=\sum_{y\in G}\alpha(xy^{-1})\beta(y).
	\]

\begin{proposition}
	Let $G$ be a finite group and $f\in L(G)$. Then 
    $f\in\cf(G)$ if and only if $f*\alpha=\alpha*f$ 
    for all $\alpha\in L(G)$. In particular, 
    $\cf(G)=Z(L(G))$. 
\end{proposition}

\begin{proof}
  If $f\in\cf(G)$, then 
  \[
    (\alpha*f)(x)=\sum_{y\in G}\alpha(xy^{-1})f(y)=\sum_{y\in G}\alpha(xy^{-1})f(xyx^{-1}).
  \]
  Let $z=xy^{-1}$. Then 
  \[
    (\alpha*f)(x)=\sum_{z\in G}\alpha(z)f(xz^{-1})=\sum_{z\in G}f(xz^{-1})\alpha(z)=(f*\alpha)(x).
  \]
  
  Conversely, if $f*\alpha=\alpha*f$ for all $\alpha$, 
  then, in particular, 
  \begin{align*}
    f(zx)&=\sum_{y\in G}\delta_{z^{-1}}(xy^{-1})f(y)=(\delta_{z^{-1}}*f)(x)
    =(f*\delta_{z^{-1}})(x)=\sum_{y\in G}f(xy^{-1})\delta_{z^{-1}}(y)
  \end{align*}
  for all $z\in G$. Thus $f(zxz^{-1})=f(z^{-1}zx)=f(x)$ for all
  $x,z\in
  G$.
\end{proof}

\begin{exercise}
\label{xca:dual}
	Let $G$ be a finite abelian group and $\widehat{G}=\Irr(G)$. 
    Prove that $\widehat{G}$ with the operation 
    \[
	(\chi\theta)(g)=\chi(g)\theta(g),\quad g\in G,
    \]
    is an abelian group of order $|G|$.
\end{exercise}
 
\begin{example}
  Let $G=\Z/n$. Then
  \[
    \widehat{G}=\{\chi\colon G\to\C^\times:\text{$\chi$ is a group homomorphism}\}=\{x\mapsto \exp(2\pi iax/n):a\in\Z\}.
  \]
\end{example}

\begin{definition}
  \index{Fourier transform}
  Let $G$ be a finite abelian group. The \emph{Fourier 
  transform} of $f\in L(G)$ is the function 
  \[
	\widehat{G}\to\C,\quad
	\chi\mapsto\widehat{f}(\chi)
	=|G|\langle f,\chi\rangle=\sum_{x\in G}f(x)\overline{\chi(x)}.
  \]
\end{definition}

\begin{example}
  We compute the Fourier transform of the map
  $f\colon\Z/n\to\C$, $f(x)=1$. If
  $\chi_j(x)=\exp(2\pi ijx/n)$, then 
  \[
	\widehat{f}(\chi_j)=n\langle f,\chi_j\rangle
	=\sum_{m\in\Z/n}f(m)\overline{\chi_j(m)}
	=\sum_{m\in\Z/n}\exp(-2\pi imj/n)
	=\begin{cases}
	  1 & \text{if $j=0$},\\
	  0 & \text{otherwise}.
	\end{cases}
  \]
\end{example}

\begin{exercise}
    Let $f(x)=\frac12(\delta_1(x)+\delta_{-1}(x))$. Prove that 
    \[
      \widehat{f}(y)=\cos (2\pi y/n). 
    \]
\end{exercise}

\begin{exercise}
    Let $f(x)=\frac13(\delta_1(x)+\delta_0(x)+\delta_{-1}(x))$. Prove that
    \[
      \widehat{f}(y)=\frac13(1+2\cos(2\pi y/n)). 
    \]
\end{exercise}

The following result is known as the 
\emph{inversion formula}. 

\begin{proposition}
  \label{pro:inversion_abelian}
  Let $G$ be a finite abelian group and $f\in L(G)$. Then 
  \[
    f=\frac{1}{|G|}\sum_{\chi\in\widehat{G}}\widehat{f}(\chi)\chi.
  \]
  In particular, the map $L(G)\to L(\widehat{G})$, 
  $f\mapsto\widehat{f}$, is a linear isomorphism. 
\end{proposition}

\begin{proof}
  Since $G$ is abelian, $\widehat{G}$ is an orthonormal basis of $L(G)$. Thus   
  \[
    f=\sum_{\chi\in\widehat{G}}\langle f,\chi\rangle\chi=\frac{1}{|G|}\sum_{\chi\in\widehat{G}}|G|\langle f,\chi\rangle \chi
    =\frac{1}{|G|}\sum_{\chi\in\widehat{G}}\widehat{f}(\chi)\chi.
  \]
  A direct calculation shows that $f\mapsto\widehat{f}$ is linear 
  and injective. It is then bijective since $\dim L(G)=\dim
  L(\widehat{G})$. 
\end{proof}

There are two products that turn $L(G)$ into a commutative ring. 
One of these is point-wise multiplication; the other is convolution.
These two ring structures are isomorphic, as the following result shows:

\begin{theorem}
  \label{thm:convolucion}
  Let $\alpha,\beta\in L(G)$. Then 
  $\widehat{(\alpha*\beta)}(\chi)=(\alpha\beta)(\chi)$ for all
  $\chi\in\widehat{G}$.
\end{theorem}

\begin{proof}
Note that 
\begin{align*}
  \widehat{(\alpha*\beta)}(\chi) &= |G|\langle \alpha*\beta,\chi\rangle 
  =\sum_{x\in G}(\alpha*\beta)(x)\overline{\chi(x)}
  =\sum_{x\in G}\sum_{y\in G}\alpha(xy^{-1})\beta(y)\overline{\chi(x)}.
\end{align*}
Letting $z=xy^{-1}$ and using that $\chi$ is a group homomorphism, 
\begin{align*}
  \widehat{(\alpha*\beta)}(\chi) &= \sum_{y\in G}\beta(y)\sum_{z\in G}\alpha(z)\overline{\chi(zy)}\\
  &=\sum_{y\in G}\beta(y)\sum_{z\in G}\alpha(z)\overline{\chi(z)}\overline{\chi(y)}
  =|G|\langle \alpha,\chi\rangle |G|\langle \beta,\chi\rangle
  =\widehat{\alpha}(\chi)\widehat{b}(\chi).\qedhere 
\end{align*}
\end{proof}

\begin{corollary}
  Let $G$ be a finite group of order $n$. 
  Then $L(G)\simeq\C^n$ as algebras. 
\end{corollary}

\begin{proof}
  Assume that $\widehat{G}=\{\chi_1,\dots,\chi_n\}$. 
  Let 
  \[
  T\colon L(G)\to\C^n,
  \quad
  f\mapsto(\widehat{f}(\chi_1),\dots,\widehat{f}(\chi_n)).
  \]
  A routine calculation shows that $T$ is linear. By 
  Proposition~\ref{pro:inversion_abelian}, 
  $T$ is injective. Hence $T$ is bijective since
  $\dim L(G)=n$. If $\alpha,\beta\in L(G)$, then Theorem~\ref{thm:convolucion} implies that 
  \begin{align*}
    T(\alpha*\beta)&=(\widehat{(\alpha*\beta)}(\chi_1),\dots,\widehat{(\alpha*\beta)}(\chi_n))\\
    &=(\widehat{\alpha}(\chi_1)\widehat{\beta}(\chi_1),\dots,\widehat{\alpha}(\chi_n)\widehat{\beta}(\chi_n))\\
    &=(\widehat{\alpha}(\chi_1),\dots,\widehat{\alpha}(\chi_n))(\widehat{\beta}(\chi_1),\dots,\widehat{\beta}(\chi_n))\\
    &=T(\alpha)T(\beta).\qedhere 
  \end{align*}
\end{proof}

The following result is known as the \emph{Plancherel formula}.

\begin{exercise}
  Let $G$ be a finite abelian group and $\alpha,\beta\in L(G)$. Prove that  \[
  |G|\langle \alpha,\beta\rangle=\langle\widehat{\alpha},\widehat{\beta}\rangle.
  \]
\end{exercise}

% \begin{proof}
%   Gracias a la fÛrmula de inversiÛn (proposiciÛn~\ref{proposition:inversion_abeliano}) y a las
%   relaciones de ortogonalidad del teorema~\ref{theorem:chi=phi},
%   \begin{align*}
% 	\langle a,b\rangle
% 	&=\frac{1}{|G|^2}\left\langle \sum_{\chi\in\widehat{G}}\widehat{a}(\chi)\chi,\sum_{\theta\in\widehat{G}}\widehat{b}(\theta)\theta\right\rangle\\
% 	&=\frac{1}{|G|^2}\sum_{\chi,\theta\in\widehat{G}}\widehat{a}(\chi)\overline{\widehat{b}(\theta)}\langle\chi,\theta\rangle
% 	=\frac{1}{|G|^2}\sum_{\chi\in\widehat{G}}\widehat{a}(\chi)\overline{\widehat{b}(\chi)}
% 	=\frac{1}{|G|}\langle \widehat{a},\widehat{b}\rangle.\qedhere 
%   \end{align*}
% \end{proof}

\subsection{Application: graph theory}

Recall that a \emph{graph} $\Gamma$ is a pair $(V,E)$, where
$E$ is a subset of the set of non-ordered pairs of $V$. The set 
$V=V(\Gamma)$ is the set of \emph{vertices} of $\Gamma$ and $E=E(\Gamma)$
is the set of \emph{edges} of $\Gamma$. If $V$ and $E$ are finite, then the graph $(V,E)$ is said to be \emph{finite}. The 
\emph{adjacency matrix} of a finite graph $\Gamma$ with $n$ vertices is the matrix 
$A=(A_{ij})_{1\leq i,j\leq n}$ given by 
\[
  A_{ij}=\begin{cases}
    1 & \text{if the vertices $v_i$ and $v_j$ are connected,}\\
    0 & \text{otherwise.}
  \end{cases}
 \]

 The adjacency matrix of a finite graph is symmetric and hence
 diagonalizable with real eigenvalues. 
 The \emph{spectrum} of  $\Gamma$ is the set $\operatorname{Spec}(\Gamma)$ of eigenvalues
 of its adjacency matrix. 

\begin{lemma}
  \label{lem:eigenvalues}
  Let $G$ be a finite abelian group and $\alpha\in L(G)$. 
  Then $A\colon L(G)\to
  L(G)$, $\beta\mapsto \alpha*\beta$, is diagonalizable. Moreover, 
  each $\chi\in\widehat{G}$
  is an eigenvector of $A$ with eigenvalue $\widehat{a}(\chi)$. 
\end{lemma}

\begin{proof}
  Let $\chi,\theta\in\widehat{G}$. 
  By Theorem~\ref{thm:convolucion}, 
  \[
   \widehat{(\alpha*\chi)}(\theta)
    =\widehat{\alpha}(\theta)\widehat{\chi}(\theta)
    =\widehat{\alpha}({\theta})|G|\delta_{\chi}(\theta),
    \quad
    \delta_{\chi}(\theta)=\begin{cases} 
      1 & \text{if $\chi=\theta$},\\ 
      0 & \text{otherwise}.
    \end{cases}
  \]
  By the inversion formula, 
  \[
	\alpha*\chi=\frac{1}{|G|}\sum_{\theta\in\widehat{G}}\widehat{(\alpha*\chi)}(\theta)\theta
	=\frac{1}{|G|}\sum_{\theta\in\widehat{G}}\widehat{\alpha}(\theta)|G|\delta_{\chi}(\theta)\theta
	=\widehat{\alpha}(\chi)\chi. 
  \]
  Thus each $\chi$ is an eigenvector of $A$ with eigenvalue $\widehat{\alpha}(\chi)$. Since the 
  $\chi\in\widehat{G}$ form an orthogonal basis of eigenvectors, $A$ is diagonalizable. 
\end{proof}

\index{Symmetric subset}
For a group $G$ and a subset $S\subseteq G$, 
we say that $S$ is a \emph{symmetric subset} if
$1\not\in S$ and for each $s\in S$ one has $s^{-1}\in S$.

\begin{definition}
\index{Cayley graph}
  Let $G$ be a finite group and $S\subseteq G$ be a symmetric subset. 
  The \emph{Cayley graph} of $G$ with respect to $S$ is the graph $X(G,S)$ with
  vertices $G$ and edges of the form $\{g,sg\}$ for $g\in G$ and $s\in S$.  
\end{definition}

%In the previous definition, 
%$\{g,h\}$ is an edge of $X(G,S)$ if and only if $gh^{-1}\in S$. 

\begin{exercise}
	Prove that $X(G,S)$ is connected if and only if
    $G=\langle S\rangle$.
\end{exercise}

\begin{theorem}
  \label{thm:spec(A)}
  Let $G=\{g_1,\dots,g_n\}$ be a finite abelian group and $S\subseteq G$ be a symmetric subset.
  Assume that $\Irr(G)=\{\chi_1,\dots,\chi_n\}$. If 
  $A$ is the adjacency matrix of $X(G,S)$, then  
  \[
    v_i=\frac{1}{\sqrt{n}}\colvec{3}{\chi_i(g_1)}{\vdots}{\chi_i(g_n)}
  \]
  is an orthonormal basis of eigenvectors of $A$, where 
  \[
    Av_i=\lambda_iv_i,\quad
    \lambda_i=\sum_{s\in S}\chi_i(s)
  \]
  for all $i\in\{1,\dots,n\}$. 
\end{theorem}

\begin{proof}
  Let $\delta_S=\sum_{s\in S}\delta_s$ be the characteristic function of $S$ 
  and 
  $F\colon L(G)\to L(G)$, $b\mapsto\delta_S*b$. By Lemma~\ref{lem:eigenvalues}, 
  $F$ is diagonalizable with eigenvalues 
  \[
    \widehat{\delta_S}(\chi_i)
    =n\langle\delta_S,\chi_i\rangle
    =\sum_{x\in G}\delta_S(x)\overline{\chi_i(x)}
    =\sum_{s\in S}\overline{\chi_i(s)}
    =\sum_{s\in S}\chi_i(s^{-1})
    =\sum_{s\in S}\chi_i(s),
  \]
  since $S$ is a symmetric subset. 
  
  The matrix $[F]$ of $F$ in the basis $\{\delta_{g_1},\dots,\delta_{g_n}\}$ 
  has eigenvectors $v_1,\dots,v_n$ with eigenvalues $\lambda_1,\dots,\lambda_n$. Moreover,  
  $[F]$ is the adjacency matrix of $X(G,S)$, that is 
  \[
    [F]_{ij}=\begin{cases}
    1 & \text{if $g_i=sg_j$,}\\
    0 & \text{otherwise,}
  \end{cases}
  \]
  since 
  \[
	F(\delta_{g_j})
	=\delta_S*\delta_{g_j}
	=\sum_{s\in S}\delta_s*\delta_{g_j}
	=\sum_{s\in S}\delta_{sg_j}, 
  \]
  by Exercise~\ref{xca:delta}. Since the $\chi_j$ are orthonormal, 
  so are the $v_j$. 
\end{proof}

\begin{definition}
\index{Circulant matrix}
  A \textbf{circulant matrix} is a matrix $A=(A_{ij})\in\C^{n\times n}$ 
  such that there exists a map $f\colon\Z/n\to\C$ with   
  $A_{ij}=f(j-i)$ for all $i,j\in\Z/n$, that is a matrix of the form 
  \[
	A=\begin{pmatrix}
	  a_0 & a_1 & a_2 & \cdots & a_{n-1}\\
	  a_{n-1} & a_0 & a_1 & \cdots &a_{n-2}\\
	  \vdots & \vdots & \vdots & \ddots & \vdots\\
	  a_2 & a_3 & a_4 & \cdots & a_1\\
	  a_1 & a_2 & a_3 & \cdots & a_{0}
	\end{pmatrix}.
  \]
\end{definition}

  \begin{example}
    A $3\times 3$ circulant matrix is of the form $
	\begin{pmatrix}
	  a_0 & a_1 & a_2\\
	  a_2 & a_0 & a_1\\
	  a_1 & a_2 & a_0
	\end{pmatrix}$.
%    Una matriz circulante de $4\times 4$:
%    \[
%	A=\begin{pmatrix}
%	  a_0 & a_1 & a_2 & a_3\\
%	  a_3 & a_0 & a_1 & a_2\\
%	  a_1 & a_3 & a_0 & a_3\\
%	  a_2 & a_2 & a_3 & a_0
%	\end{pmatrix},
%    \]
  \end{example}

\begin{exercise}
  Let $G$ be a finite group and $S$ be a symmetric subset of $G$. Prove that 
  the circulant matrix corresponding to the characteristic function 
  of $S$ is the adjacency matrix of $X(G,S)$.
\end{exercise}

\begin{corollary}
  Let $A$ be a circulant matrix of size $n\times n$, that is the adjacency matrix 
  of some $X(\Z/n,S)$. The eigenvalues of $A$ are
  \[
    \lambda_k=\sum_{m\in S}\exp(2\pi ikm/n),\quad
    k\in\{0,\dots,n-1\}.
  \]
    The corresponding orthonormal basis of eigenvectors is  
  \[
    v_k=\frac{1}{\sqrt{n}}(1,\exp(2\pi ik2/n),\dots,\exp(2\pi ik(n-1)/n))^T,\quad k\in\{0,\dots,n-1\}.
  \]
\end{corollary}

\begin{proof}
    It follows immediately from Theorem~\ref{thm:spec(A)} with $G=\Z/n$.    
\end{proof}

\subsection{Application: elementary geometry}

Let $P$ be a polygon in $\R^2$ with vertices 
$z_0,z_1,\dots,z_{k-1}$. For each 
$j\geq1$, let $d_j$ be the midpoint between $z_{j-1}$ and $z_j$, that is 
  \[
	d_j=\frac12(z_{j-1}+z_j)
  \]

Let $z\colon\Z/k\to\C$, $z(j)=z_j$, and 
  \[
    D_z=\frac12((\delta_0+\delta_1)*z)\in L(\Z/k). 
  \]
  
\begin{exercise}
\label{xca:D_z}
    Prove that $D_z(j)=d_j$ for all $j\in\{1,\dots,k-1\}$.
\end{exercise}

Now we consider the polygon $P'$ with vertices 
  $D_z(1),\dots,D_z(k-1)$. By repeatedly taking midpoints, 
  we obtain a sequence of polygons 
  \[
  P,P^{(1)},P^{(2)},P^{(3)},\dots
  \]
  where 
  $P^{(0)}=P$ and 
  $P^{(n+1)}=(P^{(n)})'$ for $n\geq0$.

    Our goal is to show that as $n \to \infty$, the polygon $P^{(n)}$ converges to the \emph{centroid} of $P$, 
    defined by
  \[
	\frac{1}{k}\sum_{j=0}^{k-1}z_j.
  \]
  
  \begin{theorem}
	Let $P$ be a plane polygon with vertices $z_0,z_1,\dots,z_{k-1}$. Then 
   	\[
	  \lim_{n\to\infty}P^{(n)}=\frac{1}{k}(z_0+\cdots+z_{k-1}).
	\]
  \end{theorem}

  \begin{proof}
    Without loss of generality we may assume that
    $P$ has its centroid at the origin, that is $\sum_{j=0}^{k-1}z(j)=0$.  Let
    $d=\frac12(\delta_0+\delta_1)$. By Exercise~\ref{xca:D_z}, 
    \[
      D_z=\frac{1}{2}(\delta_0+\delta_1)*z=d*z. 
    \]
    
    Identifying $\Z/k$ with 
    $\widehat{\Z/k}$ via $j\mapsto\chi_j$, where 
    $\chi_j(m)=\exp(2\pi ijm/k)$,
    we compute 
    \[
      \widehat{(d*z)}(j)=\widehat{d}(j)\widehat{z}(j)=\frac{1}{2}(1+\exp(-2\pi ij/k)).
    \]
    Hence  
    \[
      \lim_{n\to\infty}\widehat{d^{*n}}(j)=\begin{cases}
	0 & \text{if $j\in\{1,\dots,k_1\}$},\\
	1 & \text{if $j=0$},
      \end{cases}
    \]
    since  
    \begin{align*}
      \lim_{n\to\infty}\widehat{d^{*n}}(j)
      &=\lim_{n\to\infty}\left(\frac12(1+\exp(-2\pi ij/k)\right)^r\\
      &=\lim_{n\to\infty}\left(\frac12\exp(-\pi ij/k)(\exp(\pi ij/k)+\exp(-\pi ij/k)\right)^n\\
      &=\lim_{n\to\infty}\frac{1}{2^n}\exp(-\pi ijn/k)(2\cos\pi j/k)^n\\
      &=\lim_{n\to\infty}\exp(-\pi ijn/k)(\cos (\pi j/k)^n
    \end{align*}
    and 
    \[
    |\cos(-2\pi ij/k)|<1
    \]
    if $j\in\{1,\dots,k-1\}$. 

    Since $\widehat{z}(0)=|G|\langle z,1\rangle=\sum_{m\in\Z/k}z(m)=0$, it follows that
    \[
	\lim_{n\to\infty}\widehat{(d^{*n}*z)}(j)=\lim_{n\to\infty}\widehat{d^{*n}}(j)\widehat{z}(j)=0.
    \]
    Applying the inverse of the Fourier transform, we conclude 
    that $\lim_{n\to\infty}d^{*r}*z(j)=0$. 
  \end{proof}

\subsection{Application: Uncertainty principle}

\begin{definition}
\index{Support}
  Let $G$ be a finite group and $f\in L(G)$. The \emph{support} of 
  $f$ is the set $\supp(f)=\{x\in G:f(x)\ne0\}$. 
\end{definition}

\begin{theorem}
  Let $G$ be a finite abelian group. Then 
  $|G|\leq|\supp(f)||\supp(\widehat{f})|$.
\end{theorem}

\begin{proof}
    Note that  
    \begin{equation}
    \label{eq:incertidumbre1}
	\|f\|_2^2=\langle f,f\rangle=\frac{1}{|G|}\sum_{x\in G}|f(x)|^2=\frac{1}{|G|}\sum_{x\in\supp(f)}|f(x)|^2\leq\frac{1}{|G|}|\supp(f)\|f\|^2_{\infty},
      \end{equation}
  where $\|f\|_{\infty}=\max_{x\in G}|f(x)|$.

  By the inversion formula of Proposition~\ref{pro:inversion_abelian}, 
  \[
	|f(x)|=\left|\frac{1}{|G|}\sum_{\chi\in\widehat{G}}\widehat{f}(\chi)\chi(x)\right|
	\leq \frac{1}{|G|}\sum_{\chi\in\widehat{G}}|\widehat{f}(\chi)||\chi(x)|
	\leq \frac{1}{|G|}\sum_{\chi\in\widehat{G}}|\widehat{f}(\chi)|
  \]
  since $|\chi(x)|\leq 1$ for all $x\in G$. Hence 
  \begin{equation}
    \label{eq:incertidumbre2}
  \begin{aligned}
	\|f\|^2_{\infty}
	&\leq \frac{1}{|G|^2}\left(\sum_{\chi\in\supp(\widehat{f})}|\widehat{f}(\chi)|\right)^2\\
	&\leq \frac{1}{|G|^2}\sum_{\chi\in\supp(\widehat{f})}|\widehat{f}(\chi)|^2\sum_{\chi\in\supp(\widehat{f})}1^2
	=\frac{|\supp(\widehat{f})|}{|G|}\langle \widehat{f},\widehat{f}\rangle
	=|\supp(\widehat{f})|\|f\|^2_2,
  \end{aligned}
  \end{equation}
  where we have used the Cauchy--Schwarz inequality in the second inequality and 
  Plancherel's formula in the last equality. 

  Combining~\eqref{eq:incertidumbre1} and~\eqref{eq:incertidumbre2}, 
  \[
	\|f\|^2_{\infty}\leq |\supp\widehat{f}|\|f\|^2_2
	\leq \frac{1}{|G|}|\supp\widehat{f}||\supp f|\|f\|^2_{\infty}.
  \]
  Since $f\ne0$, the claim follows. 
\end{proof}

\begin{example}
  If $f=\delta_1$, then $\widehat{f}(\chi)=1$ for all $\chi\in\Irr(G)$.
  Here the inequality is optimal, as 
  $\supp\widehat{f}=G$ and $\supp f=\{1\}$.
\end{example}


%\section{AplicaciÛn: reciprocidad cuadr·tica}
\subsection{Fourier transform on non-abelian groups}

\begin{definition}
\index{Fourier transform}
  Let $G$ be a finite group. Let $\phi^1,\dots,\phi^s$ be the equivalence classes 
  of irreducible representations of $G$. For $k\in\{1,\dots,s\}$, let 
  $d_k=\deg\phi^k$. 
  The \emph{Fourier transform} is the map  
  \[
    T\colon L(G)\to M_{d_1}(\C)\times\cdots\times M_{d_s}(\C),
    \quad
    f\mapsto (\widehat{f}(\phi^1),\dots,\widehat{f}(\phi^s)),
  \]
  where  
  \[
    \widehat{f}(\phi^k)=\sum_{g\in G}f(g)\overline{\phi_g^k}.
  \]
\end{definition}

The matrix 
  $\widehat{f}(\phi^k)$ appearing in the Fourier transform is 
  \[
    \widehat{f}(\phi^k)_{ij}=|G|\langle f,\phi_{ij}^k\rangle=\sum_{g\in G}f(g)\overline{\phi^k(g)}_{ij}.
  \]


\begin{exercise}
  \label{xca:Tlineal}
  Prove that $T$ is a linear transformation. 
\end{exercise}

\begin{exercise}
\label{xca:delta_y}
  Prove that $\widehat{\delta_x}(\phi)=\overline{\phi_x}$ for every
  irreducible representation $\phi$.  
\end{exercise}

We now present the \emph{inversion formula}. 

\begin{theorem}
  \label{thm:inversion}
  Let $G$ be a finite group and $f\in L(G)$. Then 
  \[
    f(x)=\frac{1}{|G|}\sum_{\phi}(\deg\phi)\trace(\phi_x \widehat{f}(\phi)^T),
  \]
  where the sum is taken over all irreducible representations of $G$. 
\end{theorem}

\begin{proof}
  As the expression we need to show is linear on $f$, it is enough to 
  show that the formula is true when 
  $f=\delta_y$. Note that for each unitary $\phi$,   
  $\widehat{\delta_y}(\phi)=\overline{\phi_y}$ by Exercise~\ref{xca:delta_y}. Then 
  \begin{align*}
    \frac{1}{|G|}\sum_{\phi}(\deg\phi)\trace(\phi_x\overline{\phi_y}^T)
    &=\frac{1}{|G|}\sum_{\phi}(\deg\phi)\trace(\phi_x\phi_{y^{-1}})\\
    &=\frac{1}{|G|}\sum_{\phi}(\deg\phi)\trace(\phi_{xy^{-1}})\\
    &=\frac{1}{|G|}\sum_{\chi\in\Irr(G)}\chi(1)\chi(xy^{-1})\\
    &=\delta_y(x),
  \end{align*}
  where the last equality follows from Theorem~\ref{thm:Schur_2nd_orthogonality}.
\end{proof}

\begin{exercise}
  \label{exe:inversion}
  Let $G$ be a finite group and $f\in L(G)$. Prove that 
  \[
    f=\frac{1}{|G|}\sum_{i,j,k}d_k\widehat{f}(\phi^k)_{ij}\phi_{ij}^k.
  \]
\end{exercise}
  
%\begin{theorem}[FÛrmula de inversiÛn]
%  \label{theorem:inversion}
% Sea $G$ un grupo finito y sea $f\in L(G)$. Entonces
% \[
%	f=\frac{1}{|G|}\sum_{i,j,k}d_k\widehat{f}(\phi^k)_{ij}\phi_{ij}^k.
% \]
%\end{theorem}
%
% \begin{svgraybox}
%   Como los $\sqrt{d_k}\phi_{ij}^k$ forman una base ortonormal de $L(G)$, 
%   \[
% 	f=\sum_{i,j,k}\langle f,\sqrt{d_k}\phi_{ij}^k\rangle \sqrt{d_k}\phi_{ij}^k 
% 	=\frac{1}{|G|}\sum_{i,j,k}d_k|G|\langle f,\phi_{ij}^k\rangle\phi_{ij}^k
% 	=\frac{1}{|G|}\sum_{i,j,k}d_k\widehat{f}(\phi^k_{ij})\phi_{ij}^k.
%   \]
% \end{svgraybox}

\begin{example}
  Let $G=\Sym_3$. The group $G$ has three irreducible representations, namely the trivial one $1_G$, 
  the sign and a degree-two representation 
  $\rho\colon G\to\GL_2(\C)$ given by 
  \begin{align*}
     \id &\mapsto \begin{pmatrix}
      1 & 0\\
      0 & 1
    \end{pmatrix}
    && (12)\mapsto\begin{pmatrix}
      -1 & -1\\
      0 & 1
    \end{pmatrix},
    && (123)\mapsto\begin{pmatrix}
      -1 & -1\\
      1 & 0
    \end{pmatrix},\\
    (13)&\mapsto \begin{pmatrix}
      1 & 0\\
      -1 & -1
    \end{pmatrix}
    && (23)\mapsto\begin{pmatrix}
      0 & 1\\
      1 & 0
    \end{pmatrix},
    && (132)\mapsto\begin{pmatrix}
      0 & 1\\
      -1 & -1
    \end{pmatrix}.
  \end{align*}
  
  Let us compute the Fourier transform of $f\in L(G)$.  
  We first compute 
  \begin{align*}
    \widehat{f}(1_G)=\sum_{x\in G}f(x), && \widehat{f}(\sgn)=\sum_{x\in G}\sgn(x)f(x).
  \end{align*}
  Now we compute $\widehat{f}(\rho)$. Since $\rho$ has degree two, 
  \[ 
    \widehat{f}(\rho)=\begin{pmatrix}
      a & b\\
      c & d
    \end{pmatrix}\in M_2(\C).
  \]
  Now we compute 
  \begin{align*}
    a &= \widehat{f}(\rho)_{11}=\sum_{x\in G}f(x)\overline{\rho(x)}_{11}=f(\id)-f(12)+f(13)-f(123),\\
    b &= \widehat{f}(\rho)_{12}=\sum_{x\in G}f(x)\overline{\rho(x)}_{12}=-f(12)-f(123)+f(23)+f(132),\\
    c &= \widehat{f}(\rho)_{21}=\sum_{x\in G}f(x)\overline{\rho(x)}_{21}=f(123)-f(13)+f(23)-f(132),\\
    d &= \widehat{f}(\rho)_{22}=\sum_{x\in G}f(x)\overline{\rho(x)}_{22}=f(\id)+f(12)-f(13)-f(132).
  \end{align*}

  Let us see a concrete example. Let
  \[
  f\colon G\to\C,\quad 
    f(x)=\begin{cases}
    1 & \text{if $|x|=3$},\\
    0 & \text{otherwise.}
  \end{cases}
  \]
  Then $T(f)=(2,2,-I)$, where $I$ is the $2\times 2$ identity matrix. By the inversion formula, 
  \[
	f(x)=\frac16(2+2\sgn(x)-2\rho(x)_{11}-2\rho(x)_{22}).
  \]
\end{example}

\begin{theorem}[Wedderburn]
\index{Weddeburn theorem}
  The linear transformation $T$ is an algebra isomorphism.  
\end{theorem}

\begin{proof}
  The Fourier transform $L(G)\to M_{d_1}(\C)\times\cdots\times
  M_{d_s}(\C)$ is linear (see Exercise~\ref{xca:Tlineal}). It is injective because of 
  the inversion formula (see Exercise~\ref{exe:inversion}). Since $\dim L(G)=|G|=d_1^2+\cdots+d_s^2$, $T$
  is an isomorphism of vector spaces. We now need to show that 
  \[
    \widehat{(\alpha*\beta)}(\phi^k)=\widehat{\alpha}(\phi^k)\widehat{\beta}(\phi^k)
  \]
  for all $\alpha,\beta\in L(G)$ and $k\in\{1,\dots,s\}$. In fact, 
  \begin{align*}
    \widehat{(\alpha*\beta)}(\phi^k) &= \sum_{x\in G}(\alpha*\beta)(x)\overline{\phi^k(x)}\\
    &=\sum_{x,y\in G}\alpha(xy^{-1})\beta(y)\overline{\phi^k(x)}\\
    &=\sum_{y\in G}\beta(y)\sum_{x\in G}\alpha(xy^{-1})\overline{\phi^k(x)}\\
  &=\sum_{y\in G}\beta(y)\sum_{z\in G}\alpha(z)\overline{\phi^k(zy)}\\
  &=\sum_{y\in G}\beta(y)\overline{\phi^k(z)}\sum_{z\in G}\alpha(z)\overline{\phi^k(y)}\\
  &=\widehat{\alpha}(\phi^k)\widehat{\beta}(\phi^k).\qedhere 
  \end{align*}
\end{proof}

\begin{theorem}[Plancherel]
  Let $G$ be a finite group and $\alpha,\beta\in L(G)$. Then 
  \[
	\langle \alpha,\beta\rangle
	=\frac{1}{|G|^2}\sum_{\phi}(\deg\phi)\trace\left(\widehat{\alpha}(\phi)\widehat{\beta}(\phi)^*\right),
  \]
  where the sum runs over all irreducible representations of $G$.
\end{theorem}

\begin{proof}
  It is enough to show the theorem in the case where $\alpha=\delta_y$. The equality 
  we want to prove is equivalent to 
  %entonces 
  %\[
  %  \overline{b(y)}=\frac{1}{|G|}\sum_{\phi}(\deg\phi)\trace\left(\overline{\phi_y}\widehat{b}(\phi)^*\right),
  %\]
  %Esta igualdad es equivalente a 
  \[
    \beta(y)=\frac{1}{|G|}\sum_{\phi}(\deg\phi)\trace\left(\phi_y\widehat{\beta}(\phi)^T\right),
  \]
  which follows from Theorem~\ref{thm:inversion}.
\end{proof}
%\begin{proof}
%  Gracias a la fÛrmula de inversiÛn (ejercicio~\ref{exe:inversion}) escribimos
%  \[
%    a=\frac{1}{|G|}\sum_{i,j,k}d_k\widehat{a}(\phi^k)_{ij}(\phi^k)_{ij},
%    \quad
%    b=\frac{1}{|G|}\sum_{i,j,k}d_k\widehat{b}(\phi^k)_{ij}(\phi^k)_{ij},
%  \]
%  Entonces, como las $\sqrt{d_k}(\phi^k)_{ij}$ forman una base ortonormal de $L(G)$:
%  \begin{align*}
%    \langle a,b\rangle 
%    &= \frac{1}{|G|^2}\sum_{i,j,k}\sum_{p,q,r}d_kd_r\langle \widehat{a}(\phi^k)_{ij}(\phi^k)_{ij},\widehat{b}(\phi^r)_{pq}(\phi^r)_{pq}\rangle\\
%    &= \frac{1}{|G|^2}\sum_{i,j,k}d_k\widehat{a}(\phi^k)_{ij}\overline{\widehat{b}(\phi^k)_{ij}}
%    = \frac{1}{|G|^2}\sum_{i,j,k}d_k\widehat{a}(\phi^k)_{ij}(\widehat{b}(\phi^k))^*_{ji}\\
%    &= \frac{1}{|G|^2}\sum_{k=1}^sd_k\sum_{i=1}^{d_k}\sum_{j=1}^{d_k}\widehat{a}(\phi^k)_{ij}(\widehat{b}(\phi^k))^*_{ji}
%    =\frac{1}{|G|^2}\sum_{k=1}^sd_k\sum_{i=1}^{d_k}(\widehat{a}(\phi^k)(\widehat{b}(\phi^k))^*)_{ii},
%  \end{align*}
%  tal como querÌamos demostrar.
%\end{proof}





%\chapter{Algunas aplicaciones de la teorÌa de caracteres}
%
%\section{Clases de conjugaciÛn reales}
%
%\begin{definition}
%  Sea $G$ un grupo finito. Un caracter $\chi$ de $G$ se dice \textbf{real} si
%  $\chi=\overline{\chi}$, es decir si $\chi(g)\in\R$ para todo $g\in G$. Una
%  case de conjugaciÛn $C$ de $G$ se dice \textbf{real} si para cada $g\in C$ se
%  tiene $g^{-1}\in C$. Utilizaremos la siguiente notaciÛn: Si 
%  $C=\{xgx^{-1}:x\in G\}$ entonces $C^{-1}=\{xg^{-1}x^{-1}:x\in G\}$. 
%\end{definition}
%
%\begin{definition}
%  \label{lem:permutaciones}
%  Antes de demostrar un teorema de Burnside sobre la cantidad de clases de
%  conjugaciÛn reales y la cantidad de caracteres reales, necesitamos recordar
%  cÛmo act˙a la representaciÛn natural del grupo simÈtrico. 
%  
%  Sea $n\in\N$ y sea $\{e_1,\dots,e_n\}$ la base canÛnica de $\C^n$.  La
%  \textbf{representaciÛn natural} de $\Sym_n$ es la representaciÛn 
%  \[
%    \rho\colon\Sym_n\to\GL_n(\C),\quad
%    \sigma\mapsto\rho_{\sigma},
%  \] 
%  donde $\rho_\sigma(e_j)=e_{\sigma(j)}$ para todo $j\in\{1,\dots,n\}$. 
%  
%  La matriz de $\rho_\sigma$ en la base canÛnica est· dada por 
%  \begin{equation}
%    \label{eq:Sn_natural}
%    (\rho_\sigma)_{ij}=\begin{cases}
%      1 & \text{si $i=\sigma(j)$},\\
%      0 & \text{en otro caso}.
%    \end{cases}
%  \end{equation}
%
%  \begin{lem*}
%    Sea $n\in\N$ y sea $\phi\colon\Sym_n\to\GL_n(\C)$ la representaciÛn
%    natural. Si $A\in\C^{n\times n}$ y $\sigma\in\Sym_n$ entonces 
%    \[
%      A_{ij}=(\phi_{\sigma}A)_{\sigma(i)j}=(A\phi_{\sigma})_{\sigma^{-1}(i)j}
%    \]
%    para todo $i,j\in\{1,\dots,n\}$.
%  \end{lem*}
%
%  \begin{proof}
%    Con la fÛrmula~\eqref{eq:Sn_natural} calculamos
%	\[
%	  (A\phi_{\sigma})_{ij}=\sum_{k=1}^n A_{ik}(\phi_{\sigma})_{kj}=A_{i\sigma(j)},
%	  \quad
%	  (\phi_\sigma A)_{ij}=\sum_{k=1}^n (\phi_\sigma)_{ik}A_{kj}=A_{\sigma^{-1}(i)j}.
%	\]
%	Estas fÛrmulas son equivalentes a las que querÌamos demostrar.
%  \end{proof}
%
%\end{definition}
%
%\begin{theorem}[Burnside]
%    Sea $G$ un grupo finito. La cantidad de clases de conjugaciÛn reales es igual
%    a la cantidad de caracteres reales.
%\end{theorem}
%
%\begin{proof}
%  Sea $s$ la cantidad de clases de conjugaciÛn de $G$. Sean $C_1,\dots,C_s$ las
%  clases de conjugaciÛn de $G$ y sean $\chi_1,\dots,\chi_s$ los caracteres
%  irreducibles de $G$. 
%  Sean $\alpha,\beta\in\Sym_s$ dados por $\overline{\chi_i}=\chi_{\alpha(i)}$ y
%  $C_i^{-1}=C_{\beta(i)}$ para todo $i\in\{1,\dots,s\}$. Observar que $\chi_i$
%  es real si y sÛlo si $\alpha(i)=i$ y que $C_i$ es real si y sÛlo si
%  $\beta(i)=i$. La cantidad de puntos fijos $n$ de $\alpha$ es igual a la cantidad
%  de caracteres irreducibles de $G$ y la cantidad de puntos fijos $m$ de $\beta$ es
%  igual a la cantidad de clases reales. 
%
%  Sea $\phi\colon\Sym_s\to\GL(s,\C)$ la representaciÛn natural de $\Sym_s$. Entonces
%  $\chi_\phi(\alpha)=n$ y $\chi_\phi(\beta)=m$. Veamos que 
%  $\trace\phi_\alpha=\trace\phi_\beta$. 
%  Sea $X\in\GL(s,\C)$ la tabla de caracteres de $G$. Por el lema~\ref{lem:permutaciones}, 
%  \[
%	\phi_\alpha X=\overline{X}=X\phi_\beta.
%  \]
%  Como $X$ es una matriz inversible, $\phi_{\alpha}=X\phi_{\beta}X^{-1}$. Luego
%  \[
%    n=\chi_{\phi}(\alpha)=\trace\phi_{\alpha}=\trace\phi_{\beta}=\chi_{\phi}(\beta)=m,
%  \]
%  tal como querÌamos demostrar.
%\end{proof}
%
%\begin{corollary}
%  \label{cor:|G|impar}
%  Sea $G$ un grupo finito. Entonces $|G|$ es impar si y sÛlo si el ˙nico
%  $\chi\in\Irr(G)$ real es el caracter trivial. 
%\end{corollary}
%
%\begin{proof}
%  Supongamos que $G$ tiene una
%  clase de conjugaciÛn $C$ real no trivial y sea $g\in C$. Basta demostrar que $G$ tiene
%  un elemento de orden par. 
%  Sea $h\in G$ tal que $hgh^{-1}=g^{-1}$. Entonces $h^2\in C_G(g)$ (pues 
%  $h^2gh^{-2}=g$). Si $h\in\langle h^2\rangle\subseteq C_G(g)$, $g$ tiene orden par pues 
%  $g^{-1}=g$. Si $h\not\in\langle h^2\rangle$
%  entonces $h^2$ no es un generador de $\langle h\rangle$ y luego $2$ divide a
%  $|h|$ (pues $|h|\ne|h^2|=|h|/(|h|:2)$). 
%
%  RecÌprocamente, si $|G|$ es impar, existe $g\in G$ de orden dos y la clase de
%  conjugaciÛn de $g$ es real. 
%\end{proof}
%
%\begin{theorem}[Burnside]
%  Sea $G$ un grupo de orden impar y sea $s$ el n˙mero de clases de conjugaciÛn
%  de $G$. Entonoces 
%  \[
%    s\equiv|G|\mod 16.
%  \]
%\end{theorem}
%
%\begin{proof}
%  Como $|G|$ es impar, todo $\chi\in\Irr(G)$ no trivial es real (corolario~\ref{cor:|G|impar}). 
%  Los caracteres irreducibles de $G$ son entonces 
%  \[
%    1,\chi_1,\overline{\chi_1},\dots,\chi_k,\overline{\chi_k},
%    \quad
%    s=1+2k.
%  \]
%  Para cada $j\in\{1,\dots,k\}$ sea $d_j=\chi_j(1)$.   Como cada $d_j$ divide a $|G|$ (teorema~\ref{theorem:chi(1)||G|}) y $|G|$ es impar, los $d_j$ son
%  n˙meros impares, digamos $d_j=1+2m_j$. Entonces 
%  \begin{align*}
%    |G|&=1+\sum_{j=1}^k 2d_j^2=1+\sum_{j=1}^k2(2m_j+1)^2\\
%    &=1+\sum_{j=1}^k2(4m_j^2+4m_j+1)
%    =1+2k+8\sum_{j=1}^km_j(m_j+1).
%  \end{align*}
%  Luego $|G|\equiv s\mod{16}$ pues $s=1+2k$ y cada $m_j(m_j+1)$ es un n˙mero par. 
%\end{proof}
%
%
%\section{Teorema de Burnside}
%
%\begin{theorem}[Burnside]
%  Sea $G$ un grupo finito simple no abeliano. 
%\end{theorem}
%
%\begin{theorem}
%  Sean $p,q$ primos. Si $G$ tiene orden $p^aq^b$ entonces $G$ es resoluble.
%\end{theorem}
%
%\section{Teorema de Frobenius}
%
%\begin{theorem}[Frobenius]
%  \label{theorem:Frobenius}
%  Sean $G$ un grupo finito y $H\leq G$ con $H\cap H^x$ para todo $x\in
%  G\setminus H$. Entonces
%  \[
%	N=\left( G\setminus\bigcup_{x\in G}H^x\right)\cup\{1\}
%  \]
%  es un subgrupo normal de $G$.
%\end{theorem}
%
%\begin{proof}
%  $H$. Para cada $\chi\in\Irr(G)$, $\chi\ne1_H$ definimos
%  $\alpha_\chi=\chi-\chi(1)1_H\in\cf(H)$.
%
%  Demostremos que $(\alpha_\chi^G)_H=\alpha_\chi$.
%  Primero, $\alpha_\chi^G(1)=\alpha_{\chi}(1)=0$. Si $h\in H\setminus\{1\}$ entonces
%  \[
%    \alpha_\chi^G(h)=\frac{1}{|H|}\sum_{x\in G}\alpha_\chi^0(x^{-1}hx)
%    =\frac{1}{|H|}\sum_{x\in H}\alpha_\chi(x^{-1}hx)
%    =\frac{1}{|H|}\sum_{x\in H}\alpha_\chi(h)=\alpha_\chi(h),
%  \]
%  pues $x^{-1}hx\in H$ si y sÛlo si $h\in H\cap H^{x}$.
%
%  Por la reciprocidad de Frobenius (\ref{theorem:reciprocidad}):
%  \begin{equation}
%    \label{eq:<a,a>=1+chi2}
%    \langle\alpha_\chi^G,\alpha_\chi^G\rangle
%    =\langle\alpha_{\chi},(\alpha_{\chi}^G)_H\rangle=\langle\alpha_{\chi},\alpha_\chi\rangle
%    =1+\chi(1)^2.
%  \end{equation}
%  Si $\eta\in\Irr(G)$ entonces
%  $\langle\alpha_{\chi}^G,\eta\rangle=\langle\alpha_{\chi},\eta_H\rangle$. Si
%  descomponemos al caracter a $\eta_H$ en suma de irreducibles de $H$, digamos
%  \[
%  \eta_H=m_11_H+m_2\chi+m_3\eta_3+\cdots+m_t\eta_t,
%  \]
%  donde los $m_j$ son enteros no negativos, entonces 
%  \begin{align*}
%    \langle\alpha_{\chi}^G,\eta\rangle
%    &=\langle\chi-\chi(1)1_H,m_11_H+m_2\chi+m_3\eta_3+\cdots m_t\eta_t\rangle\\
%    &=\sum_{j=1}^t\langle\chi,m_j\eta_j\rangle-\sum_{j=1}^t\chi(1)\langle 1_H,m_j\eta_j\rangle
%    =m_2-\chi(1)m_1\in\Z.
%  \end{align*}
%  En particular, $\langle\alpha_\chi^G,1_G\rangle=-\chi(1)$ y
%  \[
%    \alpha_{\chi}^G
%    =\sum_{\theta\in\Irr(G)}\langle\alpha_\chi^G,\theta\rangle\theta
%    =-\chi(1)1_G+\sum_{\substack{\theta\in\Irr(G)\\\theta\ne1_G}}\langle\alpha_\chi^G,\theta\rangle\theta.
%  \]
%
%  Sea $\beta_\chi=\alpha_{\chi}^G+\chi(1)1_G$. 
%  Vamos a demostrar que $\beta_{\chi}\in\Irr(G)$. Si usamos~\eqref{eq:<a,a>=1+chi2} vemos que $\langle\beta_{\chi},\beta_{\chi}\rangle=1$ pues 
%  \[
%    1+\chi(1)^2=\langle\alpha_{\chi}^G,\alpha_{\chi}^G\rangle
%    =\langle\beta_{\chi}-\chi(1)1_G,\beta_{\chi}-\chi(1)1_G\rangle
%    =\langle\beta_{\chi},\beta_{\chi}\rangle+\chi(1)^2.
%  \]
%  Como adem·s $\langle\beta_{\chi},\beta_{\chi}\rangle=\sum_{\theta\ne
%  1_G}\langle\alpha_\chi^G,\theta\rangle^2$, se concluye que existe
%  $\epsilon\in\{-1,1\}$ tal que $\beta_{\chi}=\epsilon\theta$ para alg˙n
%  $\theta\in\Irr(G)$. 
%
%  Para ver que $\beta\in\Irr(G)$ basta observar que
%  $(\beta_{\chi})_H=\chi\in\Irr(H)$ ya que 
%  \[
%	\chi-\chi(1)1_H=\alpha=(\alpha^G)_H=(\beta_{\chi}-\chi(1)1_G)_H=(\beta_{\chi})_H-\chi(1)1_H.
%  \]
%
%  Vamos a demostrar que $N$ es igual a
%  \[
%	M=\bigcap_{\substack{\chi\in\Irr(H)\\\chi\ne1_H}}\ker\beta_{\chi}.
%  \]
%
%  Demostremos primero que $N\subseteq M$. 
%  Sea $n\in N\setminus\{1\}$ y sea $\chi\in\Irr(H)\setminus\{1_H\}$. Como 
%  \[
%	0=\alpha_{\chi}^G(n)=\beta_{\chi}(n)-\chi(1)=\beta_{\chi}(n)-\beta_{\chi}(1),
%  \]
%  entonces $n\in\ker\beta_{\chi}$. 
%  
%  Demostremos ahora que $M\subseteq N$. Sea $h\in M\cap H$ y sea $\chi\in\Irr(H)\setminus\{1_H\}$. Entonces
%  \[
%    \beta_{\chi}(h)-\chi(1)=\alpha_{\chi}^G(h)=\alpha_{\chi}(h)=\chi(h)-\chi(1),
%  \]
%  y luego $h\in\ker\chi$ pues 
%  \[
%    \chi(h)=\beta_{\chi}(h)=\beta_{\chi}(1)=\chi(1).
%  \]
%  Por lo tanto $h\in\cap_{\chi\ne1_H}\ker\chi=1$.  Demostremos ahora que $M\cap
%  H^x=1$ para todo $x\in G$. Sean $x\in G$ y $m\in M\cap H^x$. Como
%  $m=xhx^{-1}$ para alg˙n $h\in H$, $x^{-1}mx\in H\cap M=1$.  Esto implica que
%  $m=1$.
%\end{proof}
%
%\begin{definition}
%  Un grupo $G$ que tiene un subgrupo propositionio no trivial $H$ tal que $H\cap
%  H^x=1$ para todo $x\in G\setminus H$ se llama \textbf{grupo de Frobenius}. El
%  subgrupo $H$ se llama \textbf{complemento de Frobenius} y el subgrupo normal
%  $N$ construido en el teorema~\ref{theorem:Frobenius} se llama \textbf{n˙cleo de
%  Frobenius}.
%
%  \begin{cor*}
%    Sea $G$ un grupo finito con un subgrupo $H$ tal que $H\cap H^x=1$ para todo
%    $x\in G\setminus H$.  Entonces existe un subgrupo normal $N$ de $G$ tal que
%    $G=HN$, $H\cap N=1$.
%  \end{cor*}
%
%  \begin{proof}
%    La existencia del subgrupo normal $N$ est· garantizada por el
%    teorema~\ref{theorem:Frobenius}. Demostremos que $H\subseteq N_H(H)$: si $h\in
%    H\setminus\{1\}$ y $g\in G$ son tales que $ghg^{-1}\in H$, entonces $h\in
%    g^{-1}Hg\cap H$ y luego $g\in H$. Como entonces $H=N_G(H)$, el subgrupo $H$
%    tiene $(G:H)$ conjugados y luego $|G|=|H||N|$ pues 
%    \[
%	|N|=|G|-(G:H)(|H|-1)=(G:H).
%    \]
%    Como $N\cap H=1$, entonces $|HN|=|N||H|/|H\cap N|=|N||H|=|G|$ y luego
%    $G=NH$.
%  \end{proof}
%\end{definition}
%
%\begin{corollary}[Teorema de Frobenius, versiÛn combinatoria]
%  \label{cor:Frobenius_combinatorio}
%  Sea $X$ un conjunto finito y sea $G$ un grupo que act˙a transitivamente en
%  $X$. Supongamos que todo $g\in G\setminus\{1\}$ fija a lo sumo un punto de
%  $X$. El conjunto $N$ formado por la identidad y las permutaciones que mueven
%  todos los puntos de $X$ es un subgrupo de $G$.
%\end{corollary}
%
%\begin{proof}
%  Sea $x\in X$ y sea $H=G_x$. Veamos que si $g\in G\setminus H$ entonces $H\cap
%  gHg^{-1}=1$. Si $h\in H\cap gHg^{-1}$ entonces $h\cdot x=x$ y $g^{-1}hg\cdot
%  x=x$. Como $g\cdot x\ne x$, entonces $h$ fija dos puntos de $X$. Esto implica
%  que $h=1$ (pues todo elemento no trivial fija a lo sumo un punto de $X$). 
%
%  Por el teorema~\ref{theorem:Frobenius}, el conjunto
%  \[
%    N=\left(G\setminus\bigcup_{g\in G}gHg^{-1}\right)\cup\{1\}
%  \]
%  es un subgrupo de $G$. Veamos cÛmo son los elementos de $N$: Si
%  $h\in\cup_{g\in G}gHg^{-1}$ entonces existe $g\in G$ tal que $g^{-1}hg\in H$,
%  es decir $(g^{-1}hg)\cdot x=x$ o quivalentemente $h\in G_{g\cdot x}$. Luego,
%  a eexerciseepciÛn de la identidad, los elementos de $N$ son los elementos de $G$
%  que mueven alg˙n punto de $X$.
%\end{proof}
%
%\begin{example}
%  Sea $F$ un cuerpo finito y sea $G$ el grupo de funciones $f\colon G\to G$ de
%  la forma $f(x)=ax+b$, $a,b\in F$ con $a\ne0$. El grupo $G$ act˙a en $F$ y toda
%  $f\ne\id$ fija a lo sumo un punto de $F$ pues 
%  \[
%	x=f(x)=ax+b\implies x=1-(b/a).
%  \]
%  En este caso, $N=\{f:f(x)=x+b\,,b\in F\}$ que es
%  un subgrupo de $G$.
%\end{example}
%
%\begin{exercise}
%  Demuestre que el teorema~\ref{theorem:Frobenius} puede deducirse del
%  corolario~\ref{cor:Frobenius_combinatorio}.
%\end{exercise}
%
%
%% Wielandt 8.5.4
%% 8.5.6 para ver algo de grupos de permutaciones
%% 7.1 para ejemplo H(q)
%% 10.5.6 (Thompson) N es nilpotente, se usa 10.5.4 
%
%\chapter{La conjetura de McKay}
%
%\section{}
%
%\begin{exercise}
%  \label{exercise:X0}
%  Sea $p$ un n˙mero primo y sea $G$ un $p$-grupo. Si $G$ act˙a en un conjunto
%  finito $X$ entonces 
%  \[
%    |X|\equiv |X_0|\mod p,
%  \]
%  donde $X_0=\{x\in X:g\cdot x=x\;\forall g\in G\}$.
%\end{exercise}
%
%\begin{solution}
%  Descomponemos a $X$ en Ûrbitas: $X=O(x_1)\cup\cdots\cup O(x_n)$, donde 
%  podemos suponer que $O(x_1),\dots,O(x_k)$ son las Ûrbitas que
%  poseen solamente un elemento.  Entonces $X_0=O(x_1)\cup\dots\cup O(x_k)$. Ahora hay que mirar 
%  \[
%    |X|=|O(x_1)|+\cdots+|O(x_n)|=|X_0|+(G:G_{x_k+1})+\cdots+(G:G_{x_n}).
%  \]
%  mÛdulo $p$.
%%  El ejercicio queda resuelto al mirar esta ultima igualdad mÛdulo $p$.
%\end{solution}
%
%\begin{lemma}
%  \label{lem:N_N(H)=C_N(H)}
%Sean $G$ un grupo y $N,H$ subgrupos de $G$. Si $N\trianglelefteq G$ y $N\cap
%G=1$ entonces $N_N(H)=C_N(H)$.
%\end{lemma}
%
%\begin{proof}
% Sean $n\in N_N(H)$ y $h\in H$. Como $nhn^{-1}h^{-1}\in N\cap H=1$,
% $N_N(H)\subseteq C_N(H)$. La otra inclusiÛn es trivial. 
%\end{proof}
%
%\begin{theorem}
%  Sea $p$ un primo, $P$ un $p$-grupo y $G$ un $p'$-grupo\footnote{Un grupo
%  finito se dice un $p'$-grupo si su orden no es divisible por $p$.}.
%  Supongamos que $P$ act˙a en $G$ por automorfismos. Sea 
%  \[
%	cl_P(G)=\{K:\text{$K$ clase de conjugaciÛn de $G$, $a\cdot K=K$ para todo $a\in P$}\}
%  \]
%  el conjunto de clases de conjugaciÛn $P$-invariantes.  Entonces la funciÛn
%  \[
%	cl_P(G)\to cl(C_G(P)),\quad
%	K\mapsto K\cap C_G(P)
%  \]
%  est· bien definida y es biyectiva.
%\end{theorem}
%
%\begin{proof}
%  Sea $\Gamma=G\rtimes P$ el producto semidirecto de $G$ y $P$.  La operaciÛn
%  de $\Gamma$ es
%  \[
%	(g,a)(h,b)=(g(a\cdot h),ab).
%  \]
%  Adem·s $G\simeq G\times 1\trianglelefteq\Gamma$ y $P\simeq 1\times
%  P\leq\Gamma$. Si identificamos $G$ con $G\times 1$ y $P$ con $1\times P$
%  vemos que $g\cdot a=aga^{-1}$ para todo $g\in G$, $a\in P$. 
%
%  \begin{claim*}
%    Si $K\in cl_P(G)$ entonces $K\cap C_G(P)\ne\emptyset$. 
%  \end{claim*}
%
%  Como $P$ act˙a en $K$ el ejercicio~\ref{exercise:X0} nos dice que  $|K|\equiv
%  |K_0|\mod p$, donde 
%  \[
%    K_0=\{x\in K:a\cdot x=x\;\forall a\in P\}=K\cap C_G(P). 
%  \]
%  Si $K\cap C_G(P)=\emptyset$, entonces $p$ dividirÌa a $|K|$, que es una 
%  contradicciÛn pues $K\subseteq G$. 
%
%  \begin{claim*}
%  $K\cap C_G(P)$ es una clase de conjugaciÛn de $C_G(P)$. 
%  \end{claim*}
%
%  Supongamos que $K=\{gxg^{-1}:g\in G\}$.  Si $gxg^{-1}\in C_G(P)$ entonces
%  $g^{-1}Pg\subseteq C_{\Gamma}(x)$. Como $g^{-1}Pg\in\Syl_p(C_{\Gamma}(x))$,
%  existe $\gamma\in C_{\Gamma}(x)$ tal que $g^{-1}Pg=\gamma P\gamma^{-1}$. Si
%  escribimos $\gamma=ha$ con $h\in G$ y $a\in P$, entonces $P=(gh)P(gh)^{-1}$.
%  Sea $c=gh\in N_G(P)=C_{G}(P)$ (la igualdad es por el
%  lema~\ref{lem:N_N(H)=C_N(H)}). Entonces 
%  \[
%	cexercise^{-1}=(gh)x(gh)^{-1}=g(hxh^{-1})g^{-1}=gxg^{-1},
%  \]
%  tal como querÌamos demostrar.
%
%  \begin{claim*}
%    La funciÛn $K\mapsto K\cap C_{G}(P)$ es biyectiva. 
%  \end{claim*}
%
%  Veamos que $K\mapsto K\cap C_G(P)$ es inyectiva. Si $K,L\in cl_P(G)$ y $K\cap
%  C_G(P)=L\cap C_G(P)$ entonces $k\cap L\ne\emptyset$. Luego $K=L$ porque $K$ y
%  $L$ son clases de conjugaciÛn de $G$. Ahora veamos que $K\mapsto K\cap
%  C_G(P)$ es sobreyectiva. Sea $X\in cl(C_G(P))$, digamos $X=cl_{C_G(P)}(c)$,
%  $c\in C_G(P)$. Entonces $K=cl_G(c)\in cl_P(G)$ y adem·s $c\in K\cap C_G(P)$.
%  Luego, como $X$ y $K\cap C_G(P)$ son clases de conjugaciÛn que contienen a $c$,
%  se concluye que $K\mapsto K\cap C_G(P)=X$.
%\end{proof}
%
%%\begin{proof}
%%  Sea $\Gamma=G\rtimes P$ el producto semidirecto de $G$ y $P$.  La operaciÛn
%%  de $\Gamma$ es
%%  \[
%%	(g,a)(h,b)=g(a\cdot h),ab).
%%  \]
%%  Adem·s $G\simeq G\times 1\trianglelefteq\Gamma$ y $P\simeq 1\times
%%  P\leq\Gamma$. Si identificamos $G$ con $G\times 1$ y $P$ con $1\times P$
%%  vemos que $g\cdot a=aga^{-1}$ para todo $g\in G$, $a\in P$. 
%%
%%  Como $p$ es coprimo con el orden de $G$, $P\in\Syl_p(\Gamma)$. Sea $K\in
%%  cl_P(G)$, digamos $K=\{gxg^{-1}:g\in G\}$. 
%%
%%  \begin{claim*}
%%  $\Gamma=GC_{\Gamma}(g)$.
%%  \end{claim*}
%%
%%  Si $a\in P$ entonces, como $K=a\cdot K=cl_G(axa^{-1})$, podemos escribir
%%  $axa^{-1}=gxg^{-1}$ para alg˙n $g\in G$. Esto implica que $g^{-1}a\in
%%  C_\Gamma(x)$, es decir: 
%%  \[
%%    a=g(g^{-1}a)\in GC_\Gamma(x).
%%  \]
%%  Luego $P\subseteq GC_{\Gamma}(x)$. Como adem·s $\Gamma=GP\subseteq
%%  GC_\Gamma(x)\subseteq\Gamma$, se concluye que $\Gamma=GC_{\Gamma}(x)$.
%%
%%  \begin{claim*}
%%  $K\cap C_G(x)\ne\emptyset$. En particular, sin perder generalidad podemos
%%  suponer que $x\in K\cap C_G(P)$. 
%%  \end{claim*}
%%
%%  Sea $Q\in\Syl_p(C_\Gamma(g))$. Como $Q\in\Syl_p(\Gamma)$, existe
%%  $\gamma\in\Gamma$ tal que $Q=\gamma P\gamma^{-1}\subseteq C_{\Gamma}(x)$. Si
%%  escribimos $\gamma=ga\in GP$, entonces
%%  \[
%%	gPg^{-1}=(ga)P(ga)^{-1}=Q\subseteq C_{\Gamma}(x).
%%  \]
%%  Luego $P\subseteq g^{-1}C_\Gamma(x)g$ y por lo tanto $gxg^{-1}\in K\cap
%%  (C_{\Gamma}(x)\cap G)=K\cap C_G(x)$. 
%%
%%  \begin{claim*}
%%    Si $gxg^{-1}\in K\cap C_G(P)$ entonces $cexercise^{-1}=gxg^{-1}$ para alg˙n $c\in
%%    C_G(P)$. 
%%  \end{claim*}
%% 
%%  Si $gxg^{-1}\in C_G(P)$ entonces $g^{-1}Pg\subseteq C_{\Gamma}(x)$. Como
%%  $g^{-1}Pg\in\Syl_p(C_{\Gamma}(x))$, existe $\gamma\in C_{\Gamma}(x)$ tal que
%%  $g^{-1}Pg=\gamma P\gamma^{-1}$. Si escribimos $\gamma=ha$ con $h\in G$ y
%%  $a\in P$, entonces $P=(gh)P(gh)^{-1}$. Sea $c=gh\in N_G(P)=C_{G}(P)$ (la
%%  igualdad es por el lema~\ref{lem:N_N(H)=C_N(H)}). Entonces 
%%  \[
%%	cexercise^{-1}=(gh)x(gh)^{-1}=g(hxh^{-1})g^{-1}=gxg^{-1},
%%  \]
%%  tal como querÌamos demostrar.
%%\end{proof}
%
%
%\end{document}



\TOCpart{Part 3}
\newpage

\section*{Some other topics for final projects}

\pagestyle{plain}
\fancyhf{}
\fancyhead[LE,RO]{Representation theory of algebras}
\fancyhead[RE,LO]{Final projects}
\fancyfoot[CE,CO]{\leftmark}
\fancyfoot[LE,RO]{\thepage}

We collect here some topics for final presentations. Some topics
can also be used as bachelor or master theses. 

\subsection*{Kolchin's theorem}

If $V$ is a finite-dimensional complex 
vector space and $G$ is a subgroup of $\GL(V)$ such that every element $g$ of $G$ is unipotent (i.e., $g - 1$ is a nilpotent linear transformation), then there exists a basis of $V$ in which 
all the element of $G$ are represented by upper triangular matrices with ones on the diagonal. See my
notes for \href{https://github.com/vendramin/associative/}{Associative Algebra}) or \cite[Chapter 2]{MR1369573}.

\subsection*{Staircase groups}

This topic describes a situation similar to that of Kolchin's theorem (see the course \href{https://github.com/vendramin/associative/}{Associative Algebra}), but
more general. See \cite[Chapter 5]{MR1369573}.

\subsection*{Solvable and nilpotent groups}

The character table of a finite group
detects solvability and nilpotency of groups, see
\cite[Chapter 6]{MR1369573}.

% \subsection*{Kegel--Wielandt theorem}

% Prove Kegel--Wielandt theorem \ref{thm:KegelWielandt}. 
% For the proof see \cite[Theorem 2.13]{MR1211633}. 

% \subsection*{The Drinfeld double of a finite group}

% See \cite[Chapter IX]{MR1321145} and 
% \cite[Chapter 8]{MR3752618}.

% \subsection*{Ito's theorem}

% Ito's theorem is stated in Theorem~\ref{thm:Ito}. It  generalize Frobenius' theorem
% (Theorem \ref{thm:Frobenius_chi(1)})  
% and Schur's theorem (Theorem \ref{thm:Schur_chi(1)}). 
% The theorem states that if $\chi$ is an irreducible character
% of a finite group $G$, then $\chi(1)$ divides 
% $(G:A)$ for every normal abelian subgroup $A$ of $G$. 
% See \cite[\S8.1]{MR0450380}. 

\subsection*{Characters of $\GL_2(q)$ and $\SL_2(q)$}

One possible topic is the character table of $\GL_2(q)$, see
\cite[\S5.2]{MR2867444}. Alternatively, one can 
present the character table of the group $\SL_2(p)$  
following Humphreys's paper \cite{MR364478}. 
The character theory of $\SL_2(q)$ appears in 
\cite[\S5.2]{MR2867444}, see 
\cite[Chapter 20]{MR1650707} for details. 

\subsection*{Representations of the symmetric group}

See for example \cite[\S10]{MR2867444} and 
\cite{MR1153249}. 

\subsection*{Random walks on finite groups}

The goal is to construct the character table or 
the irreducible representations of the symmetric group. 
The topic has connections with combinatorics and applications 
to voting and card shuffling. 
See \cite[4]{MR1153249} and \cite[\S11]{MR2867444}.

\subsection*{Fourier analysis on finite groups}

See \cite[\S5]{MR2867444} for a very elementary approach and some
basic applications. Other applications 
appear in \cite{MR1695775}.

\subsection*{Mackey's irreducibility criterion}

It is not at all clear that 
induction of an irreducible character will produce an irreducible character. In fact, 
inducing the trivial character of the trivial subgroup to the whole group produces the 
regular representation, which in general is not irreducible. Mackey found a criterion 
that describes when an induced character is irreducible. See \cite[\S8.3]{MR2867444}. 

\subsection*{McKay's conjecture}

Prove McKay's conjecture \ref{conjecture:McKay} for all sporadic simple groups. 
This was first proved by Wilson in \cite{MR1643110}. 
Note that
for some ``small" sporadic simple groups this can be done
with the script presented in \S\ref{McKay}. However, 
for several sporadic simple groups a different approach is needed. One needs
to know the structure of normalizers. 
% http://www.math.rwth-aachen.de/~Thomas.Breuer/ctblocks/doc/overview.html

% \subsection*{Ore's conjecture}

% Prove Ore's conjecture \ref{conjecture:Ore} for alternating simple groups,
% see for example \cite{MR40298}. It is also interesting to prove the conjecture
% for other "small" simple groups such as $\PSL(3,2)$.  


\subsection*{Hirsh's theorem}

In \cite{MR36755} Hirsch found a generalization of Burnside's Theorem \ref{thm:Burnside_mod16}.  
If $G$ is a finite group and $d$ is the greatest common divisor of all 
the numbers $p^2-1$, where the $p$'s are prime divisors of $|G|$ and $r$ the number of conjugate sets in $G$. Then 
\[
|G|\equiv\begin{cases} 
    r\bmod 2d &\text{if $|G|$ odd,}\\
    r\bmod 3 & \text{if $|G|$ even and $\gcd(|G|,3)=1$.}
    \end{cases}
\]
The proof is elementary and does not use character theory. Is it possible
to prove Hirsch's theorem using characters?

% \subsection*{Clifford's theorem}

% Clifford’s theorem provides a description of the restriction of  irreducible representations to normal subgroups. It is a tool
% that tries to construct representations of the group
% from representations of a normal subgroup. See for example
% \cite[Chapter~7]{MR3970262}. 


\subsection*{Irreducible characters of groups of order $pq$}

Let $G$ be a non-abelian 
group of order $pq$, where 
$p$ and $q$ are prime numbers with $p>q$. Then 
$q\mid p-1$ and $G$ is a Frobenius group (see
Exercise~\ref{xca:Frobenius_pq}). 
The character table of Frobenius groups of order $pq$ can be found in~\cite[Chapter 25]{MR1864147}.

\subsection*{Irreducible characters of the simple group of order 168}

The smallest non-abelian simple group is $\Alt_5$, of
order $60$. The next smallest
is a certain group of order $168$. The character
table of this group can be found in~\cite[Chapter 27]{MR1864147}. 

\subsection*{Irreducible characters of semidirect products}

What can be said about irreducible characters
of semidirect products? The case of 
semidirect products by abelian groups is treated 
in \cite[Section 8.2]{MR0450380}.

% \subsection*{Hurwitz's theorem}

% It states that 
% if there is an identity of the form 
% 	\begin{equation*}
% 		(x_1^2+\cdots+x_n^2)(y_1^2+\cdots+y_n^2)=z_1^2+\cdots+z_n^2,
% 	\end{equation*}
% 	where the $x_j$'s and the $y_j$'s are real (or complex) numbers and
% 	each $z_k$ is a bilinear function in the $x_j$'s and the $y_j$'s, then 
% 	$n\in\{1,2,4,8\}$. There are several 
%     proofs of this result, and one of them uses representation theory! 

% Hurwitz's theorem has a nice application to elementary 
% linear algebra: When can we find  
% If $V$ is a real vector space with an inner product
% such that 
% see
% \cite{MR1534187} for more information. 

% \begin{theorem}
% 	Let $V$ be a real vector space (with an inner product) 
% 	such that $\dim
% 	V=n\geq3$. If there exists a bilinear function 
% 	$V\times V\to\R$, $(v,w)\mapsto v\times
% 	w$, such that $v\times w$ is orthogonal both 
% 	to $v$ and $w$ and 
% 	\[
% 		\|v\times w\|^2=\|v\|^2\|w\|^2-\langle v,w\rangle^2,
% 	\]
% 	where $\|v\|^2=\langle v,v\rangle$, then $n\in\{3,7\}$. 
% \end{theorem}

% \subsection*{Poincar\'e--Birkhoff--Witt theorem}

% There are several proofs of the
% Poincar\'e--Birkhoff--Witt theorem \ref{thm:PBW}, see for example 
% \cite[\S17.4]{MR499562} or \cite[Theorem 2.17]{MR938524}. 
% Bergman's proof based on the diamond lemma 
% appears in \cite{MR506890}. 

% \subsection*{Weyl's theorem}

% Weyl's theorem states that every finite-dimensional module over
% a semisimple Lie algebra is completely irreducible. 
% See \cite[Theorem 17.4]{MR2218355} for a proof. 

% \subsection*{Irreducible representations of $U_q(\sl(2,\C))$}

% Let $q\in\C\setminus\{0,1,-1\}$. 
% Let $U_q(\sl(2))$ be the (complex) algebra generated by 
% variables $E$, $F$, $K$ and $K^{-1}$ with relations
% \begin{align*}
%     &KK^{-1}=K^{-1}K=1,
%     &&
%     KEK^{-1}=q^2E,\\
%     &
%     KFK^{-1}=q^{-2}F,
%     &&
%     [E,F]=\frac{1}{(q-q^{-1})}(K-K^{-1}).
% \end{align*}

% This algebra is a \emph{deformation} of the enveloping algebra
% of $\sl(2,\C)$. The goal is to study the 
% representation theory of $U_q(\sl(2))$. This splits into
% two cases, depending on whether $q$ is a root of one or not. 
% Finite-dimensional simple $U_q(\sl(2))$-modules are studied 
% in \cite[VI]{MR1321145}. In particular, if 
% $q$ is not a root of one, finite-dimensional simple $U_q(\sl(2))$-modules
% are classified in \cite[Theorem VI.3.5]{MR1321145}. 

% \subsection*{Semisimple modules of $U_q(\sl(2,\C))$}

% Prove that if $q$ is not a root of one, any finite-dimensional
% $U_q(\sl(2))$-module is semisimple. 
% See \cite[Theorem VII.2.2]{MR1321145}. 
\section*{Some solutions}

\pagestyle{plain}
\fancyhf{}
\fancyhead[LE,RO]{Representation theory of algebras}
\fancyhead[RE,LO]{Some solutions}
\fancyfoot[CE,CO]{\leftmark}
\fancyfoot[LE,RO]{\thepage}

\addcontentsline{toc}{chapter}{Some solutions}

\begin{sol}{xca:Maschke_multiplicative1}
Let $\theta\colon U\times W\to U$, $(u,w)\mapsto u$. Then $\theta$ is a group homomorphism such that 
$\theta(u)=u$ for all $u\in U$. Since $U$ is $K$-invariant, 
\[
k^{-1}\cdot \theta(k\cdot v)\in U
\]
for all $k\in K$ and $v\in V$. 
Since $K$ is finite and $U$ is abelian, 
the map 
\[
\varphi\colon V\to U,\quad 
v\mapsto \prod_{k\in K}k^{-1}\cdot \theta(k\cdot v), 
\]
is well-defined. 
We claim that $\varphi$ is a group homomorphism. If $x,y\in V$, then 
\begin{align*}
    \varphi(xy) &= \prod_{k\in K}k^{-1}\cdot \theta(k\cdot (xy))\\
    &= \prod_{k\in K}k^{-1}\cdot (\theta(k\cdot x)\theta(k\cdot y))\\
    &= \prod_{k\in K}k^{-1}\cdot \theta(k\cdot x) \prod_{k\in K}k^{-1}\cdot \theta(k\cdot y)=\varphi(x)\varphi(y),
\end{align*}
since $U$ is abelian and $K$ acts by automorphisms on $V$. 

We claim that $N=\ker\varphi$ is $K$-invariant. 
We need to show that $\varphi(l\cdot x)=l\cdot\varphi(x)$ for all $l\in K$ and $x\in V$. 
If $l\in K$ and $x\in V$, then 
\begin{align*}
l^{-1}\cdot\varphi(l\cdot x)&=l^{-1}\cdot\left(\prod_{k\in K}k^{-1}\cdot \theta(k\cdot (l\cdot x))\right)=\prod_{k\in K}(kl)^{-1}\cdot\theta( (kl)\cdot x)=\varphi(x),
\end{align*}
since $kl$ runs over all the elements of $K$ whenever $k$ runs over all the elements of $K$.
In conclusion, $\ker\varphi$ is $K$-invariant. 

It remains to show that $V$ is the direct product of $U$ and $N$. By assumption, $U$ is normal in $V$. 
We first prove that $U\cap N=\{1\}$. If $u\in U$, then $k\cdot u\in U$ for all $k\in K$. This implies that 
$k^{-1}\cdot\theta(k\cdot u)=k^{-1}\cdot (k\cdot u)=u$. Hence $\varphi(u)=u^m$. Since this map is bijective by assumption,  
\[
U\cap N=U\cap\ker\varphi=\{1\}.
\]
We now show that $V\subseteq UN$, as the other inclusion is trivial. Since $N=\ker\varphi$,  
\[
\varphi(V)\subseteq U=\varphi(U)=\varphi(U)\varphi(N)=\varphi(UN) 
\]
and hence $V\subseteq (UN)N=UN$. 
Therefore $V$ is the direct product of $U$ and $N$, as $N$ is normal in $V$.
\end{sol}

\begin{sol}{xca:Maschke_multiplicative2}
    Let $m=|K|$. Since $m$ and $|U|$ are coprime, the map 
    $u\mapsto u^m$ is bijective in $U$. Since $V$ is a vector space over the field 
    $\Z/p$, it follows that $V=U\times W$ for some subgroup $W$ of $V$. Now the claim follows
    from the previous theorem. 
\end{sol}

\begin{sol}{xca:deg2}
  Assume that $\phi$ is not irreducible. There exists a proper non-zero $G$-invariant 
  subspace $W$ of $V$. Thus $\dim W=1$. Let $w\in W\setminus\{0\}$.
  For each $g\in G$, $\phi_g(w)\in W$. Thus $\phi_g(w)=\lambda w$ for some 
  $\lambda$. This means that $w$ is a common eigenvector for all the $\phi_g$.
  Conversely, if $\phi$ admits a common eigenvector $v\in V$, then 
  the subspace generated by $v$ is $G$-invariant.
\end{sol}

\begin{sol}{xca:Z(G)cyclic}
Let $G$ be a finite subgroup of $S^1=\{z\in\Z:z^n=1\}$. We claim that $G$ is cyclic. Let $n=|G|$. 
It is enough to show that 
$G\subseteq\{\exp(2\pi ik/n):0\leq k\leq n-1\}$, 
since $\{\exp(2\pi ik/n):0\leq k\leq n-1\}$ is
cyclic. 
Let $g\in G$. 
Since $g^n=1$, $g$ is a $n$-th root of one, say 
$g=\exp(2\pi i k/n)$ for some $k\in\{1,\dots,n-1\}$. 

Let $\rho\colon G\to\GL(V)$ be a faithful irreducible 
representation. Let $z\in Z(G)$ and $g\in G$. The map 
$T\colon V\to V$, $v\mapsto z\cdot v$ is invariant, since
\[
T(g\cdot v)=z\cdot (g\cdot v)
=(zg)\cdot v=(gz)\cdot v=g\cdot (z\cdot v)=g\cdot T(v)
\]
for all $g\in G$. By Schur's lemma, there exists $\lambda\in\C$ such that $T(v)=\lambda v$
for all $v\in V$. In particular,
$\rho(Z(G))$ is isomorphic to a subgroup of $S^1$. 
Thus $\rho(Z(G))$ is cyclic. Since $\rho$ is faithful, 
$Z(G)$ is cyclic. 
\end{sol}

\begin{sol}{xca:Solomon}
    All we need to do is carefully study the proof
    of Theorem~\ref{thm:Solomon}. Let $n=|G|$. 
    The action of $G$ on itself by conjugation induces a group homomorphism $\rho\colon G\to\GL_n(\C)$ with character 
    $\chi_\rho=\sum_{i=1}^rm_i\chi_i$. In the proof of 
    Solomon's theorem, we saw that 
    each $m_i=\sum_{j=1}^r\chi_i(g_j)$ is a non-negative integer. Thus 
    \[
    \sum_{i=1}^r\sum_{j=1}^r\chi_i(g_j)=\sum_{i=1}^rm_i\in\Z_{\geq1}. 
    \]
    To prove the other inequality, 
    \[
    n=\chi_\rho(1)=\sum_{i=1}^rm_i\chi_i(1)\geq\sum_{i=1}^km_i.
    \]

    Now assume that $\sum_{i=1}^km_i=|G|$. Then $m_i=0$ whenever 
    $\chi_i(1)>1$, that is 
    $\rho$ only have degree-one 
    irreducible components. In particular, $\rho$ is a degree-one
    representation. To see that $G/Z(G)$ is abelian, it is enough to see that $[G,G]\subseteq Z(G)$. Since $\rho$ is then
    a degree-one representation, $\rho(G)\subseteq\C^{\times}$. Thus 
    \[
    \rho([G,G])\subseteq [\rho(G),\rho(G)]=\{1\}.
    \]
    Hence $[G,G]\subseteq\ker\rho=Z(G)$. 
\end{sol}

\begin{sol}{xca:commutators}
    Let $C_1,\dots,C_t$ be the conjugacy classes of $G$. For each
    $i\in\{1,\dots,t\}$, let $g_i$ be a representative of $C_i$. Assume
    that $g_i$ is conjugate to $g$ and 
    $g_j$ is conjugate to $h$. Let $\gamma\in G$.
    Then
    \begin{align*}
        \sum_{z\in G}\chi(zg_iz^{-1}g_j) 
        &= \sum_{z\in G}\chi(\gamma zg_iz^{-1}g_j\gamma^{-1})\\
        &= \sum_{z\in G}\chi(\gamma zg_iz^{-1}\gamma^{-1}\gamma g_j\gamma^{-1})\\
        &=\sum_{y\in G}\chi(yg_iy^{-1}\gamma g_j\gamma^{-1}).
    \end{align*}
    Hence
    \[
    \sum_{z\in G}\chi(zg_iz^{-1}g_j) 
    =\frac{1}{|G|}\sum_{z,\gamma\in G}\chi(zg_iz^{-1}\gamma g_j\gamma^{-1}).
    \]
    Now $z_1g_iz_1^{-1}=z_2g_iz_2^{-1}$ if and only 
    if $z_2^{-1}z_1\in C_G(g_i)$. Thus
    \begin{align*}
        \sum_{z\in G}\chi(zg_iz^{-1}g_j) &= \frac{1}{|G|}|C_G(g_i)||C_G(g_j)|\sum_{\substack{x\in C_i\\y\in C_j}}\chi(xy)\\
        &=\frac{|G|}{|C_i||C_j|}\sum_{\substack{x\in C_i\\y\in C_j}}\chi(xy).
    \end{align*}
    Now 
    \[
    \omega_{\chi}(C_i)\omega_{\chi}(C_j)=\sum_{i=1}^t a_{ijk}\omega_{\chi}(C_k),
    \]
    where 
    \[
    \omega_{\chi}(C_i)=\frac{|C_i|\chi(C_i)}{\chi(1)}
    \]
    and
    $a_{ijk}$ is the number of solutions of the equation
    $xy=z$ with $x\in C_i$, $y\in C_j$ and $z\in C_k$. Therefore
    \begin{align*}
        \frac{\chi(1)}{|G|}\sum_{z\in G}\chi(zg_iz^{-1}g_j)
        &=\frac{\chi(1)}{|C_i||C_j|}\sum_{\substack{x\in C_i\\y\in C_j}}\chi(xy)\\
        &=\frac{\chi(1)}{|C_i||C_j|}\sum_{k=1}^t a_{ijk}\chi(g_k)|C_k|\\
        &=\frac{\chi(1)^2}{|C_i||C_j|}\sum_{k=1}^t a_{ijk}\omega_{\chi}(C_k)\\
        &=\chi(g_i)\chi(g_j).
    \end{align*}
    
    To prove the second formula, 
    set $h=g^{-1}$ in the first formula.
    Then 
    \begin{align*}
        \chi(g)\chi(g^{-1})=\frac{\chi(1)}{|G|}\sum_{z\in G}\chi(zgz^{-1}g^{-1}) &\Longleftrightarrow
        \chi(g)\overline{\chi(g)}=\frac{\chi(1)}{|G|}\sum_{z\in G}\chi([z,g])\\
        & \Longleftrightarrow |\chi(g)|^2=\frac{\chi(1)}{|G|}\sum_{z\in G}\chi([z,g])\\
        & \Longleftrightarrow \frac{|G|}{\chi(1)}|\chi(g)|^2=\sum_{z\in G}\chi([z,g]).
    \end{align*}
\end{sol}


\begin{sol}{xca:least_p}
Let $g_1,\dots,g_m$ be the representatives of non-trivial conjugacy classes. Then $C_G(g_i)$ is non-tivial for all $i$. Since $p$ is the smallest prime dividing the order of $G$, it follows that
$(G:C_G(g_i))\geq p$. Now use the class equation to get
\[
|G|\geq |Z(G)|+pm,
\]
which is equivalent to $m\leq \frac1p (|G|-|Z(G)|)$. Since $G$ is non-abelian, $G/Z(G)$ is not cyclic. Thus $(G:Z(G))\geq p^2$. Now 
\[
\frac{k(G)}{|G|}=\frac{|Z(G)|+m}{|G|}\leq \frac{(p-1)|Z(G)|+|G|}{p|G|}\leq \frac{p^2+p-1}{p^3}.
\]
This bound is reached if and only if $(G:Z(G))=p^2$.  
\end{sol}


\begin{sol}{xca:5/8}
    If $\cp(G)>5/8$, then $|[G,G]|<2$. Thus $[G,G]$ is the trivial group
    and hence $G$ is abelian. 
\end{sol}

\begin{sol}{xca:cp_NS}\
\begin{enumerate}
    \item If $\cp(G)>1/2$, then $|[G,G]|<3$ by Theorem \ref{thm:[GG]}. If $|[G,G]|=1$, 
    then $G$ is abelian and hence $G$ is nilpotent. If $|[G,G]|=2$, then 
    $[G,G]\subseteq Z(G)$. 
    %In fact, a more general fact is true. 
    %If $N$ is a normal subgroup of $G$ and
    %$|N|=2$, then $N\subseteq Z(G)$. Write $N=\{1,x\}$. If $g\in G$, then
    %$gxg^{-1}\in N$. Thus either $gxg^{-1}=1$ or $gxg^{-1}=x$. The first
    %case implies $x=1$, a contradiction. Thus $x\in Z(G)$.
    It follows that 
    $G/Z(G)$ is abelian (and hence nilpotent), so $G$ is nilpotent. 
    \item If $\cp(G)<21/80$, then 
    $|[G,G]|<60$. Thus $[G,G]$ is solvable, as groups of order $<60$ are solvable. 
    Hence $G$ is solvable. 
\end{enumerate}
\end{sol}


\begin{sol}{xca:isoclinism}\
\begin{enumerate}
    \item 
    \item Using that $\sigma$ and 
    $\tau$ are automorphisms and 
    the commutativity of the diagram~\eqref{eq:isoclinism}, 
    we compute 
    \begin{align*}
        (G:Z(G))^2\cp(G) &= \frac{1}{|Z(G)|^2}|\{(x,y)\in G\times G:xy=yx\}|\\
        &=\frac{1}{|Z(G)|^2}|\{(x,y)\in G\times G:[x,y]=1\}|\\
        &=\frac{1}{|Z(G)|^2}|\{(x,y)\in G\times G:c_G(x,y)=1\}|\\
        &=|\{(u,v)\in (G/Z(G))^2:c_G(u,v)=1\}|\\
        &=|\{(u,v)\in (G/Z(G))^2:\tau c_G(u,v)=1\}|\\
        &=|\{(u,v)\in (G/Z(G))^2:c_G(\sigma u,\sigma v)=1\}|\\
        &=|\{(a,b)\in (H/Z(H))^2:c_H(a,b)=1\}|.
    \end{align*}
    It follows that $(G:Z(G))^2\cp(G)=(H:Z(G))^2\cp(H)$. 
\end{enumerate}
\end{sol}

\begin{sol}{xca:centralizer}
    We use the second orthogonality relation and Theorem~\ref{thm:correspondence} 
    to compute
    \begin{align*}
        |C_{G/N}(gN)| &=\sum_{\chi\in\Irr(G/N)}|\chi(gN)|^2
        =\sum_{\substack{\eta\in\Irr(G)\\N\subseteq\ker\eta}} |\eta(g)|^2
        \leq\sum_{\eta\in\Irr(G)}|\eta(g)|^2=|C_G(g)|.
    \end{align*}
\end{sol}

\begin{sol}{xca:malnormal}
    We first prove that $1)\implies2)$. Let $x\not\in H$ and 
    $h\in H$ be such that $h\cdot xH=xH$. Then 
    $h\in xHx^{-1}\cap H=\{1\}$. 

    We now prove $2)\implies3)$. Let $x,y\in G$ be such that 
    $xH\ne yH$, and $g\in G$ be an element 
    that fixes $xH$ and $yH$, that is 
    $g\cdot xH=xH$ and $g\cdot yH=yH$. Then 
    $x^{-1}gx\in H\cap (x^{-1}y)H(x^{-1}y)^{-1}=\{1\}$,
    since $x^{-1}y\not\in H$. Thus $g=1$. 

    Finally, we prove that $3)\implies1)$. Let $x\not\in H$ and $h\in H\cap xHx^{-1}$. Then 
    $hx=xh_1$ for some $h_1\in H$. Hence
    \[
    h\cdot xH=hxH=xh_1H=xH,\quad 
    h\cdot H=H.
    \]
    Since $h$ has at least two fixed points
    in $G/H$, $h=1$. 
\end{sol}

\begin{sol}{xca:malnormal_no2torsion}
    Let $H\ne\{1\}$ be a malnormal subgroup of $G$ and $N=\langle n\rangle\simeq\Z$ be a
    normal subgroup of $G$. 

    Assume first that $N\cap H\ne\{1\}$. Let $k>0$ be minimal such that $n^k\in H$. If $k=1$, then
    $H=N$ is normal in $G$. Since $H$ is malnormal by assumption, $H=G$. 
    
    Assume now that $N\cap H=\{1\}$. Note that
    for every $h\in H\setminus\{1\}$, 
    $hnh^{-1}\in\{n,n^{-1}\}$. 
    If there exists 
    $h\in H\setminus\{1\}$ such that $hnh^{-1}=n$, then
    $1\ne h=nhn^{-1}\in H\cap nHn^{-1}$. Thus $H$ is not malnormal in $G$, a contradiction. 

    Since $G$ has no 2-torsion, there exist 
    $h_1,h_2\in H\setminus\{1\}$ with $h_1\ne h_2$ and 
    $h_1h_2\ne 1$. If $h_jnh_j=n$ for some $j\in\{1,2\}$, 
    then the previous argument shows that 
    $H$ is not malnormal in $G$. Thus we may assume that  $h_1nh_2=n^{-1}$ and 
    $h_2nh_2=n^{-1}$. Then 
    \[
    (h_2h_1)n(h_2h_1)^{-1}=h_2n^{-1}h_2^{-1}
    =(h_2nh_2^{-1})^{-1}=n.
    \]
    Hence, again by the previous argument, 
    $H$ is not malnormal in $G$, a contradiction.  
\end{sol}

\newpage 
\bibliographystyle{abbrv}
\bibliography{refs}
\printindex

\end{document}

