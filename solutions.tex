\section*{Some solutions}

\pagestyle{plain}
\fancyhf{}
\fancyhead[LE,RO]{Representation theory of algebras}
\fancyhead[RE,LO]{Some solutions}
\fancyfoot[CE,CO]{\leftmark}
\fancyfoot[LE,RO]{\thepage}

\begin{sol}{xca:Maschke_multiplicative1}
Let $\theta\colon U\times W\to U$, $(u,w)\mapsto u$. Then $\theta$ is a group homomorphism such that 
$\theta(u)=u$ for all $u\in U$. Since $U$ is $K$-invariant, 
\[
k^{-1}\cdot \theta(k\cdot v)\in U
\]
for all $k\in K$ and $v\in V$. 
Since $K$ is finite and $U$ is abelian, 
the map 
\[
\varphi\colon V\to U,\quad 
v\mapsto \prod_{k\in K}k^{-1}\cdot \theta(k\cdot v), 
\]
is well-defined. 
We claim that $\varphi$ is a group homomorphism. If $x,y\in V$, then 
\begin{align*}
    \varphi(xy) &= \prod_{k\in K}k^{-1}\cdot \theta(k\cdot (xy))\\
    &= \prod_{k\in K}k^{-1}\cdot (\theta(k\cdot x)\theta(k\cdot y))\\
    &= \prod_{k\in K}k^{-1}\cdot \theta(k\cdot x) \prod_{k\in K}k^{-1}\cdot \theta(k\cdot y)=\varphi(x)\varphi(y),
\end{align*}
since $U$ is abelian and $K$ acts by automorphisms on $V$. 

We claim that $N=\ker\varphi$ is $K$-invariant. 
We need to show that $\varphi(l\cdot x)=l\cdot\varphi(x)$ for all $l\in K$ and $x\in V$. 
If $l\in K$ and $x\in V$, then 
\begin{align*}
l^{-1}\cdot\varphi(l\cdot x)&=l^{-1}\cdot\left(\prod_{k\in K}k^{-1}\cdot \theta(k\cdot (l\cdot x))\right)=\prod_{k\in K}(kl)^{-1}\cdot\theta( (kl)\cdot x)=\varphi(x),
\end{align*}
since $kl$ runs over all the elements of $K$ whenever $k$ runs over all the elements of $K$.
In conclusion, $\ker\varphi$ is $K$-invariant. 

It remains to show that $V$ is the direct product of $U$ and $N$. By assumption, $U$ is normal in $V$. 
We first prove that $U\cap N=\{1\}$. If $u\in U$, then $k\cdot u\in U$ for all $k\in K$. This implies that 
$k^{-1}\cdot\theta(k\cdot u)=k^{-1}\cdot (k\cdot u)=u$. Hence $\varphi(u)=u^m$. Since this map is bijective by assumption,  
\[
U\cap N=U\cap\ker\varphi=\{1\}.
\]
We now show that $V\subseteq UN$, as the other inclusion is trivial. Since $N=\ker\varphi$,  
\[
\varphi(V)\subseteq U=\varphi(U)=\varphi(U)\varphi(N)=\varphi(UN) 
\]
and hence $V\subseteq (UN)N=UN$. 
Therefore $V$ is the direct product of $U$ and $N$, as $N$ is normal in $V$.
\end{sol}

\begin{sol}{xca:Maschke_multiplicative2}
    Let $m=|K|$. Since $m$ and $|U|$ are coprime, the map 
    $u\mapsto u^m$ is bijective in $U$. Since $V$ is a vector space over the field 
    $\Z/p$, it follows that $V=U\times W$ for some subgroup $W$ of $V$. Now the claim follows
    from the previous theorem. 
\end{sol}

\begin{sol}{xca:deg2}
  Assume that $\phi$ is not irreducible. There exists a proper non-zero $G$-invariant 
  subspace $W$ of $V$. Thus $\dim W=1$. Let $w\in W\setminus\{0\}$.
  For each $g\in G$, $\phi_g(w)\in W$. Thus $\phi_g(w)=\lambda w$ for some 
  $\lambda$. This means that $w$ is a common eigenvector for all the $\phi_g$.
  Conversely, if $\phi$ admits a common eigenvector $v\in V$, then 
  the subspace generated by $v$ is $G$-invariant.
\end{sol}

\begin{sol}{xca:Z(G)cyclic}
Let $G$ be a finite subgroup of $S^1=\{z\in\Z:z^n=1\}$. We claim that $G$ is cyclic. Let $n=|G|$. 
It is enough to show that 
$G\subseteq\{\exp(2\pi ik/n):0\leq k\leq n-1\}$, 
since $\{\exp(2\pi ik/n):0\leq k\leq n-1\}$ is
cyclic. 
Let $g\in G$. 
Since $g^n=1$, $g$ is a $n$-th root of one, say 
$g=\exp(2\pi i k/n)$ for some $k\in\{1,\dots,n-1\}$. 

Let $\rho\colon G\to\GL(V)$ be a faithful irreducible 
representation. Let $z\in Z(G)$ and $g\in G$. The map 
$T\colon V\to V$, $v\mapsto z\cdot v$ is invariant, since
\[
T(g\cdot v)=z\cdot (g\cdot v)
=(zg)\cdot v=(gz)\cdot v=g\cdot (z\cdot v)=g\cdot T(v)
\]
for all $g\in G$. By Schur's lemma, there exists $\lambda\in\C$ such that $T(v)=\lambda v$
for all $v\in V$. In particular,
$\rho(Z(G))$ is isomorphic to a subgroup of $S^1$. 
Thus $\rho(Z(G))$ is cyclic. Since $\rho$ is faithful, 
$Z(G)$ is cyclic. 
\end{sol}

\begin{sol}{xca:Solomon}
    All we need to do is carefully study the proof
    of Theorem~\ref{thm:Solomon}. Let $n=|G|$. 
    The action of $G$ on itself by conjugation induces a group homomorphism $\rho\colon G\to\GL_n(\C)$ with character 
    $\chi_\rho=\sum_{i=1}^rm_i\chi_i$. In the proof of 
    Solomon's theorem, we saw that 
    each $m_i=\sum_{j=1}^r\chi_i(g_j)$ is a non-negative integer. Thus 
    \[
    \sum_{i=1}^r\sum_{j=1}^r\chi_i(g_j)=\sum_{i=1}^rm_i\in\Z_{\geq1}. 
    \]
    To prove the other inequality, 
    \[
    n=\chi_\rho(1)=\sum_{i=1}^rm_i\chi_i(1)\geq\sum_{i=1}^km_i.
    \]

    Now assume that $\sum_{i=1}^km_i=|G|$. Then $m_i=0$ whenever 
    $\chi_i(1)>1$, that is 
    $\rho$ only have degree-one 
    irreducible components. In particular, $\rho$ is a degree-one
    representation. To see that $G/Z(G)$ is abelian, it is enough to see that $[G,G]\subseteq Z(G)$. Since $\rho$ is then
    a degree-one representation, $\rho(G)\subseteq\C^{\times}$. Thus 
    \[
    \rho([G,G])\subseteq [\rho(G),\rho(G)]=\{1\}.
    \]
    Hence $[G,G]\subseteq\ker\rho=Z(G)$. 
\end{sol}

\begin{sol}{xca:commutators}
    Let $C_1,\dots,C_t$ be the conjugacy classes of $G$. For each
    $i\in\{1,\dots,t\}$, let $g_i$ be a representative of $C_i$. Assume
    that $g_i$ is conjugate to $g$ and 
    $g_j$ is conjugate to $h$. Let $\gamma\in G$.
    Then
    \begin{align*}
        \sum_{z\in G}\chi(zg_iz^{-1}g_j) 
        &= \sum_{z\in G}\chi(\gamma zg_iz^{-1}g_j\gamma^{-1})\\
        &= \sum_{z\in G}\chi(\gamma zg_iz^{-1}\gamma^{-1}\gamma g_j\gamma^{-1})\\
        &=\sum_{y\in G}\chi(yg_iy^{-1}\gamma g_j\gamma^{-1}).
    \end{align*}
    Hence
    \[
    \sum_{z\in G}\chi(zg_iz^{-1}g_j) 
    =\frac{1}{|G|}\sum_{z,\gamma\in G}\chi(zg_iz^{-1}\gamma g_j\gamma^{-1}).
    \]
    Now $z_1g_iz_1^{-1}=z_2g_iz_2^{-1}$ if and only 
    if $z_2^{-1}z_1\in C_G(g_i)$. Thus
    \begin{align*}
        \sum_{z\in G}\chi(zg_iz^{-1}g_j) &= \frac{1}{|G|}|C_G(g_i)||C_G(g_j)|\sum_{\substack{x\in C_i\\y\in C_j}}\chi(xy)\\
        &=\frac{|G|}{|C_i||C_j|}\sum_{\substack{x\in C_i\\y\in C_j}}\chi(xy).
    \end{align*}
    Now 
    \[
    \omega_{\chi}(C_i)\omega_{\chi}(C_j)=\sum_{i=1}^t a_{ijk}\omega_{\chi}(C_k),
    \]
    where 
    \[
    \omega_{\chi}(C_i)=\frac{|C_i|\chi(C_i)}{\chi(1)}
    \]
    and
    $a_{ijk}$ is the number of solutions of the equation
    $xy=z$ with $x\in C_i$, $y\in C_j$ and $z\in C_k$. Therefore
    \begin{align*}
        \frac{\chi(1)}{|G|}\sum_{z\in G}\chi(zg_iz^{-1}g_j)
        &=\frac{\chi(1)}{|C_i||C_j|}\sum_{\substack{x\in C_i\\y\in C_j}}\chi(xy)\\
        &=\frac{\chi(1)}{|C_i||C_j|}\sum_{k=1}^t a_{ijk}\chi(g_k)|C_k|\\
        &=\frac{\chi(1)^2}{|C_i||C_j|}\sum_{k=1}^t a_{ijk}\omega_{\chi}(C_k)\\
        &=\chi(g_i)\chi(g_j).
    \end{align*}
    
    To prove the second formula, 
    set $h=g^{-1}$ in the first formula.
    Then 
    \begin{align*}
        \chi(g)\chi(g^{-1})=\frac{\chi(1)}{|G|}\sum_{z\in G}\chi(zgz^{-1}g^{-1}) &\Longleftrightarrow
        \chi(g)\overline{\chi(g)}=\frac{\chi(1)}{|G|}\sum_{z\in G}\chi([z,g])\\
        & \Longleftrightarrow |\chi(g)|^2=\frac{\chi(1)}{|G|}\sum_{z\in G}\chi([z,g])\\
        & \Longleftrightarrow \frac{|G|}{\chi(1)}|\chi(g)|^2=\sum_{z\in G}\chi([z,g]).
    \end{align*}
\end{sol}


\begin{sol}{xca:least_p}
Let $g_1,\dots,g_m$ be the representatives of non-trivial conjugacy classes. Then $C_G(g_i)$ is non-tivial for all $i$. Since $p$ is the smallest prime dividing the order of $G$, it follows that
$(G:C_G(g_i))\geq p$. Now use the class equation to get
\[
|G|\geq |Z(G)|+pm,
\]
which is equivalent to $m\leq \frac1p (|G|-|Z(G)|)$. Since $G$ is non-abelian, $G/Z(G)$ is not cyclic. Thus $(G:Z(G))\geq p^2$. Now 
\[
\frac{k(G)}{|G|}=\frac{|Z(G)|+m}{|G|}\leq \frac{(p-1)|Z(G)|+|G|}{p|G|}\leq \frac{p^2+p-1}{p^3}.
\]
This bound is reached if and only if $(G:Z(G))=p^2$.  
\end{sol}


\begin{sol}{xca:5/8}
    If $\cp(G)>5/8$, then $|[G,G]|<2$. Thus $[G,G]$ is the trivial group
    and hence $G$ is abelian. 
\end{sol}

\begin{sol}{xca:cp_NS}\
\begin{enumerate}
    \item If $\cp(G)>1/2$, then $|[G,G]|<3$ by Theorem \ref{thm:[GG]}. If $|[G,G]|=1$, 
    then $G$ is abelian and hence $G$ is nilpotent. If $|[G,G]|=2$, then 
    $[G,G]\subseteq Z(G)$. 
    %In fact, a more general fact is true. 
    %If $N$ is a normal subgroup of $G$ and
    %$|N|=2$, then $N\subseteq Z(G)$. Write $N=\{1,x\}$. If $g\in G$, then
    %$gxg^{-1}\in N$. Thus either $gxg^{-1}=1$ or $gxg^{-1}=x$. The first
    %case implies $x=1$, a contradiction. Thus $x\in Z(G)$.
    It follows that 
    $G/Z(G)$ is abelian (and hence nilpotent), so $G$ is nilpotent. 
    \item If $\cp(G)<21/80$, then 
    $|[G,G]|<60$. Thus $[G,G]$ is solvable, as groups of order $<60$ are solvable. 
    Hence $G$ is solvable. 
\end{enumerate}
\end{sol}


\begin{sol}{xca:isoclinism}\
\begin{enumerate}
    \item 
    \item Using that $\sigma$ and 
    $\tau$ are automorphisms and 
    the commutativity of the diagram~\eqref{eq:isoclinism}, 
    we compute 
    \begin{align*}
        (G:Z(G))^2\cp(G) &= \frac{1}{|Z(G)|^2}|\{(x,y)\in G\times G:xy=yx\}|\\
        &=\frac{1}{|Z(G)|^2}|\{(x,y)\in G\times G:[x,y]=1\}|\\
        &=\frac{1}{|Z(G)|^2}|\{(x,y)\in G\times G:c_G(x,y)=1\}|\\
        &=|\{(u,v)\in (G/Z(G))^2:c_G(u,v)=1\}|\\
        &=|\{(u,v)\in (G/Z(G))^2:\tau c_G(u,v)=1\}|\\
        &=|\{(u,v)\in (G/Z(G))^2:c_G(\sigma u,\sigma v)=1\}|\\
        &=|\{(a,b)\in (H/Z(H))^2:c_H(a,b)=1\}|.
    \end{align*}
    It follows that $(G:Z(G))^2\cp(G)=(H:Z(G))^2\cp(H)$. 
\end{enumerate}
\end{sol}

\begin{sol}{xca:constituent_restriction}
    Assume that $\Irr(G)=\{\chi_1,\dots,\chi_k\}$. 
    Let $L$ be the regular representation of $G$. Then 
    \[
    \chi_L(g)=\begin{cases}
    |G| & \text{if $g=1$},\\
    0 & \text{otherwise}.
    \end{cases}
    \]
    Write $\chi_L=\sum_{i=1}^k\chi_i(1)\chi_i$. Since 
    \[
    0\ne \frac{|G|}{|H|}\phi(1)=\langle \Res_H^G\chi_L,\phi\rangle=\sum_{i=1}^k\chi_i(1)\langle\Res_H^G\chi_i,\phi\rangle,
    \]
    there exists $i\in\{1,\dots,k\}$ such that 
    $\langle\Res_H^G\chi_i|_H,\phi\rangle\ne 0$. 
\end{sol}

\begin{sol}{xca:decomposing_restriction}
Note that   
\[
\sum_{i=1}^ld_i^2=\langle\Res_H^G\chi,\Res_H^G\chi\rangle=\frac{1}{|H|}\sum_{h\in H}\chi(h)\overline{\chi(h)}.
\]
Since $\chi$ is irreducible, 
\begin{align*}
1=\langle\chi,\chi\rangle&=\frac{1}{|G|}\sum_{g\in G}\chi(g)\overline{\chi(g)}\\
&=\frac{1}{|G|}\sum_{h\in H}\chi(h)\overline{\chi(h)}
+\frac{1}{|G|}\sum_{g\in G\setminus H}\chi(g)\overline{\chi(g)}\\
&=\frac{|H|}{|G|}\sum_{i=1}^l d_i^2+\frac{1}{|G|}\sum_{g\in G\setminus H}\chi(g)\overline{\chi(g)}.
\end{align*}
Since $\sum_{g\in G\setminus H}\chi(g)\overline{\chi(g)}\geq0$, we conclude that $\sum_{i=1}^ld_i^2\leq(G:H)$. Moreover, the equality
holds if and only if $\sum_{g\in G\setminus H}\chi(g)\overline{\chi(g)}=0$, 
that is, if and only if $\chi(g)=0$ for all $g\in G\setminus H$.
\end{sol}


\begin{sol}{xca:inducting_trivial}
Let $g\in G$. Then 
\[
(\Ind_H^G\Tchar_H)(g)=\sum_{x\in G}\Tchar_H^0(x^{-1}gx)
=\begin{cases}
    |G| & \text{if $g=1$,}\\
    0 & \text{otherwise},
\end{cases}
\]
since $x^{-1gx}=1$ if and only if $g=1$. 
This $\Ind_H^G\Tchar_H$ is equal to the character
of the regular representation of $G$ (see Theorem~\ref{thm:regular}). 
\end{sol}


\begin{sol}{xca:induction_G/H}
    The action of $G$ on $G/H$ by left multiplication
    yields a group homomorphism 
    $\rho\colon G\to\Sym_{G/H}$, $g\mapsto\rho_g$, where
    $\rho_g(xH)=gxH$. Now
    \[
    xH\in\Fix(g)\Longleftrightarrow
    gxH=xH\Longleftrightarrow
    x^{-1}gx\in H\Longleftrightarrow
    g\in xHx^{-1}.
    \]
    Thus 
    \[
    |\Fix(g)|=\frac{1}{|H|}|\{x\in G:x^{-1}gx\in H\}|.
    \]
    Since
    \[
    \Tchar_H^0(x^{-1}gx)=\begin{cases}
        1 & \text{if $x^{-1}gx\in H$,}\\
        0 & \text{otherwise},
    \end{cases}
    \]
    it follows that 
    $(\Ind_H^G\Tchar_H)(g)=|\Fix(g)|$, the character
    of the representation of $G$ obtained from $\rho$. 
\end{sol}

\begin{sol}{xca:induced_representations}
Write $\psi=\Ind_H^G\rho$. For $x,y\in G$ and 
$i,j\in\{1,\dots,m\}$, 
\begin{align*}
    (\psi_x\psi_y)_{ij} &= \sum_{k=1}^m 
    (\psi_x)_{ik}(\psi_y)_{kj}
    =\sum_{k=1}^m\rho_{t_i^{-1}xt_k}^0\rho_{t_k^{-1}yt_j}^0.
\end{align*}
If we want each $\rho_{t_k^{-1}yt_j}^0\ne0$, we need
$t_k^{-1}yt_j\in H$ for all $k$. This means $yt_j\in t_kH$ for all $k$. But there exists a unique $l\in\{1,\dots,m\}$ such that 
$yt_jH=t_lH$. Thus $k=l$ and then 
\[
(\psi_x\psi_y)_{ij} = \rho_{t_i^{-1}xt_l}^0\rho_{t_l^{-1}yt_j}=\rho_{t_i^{-1}xt_l}\rho_{t_l^{-1}yt_j}=\rho_{t_i^{-1}xyt_j}
\]
whenever $t_i^{-1}xt_l\in H$ (or equivalently 
$xt_l\in t_iH$). Note that 
$t_i^{-1}xyt_j\in H$, since 
$yt_j\in t_lH$ implies 
\[
xyt_j\in xt_lH=t_iH.
\]
It follows that 
\[
(\psi_x\psi_y)_{ij}=\rho_{t_i^{-1}xyt_j}^0=(\psi_{xy})_{ij}.
\]
\end{sol}


\begin{sol}{xca:centralizer}
    We use the second orthogonality relation and Theorem~\ref{thm:correspondence} 
    to compute
    \begin{align*}
        |C_{G/N}(gN)| &=\sum_{\chi\in\Irr(G/N)}|\chi(gN)|^2
        =\sum_{\substack{\eta\in\Irr(G)\\N\subseteq\ker\eta}} |\eta(g)|^2
        \leq\sum_{\eta\in\Irr(G)}|\eta(g)|^2=|C_G(g)|.
    \end{align*}
\end{sol}


\begin{sol}{xca:Frobenius_pq}
    Assume that $G$ is not abelian. By Sylow's theorems, 
    $q$ divides $p-1$ and there exists 
    a unique Sylow $p$-subgroup $P$ of $G$. Let $a,b\in G$ be such that 
    $P=\langle a\rangle\simeq\Z/p$ and $G/P=\langle bP\rangle\simeq\Z/q$. By Lagrange's theorem, 
    $G=\langle a,b\rangle$. We compute the order of $b^q$. Since 
    $G$ is not cyclic (because it is not abelian) and $b^q\in P$, 
    we conclude that $|b^q|=1$. 
    Since $P$ is normal in $G$, 
    $bab^{-1}\in P$ and hence $bab^{-1}=a^z$ for some $z\in\Z$. Therefore
    $b^qab^{-q}=a^{z^q}$. This implies that 
    $z^q\equiv1\bmod p$. The order of $z$ in $(\Z/p)^{\times}$ divides 
    $q$ and hence it is equal to $q$ (otherwise, $z=1$ and thus $bab^{-1}=a$, which implies
    that $G$ is abelian). In conclusion, 
    $G\simeq F_{p,q}$. 
\end{sol}

\begin{sol}{xca:malnormal}
    We first prove that $1)\implies2)$. Let $x\not\in H$ and 
    $h\in H$ be such that $h\cdot xH=xH$. Then 
    $h\in xHx^{-1}\cap H=\{1\}$. 

    We now prove $2)\implies3)$. Let $x,y\in G$ be such that 
    $xH\ne yH$, and $g\in G$ be an element 
    that fixes both $xH$ and $yH$, that is 
    $g\cdot xH=xH$ and $g\cdot yH=yH$. Let  
    $h=x^{-1}gx\in H$. Note that 
    $x^{-1}yH\ne H$. Since 
    \[
    h\cdot x^{-1}yH=(x^{-1}gx)\cdot x^{-1}yH=x^{-1}\cdot yH=x^{-1}yH, 
    \]
    it follows that $h=1$, which implies $g=1$. 

    Finally, we prove that $3)\implies1)$. Let $x\not\in H$ and $h\in H\cap xHx^{-1}$. Then 
    $hx=xh_1$ for some $h_1\in H$. Hence
    \[
    h\cdot xH=hxH=xh_1H=xH,\quad 
    h\cdot H=H.
    \]
    Since $h$ has at least two fixed points
    in $G/H$, $h=1$. 
\end{sol}

\begin{sol}{xca:malnormal_no2torsion}
    Let $H\ne\{1\}$ be a malnormal subgroup of $G$ and $N=\langle n\rangle\simeq\Z$ be a
    normal subgroup of $G$. 

    Assume first that $N\cap H\ne\{1\}$. Let $k>0$ be minimal such that $n^k\in H$. If $k=1$, then
    $H=N$ is normal in $G$. Since $H$ is malnormal by assumption, $H=G$. 
    
    Assume now that $N\cap H=\{1\}$. Note that
    for every $h\in H\setminus\{1\}$, 
    $hnh^{-1}\in\{n,n^{-1}\}$. 
    If there exists 
    $h\in H\setminus\{1\}$ such that $hnh^{-1}=n$, then
    $1\ne h=nhn^{-1}\in H\cap nHn^{-1}$. Thus $H$ is not malnormal in $G$, a contradiction. 

    Since $G$ has no 2-torsion, there exist 
    $h_1,h_2\in H\setminus\{1\}$ with $h_1\ne h_2$ and 
    $h_1h_2\ne 1$. If $h_jnh_j=n$ for some $j\in\{1,2\}$, 
    then the previous argument shows that 
    $H$ is not malnormal in $G$. Thus we may assume that  $h_1nh_2=n^{-1}$ and 
    $h_2nh_2=n^{-1}$. Then 
    \[
    (h_2h_1)n(h_2h_1)^{-1}=h_2n^{-1}h_2^{-1}
    =(h_2nh_2^{-1})^{-1}=n.
    \]
    Hence, again by the previous argument, 
    $H$ is not malnormal in $G$, a contradiction.  
\end{sol}

\begin{sol}{xca:Ind_chi}
    Since $|H|=n$, $(G:H)=2$. Let 
    $t_1=1$ and $t_2=s$. Then $\{t_1,t_2\}$ is a transversal of $H$ in $G$. We compute
    \begin{align*}
        t_1^{-1}rt_1=r, && t_1^{-1}st_1=s, && t_1^{-1}rt_2=sr^{-1},&& t_1^{-1}st_2=1, \\
        t_2^{-1}rt_1=sr,&& t_2^{-1}st_1=1,&&
        t_2^{-1}rt_2=r^{-1},&&t_2^{-1}st_2=s.
    \end{align*}
    Thus 
    \[
    (\rho_m)(r)=\begin{pmatrix}
    \omega_m & 0\\
    0&\omega_m^{-1}
    \end{pmatrix},\quad 
     (\rho_m)(s)=\begin{pmatrix}
     0&1\\
     1&0
    \end{pmatrix}
    \]
\end{sol}