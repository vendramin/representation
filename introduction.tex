\thispagestyle{plain}
\section*{Introduction}

The notes correspond to the master  
course \emph{Representation theory of algebras} of the 
Vrije Universiteit Brussel, 
Faculty of Sciences, 
Department of Mathematics and Data Sciences. The course
is divided into twelve two-hour lectures. 

Most of the material is based on standard 
results of the representation theory of finite groups. 
Basic texts on representation theory are \cite{MR1369573} 
and \cite{MR2270898}. 

The notes include Magma code, which we use to verify examples and offer alternative solutions to certain exercises. Magma \cite{zbMATH01077111} is a powerful software tool designed for working with algebraic structures. There is a free \href{https://magma.maths.usyd.edu.au/calc/}{online} version of Magma available.


Thanks go to Luca Descheemaeker, Wannes Malfait, Silvia Properzi, Lukas Simons.  



This version 
was compiled on \today~at~\currenttime.


 \begin{figure}[b]
     \includegraphics[scale=0.2]{VUB.jpg}
 \end{figure}