\section{Lecture: Week 7}

\subsection{The Brauer--Fowler theorem}

\index{Symmetric}
\index{Antisymmetric}
Let $\rho\colon G\to\GL(V)$ 
be a representation with character $\chi$. The $\C[G]$-module $V\otimes V$ 
has character $\chi^2$. Let 
$\{v_1,\dots,v_n\}$ be a basis of $V$ and 
\[
T\colon V\otimes V\to V\otimes V,\quad
v_i\otimes v_j\mapsto v_j\otimes v_i.
\]
It is an exercise to check that 
\[
T(v\otimes w)=w\otimes v
\]
for all 
$v,w\in V$. (Thus 
$T$ does not depend on the basis $\{v_1,\dots,v_n\}$.) Note that
$T$ is a homomorphism of $\C[G]$-modules, as
\[
T(g\cdot (v\otimes w))=T((g\cdot v)\otimes (g\cdot w))=(g\cdot w)\otimes (g\cdot v)=g\cdot T(v\otimes w)
\]
for all $g\in G$ y $v,w\in V$. 
In particular, the \emph{symmetric part} 
\begin{gather*}
S(V\otimes V)=\{x\in V\otimes V:T(x)=x\}
\shortintertext{and the \emph{antisymmetric} part}
A(V\otimes V)=\{x\in V\otimes V:T(x)=-x\}
\end{gather*}
of $V\otimes V$ are both  
$\C[G]$-submodules of $V\otimes V$. 
The terminology is motivated by the following fact:
\[
V\otimes V=S(V\otimes V)\oplus A(V\otimes V).
\]
In fact, 
$S(V\otimes V)\cap A(V\otimes V)=\{0\}$, as   
$x\in S(V\otimes V)\cap A(V\otimes V)$ implies
$x=T(x)$ and $x=-T(x)$. Hence $x=0$. Moreover, 
$V\otimes V=S(V\otimes V)+ A(V\otimes V)$, as every $x\in V\otimes V$ can be written 
as 
\[
x=\frac12(x+T(x))+\frac12(x-T(x))
\]
with $\frac12(x+T(x))\in S(V\otimes V)$ and $\frac12(x-T(x))\in A(V\otimes V)$. 

We claim that 
\[
\{v_i\otimes v_j+v_j\otimes v_i:1\leq i\leq j\leq n\}
\]
is
a basis of $S(V\otimes V)$, 
and that  
\[
\{v_i\otimes v_j-v_j\otimes v_i:1\leq i<j\leq n\}
\]
is a basis of $A(V\otimes V)$. Since both sets are linearly independent, 
\[
\dim S(V\otimes V)\geq n(n+1)/2\text{ and }
\dim A(V\otimes V)\geq n(n-1)/2.
\]
Moreover, 
\[
n^2=\dim (V\otimes V)=\dim S(V\otimes V)+\dim A(V\otimes V),
\]
so it follows that
$\dim S(V\otimes V)=n(n+1)/2$ and $\dim A(V\otimes V)=n(n-1)/2$. 

\begin{proposition}
\label{pro:SandA}
    Let $G$ be a finite group and
    $V$ be a finite-dimensional 
    $\C[G]$-module with character $\chi$. If $S(V\otimes V)$ 
    has character $\chi_S$ and $A(V\otimes V)$ has character
    $\chi_A$, then 
    \begin{align*}
        &\chi_S(g)=\frac12(\chi^2(g)+\chi(g^2)) && \text{and} &&
        \chi_A(g)=\frac12(\chi^2(g)-\chi(g^2)).
    \end{align*}
\end{proposition}

\begin{proof}
    Let $g\in G$ and $\rho\colon G\to\GL(V)$ be the representation
    associated with $V$, that is $\rho(g)(v)=\rho_g(v)=g\cdot v$. 
    Since $\rho_g$ is diagonalizable, let $\{e_1,\dots,e_n\}$ 
    be a basis of eigenvectors of $\rho_g$, say
    $g\cdot e_i=\lambda_ie_i$ with $\lambda_i\in\C$ for all $i\in\{1,\dots,n\}$. In particular, $\chi(g)=\sum_{i=1}^n\lambda_i$. 
    
    Since $\{e_i\otimes e_j-e_j\otimes e_i:1\leq i<j\leq n\}$ is a basis of
    $A(V\otimes V)$ and 
    \[
    g\cdot (e_i\otimes e_j-e_j\otimes e_i)=\lambda_i\lambda_j(e_i\otimes e_j-e_j\otimes e_i),
    \]
    it follows that
    \[
    \chi_A(g)=\sum_{1\leq i<j\leq n}\lambda_i\lambda_j.
    \]
    On the other hand,
    $g^2\cdot e_i=\lambda_i^2e_i$ for all $i$,
    $\chi(g^2)=\sum_{i=1}^n\lambda_i^2$. Thus 
    \[
    \chi^2(g)=\chi(g)^2=\sum_{i=1}^n\sum_{j=1}^n\lambda_i\lambda_j=2\sum_{1\leq i<j\leq n}\lambda_i\lambda_j+\sum_{i=1}^n\lambda_i^2=2\chi_A(g)+\chi(g^2).
    \]
    Since $V\otimes V=S(V\otimes V)\oplus A(V\otimes V)$, it follows that  
    $\chi^2(g)=\chi_S(g)+\chi_A(g)$, that is 
    $\chi_S(g)=\frac12(\chi^2(g)+\chi(g^2))$.
\end{proof}

\index{Involution}
An \emph{involution} of a group is an element $x\ne 1$ such that $x^2=1$. 
It is possible to use the character table to count the number
of involutions.

\begin{proposition}
\label{pro:involutions}
    If $G$ is a finite group with $t$ involutions, then
    \[
        1+t=\sum_{\chi\in\Irr(G)}\langle\chi_S-\chi_A,\chi_1\rangle\chi(1),
    \]
    where $\chi_1$ is 
    the trivial character of $G$.
\end{proposition}

\begin{proof}
    Assume that $\Irr(G)=\{\chi_1,\dots,\chi_k\}$.  
    For $x\in G$ let 
    \[
    \theta(x)=|\{y\in G:y^2=x\}|.
    \]
    Since $\theta$ is a class function, 
    $\theta$ is a linear combination of the $\chi_j$'s, say 
    \[
    \theta=\sum_{\chi\in\Irr(G)}\langle\theta,\chi\rangle\chi.
    \]
    For every $\chi\in\Irr(G)$ we compute: 
    \begin{align*}
        \langle\chi_S-\chi_A,\chi_1\rangle 
        &=\frac{1}{|G|}\sum_{g\in G}\chi(g^2)\\
        &=\frac{1}{|G|}\sum_{x\in G}\sum_{\substack{g\in G\\g^2=x}}\chi(g^2)
        =\frac{1}{|G|}\sum_{x\in G}\theta(x)\chi(x)=\langle\theta,\chi\rangle.
    \end{align*}
    Thus $\theta=\sum_{\chi\in\Irr(G)}\langle\chi_S-\chi_A,\chi_1\rangle\chi$. Now
    the claim follows after evaluating this expression in 
    $x=1$. 
\end{proof}

\begin{example}
    We know that $\Sym_3$ has three involutions, namely $(12)$, $(23)$ and $(13)$. Thus $t=3$. 
    Let us use Proposition~\ref{pro:involutions} to verify this. 
    We have already computed the  character table
    of $\Sym_3$: 
\bigskip 
    \begin{center}
		\begin{tabular}{|c|ccc|}
			\hline
			& $1$ & $(12)$ & $(123)$ \tabularnewline
			\hline 
			$\chi_{1}$ & $1$ & $1$ & $1$\tabularnewline
			$\chi_{2}$ & $1$ & $-1$ & $1$ \tabularnewline
			$\chi_{3}$ & $2$ & $0$ & $-1$ \tabularnewline
			\hline
		\end{tabular}
	\end{center}
\bigskip 
A direct calculation shows that 
\[
(\chi_1)_S=(\chi_2)_S=\chi_1\quad\text{and}
\quad (\chi_1)_A=(\chi_2)_A=0.
\]
Moreover, the values of $(\chi_3)_S$ and $(\chi_3)_A$ 
are
given by the following table: 
\bigskip 
    \begin{center}
		\begin{tabular}{|c|ccc|}
			\hline
			& $1$ & $(12)$ & $(123)$ \tabularnewline
			\hline 
			$(\chi_{3})_S$ & $3$ & $1$ & $0$ \tabularnewline
			$(\chi_{3})_A$ & $1$ & $-1$ & $1$ \tabularnewline
			\hline
		\end{tabular}
	\end{center}
\bigskip 
Let $t$ be the number of elements 
of order two of $\Sym_3$. 
Since 
\begin{align*}
\langle\chi_S-\chi_A,\chi_1\rangle=1
\end{align*}
for all $\chi\in\{\chi_1,\chi_2\}$ and
\[
\langle (\chi_3)_S-(\chi_3)_A,\chi_1\rangle
=\frac{1}{6}(12+6-2)=\frac16(2+6-2)=1,
\]
Proposition~\ref{pro:involutions} yields 
\begin{align*} 
1+t&=\langle (\chi_1)_S-(\chi_1)_A,\chi_1\rangle\chi_1(1)
+\langle (\chi_2)_S-(\chi_2)_A,\chi_1\rangle\chi_2(1)
+\langle (\chi_3)_S-(\chi_3)_A,\chi_1\rangle\chi_3(1)\\
&=1+1+2.
\end{align*}
Thus $t=3$. 
\end{example}

Before proving the Brauer--Fowler theorem, we
need a lemma. 

\begin{lemma}
    Let $G$ be a finite group with $k$ conjugacy classes. 
    If $t$ is the number of involutions of $G$, then
    $t^2\leq (k-1)(|G|-1)$. 
\end{lemma}

\begin{proof}
    Assume that $\Irr(G)=\{\chi_1,\dots,\chi_k\}$, where $\chi_1$ is the
    trivial character of $G$. 
    If $\chi\in\Irr(G)$, then 
    \[
        \langle\chi^2,\chi_1\rangle=\frac{1}{|G|}\sum_{g\in G}\chi(g)\chi(g)=\langle\chi,\overline{\chi}\rangle=\begin{cases}
        1 & \text{if $\chi=\overline{\chi}$},\\
        0 & \text{otherwise}.
        \end{cases}
    \]
    Since $\chi^2=\chi_S+\chi_A$, if $\langle\chi^2,\chi_1\rangle=1$, then
    the trivial character is an irreducible component either of $\chi_S$ or $\chi_A$, but not both. 
    Thus
    \[
    \langle\chi_S-\chi_A,\chi_1\rangle\in\{-1,1,0\}.
    \]
    
    We claim that 
    $t\leq\sum_{i=2}^k\chi_i(1)$. In fact, since 
    $|\langle\chi_S-\chi_A,\chi_1\rangle|\leq 1$, 
    \begin{align*}
        1+t=\theta(1)
        &=\left|\sum_{\chi\in\Irr(G)}\langle\chi_S-\chi_A,\chi_1\rangle\chi(1)\right|\\
        &\leq\sum_{\chi\in\Irr(G)}|\langle\chi_S-\chi_A,\chi_1\rangle|\chi(1)
        \leq\sum_{\chi\in\Irr(G)}\chi(1).
    \end{align*}
    It follows that $t\leq\sum_{i=2}^k\chi_i(1)$. 
    By the Cauchy--Schwarz inequality, 
    \[
        t^2\leq\left(\sum_{i=2}^k\chi_i(1)\right)^2
        \leq(k-1)\sum_{i=2}^k\chi_i(1)^2=(k-1)(|G|-1).\qedhere
    \]
\end{proof}

Now we prove the Brauer--Fowler theorem. 

\begin{theorem}[Brauer--Fowler]
    \index{Brauer--Fowler theorem}
    Let $G$ be a finite simple group and $x$ be an involution of $G$. If $|C_G(x)|=n$, then $|G|\leq (n^2)!$	
\end{theorem}

\begin{proof}
    If $G$ is abelian, the claim is trivial. Let $G$ be a finite non-abelian simple group.
    We first assume the existence of a proper subgroup $H$ of $G$ 
    such that 
    \[
    (G:H)\leq n^2.
    \]
    Let $G$ act on $G/H$ 
    by left multiplication, and let 
    $\rho\colon G\to\Sym_{n^2}$ be the corresponding
    group homomorphism. Since $G$ is simple, either 
    $\ker\rho=\{1\}$ or $\ker\rho=G$. If $\ker\rho=G$, then
    $\rho(g)(yH)=yH$ for all $g\in G$ and $y\in G$. 
    Hence $H=G$, a contradiction. Therefore $\rho$ is injective
    and hence $G$ is isomorphic to a subgroup of $\Sym_{n^2}$. 
    In particular, $|G|$ divides $(n^2)!$. 

    Let $m=(|G|-1)/t$, where $t$ is the number of involutions of $G$. 
    Since $|C_G(x)|=n$, the group $G$ has at least $|G|/n$ involutions (because
    the conjugacy class of $x$ has size $|G|/n$ and all its elements are involutions), 
    that is $t\geq |G|/n$. Hence 
    \[
    m=(|G|-1)/t<n.
    \]
    It is enough to show that
    $G$ contains a subgroup of index $\leq m^2$. 

    Let $C_1,\dots,C_k$ be the conjugacy classes of $G$, where $C_1=\{1\}$. 
    Since $G$ is simple and non-abelian, $|C_i|>1$ 
    for all $i\in\{2,\dots,k\}$. By the previous lemma, 
    \[
    t^2\leq(k-1)(|G|-1)\implies |G|-1=\frac{mt^2}{t}\leq\frac{(k-1)(|G|-1)^2}{t^2}=(k-1)m^2.
    \]
    If $|C_i|>m^2$ for all $i\in\{2,\dots,k\}$, then
    \[
    |G|-1=\sum_{i=2}^k|C_i|>(k-1)m^2,
    \]
    a contradiction. Thus there exists a non-trivial conjugacy class
    $C$ of $G$ such that $|C|\leq m^2$. If $g\in C$, then
    $C_G(g)$ is a proper subgroup of $G$ of index $|C|\leq m^2$.
\end{proof}

The bound of the Brauer--Fowler theorem is not essential.
What matters is the following consequence:

\begin{corollary}
    Let $n\geq 1$ be an integer. There are at most finitely many 
    finite simple groups with an involution with a centralizer of order $n$.
\end{corollary}

As an exercise, a simple application: 

\begin{exercise}
    If $G$ is a finite simple group and $x$ is an involution with
    centralizer of order two, then  
    $G\simeq\Z/2$. 
\end{exercise}

\subsection{The character table of $\Sym_5$}
\index{Character table!of $\Sym_5$}
Let $G=\Sym_5$. The conjugacy classes 
of $G$ are given in the following table:

\bigskip 
\begin{center}
    \begin{tabular}{c|ccccccc}
        Representative & $\id$ & $(12)$ & $(123)$ & $(12)(34)$ & $(1234)$ & $(123)(45)$  & $(12345)$ \\
        \hline 
        Size & $1$ & $10$ & $20$ & $15$ & $30$ & $20$ & $24$ \\
    \end{tabular}
\end{center}
\bigskip 

Thus there are seven irreducible characters. The trivial character $\chi_1$ and the sign $\chi_2$ are degree-one (hence irreducible) 
characters. 

\bigskip 
\begin{center}
    \begin{tabular}{|c|ccccccc|}
        \hline 
        & $\id$ & $(12)$ & $(123)$ & $(12)(34)$ & $(1234)$ & $(123)(45)$  & $(12345)$ \\
        \hline 
        $\chi_1$ & $1$ & $1$ & $1$ & $1$ & $1$ & $1$ & $1$ \\
        $\sgn$ & $1$ & $-1$ & $1$ & $1$ & $-1$ & $-1$ & $1$ \\
        \hline 
    \end{tabular}
\end{center}
\bigskip 

Since 
$[G,G]=\Alt_5$ and $|G/[G,G]|=2$, it follows
from Exercise~\ref{xca:degree-one} that $\chi_1$ and $\sgn$ 
are the only degree-one characters. 

Since $G$ acts 2-transitively on $\{1,\dots,5\}$, Proposition~\ref{pro:2transitive} implies that 
$\varsigma(g)=|\Fix(g)|-1$ is an irreducible character. 
A direct
calculation yields the values of $\varsigma$: 
\bigskip 
\begin{center}
    \begin{tabular}{|c|ccccccc|}
        \hline 
        & $\id$ & $(12)$ & $(123)$ & $(12)(34)$ & $(1234)$ & $(123)(45)$  & $(12345)$ \\
        \hline 
        $\varsigma$ & $4$ & $2$ & $1$ & $0$ & $0$ & $-1$ & $-1$ \\
        \hline 
    \end{tabular}
\end{center}
\bigskip 

The values of the product 
$\sgn\varsigma$ are easily computed: 
\bigskip 
\begin{center}
    \begin{tabular}{|c|ccccccc|}
        \hline 
        & $\id$ & $(12)$ & $(123)$ & $(12)(34)$ & $(1234)$ & $(123)(45)$  & $(12345)$ \\
        \hline 
        $\sgn\varsigma$ & $4$ & $-2$ & $1$ & $0$ & $0$ & $1$ & $-1$ \\
        \hline 
    \end{tabular}
\end{center}
\bigskip 

Since 
\begin{align*}
\langle\sgn\varsigma,\sgn\varsigma\rangle&=
\frac{1}{120}(4^2+10(-2)^2+20+15\cdot 0+30\cdot 0+20+24)\\
&=\frac{1}{120}(16+40+20+20+24)=1,
\end{align*}
it follows that $\sgn\varsigma\in\Irr(G)$. 

We now consider the characters 
\[
\psi(g)=\frac12(\varsigma^2(g)+\varsigma(g^2))\quad\text{and}\quad  
\eta(g)=\frac12(\varsigma^2(g)-\varsigma(g^2)),
\]
where $\varsigma^2(g)=\varsigma(g)\varsigma(g)=\varsigma(g)^2$ (see Proposition~\ref{pro:SandA}). 
A straightforward 
calculation shows that 
\bigskip 
\begin{center}
    \begin{tabular}{|c|ccccccc|}
        \hline 
        & $\id$ & $(12)$ & $(123)$ & $(12)(34)$ & $(1234)$ & $(123)(45)$  & $(12345)$ \\
        \hline 
        $\psi$ & $10$ & $4$ & $1$ & $2$ & $0$ & $1$ & $0$ \\
        $\eta$ & $6$ & $0$ & $0$ & $-2$ & $0$ & $0$ & $1$ \\
        \hline 
    \end{tabular}
\end{center}
\bigskip 

Since 
\[
\langle\eta,\eta\rangle
=\frac{1}{120}(6^2+15(-2)^2+24)=1,
\]
it follows that $\eta\in\Irr(G)$. On the other hand,
\[
\langle\psi,\psi\rangle
=\frac{1}{120}(10^2+10\cdot16+20+15\cdot 4+20)=3. 
\]
Thus $\psi$ is the sum of three irreducible characters (see Exercise~\ref{xca:n_irreducible}). Since 
\begin{align*}
\langle\psi,\chi_1\rangle&=\frac{1}{120}(10+10\cdot 4+20+15\cdot 2+20)=1,\\
\langle\psi,\varsigma\rangle&=\frac{1}{120}(10\cdot 4+10\cdot 4\cdot 2+20-20)=1,
\end{align*}
it follows that 
$\psi=\chi_1+\varsigma+\chi$ for some $\chi\in\Irr(G)$. Thus
we can compute $\chi$:
\bigskip 
\begin{center}
    \begin{tabular}{|c|ccccccc|}
        \hline 
        & $\id$ & $(12)$ & $(123)$ & $(12)(34)$ & $(1234)$ & $(123)(45)$  & $(12345)$ \\
        \hline 
        $\chi$ & $5$ & $1$ & $-1$ & $1$ & $-1$ & $1$ & $0$ \\
        \hline 
    \end{tabular}
\end{center}
\bigskip 
We are missing one irreducible character. Let $n$ be 
the degree of this character. Since 
$120=1+1+16+16+36+25+n^2$, it follows that 
$n=5$. Since we need a degree-five
irreducible character, we can try with
$\xi=\sgn\chi$:
\bigskip 
\begin{center}
    \begin{tabular}{|c|ccccccc|}
        \hline 
        & $\id$ & $(12)$ & $(123)$ & $(12)(34)$ & $(1234)$ & $(123)(45)$  & $(12345)$ \\
        \hline 
        $\xi$ & $5$ & $-1$ & $-1$ & $1$ & $1$ & $-1$ & $0$ \\
        \hline 
    \end{tabular}
\end{center}
\bigskip 

Since 
\[
\langle\xi,\xi\rangle=\frac{1}{120}(25+10(-1)^2+20(-1)^2+15+30+20(-1)^2)
=1,
\]
it follows that $\xi\in\Irr(G)$. We have found the character table of $G$. 

\begin{table}[h]
    \caption{The character table of $\Sym_5$.}
    \begin{tabular}{|c|ccccccc|}
        \hline 
        & $1$ & $10$ & $20$ & $15$ & $30$ & $20$ & $24$ \\
        & $\id$ & $(12)$ & $(123)$ & $(12)(34)$ & $(1234)$ & $(123)(45)$  & $(12345)$ \\
        \hline 
        $\chi_1$ & $1$ & $1$ & $1$ & $1$ & $1$ & $1$ & $1$ \\
        $\sgn$ & $1$ & $-1$ & $1$ & $1$ & $-1$ & $-1$ & $1$ \\
        $\varsigma$ & $4$ & $2$ & $1$ & $0$ & $0$ & $-1$ & $-1$ \\
        $\sgn\varsigma$ & $4$ & $-2$ & $1$ & $0$ & $0$ & $1$ & $-1$ \\
        $\eta$ & $6$ & $0$ & $0$ & $-2$ & $0$ & $0$ & $1$ \\
        $\chi$ & $5$ & $1$ & $-1$ & $1$ & $-1$ & $1$ & $0$ \\
        $\xi$ & $5$ & $-1$ & $-1$ & $1$ & $1$ & $-1$ & $0$ \\
        \hline 
    \end{tabular}
    \end{table}

% g=id,(12),(123),(12)(34),(1234),(123)(45),(12345)
% chi_3^2(g) : 16,4,1,0,0,1,1
% chi_3(g^2) : 4,4,1,4,0,1,-1

\begin{optional}
\subsection{An elementary proof of the Brauer--Fowler theorem}

We need to find a subgroup of index $\leq 2n^2$. 
Let $X$ be the conjugacy class of $x$. For $g\in G$ let
\[
J(g)=\{z\in X:zgz^{-1}=g^{-1}\}.
\]
We claim that $|J(g)|\leq|C_G(g)|$. The map $J(g)\to C_G(g)$, $z\mapsto gz$, 
is well-defined,~as 
\[
(gz)g(gz)^{-1}=g(zgz^{-1})g^{-1}=g^{-1}\in C_G(g).
\]
It is injective, as $gz=gz_1$ implies $z=z_1$.

Let $J=\{(g,z)\in G\times X:zgz^{-1}=g^{-1}\}$.  
Since $X\times X\to J$, $(y,z)\mapsto (yz,z)$, 
is well-defined (since $z(yz)z^{-1}=zy=(yz)^{-1}$) and
it is trivially injective, 
\[
|X|^2\leq |J|=\sum_{(g,z)\in J}1\leq\sum_{g\in G}|J(g)|
\leq\sum_{g\in G}|C_G(g)|=k|G|,
\]
where $k$ is the number of conjugacy classes of $G$, 
as $(g,z)\in J$ if and only if $z\in J(g)$. Thus $|G|\leq kn^2$, as
\[
\left(\frac{|G|}{|C_G(x)|}\right)^2=|X|^2=\frac{|G|^2}{n^2}\leq k|G|.
\]

\begin{claim}
    There exists a non-trivial conjugacy class with $\leq 2n^2$ elements.
\end{claim}

Assume that the claim is not true. Let
$C_1,\dots,C_k$ be the conjugacy classes of $G$, where 
$C_1=\{1\}$ and $|C_i|>2n^2$ for all $i\in\{2,\dots,k\}$. Then
\[
|G|=1+\sum_{i=2}^k|C_i|>1+\sum_{i=2}^k2n^2=1+(k-1)2n^2\geq |G|,
\]
a contradiction. 

\begin{claim}
    There exists a subgroup $H$ of $G$ such that
    $(G:H)\leq 2n^2$.
\end{claim}

Let $C$ be a conjugacy class of $G$ such that
$|C|\leq 2n^2$. Let $g\in C$.
Then $H=C_G(g)$ is a subgroup of $G$ such that
$(G:H)\leq 2n^2$.
\end{optional}


\subsection{Restriction of characters}

\begin{definition}
    Let $G$ be a finite group and $f\colon G\to\C$ be
    a map. For a subgroup $H$ of $G$, the \emph{restriction}
    of $f$ to $H$ is the map 
    $\Res_H^G=f|_H\colon H\to\C$, $h\mapsto f(h)$. 
\end{definition}

\begin{exercise}
\label{xca:restriction}
    Let $G$ be a finite group. Prove that
    the map 
    \[
    \Res_H^G\colon\cf(G)\to\cf(H),\quad  f\mapsto\Res_H^G f,
    \]
    is a well-defined linear map. 
\end{exercise}

One important property is the following: 

\begin{exercise}
\label{xca:Res}
Let $G$ be a finite group, $H$ a subgroup of $G$ and $\chi \in \Char(G)$. Prove that $\Res_H^G\chi \in \Char(H)$.
\end{exercise}

The restriction of an irreducible representations
does not need to be irreducible: 

\begin{example}
    Let $G=\D_4=\langle r,s:r^4=s^2=1,\,srs=r^{-1}\rangle$ the dihedral group
    of eight elements. Let $V$ be a complex vector space
    with basis $\{v_1,v_2\}$. Then $V$ is a $\C[G]$-module with 
    \[
    r\cdot v_1=v_2,\quad
    r\cdot v_2=-v_1,\quad
    s\cdot v_1=v_1,\quad
    s\cdot v_2=-v_2.
    \]
    The character of $V$ is 
    \[
    \chi(g)=\begin{cases}
    2 & \text{if $g=1$},\\
    -2 & \text{if $g=r^2$},\\
    0 & \text{otherwise}.
    \end{cases}
    \]
    Since $\langle\chi,\chi\rangle=1$, the character $\chi$ 
    is irreducible. Thus $V$ is simple as a $\C[G]$-module. 
    
    Let  
    $H=\langle r^2,s\rangle=\{1,r^2,s,r^2s\}$. Then  $\Res_H^GV$ is $V$ as a vector space, with
    the $\C[H]$-module structure given by 
    \[
    r^2\cdot v_1=-v_1,\quad
    r^2\cdot v_2=-v_1,\quad
    s\cdot v_1=-v_1,\quad
    s\cdot v_2=-v_2.
    \]
    The character of $\Res_H^GV$ is
    \[
    \Res_H^G\chi(h)
    =\begin{cases}
    2 & \text{if $h=1$},\\
    -2 & \text{if $h=r^2$},\\
    0 & \text{otherwise}.
    \end{cases}
    \]
    Since $\langle\Res_H^G\chi,\Res_H^G\chi\rangle=0$, 
    the character of $\Res_H^G\chi$ is not irreducible. Thus $V$ 
    is not simple as a $\C[H]$-module. 
\end{example}

\begin{exercise}
\label{xca:constituent_restriction}
    Let $H$ be a subgroup of $G$ and 
    $\phi\in\Char(H)$. Prove that there exists 
    $\chi\in\Irr(G)$ such that
    $\langle\Res_H^G\chi,\phi\rangle\ne 0$.
\end{exercise}

\begin{exercise}
\label{xca:decomposing_restriction}
    Let $H$ be a subgroup of $G$, 
    $\Irr(H)=\{\phi_1,\dots,\phi_l\}$, 
    and $\chi\in\Irr(G)$. Prove that 
    \[
    \Res_H^G\chi=\sum_{i=1}^ld_i\phi_i,
    \]
    where $\sum_{i=1}^l d_i^2\leq (G:H)$. Moreover,  $\sum_{i=1}^l d_i^2=(G:H)$ 
    if and only if $\chi(g)=0$ for all $g\in G\setminus H$. 
\end{exercise}

\subsection{Induction of characters}

We now define the induction of class functions. Let $G$ be a finite group
and $H$ be a subgroup of $G$. If $f\colon H\to\C$ is a map, 
then 
\[
f^0(x)=\begin{cases}
    f(x) & \text{if $x\in H$},\\
    0 & \text{otherwise}.
    \end{cases}
\]
It is an exercise to prove that
the map $f\mapsto f^0$ is linear. 

\begin{definition}
    Let $G$ be a finite group and $H$ be a subgroup of $G$. Let
    $f\colon H\to\C$ be
    a map. The \emph{induction}
    of $f$ to $G$ is the map 
    \begin{align*}
      g\mapsto\Ind_H^Gf(g)=\frac{1}{|H|}\sum_{x\in G}f^0(x^{-1}gx).
    \end{align*}
\end{definition}

\begin{exercise}
\label{xca:induction}
    Let $G$ be a finite group. Prove that
    the map 
    \[
    \Ind_H^G\colon\cf(H)\to\cf(G),\quad  f\mapsto\Ind_H^G(f),
    \]
    is a well-defined linear map. 
\end{exercise}

For a finite group $G$, $\Tchar_G$ denotes the
trivial character of $G$. 

\begin{exercise}
\label{xca:inducting_trivial}
    Let $G$ be a finite group and $H=\{1\}$. Compute
    $\Ind_H^G\Tchar_H$. 
\end{exercise}

\begin{exercise}
\label{xca:induction_G/H}
    Let $G$ be a finite group, $H$ be a 
    subgroup of $G$. Prove that
    $\Ind_H^G\Tchar_H$ is the character of the
    representation induced by the action of $G$ on $G/H$ by left multiplication. 
\end{exercise}


The following exercise shows that the induction of a character yields a character; in fact, it demonstrates even more.

\begin{exercise}
    \label{xca:induced_representations}
    Let $H$ be a subgroup of $G$ and 
    $t_1,\dots,t_m$ be a left transversal of $H$ in $G$. For a representation $\rho\colon H\to\GL_n(\C)$ with character $\chi$ 
    and $x\in G$, let 
    \[
    \rho_x^0=\begin{cases}
        \rho_x & \text{if $x\in H$,}\\
        0_{n\times n} & \text{otherwise,}
    \end{cases}
    \]
    where $0_{n\times n}$ represents the zero
    $n\times n$ matrix. Prove that 
    \[
    \Ind_H^G\rho\colon G\to \GL_{mn}(\C),\quad 
    g\mapsto \left(\rho_{t_i^{-1}gt_j}^0\right)_{1\leq i,j\leq m}\in\C^{nm\times nm}
    \]
    is a representation with character $\Ind_H^G\chi$. 
\end{exercise}

Let us show an easy application of Exercise~\ref{xca:induced_representations}. 

\begin{example}
    Let $G=\{\pm1,\pm i,\pm j,\pm k\}$ the quaternion group. Then $[G,G]=\{1,-1\}$ and $G/[G,G]\simeq C_2\times C_2$ (since $G$ is non-abelian, $G/[G,G]$ cannot be cyclic). Thus 
    there are four degree-one representations of $G$ (see Exercise~\ref{xca:degree-one}). Let $H=\langle i\rangle=\{1,-1,i,-i\}$. Thus $|H|=4$ and $(G:H)=2$. 
    Let $t_1=1$ and $t_2=j$. Then $\{t_1,t_2\}$ is a left
    transversal of $H$ in $G$. Let 
    \[
    \rho\colon H\to\C^{\times},\quad i\mapsto\sqrt{-1}.
    \]
    Then $\rho$ is a representation. Let us compute $\Ind_H^G\rho$:
    \begin{align*}
        &(\Ind_H^G\rho)(\pm1)=\pm\begin{pmatrix}
            1&0\\0&1
        \end{pmatrix},
        &&
        (\Ind_H^G\rho)(\pm i)=\pm\begin{pmatrix}
            \sqrt{-1}&0\\0&-\sqrt{-1}
        \end{pmatrix},\\
        &
        (\Ind_H^G\rho)(\pm j)=\pm\begin{pmatrix}
            0&-1\\1&0
        \end{pmatrix},
        &&
        (\Ind_H^G\rho)(\pm k)=\pm\begin{pmatrix}
            0&-\sqrt{-1}\\-\sqrt{-1}&0
        \end{pmatrix}.
    \end{align*}
    As an exercise, the reader is invited to verify that the representation $\Ind_H^G\rho$ is irreducible.
\end{example}

\subsection{Frobenius' reciprocity}

Before proving that the induction of a character is a character, we  mention the following crucial property:

\begin{theorem}[Frobenius' reciprocity]
\index{Frobenius!reciprocity theorem}
    Let $G$ be a finite group and $H$ be a subgroup of $G$. 
    If $a\in\cf(H)$ and $b\in\cf(G)$, then
    \[
    \langle\Ind_H^Ga,b\rangle=\langle a,\Res_H^Gb\rangle
    \quad\text{and}\quad
    \langle\Res_H^Ga,b\rangle=\langle a,\Ind_H^Gb\rangle.
    \]
\end{theorem}

\begin{proof}
    We only need to prove the first equality. We compute 
    \begin{equation}
    \label{eq:reciprocity}
    \begin{aligned}
        \langle\Ind_H^Ga,b\rangle 
        &= \frac{1}{|G|}\sum_{x\in G}\Ind_H^Ga(x)\overline{b(x)}
        = \frac{1}{|G|}\frac{1}{|H|}\sum_{x,y\in G}a^0(y^{-1}xy)\overline{b(x)}.
    \end{aligned}
    \end{equation}
    Setting $h=y^{-1}xy$, 
    % Since 
    % \[
    % a^0(y^{-1}xy)\ne 0\Longrightarrow
    % y^{-1}xy\in H\Longleftrightarrow x\in yHy^{-1},
    % \]
    % setting $h=y^{-1}xy$ 
    we can write \eqref{eq:reciprocity} as 
    \begin{align*}
        \langle\Ind_H^Ga,b\rangle
        &=\frac{1}{|G|}\frac{1}{|H|}\sum_{y\in G}\sum_{h\in H}a(h)\overline{b(yhy^{-1})}\\
        &=\frac{1}{|G|}\frac{1}{|H|}\sum_{y\in G}\sum_{h\in H}a(h)\overline{b(h)}\\
        &=\frac{1}{|G|}\sum_{y\in G}\langle a,\Res_H^Gb\rangle.\qedhere 
    \end{align*}
\end{proof}

\begin{corollary}
\label{cor:reciprocity}
    Let $G$ be a finite group and $H$ be a subgroup of $G$. 
    Let $\chi\in\Char(H)$ be such that 
    $\chi(1)=n$. Then 
    $\Ind_H^G\chi\in\Char(G)$ and 
    has degree $n(G:H)$. 
\end{corollary}

\begin{proof}
    It is enough to show that
    each $m_\psi=\langle\Ind_H^G\chi,\psi\rangle\in\Z_{\geq0}$ 
    for all $\psi\in\Irr(G)$. Let $\psi\in\Irr(G)$. By 
    Frobenius' reciprocity theorem, 
    \[
    m_\psi=\langle\Ind_H^G\chi,\psi\rangle
    =\langle\chi,\Res_H^G\psi\rangle\in\Z_{\geq0}
    \]
    because both $\chi$ and $\Res_H^G\psi$ are
    characters of $H$. To prove this, let $\Irr(H)=\{\theta_1,\dots,\theta_k\}$. Since $\chi\in\Char(H)$ and 
    $\Res_H^G\psi\in\Char(H)$, there are 
    non-negative integers $a_1,\dots,a_k$ and $b_1,\dots,b_k$ such that
    $\chi=\sum_{i=1}^ka_i\theta_i$ and 
    $\Res_H^G\psi=\sum_{j=1}^kb_j\theta_j$. Then
    \[
\langle\chi,\Res_H^G\psi\rangle=\sum_{i=1}^k\sum_{j=1}^ka_ib_j\langle\theta_i,\theta_j\rangle=\sum_{i=1}^ka_ib_i\in\Z_{\geq0}.
    \]
    Therefore   
    \[
    \Ind_H^G\chi=\sum_{\psi\in\Irr(G)}m_{\psi}\psi\in\Char(G). 
    \]
    In particular, 
    \[
    \left(\Ind_H^G\chi\right)(1)=\frac{1}{|H|}\sum_{x\in G}\chi^0(1)=\frac{1}{|H|}|G|\chi(1)=\chi(1)(G:H).\qedhere 
    \]
\end{proof}



\begin{exercise}
\label{xca:induction_S3}
    Let $G=\Sym_3$ and $H=\langle (12)\rangle$. 
    Let $\varphi=\sgn|_H$ be the restriction sign homomorphism 
    to the subgroup $H$. Compute 
    $\Ind_H^G\varphi$. 
\end{exercise}

There are some useful properties that are easy to show. 

\begin{exercise}
\label{xca:indres}
    Let $G$ be a finite group, $H$ be a subgroup of $G$, 
    $a\in\cf(H)$ and $b\in\cf(G)$. Prove that 
    \[
    \Ind_H^G((\Res_H^Gb)a)=b(\Ind_H^Ga).
    \]
\end{exercise}

\begin{exercise}[Transitivity of induction]
\label{xca:transitivity_induction}
    Let $G$ be a finite group, $H\subseteq K$ be 
    subgroups of $G$ and $a\in\cf(H)$. 
    Prove that 
    \[
    \Ind_K^G\Ind_H^Ka=\Ind_H^Ga.
    \]
\end{exercise}

\begin{exercise}
\label{xca:useful_Ind}
    Let $G$ be a finite group, $H$ be a subgroup
    of $G$ and $t_1,\dots,t_k$ be a transversal
    of $H$ in $G$. Prove that
    \[
    (\Ind_H^G\alpha)(g)=\sum_{i=1}^k\alpha^0(t_i^{-1}gt_i)
    \]
    for all $\alpha\in\cf(H)$.
\end{exercise}



% 8.1.4 de Steinberg
% Seccion 8.2 define la inducción de representaciones
% 8.2.1 Induce la trivial del trivial a todo el grupo o obtiene la regular
% 8.2.2 Induce la representación por permutaciones
% Define la matriz y mira el dihedral de orden 2n (yo tengo un caso particular)
% Hace el ejemplo de Q8
% En el teorema 8.2.5 prueba la inducción da una representación

