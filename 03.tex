\chapter{} 

\topic{Representations}

Unless we state differently, we will always work
with finite groups. All our vector spaces will
be complex vector spaces. 

\begin{definition}
\index{Representation}
    Let $G$ be a finite group. A \textbf{representation}
    of $G$ is a group homomorphism $\rho\colon G\to\GL(V)$, where
    $V$ is a finite-dimensional vector space. The \textbf{degree} (or dimension) 
    of the representation is the integer $\deg\rho=\dim V$. 
\end{definition}

\index{Matrix representation}
Let $G\to\GL(V)$ be a representation. 
If we fix a basis of $V$, then we obtain
a \textbf{matrix representation} of $G$, that is a 
group homomorphism 
\[
\rho\colon G\to\GL(V)\simeq\GL_n(\C),
\quad 
g\mapsto\rho_g,
\]
where
$n=\dim V$. 

\begin{example}
Since $\Sym_3=\langle (12),(123)\rangle$, the map $\rho\colon \Sym_3\to\GL_3(\C)$,
\[
(12)\mapsto\begin{pmatrix}
0 & 1 & 0\\
1 & 0 & 0\\
0 & 0 & 1
\end{pmatrix},\quad
(123)\mapsto\begin{pmatrix}
0 & 0 & 1\\
1 & 0 & 0\\
0 & 1 & 0
\end{pmatrix},
\] 
is a representation of $\Sym_3$. 
\end{example}

\begin{example}
Let $G=\langle g\rangle$ be cyclic of order six. 
The map $\rho\colon G\to\GL_2(\C)$, 
\[
g\mapsto
\begin{pmatrix}
1&1\\
-1&0
\end{pmatrix}
\] 
is a representation of $G$. 
\end{example}

\begin{example}
Let $G=\langle g\rangle$ be cyclic of order four. 
The map $\rho\colon G\to\GL_2(\C)$, 
\[
g\mapsto
\begin{pmatrix}
0&-1\\
1&0
\end{pmatrix}
\] 
is a representation of $G$. 
\end{example}

\begin{example}
  Let $G=\langle a,b:a^2=b^3=(ab)^3=1\rangle$. The map 
  \[
    a\mapsto\begin{pmatrix}
    0 & 1 & -1\\
    1 & 0 & -1\\
    0 & 0 & -1
    \end{pmatrix},
    \quad
    b\mapsto\begin{pmatrix}
      0 & 0 & 1\\
      1 & 0 & 0\\
      0 & 1 & 0
    \end{pmatrix},
  \]
  defines a representation $G\to\GL_3(\C)$. 
\end{example}

\begin{example}
  Let $G$ be a finite group that acts on a finite set $X$. 
  Let $V=\C X$ the complex vector space with basis $\{x:x\in
  X\}$. The map 
  \[
	\rho\colon G\to\GL(V),\quad
	\rho_g\left(\sum_{x\in X}\lambda_x x\right)
	=\sum_{x\in X}\lambda_x\rho_g(x)
	=\sum_{x\in X}\lambda_{g^{-1}\cdot x}x, 
  \]
  is a representation of degree $|X|$.
\end{example}

\begin{example}
    The sign $\sgn\colon\Sym_n\to\GL_1(\C)=\C^{\times}$ is a representation of $\Sym_n$.
\end{example}

An important fact is that there exists a bijective
correspondence 
between 
representations of a finite group $G$ 
and 
finite-dimensional modules over $\C[G]$. The correspondence
is given as follows. If $\rho\colon G\to\GL(V)$ is a representation, 
then $V$ is a $\C[G]$-module with
\[
\left(\sum_{g\in G}\lambda_gg\right)\cdot v=\sum_{g\in G}\lambda_g\rho_g(v).
\]
Conversely, if $V$ is a $\C[G]$-module, then
$\rho\colon G\to\GL(V)$, $\rho_g\colon V\to V$, $v\mapsto g\cdot v$, 
is a representation. 

\begin{exercise}
    Let $G$ be a finite group and 
    $\rho\colon G\to\GL(V)$ be a representation. Prove that 
    each $\rho_g$ is diagonalizable. 
\end{exercise}

The previous exercise uses properties of the minimal polynomial. We will 
see a different proof later. 

\begin{definition}
\index{Equivalent representations}
Let $G$ be a group and $\phi\colon G\to\GL(V)$ and $\psi\colon G\to\GL(W)$ be representations of $G$. 
We say that $\phi$ and $\psi$ are \textbf{equivalent} if 
there exists a linear isomorphism $T\colon V\to W$ such that 
\[
	\psi_g T=T \phi_g
\]
for all $g\in G$. In this case, we write $\phi\simeq\psi$. 
\end{definition}

Note that $\phi\simeq\psi$ if and only if $V$
and $W$ are isomorphic as $\C[G]$-modules.

\begin{example}
  The representation 
  \begin{gather*}
  \phi\colon\Z/n\to\GL_2(\C),\quad
  \phi(m)=
  \begin{pmatrix}
    \cos(2\pi m/n) & -\sin(2\pi m/n)\\
    \sin(2\pi m/n) & \cos(2\pi m/n)
  \end{pmatrix},
  \shortintertext{is equivalent to the representation}
  \psi\colon\Z/n\to\GL_2(\C),
  \quad 
  \psi(m)=\begin{pmatrix}
    e^{2\pi im/n} & 0\\
    0 & e^{-2\pi im/n}
  \end{pmatrix}.
  \end{gather*}
  The equivalence is obtained with the matrix $T=\begin{pmatrix} i & -i\\
    1&1\end{pmatrix}$, as a direct calculation shows that
    $\phi_m T=T\psi_m$ for all $m$.
\end{example}

\begin{definition}
    Let $\phi\colon G\to\GL(V)$ be a representation. A subspace 
    $W\subseteq V$ is said to be \textbf{$G$-invariant} if
    $\phi_g(W)\subseteq W$ for all $g\in G$.  
\end{definition}

Let $\rho\colon G\to\GL(V)$ be a representation. 
If $W$ is a $G$-invariant subspace of $V$, 
then the restriction $\rho|_W\colon G\to\GL(W)$
is a representation. In particular, $W$ is a submodule (over $\C[G]$) 
of $V$. 

\begin{definition}
\index{Representation!irreducible}
\index{Module!simple}
    A representation $\rho\colon G\to\GL(V)$ is 
    said to be \textbf{irreducible} if 
    $\{0\}$ and $V$ are the only 
    $G$-invariant subspaces of $V$. 
\end{definition}

Note that a representation $\rho\colon G\to\GL(V)$ is irreducible
if and only if $V$ is simple. 

\begin{example}
    Degree-one representations are irreducible. 
\end{example}

In the following example we work over the real numbers. 

\begin{example}
Let $G=\langle g\rangle$ be the cyclic group of three elements and 
\[
\rho\colon G\to\GL_3(\R),\quad
g\mapsto\begin{pmatrix}
  0&1&0\\
  0&0&1\\
  1&0&0
\end{pmatrix}.
\]
Thus $g$ acts on $\R^{3}$ by left matrix multiplication,
\[
g\cdot (x,y,z)=
\begin{pmatrix}
  0&1&0\\
  0&0&1\\
  1&0&0
\end{pmatrix}\begin{pmatrix}
x\\
y\\
z
\end{pmatrix}.
\]
The set 
\[
N=\{(x,y,z)\in\R^{3}:x+y+z=0\}
\]
is a $G$-invariant subspace of $\R^3$. 

We claim that $N$ is irreducible. 
If $N$ contains a non-zero $G$-invariant subspace $S$, 
let $(x_0,y_0,z_0)\in S\setminus\{(0,0,0)\}$. Since $S$ is $G$-invariant, 
\[
\begin{pmatrix}
y_0\\
z_0\\
x_0
\end{pmatrix}
=
\begin{pmatrix}
  0&1&0\\
  0&0&1\\
  1&0&0
\end{pmatrix}
\begin{pmatrix}
x_0\\
y_0\\
z_0
\end{pmatrix}\in S.
\]
We claim that $\{(x_0,y_0,z_0),(y_0,z_0,x_0)\}$ is linearly independent. If there exists $\lambda\in\R$ 
such that $\lambda(x_0,y_0,z_0)=(y_0,z_0,x_0)$, then $x_0=\lambda^3 x_0$. Since $x_0=0$ implies 
$y_0=z_0=0$, it follows that $\lambda=1$. In particular, $x_0=y_0=z_0$, a contradiction, as $x_0+y_0+z_0=0$. 
Hence $\dim S=2$ and therefore $S=N$. 
\end{example}

What happens in the previous example if we consider complex numbers?

\begin{exercise}
  \label{xca:deg2}
  Let $\phi\colon G\to \GL(V)$, $g\mapsto\phi_g$, be a degree-two representation. Prove that
  $\phi$ is ireeducible if and only if there is no common eigenvector for all the $\phi_g$.
\end{exercise}

\begin{example}
  Recall that $\Sym_3$ is generated by $(12)$ and $(23)$. The map 
  \[(12)\mapsto\begin{pmatrix}
    -1 & 1\\
    0 & 1
  \end{pmatrix},
  \quad
  (23)\mapsto\begin{pmatrix}
    1 & 0\\
    1 & -1
  \end{pmatrix},
  \]
  defines a representation $\phi$ of $\Sym_3$. 
  Exercise \ref{xca:deg2} shows that $\phi$ is  
  irreducible.
\end{example}