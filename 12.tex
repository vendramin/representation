\section{Lecture: Week 12}

\subsection{Kronecker's theorem}

We begin with a classical theorem of Kronecker on algebraic integers. Recall 
that $\alpha\in\C$ is an \emph{algebraic integer} if there is a monic
polynomial $f\in\Z[X]$ such that $f(\alpha)=0$ (see Definition~\ref{def:algebraic_integer}). Let $\A$ 
be the set of algebraic integers. 

\begin{exercise}
    \label{xca:irreducible}
    Let $\alpha\in\A$. Prove that there exists a monic polynomial $f\in\Z[X]$, 
    irreducible $f\in\Q[X]$ such that $f(\alpha)=0$. 
\end{exercise}

The polynomial of Exercise~\ref{xca:irreducible} is called the \emph{minimal polynomial} of $\alpha$. 

\begin{exercise}
\label{xca:distinct}
    Let $\alpha\in\A$. Prove that the roots of the
    minimal polynomial of $\alpha$ are pairwise distinct. 
\end{exercise}

The \emph{conjugates} of $\alpha$ are the roots of the minimal polynomial of $\alpha$. 

Recall that for an $n\times n$ matrix $A=(a_{ij})$, its \emph{norm} is defined 
as the maximum absolute row sum of the matrix, that is 
\[
\|A\|_\infty=\max_{1\leq i\leq n}\sum_{j=1}^n|a_{ij}|.
\]
For $A,B\in\C^{n\times n}$ and $\lambda\in\C$, 
the following properties hold: 
\begin{enumerate}
    \item $\|A\|_\infty\geq0$.
    \item $\|A\|_\infty=0$ if and only if $A$ is the $n\times n$ zero matrix. 
    \item $\lambda\|A\|_\infty=|\lambda|\|A\|_\infty$. 
    \item $\|A+B\|_\infty\leq\|A\|_\infty+\|B\|_\infty$. 
    \item $\|AB\|_\infty\leq\|A\|_\infty\|B\|_\infty$. 
\end{enumerate}

\begin{theorem}[Kronecker]
\index{Kronecker theorem}
\label{thm:Kronecker}
Let $\alpha\in\A$. Assume that all the conjugates of $\alpha$ 
have absolute value at most one. Then either $\alpha=0$ or $\alpha$ is a root of one. 
\end{theorem}

\begin{proof}
    Assume that $\alpha\ne 0$. 
    Let $f\in\Z[X]$ be the minimal polynomial of $\alpha$, say 
    \[
    f=X^n+a_{n-1}X^{n-1}+\cdots+a_1X+a_0
    \]
    for integers $a_0,a_1,\dots,a_{n-1}\in\Z$. Then $f(0)\ne0$ because $f$ is irreducible in $\Q[X]$ (see Exercise~\ref{xca:irreducible}).  
    Let 
    \[
    F=\begin{pmatrix}
        0 & 0 & \cdots & 0 & -a_0\\
        1 & 0 & \cdots & 0 & -a_1\\
        0 & 1 & \cdots & 0 & -a_2\\
        \vdots & \vdots & \ddots & \vdots & \vdots\\
        0 & 0 & \cdots & 1 & -a_{n-1}
    \end{pmatrix}\in\Z^{n\times n}
    \]
    be the \emph{companion matrix} of $f$. The characteristic polynomial and the minimal polynomial of 
    the matrix $F$ are equal to $f$. Moreover, the roots of $f$ are the eigenvalues of $F$. Since 
    all the roots of $f$ are distinct, all the eigenvalues of $F$ are different. Thus 
    $F$ is diagonalizable, so there exists $P\in\GL_n(\C)$ such that $F=PDP^{-1}$, where
    $D$ is the $n\times n$ diagonal matrix with diagonal 
    entries $\alpha=\alpha_1,\alpha_2,\dots,\alpha_n$, 
    the roots of $f$ (i.e., the conjugates of $\alpha$), 
    so all with absolute value at most one. Thus 
    $\|D\|_\infty\leq 1$. 
    Since $0\not\in\{\alpha_1,\dots,\alpha_n\}$, the matrix
    $F$ is invertible. Moreover, 
    \[
    F^k=(PDP^{-1})^k=PD^kP^{-1}
    \]
    for all $k\geq1$. Note that the set 
    $X=\{F^k:k\geq 1\}\subseteq M_n(\Z)$ is bounded in $M_n(\C)$, 
    as 
    \[
    \|F^k\|_\infty=\|PD^kP^{-1}\|_\infty\leq 
    \|P\|_\infty\|D\|_\infty^k\| P^{-1}\|_\infty
    \leq \underbrace{\|P\|_\infty\| P^{-1}\|_\infty}_{\text{This is independent of $k$}}.
    \]
    Thus $X$ is finite. In particular, there are integers $i<j$ such that 
    $F^i=F^j$. Since $F$ is invertible, $F^{j-i}$ is the $n\times n$ 
    identity matrix. Since $\alpha$ is an eigenvalue of $F$, it follows
    that $\alpha^{j-i}=1$.  
    % Let $\{e_1,\dots,e_n\}$ be the standard basis of $\C^{n\times1}$. 
    % A direct calculation shows that $Fe_{j}=e_{j+1}$ for all $j\in\{2,\dots,n-1\}$ and 
    % \begin{align*}
    %     Fe_n&=-a_0e_1-a_1e_2-\cdots-a_{n-1}e_n.
    % \end{align*} 
\end{proof}

The proof of the theorem presented here goes back to Greiter~\cite{MR514044}. 
Kronecker’s original proof is somewhat similar, relying on 
Vieta’s formulas and estimates involving binomial coefficients; see~\cite{MR1834706}.


\subsection{Solvable groups and Burnside's theorem}

\index{Derived series}
For a group $G$ let 
$G^{(0)}=G$ and 
$G^{(i+1)}=[G^{(i)},G^{(i)}]$ for $i\geq0$.
The \emph{derived series} of $G$ is the sequence
\[
G=G^{(0)}\supseteq G^{(1)}\supseteq G^{(2)}\supseteq\cdots
\]
Each $G^{(i)}$ is a characteristic subgroup of $G$. We say that 
$G$ is \emph{solvable} if $G^{(n)}=\{1\}$ for some $n$.  

\begin{example}
	Abelian groups are solvable. 
\end{example}

\begin{example}
	The group $\SL_2(3)$ is solvable. 
	Let us see what the computer says:
\begin{lstlisting}
> G := SL(2,3);;
> IsSolvable(G);
true
> [GroupName(x) : x in DerivedSeries(G)];
[ SL(2,3), Q8, C2, C1 ]
\end{lstlisting}
\end{example}

\begin{example}
	Non-abelian simple groups cannot be solvable. 
\end{example}

For $n\geq5$, the group $\Alt_n$ is not solvable.

\begin{exercise}
	\label{xca:solvable}
	Let $G$ be a group. Prove the following statements:
	\begin{enumerate}
		\item A subgroup $H$ of $G$ is solvable, when $G$ is solvable.
		\item Let $K$ be a normal subgroup of $G$. 
		    Then $G$ is solvable if and only if $K$ and $G/K$ are solvable.
	\end{enumerate}
\end{exercise}

For $n\geq5$, the group $\Sym_5$ is not solvable. 

\begin{exercise}
\label{xca:pgroups_solvable}
	Let $p$ be a prime number. Prove that 
	finite $p$-groups are solvable.
\end{exercise}

Exercises~\ref{xca:solvable} and~\ref{xca:pgroups_solvable} may be omitted if the reader is already familiar with solvable groups.

\begin{theorem}[Burnside]
	\index{Burnside theorem}
	\label{thm:Burnside_auxiliar}
	Let $G$ be a finite group. If $\phi\colon G\to\GL_n(\C)$ is a representation
	with character $\chi$ and $C$ is a conjugacy class of $G$ such that 
	$\gcd(|C|,n)=1$, then for every $g\in C$ either 
	$\chi(g)=0$ or $\phi_g$ is a scalar matrix. 
\end{theorem}

% We need a lemma.

% \begin{lemma}
% 	\label{lem:4Burnside}
% 	Let $\epsilon_1,\dots,\epsilon_n$ be roots of one such that 
% 	$(\epsilon_1+\cdots+\epsilon_n)/n\in\A$. Then either 
% 	$\epsilon_1=\cdots=\epsilon_n$ or 
% 	$\epsilon_1+\cdots+\epsilon_n=0$.
% \end{lemma}

% \begin{proof}
% 	Let $\alpha=(\epsilon_1+\cdots+\epsilon_n)/n$.
% 	If the $\epsilon_j$s are not all equal, then $\|\alpha\|<1$. Moreover, 
% 	$\|\beta\|<1$ for every algebraic conjugate $\beta$ of $\alpha$. Since the product 
% 	of the algebraic conjugates of $\alpha$ is an integer of absolute value 
% 	$<1$, it follows that it is zero. 
% \end{proof}

Now we prove the theorem.

\begin{proof}[Proof of Theorem \ref{thm:Burnside_auxiliar}]
	Let $\epsilon_1,\dots,\epsilon_n$ be the eigenvalues of $\phi_g$. Then 
    $\epsilon_1,\dots,\epsilon_n$ are roots of one. 
    By assumption, 
	$\gcd(|C|,n)=1$, there exist $a,b\in\Z$ such that $a|C|+bn=1$. Since 
	$|C|\chi(g)/n\in\A$, after multiplying by $\chi(g)/n$ we obtain that  
	\[
		a|C|\frac{\chi(g)}{n}+b\chi(g)=\frac{\chi(g)}{n}=\frac{1}{n}(\epsilon_1+\cdots+\epsilon_n)\in\A.
	\]
    Let $\alpha_1=\chi(g)/n\in\A$ and $\alpha_2,\dots,\alpha_n$ be its conjugates. Since $|\alpha_1|\leq 1$ 
    and $\alpha_2,\dots,\alpha_n$ are conjugates of $\alpha_1$, it follows that  
    $|\alpha_j|\leq 1$ for all $j\in\{1,\dots,n\}$. By Kronecker's theorem, 
    either $\alpha_1=0$ or $\alpha_1$ is a root of one. If $\alpha_1=0$, then $\chi(g)=0$. If 
    $\alpha_1$ is a root of one, then 
    \[
    1=|\alpha_1|=\frac{|\chi(g)|}{n}=\frac1{n}.
    \]
    Thus $|\chi(g)|=n=\chi(1)$. This means that $g\in\Z(\chi)$. By Exercise~\ref{xca:center}, 
    $\phi_g$ is a scalar matrix. 
	% The previous lemma implies that there are two cases to consider: 
	% either $\epsilon_1=\cdots=\epsilon_n$ or $\epsilon_1+\cdots+\epsilon_n=0$. In the first
	% case, since $\phi_g$ is diagonalizable, $\phi_g$ is a scalar matrix. 
	% In the second case, $\chi(g)=0$.
\end{proof}

\begin{theorem}[Burnside]
	\index{Burnside theorem}
    \label{thm:pq_notsimple}
	Let $p$ be a prime number. If $G$ is a finite group and 
	$C$ is a conjugacy class of $G$ with $p^k>1$ elements, then $G$ 
	is not simple.
\end{theorem}

\begin{proof}
	Let $g\in C\setminus\{1\}$. Column orthogonality implies that 
	\begin{equation}
	\label{eq:Burnside}
	\begin{aligned}
		0&=\sum_{\chi\in\Irr(G)}\chi(1)\chi(g)\\
		&=\sum_{p\mid\chi(1)}\chi(1)\chi(g)+\sum_{p\nmid\chi(1):\chi\ne\chi_1}\chi(1)\chi(g)+1,
	\end{aligned}
	\end{equation}
	where the one corresponds to the trivial representation of
	$G$.
	
	Look at this equation modulo $p$. If $\chi(g)=0$ for all
	$\chi\in\Irr(G)$
	such that $\chi\ne\chi_1$ and $p\nmid\chi(1)$, then
	\[
	-\frac{1}{p}=\sum\frac{\chi(1)}{p}\chi(g)\in\A\cap\Q=\Z,
	\]
	where the sum is taken over all non-trivial irreducibles
	of $G$ of degree divisible by $p$, a contradiction. Hence 
	there exists an irreducible non-trivial representation 
	$\phi$ with character $\chi$ such that $p$ does not divide
	$\chi(1)$ and $\chi(g)\ne0$. By the previous theorem, 
	$\phi_g$ is a scalar matrix. If $\phi$ is faithful, then 
	$g$ is a non-trivial central element, a contradiction since 
    $|C|>1$. If $\phi$ is not faithful, then 
    $G$ is not simple (because 
	$\ker\phi$ is a non-trivial proper normal subgroup of $G$). 
\end{proof}

\begin{theorem}[Burnside]
  \index{Burnside $p^aq^b$-theorem}
  \label{thm:pq}
  Let $p$ and $q$ be prime numbers. If $G$ has order $p^aq^b$, then $G$ is solvable.
\end{theorem}

\begin{proof}
	If $G$ is abelian, then it is solvable.
	Suppose now $G$ is non-abelian.
	Let us assume that the theorem is not true. Let $G$ be a group
	of minimal order $p^aq^b$
	that is not solvable. Since $|G|$ is minimal, $G$ is a non-abelian simple group.
	By the previous theorem, 
	$G$ has no conjugacy classes of size $p^k$ nor 
	conjugacy classes of size $q^l$ with $k,l\geq1$. The size
	of every conjugacy class of $G$ is one or divisible by $pq$. 
	Note that, since $G$ is a non-abelian simple group,
	the center of $G$ is trivial.
	Thus there is only one conjugacy class of size one.
	By the class
	equation,
	\[
		|G|=1+\sum_{C:|C|>1}|C|\equiv 1 \bmod pq,
	\]
	where the sum is taken over all conjugacy classes of $G$ 
	with more than one element, a contradiction.
\end{proof}

\subsection{Some generalizations of Burnside's theorem}

If the reader does not know what nilpotent groups are, this section can be safely omitted.

\begin{theorem}[Kegel--Wielandt]
    \index{Kegel--Wielandt theorem}
    \label{thm:KegelWielandt}
    If $G$ is a finite group and there are nilpotent subgroups 
    $A$ and $B$ of $G$ such that 
    $G=AB$, then $G$ is solvable.
\end{theorem}

See~\cite[Theorem 2.4.3]{MR1211633} for the proof.

\begin{exercise}
    Prove that Theorem~\ref{thm:KegelWielandt} 
    implies Theorem~\ref{thm:pq}.
\end{exercise}

Another generalization of Burnside's theorem
is based on \emph{word maps}. A word map
of a group $G$ is a map 
\[
G^k\to G,\quad 
(x_1,\dots,x_k)\mapsto w(x_1,\dots,x_k)
\]
for some word $w(x_1,\dots,x_k)$ of the free group $F_k$ of rank $k$. 
Some word maps are surjective in certain families of groups. For example, 
Ore's conjecture is precisely the surjectivity of the word map
$(x,y)\mapsto [x,y]=xyx^{-1}y^{-1}$ in every finite non-abelian simple 
group. 

\begin{theorem}[Guralnick--Liebeck--O'Brien--Shalev--Tiep]
    Let $a,b\geq0$, $p$ and $q$ be prime numbers and $N=p^aq^b$. The map 
    $(x,y)\mapsto x^Ny^N$ is surjective in every finite simple group. 
\end{theorem}

The proof appears in~\cite{MR3827208}. 

The theorem implies Burnside's theorem. Let $G$ be a group of order
$N=p^aq^b$. Assume that $G$ is not solvable. 
Fix a composition series of $G$. There is a non-abelian factor $S$ 
of order that divides $N$. Since 
$S$ is simple non-abelian and $s^N=1$, it follows that the word map
$(x,y)\mapsto x^Ny^N$ has trivial image in $S$, a contradiction 
to the theorem. 

\subsection{The Feit--Thompson theorem}

\begin{theorem}[Feit--Thompson]
    \index{Feit--Thompson!theorem}
    Groups of odd order are solvable. 
\end{theorem}

The proof of Feit--Thompson theorem is extremely hard. 
It occupies a full volume of the 
\emph{Pacific Journal of Mathematics}~\cite{MR166261}. 
A formal verification of the proof 
(based on the computer software Coq) 
was announced in~\cite{MR3111271}.  

Back in the day it was believed that if a certain divisibility 
conjecture is true, 
the proof of Feit--Thompson theorem 
could be simplified. 

\begin{conjecture}[Feit--Thompson]
\index{Feit--Thompson!conjecture}
    There are no prime numbers $p$ and $q$ such that
    $\frac {p^{q}-1}{p-1}$ divides $\frac{q^{p} - 1}{q - 1}$. 
\end{conjecture}

The conjecture remains open. However, now we know that 
proving the conjecture will not simplify further
the proof of Feit--Thompson theorem. 

In 2012 Le proved that the conjecture is true for $q=3$, see 
\cite{MR2900154}. 


In~\cite{MR297686} 
Stephens proved that a certain stronger version of the conjecture 
does not hold, as the integers 
$\frac {p^{q}-1}{p-1}$ and $\frac{q^{p} - 1}{q - 1}$ 
could have common factors. In fact, if $p=17$ and $q=3313$, 
then 
\[
\gcd\left(\frac {p^{q}-1}{p-1},\frac{q^{p} - 1}{q - 1}\right)=112643.
\]
Nowadays we can check this easily in almost every desktop computer:
% \begin{lstlisting}
% gap> Gcd((17^3313-1)/16,(3313^17-1)/3312);
% 112643
% \end{lstlisting}
\begin{lstlisting}
> p := 17; 
> q := 3313;
> bool, a := IsCoercible(Integers(), (p^q-1)/(p-1));
> bool, b := IsCoercible(Integers(), (q^p-1)/(q-1));
> Gcd(a,b);
112643    
\end{lstlisting}
No other counterexamples have been found of Stephen’s stronger version of the conjecture.



\subsection{The character table of \texorpdfstring{$\Alt_5$}{A5}}

\index{Character table!of $\Alt_5$}
Let $G=\Alt_5$. 
The group $G$ is a non-abelian simple group of order 60. It has five conjugacy classes, namely

\bigskip 
\begin{center}
    \begin{tabular}{c|ccccc}
         Representative & $\id$  & $(12)(34)$ & $(123)$  & $(12345)$ & $(12354)$\\
         \hline 
         Size & $1$ & $15$ & $20$ & $12$ & $12$ \\
    \end{tabular}
\end{center}
\bigskip 

One can easily get the conjugacy classes of 
$\Alt_5$ with Magma:
\begin{lstlisting}
> A5 := Alt(5);
> ConjugacyClasses(A5);
Conjugacy Classes of group A5
-----------------------------
[1]     Order 1       Length 1
        Id(A5)

[2]     Order 2       Length 15
        (1, 2)(3, 4)

[3]     Order 3       Length 20
        (1, 2, 3)

[4]     Order 5       Length 12
        (1, 2, 3, 4, 5)

[5]     Order 5       Length 12
        (1, 3, 4, 5, 2)    
\end{lstlisting}

Let us see how to obtain all conjugacy classes
of $\Alt_5$ without computers. Let $\sigma\in\Alt_5$ and $C$ be its
conjugacy class in $\Sym_5$. Thus $|C|=(\Sym_5:C_{\Sym_5}(\sigma))$. There are two cases to consider

Assume first that $C_{\Sym_5}(\sigma)\not\subseteq\Alt_5$. Since $\Alt_5$ is a maximal subgroup of $\Sym_5$, it follows that 
$\Alt_5C_{\Sym_5}(\sigma)=\Sym_5$. Using the isomorphism theorems, 
\[
\Sym_5/\Alt_5=\Alt_5C_{\Sym_5(\sigma)}/\Alt_5
\simeq C_{\Sym_5}(\sigma)/(C_{\Sym_5}(\sigma)\cap\Alt_5)
=C_{\Sym_5}(\sigma)/C_{\Alt_5}(\sigma).
\]
Hence 
\[
(\Alt_5:C_{\Alt_5}(\sigma))=\frac{(\Sym_5:C_{\Alt_5}(\sigma))}{(\Sym_5:\Alt_5)}
=\frac{(\Sym_5:C_{\Alt_5}(\sigma))}{(C_{\Sym_5}(\sigma):C_{\Alt_5}(\sigma))}
=(\Sym_5:C_{\Sym_5}(\sigma))=|C|.
\]
Therefore $C$ is the class of $\sigma$ in $\Alt_5$. 

Assume now that $C_{\Sym_5}(\sigma)\subseteq\Alt_5$. Then 
$C_{\Alt_5}(\sigma)=C_{\Sym_5}(\sigma)\cap\Alt_5=C_{\Sym_5}(\sigma)$
and therefore 
\[
(\Alt_5:C_{\Alt_5}(\sigma))=(\Alt_5:C_{\Sym_5}(\sigma))
=\frac12(\Sym_5:C_{\Sym_5}(\sigma))=\frac12|C|.
\]
Thus $C$ splits into two conjugacy classes of $\Alt_5$ of equal size. 

The identity permutation is central. The even permutations 
$(12)(34)$ and $(123)$ both commutes with some odd permutation in $\Sym_5$ (e.g. 
$[(12)(34),(34)]=[(123),(45)]=\id$). Thus these classes do not split
in $\Alt_5$. There are twenty-four 5-cycles in $\Sym_5$. Since $24$ does not
divide $|\Alt_5|=60$, it follows that the class of 5-cycles
splits in $\Alt_5$. As representatives of these classes
we can take $(12345)$ and $(12354)$. 

Since $\Alt_5$ has five conjugacy classes, $|\Irr(G)|=5$. Assume that 
\[
\Irr(G)=\{\chi_1,\chi_2,\chi_3,\chi_4,\chi_5\}, 
\]
where $\chi_1$ is the trivial character. 

Let $H=\Alt_4$. We compute $\Ind_H^G\chi_1$. By Corollary~\ref{cor:reciprocity}, 
\[
\left(\Ind_H^G\chi_1\right)(\id) = 5.
\]
And a direct calculation shows
\begin{align*}
    &\left(\Ind_H^G\chi_1\right)((12)(34)) = 1,\\
    &\left(\Ind_H^G\chi_1\right)((123)) = 2,\\
    &\left(\Ind_H^G\chi_1\right)((12345)) = 0\\ 
    &\left(\Ind_H^G\chi_1\right)((12354)) = 0.
\end{align*}

Now, using Frobenius' reciprocity and the fact that 
$\Res_H^G\chi_1$ is the trivial character of $H$, 
\begin{align*}
    \langle\Ind_H^G\chi_1,\chi_1\rangle = \langle\chi_1,\Res_H^G\chi_1\rangle=1.
\end{align*}

Let $\chi_2=\Ind_H^G\chi_1-\chi_1$. Since 
\[
\langle\Ind_H^G\chi_1-\chi_1,\Ind_H^G\chi_1-\chi_1\rangle=1, 
\]
it follows that $\chi_2\in\Irr(G)$. 

\begin{exercise}
\label{xca:A5_chi2}
    Use Proposition~\ref{pro:2transitive} to derive (once again) the values of $\chi_2$.
\end{exercise}

So far we have the 
following table: 

\bigskip 
\begin{center}
        \begin{tabular}{|c|ccccc|}
        \hline  
         & $\id$ & $(12)(34)$ & $(123)$ & $(12345)$ & $(12354)$\\
        \hline 
        $\chi_1$ & $1$ & $1$ & $1$ & $1$ & $1$\\
        $\chi_2$ & $4$ & $0$ & $1$ & $-1$ & $-1$\\
        $\chi_3$ & $n_3$ & $\cdot$ & $\cdot$ & $\cdot$& $\cdot$\\
        $\chi_4$ & $n_4$ & $\cdot$ & $\cdot$ & $\cdot$& $\cdot$\\
        $\chi_5$ & $n_5$ & $\cdot$ & $\cdot$ & $\cdot$& $\cdot$\\
        \hline 
    \end{tabular}
\end{center}
\bigskip 

As $G$ is simple non-abelian, 
$|G/[G,G]|=1$. It follows that
$\chi_1$ is the only linear character of $G$. Moreover, 
$\chi_j(1)\geq3$ by Theorem~\ref{thm:simple}. Since 
\[
60=1+16+n_3^2+n_4^2+n_5^2
\]
and each $n_j$ divides $|G|=60$ 
(see Theorem \ref{thm:Frobenius_chi(1)}), it follows that 
$n_j\in\{3,4,5,6\}$. If some $n_j=6$, say without
loss of generality $n_3=6$, then 
\[
7=43-36=n_2^2+n_3^2, 
\]
a contradiction. Thus $n_j\in\{3,4,5\}$ for 
all $j\in\{3,4,5\}$. Without loss of generality, 
we may assume that $n_3=n_4=3$ and $n_5=5$. 

\bigskip 
\begin{center}
        \begin{tabular}{|c|ccccc|}
        \hline  
         & $\id$ & $(12)(34)$ & $(123)$ & $(12345)$ & $(12354)$\\
        \hline 
        $\chi_1$ & $1$ & $1$ & $1$ & $1$ & $1$\\
        $\chi_2$ & $4$ & $0$ & $1$ & $-1$ & $-1$\\
        $\chi_3$ & $3$ & $\cdot$ & $\cdot$ & $\cdot$& $\cdot$\\
        $\chi_4$ & $3$ & $\cdot$ & $\cdot$ & $\cdot$& $\cdot$\\
        $\chi_5$ & $5$ & $\cdot$ & $\cdot$ & $\cdot$& $\cdot$\\
        \hline 
    \end{tabular}
\end{center}
\bigskip 

The group $\Alt_5$ acts on the set $Y$ of subsets 
of $\{1,2,\dots,5\}$ of two elements, namely
\[
g\cdot \{a,b\}=\{g\cdot a,g\cdot b\}.
\]
Note that $|Y|=\binom{5}{2}=10$. Moreover, 
this action is transitive. Let us compute 
the character $\psi$ of the corresponding 
$\C\Alt_5$-module and the difference 
$\psi-\chi_1$ (We know $\psi$ counts
fixed points.)

\bigskip 
\begin{center}
        \begin{tabular}{|c|ccccc|}
        \hline  
         & $\id$ & $(12)(34)$ & $(123)$ & $(12345)$ & $(12354)$\\
        \hline 
        $\psi$ & $10$ & $2$ & $1$ & $0$ & $0$\\
        $\psi-\chi_1$ & $9$ & $1$ & $0$ & $-1$ & $-1$\\
        \hline 
    \end{tabular}
\end{center}
\bigskip 

The identity, of course, fixes all the ten elements
of $Y$. The permutation 
$(12)(34)$ fixed two two-elements subsets, namely
$\{1,2\}$ and $\{3,4\}$. The permutation 
$(123)$ fixes only one two-elements subset, namely
$\{4,5\}$. Finally, $(12345)$ and 
$(12354)$ fix no two-element subsets. 

Now we compute 
\[
\langle \psi-\chi_1,\psi-\chi_1\rangle=2
\]
and hence $\psi-\chi_1$ is the sum of two irreducible
characters (see Exercise~\ref{xca:n_irreducible}). Since
\[
\langle \psi-\chi_1,\chi_2\rangle=1,
\]
it follows that $\psi-\chi_1-\chi_2\Irr(G)$. Let 
$\chi_5=\psi-\chi_1-\chi_2$. Then

\bigskip 
\begin{center}
        \begin{tabular}{|c|ccccc|}
        \hline  
         & $\id$ & $(12)(34)$ & $(123)$ & $(12345)$ & $(12354)$\\
        \hline 
        $\chi_1$ & $1$ & $1$ & $1$ & $1$ & $1$\\
        $\chi_2$ & $4$ & $0$ & $1$ & $-1$ & $-1$\\
        $\chi_3$ & $3$ & $\cdot$ & $\cdot$ & $\cdot$& $\cdot$\\
        $\chi_4$ & $3$ & $\cdot$ & $\cdot$ & $\cdot$& $\cdot$\\
        $\chi_5$ & $5$ & $1$ & $-1$ & $0$& $0$\\
        \hline 
    \end{tabular}
\end{center}
\bigskip 

Let $K=\langle(12345)\rangle$ and 
$\eta\in\Irr(K)$ be such that $\eta((12345))=\zeta$, where
$\zeta=\exp(2\pi i/5)$ is a primitive $5$-th root of one. We can then compute 
$\Ind_K^G$. 

\bigskip 
\begin{center}
        \begin{tabular}{|c|ccccc|}
        \hline  
         & $\id$ & $(12)(34)$ & $(123)$ & $(12345)$ & $(12354)$\\
         \hline 
         $\Ind_K^G\psi$ & $12$ & $0$ & $0$ & $\zeta^2+\zeta^3$ & $\zeta+\zeta^4$\\
         \hline 
\end{tabular}
\end{center}
\bigskip 

Since 
$\langle\Ind_K^G\psi,\chi_2\rangle=1=\langle\Ind_H^G,\chi_5\rangle$,
it follows that 
\bigskip 
\begin{center}
        \begin{tabular}{|c|ccccc|}
        \hline  
         & $\id$ & $(12)(34)$ & $(123)$ & $(12345)$ & $(12354)$\\
         \hline 
         $\Ind_K^G\psi-\chi_2-\chi_5$ & $3$ & $-1$ & $0$ & $-\zeta-\zeta^4$ & $-\zeta^2-\zeta^3$\\
         \hline 
\end{tabular}
\end{center}
\bigskip 
Let $\chi_3=\Ind_K^G\psi-\chi_2-\chi_5$. Then $\chi_3\in\Irr(G)$, because it is
not the sum of three copies of the trivial character. 
\bigskip
\begin{center}
        \begin{tabular}{|c|ccccc|}
        \hline  
         & $\id$ & $(12)(34)$ & $(123)$ & $(12345)$ & $(12354)$\\
        \hline 
        $\chi_1$ & $1$ & $1$ & $1$ & $1$ & $1$\\
        $\chi_2$ & $4$ & $0$ & $1$ & $-1$ & $-1$\\
        $\chi_3$ & $3$ & $-1$ & $0$ & $-\zeta-\zeta^4$ & $-\zeta^2-\zeta^3$\\
        $\chi_4$ & $3$ & $\cdot$ & $\cdot$ & $\cdot$& $\cdot$\\
        $\chi_5$ & $5$ & $1$ & $-1$ & $0$& $0$\\
        \hline 
    \end{tabular}
\end{center}
\bigskip 

\begin{exercise}
    Use the orthogonality relations
    to compute the missing row of the character table
    of $\Alt_5$. 
\end{exercise}

The previous exercise finishes the calculation
of the character table of $\Alt_5$; see Table~\ref{tab:A5}. 


\begin{table}[h]
\caption{The character table of $\Alt_5$.}
\label{tab:A5}
        \begin{tabular}{|c|ccccc|}
        \hline  
        & $1$ & $15$ & $20$ & $12$ & $12$ \\
         & $\id$ & $(12)(34)$ & $(123)$ & $(12345)$ & $(12354)$\\
        \hline 
        $\chi_1$ & $1$ & $1$ & $1$ & $1$ & $1$\\
        $\chi_2$ & $4$ & $0$ & $1$ & $-1$ & $-1$\\
        $\chi_3$ & $3$ & $-1$ & $0$ & $-\zeta-\zeta^4$ & $-\zeta^2-\zeta^3$\\
        $\chi_4$ & $3$ &  $-1$ & $0$ & $-\zeta^2-\zeta^3$ & $-\zeta-\zeta^4$ \\
        $\chi_5$ & $5$ & $1$ & $-1$ & $0$& $0$\\
        \hline 
    \end{tabular}
\end{table}

One last observation: 
Since $\zeta=\exp(2\pi i/5)$, it follows
that 
\[
-\zeta-\zeta^4=\frac{1-\sqrt{5}}{2},
\quad 
-\zeta^2-\zeta^3=\frac{1+\sqrt{5}}{2}.
\]

% Let us see what Magma says:

% \begin{lstlisting}
% > CharacterTable(Alt(5));


% Character Table
% ---------------


% ---------------------------
% Class |   1  2  3    4    5
% Size  |   1 15 20   12   12
% Order |   1  2  3    5    5
% ---------------------------
% p  =  2   1  1  3    5    4
% p  =  3   1  2  1    5    4
% p  =  5   1  2  3    1    1
% ---------------------------
% X.1   +   1  1  1    1    1
% X.2   +   3 -1  0   Z1 Z1#2
% X.3   +   3 -1  0 Z1#2   Z1
% X.4   +   4  0  1   -1   -1
% X.5   +   5  1 -1    0    0


% Explanation of Character Value Symbols
% --------------------------------------

% # denotes algebraic conjugation, that is,
% #k indicates replacing the root of unity w by w^k

% Z1     = (CyclotomicField(5: Sparse := true)) ! [ RationalField() | 0, 0, -1, -1 ]    
% \end{lstlisting}