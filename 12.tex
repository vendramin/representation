\section{Lecture: Week 12}

\subsection{Clifford theory}

We begin with a routine exercise. 

%\begin{exercise}
%$    Let $G$ be a finite group and $\chi\in\Char(G)$. 
%$    Let $f\colon G\to H$ be an isomorphism of 
%$    groups. Then 
%$    \[
%$    (f\cdot )(f(g))
%$    
%\end{exercise}

\begin{exercise}
\label{xca:conjugate_chars1}
Let $G$ be a finite group and $N$ be a normal subgroup
of $G$. Prove that $G$ acts on $\Irr(N)$ via 
\[
(g\cdot\theta)(n)=\theta(g^{-1}ng),\quad 
g\in G,\theta\in\Irr(N),n\in N.
\]
\end{exercise}

\begin{exercise}
\label{xca:conjugate_chars2}
Let $G$ be a finite group and $N$ be a normal subgroup of $G$. 
Let $\chi\in\cf(G)$, $\theta\in\cf(N)$ and $g\in G$. Prove that
\[
\langle\Res_N^G\chi,g\cdot\theta\rangle=\langle\Res_N^G\chi,\theta\rangle.
\]
% for all $\chi\in\cf(G)$.
%     \begin{enumerate}
%         \item $\langle g\cdot\alpha,g\cdot\beta\rangle=\langle\alpha,\beta\rangle$.
%         \item $\langle\Res_N^G\chi,g\cdot\alpha\rangle=\langle\Res_N^G\chi,\alpha\rangle$ for all $\chi\in\cf(G)$. 
%     \end{enumerate}
\end{exercise}

\index{Irreducible constituent}
Recall that every character $\chi$ of a finite group is uniquely 
a sum of irreducible characters. These are called
the \emph{irreducible constituents} of $\chi$. The set 
of irreducible constituents of $\chi$ is the set  
\[
\{\eta\in\Irr(G):\langle\chi,\eta\rangle>0\}.
\]

\begin{theorem}[Clifford]
\label{thm:Clifford}
\index{Clifford's theorem}
    Let $G$ be a finite group and $N$ be a normal
    subgroup of $G$. Let $\chi\in\Irr(G)$ and $\theta\in\Irr(N)$ be 
    an irreducible constituent of $\Res_N^G\chi$. 
    Then 
    \[
    \Res_N^G\chi = e(\theta_1+\cdots+\theta_t),
    \]
    where $\theta=\theta_1,\dots,\theta_t$ are the conjugates 
    of $\theta$ in $G$, 
    and $e=\langle\Res_N^G\chi,\theta\rangle$ is a positive integer. In particular, all the constituents of $\Res_N^G\chi$ have the same degree. 
\end{theorem}

\begin{proof}
    We claim that 
    \[
    \Res_N^G\Ind_N^G\theta=\frac{1}{|N|}\sum_{g\in G}(g\cdot\theta).
    \]
    To prove this formula, let $n\in N$. Then 
    \[
    \left(\Ind_N^G\theta\right)(n)
    =\frac{1}{|N|}\sum_{g\in G}\theta^0(g^{-1}ng)
    =\frac{1}{|N|}\sum_{g\in G}\theta(g^{-1}ng)
    =\frac{1}{|N|}\sum_{g\in G}(g\cdot\theta)(n).
    \]
    By Frobenius' reciprocity, 
    $\langle\Ind_N^G\theta,\chi\rangle=\langle\theta,\Res_N^G\chi\rangle>0$. 
    Thus $\chi$ is an irreducible constituent of $\Ind_N^G\theta$. 
    
    % Moreover, $g\cdot\chi=\chi$ 
    % for all $g\in G$, since $\chi\in\Irr(G)\subseteq \cf(G)$. 
    Now 
    let $\varphi\in\Irr(N)\setminus\{\theta_1,\dots,\theta_t\}$. Then
    \[
    \langle \Ind_N^G\theta, \Ind_N^G\varphi\rangle
    =\langle \Res_N^G\Ind_N^G\theta,\varphi\rangle
    =\sum_{g\in G}\langle g\cdot\theta,\varphi\rangle=0,
    \]
    since each $g\cdot\theta=\theta_j$ for some $j\in\{1,\dots,t\}$. 
    It follows that $\langle\Res_N^G\chi,\varphi\rangle=0$. 
    Thus all irreducible constituents of $\Res_N^G\chi$ belong
    to $\{\theta_1,\dots,\theta_t\}$, that is
    \[
\Res_N^G\chi=\sum_{i=1}^t\langle\Res_N^G\chi,\theta_i\rangle\theta_i.
    \]
    Moreover, for each $i\in\{1,\dots,t\}$, 
    there exists $g_i\in G$ such that
    $\theta_i=g_i\cdot\theta$. Thus, using Exercise~\ref{xca:conjugate_chars2}, 
    \[
    \langle\Res_N^G\chi,\theta_i\rangle=\langle\Res_N^G\chi,g_i\cdot\theta\rangle=\langle\Res_N^G\chi,\theta\rangle=e.\]
    From this the theorem follows. 
\end{proof}

\index{Ramification index}
The integer $e$ in Theorem~\ref{thm:Clifford} is known as the \emph{ramification index} of $\chi$ on $N$. In general, the number $e$ is not easy to control. 

\begin{exercise}
\label{xca:Clifford_divisibility}
    Let $G$ be a finite group and $N$ be a normal subgroup of $G$. Let $\chi\in\Irr(G)$ and $\theta$ 
    be an irreducible constituent of $\Res_N^G\chi$. 
    Prove that $\theta(1)$ divides $\chi(1)$. 
\end{exercise}

\index{Inertia subgroup}
Let $G$ be a group and $\theta\in\Irr(G)$. The set 
\[
I_G(\theta)=\{g\in G:g\cdot\theta=\theta\}
\]
is a subgroup of $G$ and is called \emph{inertia subgroup} of $\theta$ in $G$. Note that the inertia
subgroup is the stabilizer of the action of $G$ 
on characters by conjugation (see 
of Exercise~\ref{xca:conjugate_chars1}). In particular, 
$\theta$ has 
$(G:I_G(\theta))$ conjugates. 

\begin{theorem}[Clifford correspondence]
    Let $G$ be a finite group and $N$ be a normal subgroup of $G$. Let $\theta\in\Irr(N)$ and $I=I_G(\theta)$.  Then 
    the map 
    \[
    \{\psi\in\Irr(I):\langle\Res_N^I\psi,\theta\rangle>0\}\to 
    \{\chi\in\Irr(G):\langle\Res_N^G\chi,\theta\rangle>0\},\quad 
    \psi\mapsto\Ind_I^G\psi,
    \]
    is bijective. Moreover, if $\psi$ is a constituent of $\Res_N^I\theta$, then 
    $\langle\Res_N^I\psi,\theta\rangle=\langle\Res_N^G\chi,\theta\rangle$. 
\end{theorem}

\begin{proof}
    There are several things to prove. 

    \begin{claim}
        The map is $\psi\mapsto\Ind_I^G\psi$ well-defined. 
    \end{claim}

    Let $\psi\in\Irr(I)$ be such that $e=\langle\Res_N^I\psi,\theta\rangle>0$ and 
    let $\chi\in\Irr(G)$ be a constituent of $\Ind_I^G\psi$. By Frobenius' reciprocity, 
    \[
    \langle\psi,\Res_I^G\chi\rangle=\langle\Ind_I^G\psi,\chi\rangle\ne0.
    \]
    Thus $\psi$ is a constituent of $\Res_I^G\chi$. 
    This implies that 
    $\Res_N^I\psi$ is a constituent of $\Res_N^I\Res_I^G\chi$. 
    Moreover, 
    \[
    \chi(1)\leq(\Ind_I^G\psi)(1)=(G:I)\psi(1).
    \]
    Let $f=\langle\Res_N^G\chi,\theta\rangle$. Then  
    \[
    f=\langle\Res_N^G\chi,\theta\rangle=\langle\Res_N^I\Res_I^G\chi,\theta\rangle
    \geq\langle\Res_N^I\psi,\theta\rangle=e>0.
    \]
    Since $\Res_N^G\chi=f(\theta_1+\cdots+\theta_t)$, where $G\cdot\theta=\{\theta_1,\dots,\theta_t\}$ is 
    the orbit of $\theta$ under the action of $G$ and $t=(G:I)$,  
    \[
    ft\theta(1)=\chi(1)=(\Res_N^G\chi)(1)\leq(\Ind_I^G\psi)(1)=t\psi(1)=et\theta(1)\leq ft\theta(1),
    \]
    where the last equality follows since 
    $\Res_N^I\psi=e\theta$ by Clifford's theorem. Therefore $e=f$ and 
    $\Ind_I^G\psi=\chi$. 
\end{proof}

We now present several consequences of Clifford’s theorem.

\begin{theorem}
    Let $G$ be a finite group and $N$ be a normal subgroup of prime index $p$. Then either $\Res_N^G\chi$ is a sum of $p$ 
    distinct irreducible characters or $\Res_N^G\chi$ is irreducible. 
\end{theorem}

\begin{proof}
    Note that $t$ divides $(G:N)$ (see Exercise~\ref{xca:t|G/N}). 
    So there are two cases
    to consider, namely $t\in\{1,p\}$. 

    Assume first that $t=p$. Let $\theta$ be an irreducible constituent of $\Res_N^G\chi$. Then $e=\langle\Res_N^G\chi,\theta\rangle$.... 
\end{proof}

To simplify our presentation, we will introduce an additional assumption regarding the solvability of a certain quotient group. However, the results presented remain valid even without this assumption.

\begin{theorem}
    Let $G$ be a finite group and $N$ be a normal subgroup of $G$. Assume that $G/N$ is solvable. Let $\chi\in\Irr(G)$ 
    and $\theta$ be an irreducible constituent of $\Res_N^G\chi$. Then 
    $\frac{\chi(1)}{\theta(1)}$ divides $(G:N)$. 
\end{theorem}

\begin{proof}
    By Clifford's theorem, $\chi(1)/\theta(1)=et\in\Z$. Assume 
    that the theorem does not hold and let $G$ 
    be a minimal counterexample. 
    If $(G:N)=1$, then the result trivially holds. Thus $N\ne G$. 

    We first assume that $G/N$ is simple. Since $G/N$ is solvable, 
    then $G/N\simeq\Z/p$ for some prime number $p$. Thus 
    $\chi(1)/\theta(1)\in\{1,p\}$. 

    Assume now that $G/N$ is not simple. Let $M$ be a normal
    subgroup of $G$ such that $N\leq M\leq G$. Let 
    $\psi\Irr(M)$ be such that...

    By the inductive hypothesis, 
    $\chi(1)/\psi(1)$ divides $(G:M)$ and 
    $\psi(1)/\theta(1)$ divides $(M:N)$. Thus 
    \[
    \frac{\chi(1)}{\theta(1)}=\frac{\chi(1)}{\psi(1)}\frac{\psi(1)}{\theta(1)}
    \]
    divides $(G:M)(M:N)=(G:N)$. 
\end{proof}

We now present a weaker version of Ito’s theorem (Theorem~\ref{thm:Ito}) as a corollary of Clifford’s theorem and
the previous results. 

\begin{theorem}[Ito]
\index{Ito's theorem}
    Let $G$ be a finite group and $A$ be an abelian normal 
    subgroup of $G$. Assume that $G/A$ is solvable. 
    If $\chi\in\Irr(G)$, then $\chi(1)$ divides $(G:A)$. 
\end{theorem}

\begin{proof}
    Let $\theta$ be an irreducible constituent of $\Res_A^G\chi$. 
    Since $A$ is abelian, $\theta(1)=1$. Thus 
    $\chi(1)=\chi(1)/\theta(1)$ divides $(G:A)$ by 
    the previous theorem. 
\end{proof}