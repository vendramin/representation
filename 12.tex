\chapter{}

\topic{Lie algebras}

\begin{definition}
    \index{Lie algebra}
    Let $K$ be a field. 
    A \textbf{Lie algebra} (over $K$) is a $K$-vector space
    $L$ together with a bilinear map 
    $L\times L\to L$, $(x,y)\mapsto [x,y]$,
    such that
    \begin{align}
        \label{eq:[xx]=0}&[x,x]=0\quad\text{for all $x\in L$},\\ 
        \label{eq:Jacobi}&[x,[y,z]]+[y,[z,x]]+[z,[x,y]]=0\quad\text{for all $x,y,z\in L$}.
    \end{align}
\end{definition}

\index{Jacobi identity}
Equality \eqref{eq:Jacobi} is known as the \textbf{Jacobi identity}. 

\begin{exercise}
    Prove that \eqref{eq:[xx]=0} implies $[x,y]=-[y,x]$ for all
    $x,y\in L$. 
\end{exercise}

\index{Abelian Lie algebra}
A Lie algebra $L$ is said to be \textbf{abelian} if $[x,y]=0$ for 
all $x,y\in L$. 

\begin{exercise}
    If $L$ and $L_1$ are Lie algebras, then 
    $L\oplus L_1$ is a Lie algebra with
    $[(x,x_1),(y,y_1)]=([x,y),(x_1,y_1)]$ for $x,y\in L$ and
    $x_1,y_1\in L_1$. 
\end{exercise}

\begin{exercise}
    Prove that $\R^3$ with the usual vector product 
    \[
    [(x_1,x_2,x_3),(y_1,y_2,y_3)]=(x_2y_3-x_3y_2,x_3y_1-x_1y_3,x_1y_2-x_2y_1)
    \]
    is a (real) Lie algebra.     
\end{exercise}

We will main work with finite-dimensional complex Lie algebras.

\begin{example}[general linear Lie algebra]
\index{General linear Lie algebra}
    Let $V$ be a finite-dimensional vector space and 
    $\gl(V)$ be the set of linear maps $V\to V$. Then 
    $\gl(V)$ with $[x,y]=xy-yx$ is a Lie algebra. 
\end{example}

A matrix version of the previous example: We write $\gl(n,\C)$ 
to denote the vector space of all $n\times n$ complex 
matrices with Lie bracket $[x,y]=xy-yx$. The vector space
$\gl(n,\C)$ has a basis $\{e_{ij}:1\leq i,j\leq n\}$, where
\[
(e_{ij})_{kl}=\begin{cases} 
    1 & \text{if $(i,j)=(k,l)$},\\
    0 & \text{otherwise}.
    \end{cases}
\]

\begin{exercise}
    Compute $[e_{ij},e_{ik}]$.
\end{exercise}

\begin{example}[special linear Lie algebra]
\index{Special linear Lie algebra}
    Let $\sl(n,\C)$ be the subspace of $\gl(n,\C)$ consisting
    of all matrices with trace zero. 
\end{example}

\begin{exercise}
    Find a basis of $\sl(n,\C)$. 
\end{exercise}

%\begin{example}
%    Let $\bl(n,\C)$ be the subspace of all upper triangular matrices
%    in $\gl(n,\C)$. Then $\bl(n,\C)$ is a Lie algebra. 
%\end{example}

\begin{definition}
    \index{Lie!subalgebra}
    A Lie \textbf{subalgebra} of $L$ is a vector space $L_1$ of $L$ 
    such that $[x,y]\in L_1$ for all $x,y\in L_1$. 
\end{definition}

Of course, $\sl(n,\C)$ is a subalgebra of $\gl(n,\C)$. 

\begin{definition}
\index{Ideal!of a Lie algebra}
    An \textbf{ideal} of a Lie algebra $L$ is a subspace $I$ of $L$ 
    such that $[x,y]\in I$ for all $x\in L$ and $y\in I$. 
\end{definition}

Trivial examples of ideals of a Lie algebra $L$ are
$\{0\}$ and $L$.

\begin{example}
\index{Center!of a Lie algebra}
    Let $L$ be a Lie algebra. Then 
    the \textbf{center} 
    \[
    Z(L)=\{x\in L:[x,y]=0\text{ for all $y\in L$}\}.
    \]
    is an ideal of $L$. 
\end{example}

\begin{example}
\index{Derived algebra!of a Lie algebra}
    Let $L$ be a Lie algebra. 
    The \textbf{derived algebra} $[L,L]$
    consists of all linear combinations of commutators $[x,y]$ 
    is an ideal of $L$. 
\end{example}

\begin{exercise}
    Compute $Z(\sl(n,\C))$. 
\end{exercise}

\begin{exercise}
    Prove that $\sl(2,\C)$ has no non-trivial ideals. 
\end{exercise}

One easily checks that $\sl(n,\C)$ is an ideal of $\gl(n,\C)$. In fact, 
an ideal is always a subalgebra. The converse is not true. 
Can you find an example?

\begin{definition}
\index{Homomorphism!of Lie algebras}
    Let $L$ and $L_1$ be Lie algebras. A map $f\colon L\to L_1$ is a 
    \textbf{Lie algebra homomorphism} if $f([x,y])=[f(x),f(y)]$ for all
    $x,y\in L$. 
\end{definition}

As usual, an isomorphism between Lie algebras will be
a bijective homomorphism of Lie algebras. 

\begin{example}
    Let $L$ and $L_1$ be Lie algebras. The canonical injections
    $L\to L\oplus L_1$ and $L_1\to L_\oplus L_1$ and
    the canonical surjections $L\oplus L_1\to L$ and 
    $L\oplus L_1\to L_1$ are Lie algebras homomorphisms.  
\end{example}

\begin{example}
    Let $L$ be a Lie algebra. The \textbf{opposite Lie algebra} 
    $L^{\op}$ is the vector space $L$ with 
    $[x,y]^{\op}=-[x,y]$. Then $L\to L^{\op}$, $x\mapsto -x$, 
    is an isomorphism of Lie algebras.
\end{example}

\begin{exercise}
    Let $f\colon L\to L_1$ be a Lie algebra homomorphism. Prove
    that the \textbf{kernel} of $f$, 
    $\ker f=\{x\in L:f(x)=0\}$ is an ideal
    of $L$, and that the \textbf{image} of $f$ 
    is a subalgebra of $L_1$. 
\end{exercise}

\begin{example}
\index{Adjoint homomorphism}
    Let $L$ be a Lie algebra. 
    The \textbf{adjoint homomorphism} is the map 
    \[
    \ad\colon L\to\gl(L),\quad
    (\ad x)(y)=[x,y].
    \]
\end{example}

\begin{exercise}
    Let $x=\begin{pmatrix}
        0&1\\
        0&0\end{pmatrix}$, $h=\begin{pmatrix}
        1&0\\
        0&-1\end{pmatrix}$ and $y=\begin{pmatrix}0&0\\1&0\end{pmatrix}$ 
        be an ordered basis for $\sl(2,\C)$. Compute the matrices
        of $\ad x$, $\ad h$ and $\ad y$ with respect to this basis. 
\end{exercise}

Let $L$ be a Lie algebra and $I$ be an ideal of $L$. Then 
the quotient vector space $L/I$ is a Lie algebra
with $[x+I,y+I]=[x,y]+I$. The canonical map 
$L\to L/I$, $x\mapsto x+I$, 
is a surjective Lie algebra homomorphism. 

\begin{exercise}
    Let $f\colon L\to L_1$ be a Lie algebra homomorphism.
    Prove that $f/\ker f\simeq f(L)$. 
\end{exercise}

\begin{definition}
\index{Simple Lie algebra}
    A Lie algebra $L$ is said to be \textbf{simple} if 
    $[L,L]\ne\{0\}$ and $\{0\}$ and $L$ are the only ideals of $L$. 
\end{definition}

If $L$ is a simple Lie algebra, then $Z(L)=\{0\}$ and $L=[L,L]$. 

\begin{exercise}
    Prove that every simple Lie algebra is isomorphic to 
    a linear Lie algebra. 
\end{exercise}

\topic{Representations of Lie algebras}

\begin{definition}
\index{Representation!of a Lie algebra}
    A \textbf{representation} of a Lie algebra $L$ 
    is a Lie homomorphism $\rho\colon L\to\gl(V)$, where $V$ is a vector space. 
\end{definition}

\topic{Enveloping algebras}

