\section{Lecture: Week ?}

\subsection{Categories}

\begin{definition}
    \index{Category}    
    \label{defn:category}
    A \emph{category} $\cC$ is a class of elements, called 
	\textbf{objects} together with a set
    $\cC(A,B)$ of \textbf{morphisms} 
    for every pair of objects $A$ and $B$, 
    such that
    the following properties hold:
    \begin{enumerate}
		\item If $f\in\Hom_\cC(A,B)$ and 
			$g\in\cC(B,C)$, the composition $g\circ f\in\cC(A,C)$ is defined. 
		\item The composition of morphisms is associative: if 
			$f\in\cC(A,B)$, $g\in\cC(B,C)$ and $h\in\cC(C,D)$, then 
			$f\circ(g\circ h)=(f\circ g)\circ h$.
		\item For each object $A$ of $\cC$ there exists       the \textbf{identity morphism}    $\id_A\in\cC(A,A)$ satisfying $f\circ \id_A=f$ and 
			$\id_A\circ g=g$ for all $f\in\cC(A,B)$ and 
			$g\in\Hom_\cC(B,A)$.
	\end{enumerate}
\end{definition}

Usually we write $X\in\cC$ to mean that
$X$ is an object of the category $\cC$. 

Definition~\ref{defn:category}, where the morphism 
between two objects form a set, is the definition 
of a \emph{locally small category}. In this course, 
our categories 
will always be locally small. 

Note that we needed to use
the word \textit{class} and we cannot simply use 
sets. This is because there is no set of all sets (assuming this exists will produce a contradiction). We use, however, sets of morphisms between two given objects.

The associative law used in Definition~\ref{defn:category} implies the \emph{general 
associative law}. For example, the associativity 
condition implies 
\[
    (x_1((x_2x_3)x_4))x_5=(x_1x_2)((x_3x_4)x_5).
\]

\begin{bonus}
\label{xca:Catalan}
\index{Catalan number}
    For $n+1$ letters, prove that the number of different possible 
    brackets is equal to the \emph{Catalan number}
    \[
    C_n=\frac{1}{n+1}\binom{2n}{n}.
    \]
\end{bonus}

\begin{example}
    The category $\Sets$ of sets; the morphisms are
    arbitrary maps. 
\end{example}

\begin{example}
    The categories $\Groups$ and 
    $\Rings$; the morphisms are the homomorphisms. 
\end{example}

\begin{example}
    Let $K$ be a field. The category $\Vect_K$ 
    of $K$-vector spaces; the morphisms are 
    the $K$-linear transformations. 
\end{example}

\begin{example}
    Let $A$ be an algebra. The category 
    $\mathrm{Rep}(A)$ of representations of $A$. 
\end{example}

\begin{example}
\index{Category!opposite}
\index{Category!dual}
    Let $\cC$ be a category. The \emph{opposite category} $\cC^{\op}$ is the category 
    $\cC$ after reversing the arrows. This means
    that the morphism $A\to B$ in $\cC$ correspond
    to the morphism $B\to A$ in $\cC^{\op}$. 
\end{example}

\begin{definition}
\index{Isomorphism}
    Let $\cC$ be a category. A morphism $f\in\cC(A,B)$ 
    is an \emph{isomorphism} if there exists 
    $g\in\cC(B,A)$ such that $gf=\id_A$ and 
    $fg=\id_B$. 
\end{definition}

A straightforward computation shows that 
isomorphisms in $\Sets$ are bijections. Similarly, 
isomorphisms in $\Groups$ are group isomorphisms. 

\begin{example}
\index{Product of categories}
    Let $\cC$ and $\cD$ be categories. 
    The \emph{product category} $\cC\times\cD$ 
    is the category whose objects
    are pairs $(X,Y)$, where $X\in\cC$ and $Y\in\cD$, 
    and whose morphisms 
    are pairs $(f,g)$...
\end{example}

Not all examples of interesting categories 
are of categories whose objects are sets structure
and morphisms are maps preserving the structure. Objects
need not to be sets and morphisms 
need not to be functions! 

\begin{example}
    A group is a category $\cC$ with only one object, say $X$, 
    and in which all morhisms are isomorphisms. In this category, 
    the morphisms are the elements of the group,  the composition
    of morphisms is the group operation and the identity map
    $\id_X$ is the neutral element of the group. 
x\end{example}

\begin{example}
The picture 
\[\begin{tikzcd}
	X & Y
	\arrow[from=1-1, to=1-2]
\end{tikzcd}\]
represents a category with only two objects, say $X$ and $Y$, 
and only one
non-identity morphism $X\to Y$. Here the letters $X$ and $Y$ 
used for
our objects play no role whatsoever. We could then 
draw the category as follows:
\[\begin{tikzcd}
	\bullet & \bullet
	\arrow[from=1-1, to=1-2]
\end{tikzcd}\]
\end{example}

\section{Functors}

\begin{definition}
    Let $\cC$ and $\cD$ be categories. A functor $F\colon\cC\to\cD$ 
    consists of a map 
    $\operatorname{ob}(\cC)\to\operatorname{ob}(\cD)$, $X\mapsto F(X)$, and 
    a map 
\end{definition}

\begin{example}
    The following are examples of \emph{forgetful functors}:
    \begin{enumerate}
        \item $\Groups\to\Sets$.
        \item $\Rings\to\Sets$.
        \item $\Rings\to\AbelianGroups$.
        \item $\AbelianGroups\to\Groups$. 
    \end{enumerate}
\end{example}

\begin{definition}
    A \emph{contravariant functor} between 
    the categories $\cC$ and $\cD$ is 
    a functor $\cC^{\op}\to\cD$. 
\end{definition}

\section{Application: The Skolem--Noether theorem}

\begin{proposition}
\label{pro:equivalence}
    Let $K$ be a field. Then 
    $M_n(K)-\operatorname{Mod}\simeq\Vect_K$.
\end{proposition}

\begin{proof}
    
\end{proof}

\begin{definition}
    Let $R$ be a commutative ring. A category $\cC$ is said to be 
    \emph{$R$-linear} if the following conditions hold:
    \begin{enumerate}
        \item There exists a null object $0\in\cC$. 
        \item Every set $\cC(X,Y)$ is an $R$-module 
        such that the composition of morphisms is $R$-linear and 
        $\cC(0,0)=0$. 
        \item For $X,Y\in\cC$ there exists...
    \end{enumerate}
\end{definition}

A category is called \emph{additive} if it is $\Z$-linear. 

\begin{definition}
\index{Initial object}
    Let $\cC$ be a category. An object $I\in\cC$ is said to be
    an \emph{initial object} if for every $X\in\cC$ 
    there exists a unique morphism $I\to X$. 
\end{definition}

\index{Terminal object}
Similarly, one defines a \emph{terminal object} as 
an initial object of the opposite category. 

\begin{definition}
    Let $R$ be a commutative ring and 
    $\cC$ be an $R$-linear category. An object $X\in\cC$ is said 
    to be \emph{indecomposable} if $X\ne0$ and $X$ is not 
    isomorphic to a direct sum of two non-null objects. 
\end{definition}

\begin{theorem}[Skolem--Noether]
\index{Skolem--Noether theorem}
    Let $K$ be a field. Any $K$-algebra automorphism 
    if $M_n(K)$ is inner.
\end{theorem}

\begin{proof}
    Up to isomorphism, the category $\Vect_K$ has a unique indecomposable object, namely 
    the one-dimensional $K$-vector space. By the equivalence
    of Proposition~\ref{pro:equivalence}, 
    $M_n(K)-\operatorname{Mod}$ also has only one
    indecomposable object up to isomorphism, namely 
    $K^{n\times 1}$. 
    Let $\alpha$ be a $K$-algebra automorphism 
    of $M_n(K)$. If $X$ is any $M_n(K)$-module, 
    then $X^{\alpha}$, defined as 
    the $K$-vector space $X$ 
    with 
    \[
    a\cdot m=\alpha(a)\cdot m\quad a\in M_n(K),m\in M,
    \]
    is also a $M_n(K)$-module. In particular, 
    $(K^{n\times 1})^\alpha$ is an indecomposable 
    $M_n(K)$-module. Thus $(K^{n\times 1})^\alpha\simeq K^{n\times1}$. Let $g\colon K^{n\times 1}\to (K^{n\times 1})^\alpha$ be an 
    isomorphism of $M_n(K)$-modules. Then 
    \[
    g(a\cdot v)=\alpha(a)\cdot g(v)
    \]
    for all $a\in M_n(K)$ and $v\in K^{n\times 1}$. Hence 
    $\alpha(a)=gag^{-1}$ for all $a\in M_n(K)$. 
\end{proof}

