\chapter{}

\topic{Lie algebras}

\begin{definition}
    \index{Lie algebra}
    Let $K$ be a field. 
    A \textbf{Lie algebra} (over $K$) is a $K$-vector space
    $L$ together with a bilinear map 
    $L\times L\to L$, $(x,y)\mapsto [x,y]$,
    such that
    \begin{align}
        \label{eq:[xx]=0}&[x,x]=0\quad\text{for all $x\in L$},\\ 
        \label{eq:Jacobi}&[x,[y,z]]+[y,[z,x]]+[z,[x,y]]=0\quad\text{for all $x,y,z\in L$}.
    \end{align}
\end{definition}

\index{Jacobi identity}
Equality \eqref{eq:Jacobi} is known as the \textbf{Jacobi identity}. 

\begin{exercise}
    Prove that \eqref{eq:[xx]=0} implies $[x,y]=-[y,x]$ for all
    $x,y\in L$. 
\end{exercise}

\index{Abelian Lie algebra}
A Lie algebra $L$ is said to be \textbf{abelian} if $[x,y]=0$ for 
all $x,y\in L$. 

\begin{exercise}
    If $L$ and $L_1$ are Lie algebras, then 
    $L\oplus L_1$ is a Lie algebra with
    $[(x,x_1),(y,y_1)]=([x,y),(x_1,y_1)]$ for $x,y\in L$ and
    $x_1,y_1\in L_1$. 
\end{exercise}

\begin{exercise}
    Prove that $\R^3$ with the usual vector product 
    \[
    [(x_1,x_2,x_3),(y_1,y_2,y_3)]=(x_2y_3-x_3y_2,x_3y_1-x_1y_3,x_1y_2-x_2y_1)
    \]
    is a (real) Lie algebra.     
\end{exercise}

We will main work with finite-dimensional complex Lie algebras.

\begin{example}[general linear Lie algebra]
\index{General linear Lie algebra}
    Let $V$ be a finite-dimensional vector space and 
    $\gl(V)$ be the set of linear maps $V\to V$. Then 
    $\gl(V)$ with $[x,y]=xy-yx$ is a Lie algebra. 
\end{example}

A matrix version of the previous example: We write $\gl(n,\C)$ 
to denote the vector space of all $n\times n$ complex 
matrices with Lie bracket $[x,y]=xy-yx$. The vector space
$\gl(n,\C)$ has a basis $\{e_{ij}:1\leq i,j\leq n\}$, where
\[
(e_{ij})_{kl}=\begin{cases} 
    1 & \text{if $(i,j)=(k,l)$},\\
    0 & \text{otherwise}.
    \end{cases}
\]

\begin{exercise}
    Compute $[e_{ij},e_{ik}]$.
\end{exercise}

\begin{example}[special linear Lie algebra]
\index{Special linear Lie algebra}
    Let $\sl(n,\C)$ be the subspace of $\gl(n,\C)$ consisting
    of all matrices with trace zero. 
\end{example}

\begin{exercise}
    Find a basis of $\sl(n,\C)$. 
\end{exercise}

We discuss a particular important case,
\[
\sl(2,\C)=\left\{\begin{pmatrix}
    a & b\\
    c & -a
    \end{pmatrix}:a,b,c\in\C\right\}
\]
Note that 
$e=\begin{pmatrix}
        0&1\\
        0&0\end{pmatrix}$, $h=\begin{pmatrix}
        1&0\\
        0&-1\end{pmatrix}$ and $f=\begin{pmatrix}0&0\\1&0\end{pmatrix}$ 
        is an ordered basis for $\sl(2,\C)$. In this basis,
\[
[h,e]=2e,\quad
[h.f]=-2f,\quad
[e,f]=h.
\]

%\begin{example}
%    Let $\bl(n,\C)$ be the subspace of all upper triangular matrices
%    in $\gl(n,\C)$. Then $\bl(n,\C)$ is a Lie algebra. 
%\end{example}

\begin{definition}
    \index{Lie!subalgebra}
    A Lie \textbf{subalgebra} of $L$ is a vector space $L_1$ of $L$ 
    such that $[x,y]\in L_1$ for all $x,y\in L_1$. 
\end{definition}

Of course, $\sl(n,\C)$ is a subalgebra of $\gl(n,\C)$. 

\begin{definition}
\index{Ideal!of a Lie algebra}
    An \textbf{ideal} of a Lie algebra $L$ is a subspace $I$ of $L$ 
    such that $[x,y]\in I$ for all $x\in L$ and $y\in I$. 
\end{definition}

Trivial examples of ideals of a Lie algebra $L$ are
$\{0\}$ and $L$.

\begin{example}
\index{Center!of a Lie algebra}
    Let $L$ be a Lie algebra. Then 
    the \textbf{center} 
    \[
    Z(L)=\{x\in L:[x,y]=0\text{ for all $y\in L$}\}.
    \]
    is an ideal of $L$. 
\end{example}

\begin{example}
\index{Derived algebra!of a Lie algebra}
    Let $L$ be a Lie algebra. 
    The \textbf{derived algebra} $[L,L]$
    consists of all linear combinations of commutators $[x,y]$ 
    is an ideal of $L$. 
\end{example}

\begin{exercise}
    Compute $Z(\sl(n,\C))$. 
\end{exercise}

\begin{exercise}
    Prove that $\sl(2,\C)$ has no non-trivial ideals. 
\end{exercise}

One easily checks that $\sl(n,\C)$ is an ideal of $\gl(n,\C)$. In fact, 
an ideal is always a subalgebra. The converse is not true. 
Can you find an example?

\begin{definition}
\index{Homomorphism!of Lie algebras}
    Let $L$ and $L_1$ be Lie algebras. A map $f\colon L\to L_1$ is a 
    \textbf{Lie algebra homomorphism} if $f([x,y])=[f(x),f(y)]$ for all
    $x,y\in L$. 
\end{definition}

As usual, an isomorphism between Lie algebras will be
a bijective homomorphism of Lie algebras. 

\begin{example}
    Let $L$ and $L_1$ be Lie algebras. The canonical injections
    $L\to L\oplus L_1$ and $L_1\to L_\oplus L_1$ and
    the canonical surjections $L\oplus L_1\to L$ and 
    $L\oplus L_1\to L_1$ are Lie algebras homomorphisms.  
\end{example}

\begin{example}
    Let $L$ be a Lie algebra. The \textbf{opposite Lie algebra} 
    $L^{\op}$ is the vector space $L$ with 
    $[x,y]^{\op}=-[x,y]$. Then $L\to L^{\op}$, $x\mapsto -x$, 
    is an isomorphism of Lie algebras.
\end{example}

\begin{exercise}
    Let $f\colon L\to L_1$ be a Lie algebra homomorphism. Prove
    that the \textbf{kernel} of $f$, 
    $\ker f=\{x\in L:f(x)=0\}$ is an ideal
    of $L$, and that the \textbf{image} of $f$ 
    is a subalgebra of $L_1$. 
\end{exercise}

\begin{example}
\index{Adjoint homomorphism}
    Let $L$ be a Lie algebra. 
    The \textbf{adjoint homomorphism} is the map 
    \[
    \ad\colon L\to\gl(L),\quad
    (\ad x)(y)=[x,y].
    \]
\end{example}

Let $L$ be a Lie algebra and $I$ be an ideal of $L$. Then 
the quotient vector space $L/I$ is a Lie algebra
with $[x+I,y+I]=[x,y]+I$. The canonical map 
$L\to L/I$, $x\mapsto x+I$, 
is a surjective Lie algebra homomorphism. 

\begin{exercise}
    Let $f\colon L\to L_1$ be a Lie algebra homomorphism.
    Prove that $f/\ker f\simeq f(L)$. 
\end{exercise}

\begin{definition}
\index{Simple Lie algebra}
    A Lie algebra $L$ is said to be \textbf{simple} if 
    $[L,L]\ne\{0\}$ and $\{0\}$ and $L$ are the only ideals of $L$. 
\end{definition}

If $L$ is a simple Lie algebra, then $Z(L)=\{0\}$ and $L=[L,L]$. 

\begin{exercise}
    Prove that every simple Lie algebra is isomorphic to 
    a linear Lie algebra. 
\end{exercise}

\topic{Representations of Lie algebras}

\begin{definition}
\index{Representation!of a Lie algebra}
    A \textbf{representation} of a Lie algebra $L$ 
    is a Lie homomorphism $\rho\colon L\to\gl(V)$, where $V$ is a vector space. 
\end{definition}

If $L\to\gl(V)$ is a representation of a Lie algebra $L$, 
fixing a basis for $V$ 
we obtain a \textbf{matrix representation}
$L\to\gl(n,\C)$. 

\begin{example}
    Let $L$ be a Lie algebra. 
    The map $\ad\colon L\to\gl(L)$, $x\mapsto (\ad x)$, is a lie 
    homomorphism. 
\end{example}

\begin{definition}
Let $L$ be a Lie algebra. 
A (left) Lie $L$-module is a vector space $V$ 
together with a map $L\times V\to V$, $(x,v)\mapsto xv$, 
such that $(x,v)\mapsto xv$ is bilinear 
and 
\[
    [x,y]v=x(yv)-y(xv)
\]
for all $x,y\in L$ and $v\in V$. 
\end{definition}

As it happens in the case of groups, Lie modules are
in bijective correspondence with representations. 

\begin{example}
    Let $L$ be a subalgebra of $\gl(V)$. Then 
    $L$ is an $L$-module. 
\end{example}

\begin{definition}
    Let $L$ be a Lie algebra and $V$ be a Lie $L$-module. 
    A \textbf{submodule} of $V$ is a subspace $W$ 
    such that $xw\in W$ for all $x\in L$ and $w\in W$. 
\end{definition}

\begin{example}
We know that $L$ is an $L$-module 
with the adjoint representation. The submodules of $L$ are
the ideals of $L$. 
\end{example}

If $W$ is a submodule of $V$, then $V/W$ 
with $x(v+W)=xv+W$ is a module. 

\begin{definition}
    Let $L$ be a Lie algebra. An $L$-module $V$ 
    is said to be \textbf{simple} (or irreducible) 
    if $V\ne \{0\}$ and it has no submodules other than $\{0\}$ and $V$. 
\end{definition}

One-dimensional modules are simple. In particular, 
the trivial module is always simple. 

\begin{example}
    Let $L$ be a simple Lie algebra (e.g. $\sl(2,\C)$). Then 
    the adjoint representation is irreducible, that is $L$ is a simple $L$-module.  
\end{example}

\begin{definition}
    Let $L$ be a Lie algebra and $V$ be an $L$-module. We say 
    that $V$ is \textbf{indecomposable} if 
    there are no non-zero submodules $U$ and $W$ such that
    $V=U\oplus W$.
\end{definition}

Clearly, irreducible modules are indecomposable. 
The converse is not true.

\begin{definition}
    Let $L$ be a Lie algebra and $V$ be an $L$-module. We say that
    $V$ is \textbf{completely reducible} if $V=S_1\oplus\cdots\oplus S_k$
    for simple modules $S_1,\dots,S_k$. 
\end{definition}

\begin{exercise}
    Let $\mathfrak{b}(n,\C)$ be the set of $n\times n$ 
    upper triangular 
    matrices in $\gl(n,\C)$. Prove that $V=\C^n$ is indecomposable,  
    not irreducible. 
\end{exercise}

\begin{definition}
    Let $L$ be a Lie algebra and $f\colon V\to W$ be a map. 
    We say that $f$ is an $L$-module \textbf{homomorphism} if
    $f(xv)=xf(v)$ for all $x\in L$ and $v\in V$. 
\end{definition}

As usual, an isomorphism is a bijective module homomorphism. 

\begin{exercise}
    State and prove the isomorphism theorems for modules
    over Lie algebras. 
\end{exercise}

\begin{exercise}[Schur lemma]
    Let $L$ be a Lie algebra.
    \begin{enumerate}
        \item Let $S$ and $T$ be simple $L$-modules.
            Prove that a non-zero homomorphism $f\colon S\to T$ 
            is an isomorphism.
        \item Let $S$ be a finite-dimensional simple $L$-module. 
            Prove that if $f\colon S\to S$ is a homomorphism, then
                $f=\lambda\id$ for some $\lambda\in\C$. 
    \end{enumerate} 
\end{exercise}

As an example, if $V$ is a simple module, then $z$
acts by scalar multiplication on $V$, that is
$zv=\lambda v$ for some $\lambda\in\C$. 

\topic{Representations of $\sl(2,\C)$}

Consider the polynomial ring $\C[X,Y]$ in two commuting variables
$X$ and $Y$. Let $V_d$ be the subspace of homogeneous polynomials
of degree $d$. Then 
\[
\dim V_d=\begin{cases}
    1 & \text{if $d=0$},\\
    d+1 & \text{otherwise},
    \end{cases}
\]
as a basis of $V_d$ is given 
by $\{X^d,X^{d-1}Y,X^{d-2}Y^2,\dots,XY^{d-1},Y^d\}$. 

\begin{exercise}
    Prove that $\varphi\colon\sl(2,\C)\to\gl(V_d)$, 
    \begin{align}
        \varphi(e)=X\frac{\partial}{\partial Y},
        &&
        \varphi(f)=Y\frac{\partial}{\partial X},
        &&
        \varphi(h)=X\frac{\partial}{\partial X}-Y\frac{\partial}{\partial Y},
    \end{align}
    is a representation of $\sl(2,\C)$. 
    This means that 
    \begin{gather*}
    \varphi(e)(X^aY^b)=bX^{a+1}Y^{b-1},
    \quad
    \varphi(f)(X^aY^b)=aX^{a-1}Y^{b+1},
    \shortintertext{and that}
    \varphi(h)(X^aY^b)=(a-b)X^aY^b.
    \end{gather*}
\end{exercise}

In the basis $\{X^d,X^{d-1}Y,X^{d-2}Y^2,\dots,XY^{d-1},Y^d\}$, 
\begin{align*}
\varphi(e)=\left(\begin{smallmatrix}
0 & 1 & 0 & \cdots & 0\\
0 & 0 & 2 & \cdots & 0\\
\vdots & \vdots & \vdots & \ddots & \vdots\\
0 & 0 & 0 & \cdots & d\\
0 & 0 & 0 & \cdots & 0
\end{smallmatrix}\right),
&& 
\varphi(f)=\left(\begin{smallmatrix}
0 & 0 & \cdots & 0 & 0\\
d & 0 & \cdots & 0 & 0\\
0 & d-1 & \cdots & 0 & 0\\
\vdots & \vdots & \ddots & \vdots & \vdots\\
0 & 0 & \cdots & 1 & 0
\end{smallmatrix}\right),
&&
\varphi(h)=\left(\begin{smallmatrix}
d & 0 & \cdots & 0 & 0\\
0 & d-2 & \cdots & 0 & 0\\
\vdots & \vdots & \ddots & \vdots & \vdots\\
0 & 0 & \cdots & -d+2 & 0\\
0 & 0 & \cdots & 0 & -d
\end{smallmatrix}\right).
\end{align*}

\begin{exercise}
    Prove that $V_d$ is generated (as an $\sl(2,\C)$-module) by
    $X^aY^b$ for some $a$ and $b$ such that $a+b=d$. 
\end{exercise}

The following exercise is important:

\begin{exercise}
    Prove that each $V_d$ is a simple $\sl(2,\C)$-module.
\end{exercise}

Now we prove one of the main results of this section. 

\begin{theorem}
\label{thm:irreducibles_sl2}
    Let $V$ be a finite-dimensional 
    simple $\sl(2,\C)$-module. Then $V\simeq V_d$ for some $d$. 
\end{theorem}

We use the notation $e^2v=e(ev)$.

We need some lemmas. 

\begin{lemma}
    Let $V$ be an $\sl(2,\C)$-module and $v\in V$ be 
    an eigenvector of $h$ with eigenvalue $\lambda$. 
    \begin{enumerate}
        \item Either $ev=0$ or $ev$ is an eigenvector of $h$ with
            eigenvalue $\lambda+2$.
        \item Either $fv=0$ or $fv$ is an eigenvector of $h$ with
            eigenvalue $\lambda-2$.
    \end{enumerate} 
\end{lemma}

\begin{proof}
    We only prove 1): $h(ev)=e(hv)+[h,e]v=e(\lambda v)+2ev=(\lambda+2)ev$.
\end{proof}

\begin{lemma}
    Let $V$ be a finite-dimensional $\sl(2,\C)$-module.
    There exists an eigenvector 
    $w\in V$ of $h$ such that $ew=0$. 
\end{lemma}

\begin{proof}
    The linear map $h\colon V\to V$ has at least one eigenvector 
    $v$ with eigenvalue $\lambda$. If the elements 
    $v,ev,e^2v,\dots$ are non-zero, they are linearly independent, as they 
    form a sequence of eigenvectors of $h$ with different eigenvalues.  
    As $\dim V<\infty$, it follows that there exists $k$ 
    such that $e^kv\ne 0$ and $e^{k+1}v=0$. Let $w=e^kv\ne 0$. 
    Then
    $hw=(\lambda+2k)w$ and $ew=0$. 
\end{proof}

Now we prove the theorem.

\begin{proof}[Proof of Theorem \ref{thm:irreducibles_sl2}]
    By the previous lemma, there exists an eigenvector $w$ 
    of $h$ of eigenvalue $\lambda$ such that $ew=0$. Since $V$ is finite-dimensional, 
    after considering
    the sequence $w,fw,f^2w\dots$ we find that there
    exists $d\geq0$ such that 
    $f^dw\ne 0$ and $f^{d+1}w=0$. 
    
    \begin{claim}
        $\{w,fw,\dots,f^dw\}$ is a basis of a submodule of $V$.
    \end{claim}
    
    The elements are linearly independent, as they are eigenvectors of $h$ with 
    different eigenvalues. The subspace $W=\langle w,fw,\dots,f^dw\rangle$ 
    is invariant under $h$ and $f$. Let us prove that $W$ is invariant 
    under $e$, that is $eW\subseteq W$. We need to prove that
    $e(f^kw)\in W$ for all $k$. We proceed by induction on $k$. 
    The case $k=0$ is trivial. 
    If the claim holds for some $k$, then $ef^kw\in W$ by 
    the inductive hypothesis. Thus 
    \[
    e(f^{k+1}w)=ef(f^{k}w)=(fe+h)f^{k}w\in W,
    \]
    as $hf^kw\in W$. 
    
    \begin{claim}
        $\lambda=d$. 
    \end{claim}
    
    The matrix of $h$ with respect to $\{w,fw,\dots,f^dw\}$ 
    is diagonal with trace 
    \[
    \lambda+(\lambda-2)+\cdots+(\lambda-2d)=(d+1)(\lambda-d).
    \]
    Since $[e,f]=h$ has trace zero, it follows that $\lambda=d$. 
    
    \begin{claim}
        $V\simeq V_d$.
    \end{claim}
    
    The vector spaces are isomorphic, as $V$ has basis $\{w,fw,\dots,f^dw\}$ and 
    $V_d$ has basis $\{X^d,fX^d,\dots,f^dX^d\}$, where
    $f^kX^d\in\C X^{d-k}Y^k$. The eigenvalues of $h$ on $f^kw$ 
    are the same as the eigenvalues of $h$ on $f^kX^d$. Let 
    \[
    \varphi\colon V\to V_d,\quad
    f^kw\mapsto f^kX^d.
    \]
    This bijective linear map commutes with the action of $h$ and $f$. 
    It also satisfies $\varphi(ew)=...$
    and 
    \[
    \varphi(hf^kw)=...
    \varphi(ef^kw)=...
    \]
\end{proof}

We now summarize our results in terms of \textbf{highest weight vectors}
and \textbf{highest weights}.

\begin{corollary}
    Let $V$ be a finite-dimensional $\sl(2,\C)$-module and $w\in V$ 
    (a highest vector of $V$)
    be an eigenvector of $h$ 
    such that $ew=0$. Then $hw=dw$ for some
    non-negative integer $d$ (a highest weight). 
    Moreover, the submodule
    of $V$ generated by $w$ is isomorphic to $V_d$. 
\end{corollary}

\begin{proof}
    
\end{proof}

The following result is a particular
case of Weyl's theorem in the context of $\sl(2,\C)$-modules. 

\begin{theorem}
    Any finite-dimensional $\sl(2,\C)$-module is absolutely reducible. 
\end{theorem}

\begin{proof}
    Let $V$ be a finite-dimensional $\sl(2,\C)$-module. 
    We proceed in several steps.
    
    \begin{claim}
        The elmenet $Z=\frac12h^2+h+2fe$ commutes with every $X\in\sl(2,\C)$. 
    \end{claim}
    
    We first compute
    \begin{equation}
        \begin{aligned}
            \label{eq:Casimir}
            ZX-XZ &= \frac12h^2X-\frac12Xh^2+[h,X]+2feX-2Xfe\\
            &=\frac12h[e,X]-\frac12[X,h]h+[h,X]+2f[e,X]-2[X,f]e.
        \end{aligned}
    \end{equation}
    Now one checks that 
    Equation \eqref{eq:Casimir} is zero if $X\in\{h,e,f\}$ and
    the claim follows. 
    %See \cite[Theorem V.4.6]{MR1321145} or \cite[Theorem 1.67]{MR1920389}. 
    
    \begin{claim}
        If $\dim V=n+1$, then $Z$ acts as the scalar 
        $\frac12n^2+n$, which is not zero unless $V$ is the trivial module. 
    \end{claim}
    
    By Schur's lemma, $Z$ acts by a scalar. Since $V$ is a simple
    $\sl(2,\C)$-module of dimension $n+1$, $V\simeq V_{n+1}$. In particular, 
    $hv_0=nv_0$ and $ev_0=0$. 
    
    \begin{claim}
        Let $U\subseteq V$ be a submodule of codimension one. Then 
        there exists a submodule $W$ of $V$ such that $V=U\oplus W$ 
        and $\dim W=1$. 
    \end{claim}
    
    We split the proof of the claim into several steps. 
    
    First we assume that
    $\dim U=1$. The quotient module $V/U$ is one-dimensional and hence simple. 
    Thus $\sl(2,\C)V\subseteq U$ and $\sl(2,\C)U=\{0\}$. Hence 
    \[
    [X,Y]V\subseteq XYV-YXV\subseteq XU+YU=\{0\}.
    \]
    Since $\sl(2,\C)=[\sl(2,\C),\sl(2,\C)]$, we conclude that
    $\sl(2,\C)V=\{0\}$. Thus any complement of $U$ will serve as $W$. 
    
    \medskip
    We now finish the proof of the theorem. 
\end{proof}

\topic{Enveloping algebras}

