\section{}

\subsection{Group algebras}

\index{Group algebra}
Let $G$ be a finite group. The (complex) \emph{group
algebra} $\C[G]$ is the $\C$-vector space with
basis $\{g:g\in G\}$ and multiplication
\[
\left(\sum_{g\in G}\lambda_gg\right)\left(\sum_{h\in G}\mu_hh\right)
=\sum_{g,h\in G}\lambda_g\mu_h(gh).
\]

Clearly, $\dim \C[G]=|G|$. Moreover, 
$\C[G]$ is commutative if and only if $G$ is abelian. 

\index{Augmentation ideal}
If $G$ is non-trivial, 
then $\C[G]$ contains proper non-trivial ideals. For example, 
the \emph{augmentation ideal} 
\[
I(G)=\left\{\sum_{g\in G}\lambda_gg\in \C[G]:\sum_{g\in G}\lambda_g=0\right\}
\]
is a non-zero proper ideal of $\C[G]$. 

\begin{exercise}
Let $C_n$ be the cyclic group of order $n$ (written multiplicatively).
Prove that $\C[G]\simeq \C[X]/(X^n-1)$. 
\end{exercise}

\begin{exercise}
    Let $G$ be a finite non-trivial group. Prove that
    $\C[G]$ has zero divisors. 
\end{exercise}

\begin{exercise}
    Let $G$ be a finite group. The set
    \[
    \Fun(G,\C)=\{\alpha\colon G\to\C\}
    \]
    is a complex vector space with 
    the operations 
    \[
    (\alpha+\beta)(x)=\alpha(x)+\beta(x),
    \quad
    (\lambda\alpha)(x)=\lambda\alpha(x),
    \]
    for all $\alpha,\beta\in\Fun(G,\C)$, $x\in G$ 
    and $\lambda\in\C$. It is an algebra
    with the \emph{convolution product} 
    \[
    (\alpha*\beta)(x)=\sum_{y\in G}\alpha(xy^{-1})\beta(y).
    \]
    Let 
    \[
    \delta_x(y)=\begin{cases}
            1 & \text{if $x=y$},\\
            0 & \text{otherwise}.
        \end{cases}
    \]
    Prove the following statements:
    \begin{enumerate}
        \item 
        The set $\{\delta_x:x\in G\}$ is a basis
        of $\Fun(G,\C)$. 
        \item The map $\C[G]\to\Fun(G,\C)$, $g\mapsto\delta_g$, 
            extends linearly to an algebra isomorphism. 
    \end{enumerate}
\end{exercise}


\index{Module!semisimple}
Recall that a finite-dimensional module $M$ is semisimple 
if and only if for every submodule $S$ of $M$ there 
is a submodule $T$ of $M$ such that $M=S\oplus T$.    

\begin{theorem}[Maschke]
\index{Maschke's theorem}
    Let $G$ be a finite
    group and $M$ be a finite-dimensional $\C[G]$-module.
    Then $M$ is semisimple. 
\end{theorem}

\begin{proof}
We must show that every submodule $S$ of $M$ admits a complement. 
Since $S$ is a subspace of $M$, there exists a subspace $T_0$ of $M$ 
such that $M=S\oplus T_0$ (as vector spaces). We use 
$T_0$ to construct a submodule $T$ of $M$ that complements $S$. Since $M=S\oplus T_0$, 
every $m\in M$ can be written uniquely as $m=s+t_0$ for some $s\in S$ and $t_0\in T$. 
Let 
\[
p_0\colon M\to S,\quad
p_0(m)=s,
\]
where $m=s+t_0$ with $s\in S$ and $t_0\in T$. 
If $s\in S$, then $p_0(s)=s$. In particular, $p_0^2=p_0$, as 
$p_0(m)\in S$. 

Generally, $p_0$ is not a $\C[G]$-modules homomorphism. 
Let 
\[
p\colon M\to S,\quad
p(m)=\frac{1}{|G|}\sum_{g\in G}g^{-1}\cdot p_0(g\cdot m).
\]

We claim that $p$ is a homomorphism of $\C[G]$-modules. For that purpose, we need to show that 
$p(g\cdot m)=g\cdot p(m)$ for all $g\in G$ and $m\in M$. In fact, 
\[
p(g\cdot m)=\frac{1}{|G|}\sum_{h\in G}h^{-1}\cdot p_0(h\cdot (g\cdot m))
=\frac{1}{|G|}\sum_{h\in G}(gh^{-1})\cdot p_0(h\cdot m)=g\cdot p(m).
\]

We now claim that $p(M)=S$. The inclusion $\subseteq$ is trivial to prove, as $S$ is a submodule of $M$ 
and $p_0(M)\subseteq S$. Conversely, if $s\in S$, then $g\cdot s\in S$, as 
$S$ is a submodule. Thus 
$s=g^{-1}\cdot (g\cdot s)=g^{-1}\cdot p_0(g\cdot s)$ and hence 
\[
s=\frac{1}{|G|}\sum_{g\in G}g^{-1}\cdot (g\cdot s)=\frac{1}{|G|}\sum_{g\in G}g^{-1}\cdot (p_0(g\cdot s))=p(s).
\]
Since $p(m)\in S$ for all $m\in M$, it follows that $p^2(m)=p(m)$, so $p$ is a projector onto $S$. 
Hence $S$ admits a complement in $M$, that is $M=S\oplus\ker(p)$.
\end{proof}

\begin{exercise}
Let $G=\langle g\rangle$ be the cyclic group 
of order four and $\rho_g=\begin{pmatrix}
0&-1\\
1&0\end{pmatrix}$. 
Let $M=\C^{2\times 1}$ as an $\C[G]$-module with 
\[
g\cdot\begin{pmatrix}u\\v\end{pmatrix}
=\begin{pmatrix}-v\\u\end{pmatrix}.
\]
Prove that $M$ is a semisimple non-simple $\C[G]$-module.
\end{exercise}

\begin{exercise}
Let $G=\langle g\rangle$ be the cyclic group 
of order four and $\rho_g=\begin{pmatrix}
0&-1\\
1&0\end{pmatrix}$. 
Let $M=\R^{2\times 1}$ as an $\R[G]$-module with 
\[
g\cdot\begin{pmatrix}u\\v\end{pmatrix}
=\begin{pmatrix}-v\\u\end{pmatrix}.
\]
Prove that $M$ is a simple $\R[G]$-module. 
\end{exercise}

If $G$ is a finite group, 
then $\C[G]$ is semisimple. By Artin--Wedderburn theorem, 
\[
\C[G]\simeq\prod_{i=1}^r M_{n_i}(\C),
\]
where $r$ is the number of isomorphism classes of simple modules of $\C[G]$. Moreover, 
$|G|=\dim\C[G]=\sum_{i=1}^r n_i^2$. By convention, 
we always assume that $n_1=1$. 
This corresponds, of course, to the \emph{trivial module}. 

\begin{theorem}
    Let $G$ be a finite group. The number of simple 
    modules of $\C[G]$ coincides with the number of conjugacy classes of $G$. 
\end{theorem}

\begin{proof}
    By Artin--Wedderburn theorem, $\C[G]\simeq\prod_{i=1}^rM_{n_i}(\C)$. Thus 
    \[
		Z(\C[G])\simeq\prod_{i=1}^rZ(M_{n_i}(\C))\simeq\C^r.
	\]
	In particular, $\dim Z(\C[G])=r$. If $\alpha=\sum_{g\in
	G}\lambda_gg\in Z(\C[G])$, then $h^{-1}\alpha h=\alpha$ for all $h\in
	G$. Thus 
	\[
		\sum_{g\in G}\lambda_{hgh^{-1}}g=
		\sum_{g\in g}\lambda_g h^{-1}gh=\sum_{g\in G}\lambda_gg
	\]
	and hence $\lambda_{g}=\lambda_{hgh^{-1}}$ for all $g,h\in G$. A basis for 
	$Z(\C[G])$ is given by elements of the form 
	\[
		\sum_{g\in K}g,
	\]
	where $K$ is a conjugacy class of $G$. Therefore $\dim Z(\C[G])$ equals 
	the number of conjugacy classes of $G$.
\end{proof}

\begin{exercise}
    Let $G$ be a finite group of order $n$ with $k$ conjugacy classes.
    Let $m=(G:[G,G])$. Prove that $n+3m\geq4k$. 
\end{exercise}

If $G$ is a finite group,
then 
\[
\C[G]\simeq \prod_{i=1}^k M_{n_i}(\C),
\]
where $k$ is the number of conjugacy classes of $G$. 
In particular, 
\[
|G|=\dim\C[G]=\sum_{i=1}^k n_i^2.
\]

For $n\in\Z_{\geq2}$, we write $C_n$ to denote the (multiplicative) cyclic group of order $n$. 

\begin{exercise}
    Prove that $\C[C_4]\simeq\C^4$. 
\end{exercise}

For $n\geq1$, let $\Sym_n$ denote the symmetric group in $n$ letters. 

\begin{example}
    The group $\Sym_3$ has three conjugacy classes:
    $\{\id\}$, $\{(12),(13),(23)\}$ and $\{(123),(132)\}$. 
    Since $6=1^2+a^2+b^2$, it follows that 
    $\C[G]\simeq\C\times\C\times M_2(\C)$. 
\end{example}    

\begin{problem}[Brauer]
\index{Brauer's problem}
    Which algebras are group algebras?
\end{problem}

This question might be impossible to answer, but it is extremely interesting.

\begin{example}
    The algebra $\C^2\times M_2(\C)\times M_3(\C)$ is not complex group algebra, as all groups of order 15 are abelian. 
\end{example}

\subsection{Some comments}

There is a multiplicative version of Maschke's theorem. A group $G$ acts 
by automorphisms on $A$ if there is a group homomorphism 
$\lambda\colon G\to\Aut(A)$. In this case, a subgroup $B$ of $A$ is said to be 
$G$-invariant if $\lambda(B)\subseteq B$. 

\begin{theorem}
\index{Maschke's theorem!multiplicative version}
    Let $K$ be a finite group of order $m$. Assume that 
    $K$ acts by automorphisms on $V=U\times W$, where
    $U$ and $W$ are subgroups of $V$ and $U$ is abelian and $K$-invariant. 
    If the map $U\to U$, $u\mapsto u^m$, is bijective, 
    then there exists a normal $K$-invariant subgroup $N$ of $V$ 
    such that $V=U\times N$. 
\end{theorem}

\begin{proof}
Let $\theta\colon U\times W\to U$, $(u,w)\mapsto u$. Then $\theta$ is a group homomorphism such that 
$\theta(u)=u$ for all $u\in U$. Since $U$ is $K$-invariant, 
\[
k^{-1}\cdot \theta(k\cdot v)\in U
\]
for all $k\in K$ and $v\in V$. 
Since $K$ is finite and $U$ is abelian, 
the map 
\[
\varphi\colon V\to U,\quad 
v\mapsto \prod_{k\in K}k^{-1}\cdot \theta(k\cdot v), 
\]
is well-defined. 
We claim that $\varphi$ is a group homomorphism. If $x,y\in V$, then 
\begin{align*}
    \varphi(xy) &= \prod_{k\in K}k^{-1}\cdot \theta(k\cdot (xy))\\
    &= \prod_{k\in K}k^{-1}\cdot (\theta(k\cdot x)\theta(k\cdot y))\\
    &= \prod_{k\in K}k^{-1}\cdot \theta(k\cdot x) \prod_{k\in K}k^{-1}\cdot \theta(k\cdot y)=\varphi(x)\varphi(y),
\end{align*}
since $U$ is abelian and $K$ acts by automorphisms on $V$. 

We claim that $N=\ker\varphi$ is $K$-invariant. 
We need to show that $\varphi(l\cdot x)=l\cdot\varphi(x)$ for all $l\in K$ and $x\in V$. 
If $l\in K$ and $x\in V$, then 
\begin{align*}
l^{-1}\cdot\varphi(l\cdot x)&=l^{-1}\cdot\left(\prod_{k\in K}k^{-1}\cdot \theta(k\cdot (l\cdot x))\right)=\prod_{k\in K}(kl)^{-1}\cdot\theta( (kl)\cdot x)=\varphi(x),
\end{align*}
since $kl$ runs over all the elements of $K$ whenever $k$ runs over all the elements of $K$.
In conclusion, $\ker\varphi$ is $K$-invariant. 

It remains to show that $V$ is the direct product of $U$ and $N$. By assumption, $U$ is normal in $V$. 
We first prove that $U\cap N=\{1\}$. If $u\in U$, then $k\cdot u\in U$ for all $k\in K$. This implies that 
$k^{-1}\cdot\theta(k\cdot u)=k^{-1}\cdot (k\cdot u)=u$. Hence $\varphi(u)=u^m$. Since this map is bijective by assumption,  
\[
U\cap N=U\cap\ker\varphi=\{1\}.
\]
We now show that $V\subseteq UN$, as the other inclusion is trivial. Since $N=\ker\varphi$,  
\[
\varphi(V)\subseteq U=\varphi(U)=\varphi(U)\varphi(N)=\varphi(UN) 
\]
and hence $V\subseteq (UN)N=UN$. 
Therefore $V$ is the direct product of $U$ and $N$, as $N$ is normal in $V$.
\end{proof}

\begin{corollary}
    Let $p$ be a prime number and $K$ be a finite
    group with order not divisible by $p$. Let $V$ be
    a $p$-elementary abelian group. Assume that $K$ acts
    by automorphism on $V$. If $U$ be a $K$-invariant subgroup of $V$, 
    then there exists a $K$-invariant subgroup $N$ of $V$ 
    such that $V=U\times N$. 
\end{corollary}

\begin{proof}
    Let $m=|K|$. Since $m$ and $|U|$ are coprime, the map 
    $u\mapsto u^m$ is bijective in $U$. Since $V$ is a vector space over the field 
    $\Z/p$, it follows that $V=U\times W$ for some subgroup $W$ of $V$. Now the claim follows
    from the previous theorem. 
\end{proof}


\subsection{Representations}

Unless we state differently, we will always work
with finite groups. All our vector spaces will
be complex vector spaces. 

\begin{definition}
\index{Representation}
    Let $G$ be a finite group. A \emph{representation}
    of $G$ is a group homomorphism $\rho\colon G\to\GL(V)$, where
    $V$ is a finite-dimensional vector space. The \emph{degree} (or dimension) 
    of the representation is the integer $\deg\rho=\dim V$. 
\end{definition}

\index{Matrix representation}
Let $G\to\GL(V)$ be a representation. 
If we fix a basis of $V$, then we obtain
a \emph{matrix representation} of $G$, that is a 
group homomorphism 
\[
\rho\colon G\to\GL(V)\simeq\GL_n(\C),
\quad 
g\mapsto\rho_g,
\]
where
$n=\dim V$. 

\begin{example}
Since $\Sym_3=\langle (12),(123)\rangle$, the map $\rho\colon \Sym_3\to\GL_3(\C)$,
\[
(12)\mapsto\begin{pmatrix}
0 & 1 & 0\\
1 & 0 & 0\\
0 & 0 & 1
\end{pmatrix},\quad
(123)\mapsto\begin{pmatrix}
0 & 0 & 1\\
1 & 0 & 0\\
0 & 1 & 0
\end{pmatrix},
\] 
is a representation of $\Sym_3$. 
\end{example}

\begin{example}
Let $G=\langle g\rangle$ be cyclic of order six. 
The map $\rho\colon G\to\GL_2(\C)$, 
\[
g\mapsto
\begin{pmatrix}
1&1\\
-1&0
\end{pmatrix}
\] 
is a representation of $G$. 
\end{example}

\begin{example}
Let $G=\langle g\rangle$ be cyclic of order four. 
The map $\rho\colon G\to\GL_2(\C)$, 
\[
g\mapsto
\begin{pmatrix}
0&-1\\
1&0
\end{pmatrix}
\] 
is a representation of $G$. 
\end{example}

\begin{example}
  Let $G=\langle a,b:a^2=b^3=(ab)^3=1\rangle$. The map 
  \[
    a\mapsto\begin{pmatrix}
    0 & 1 & -1\\
    1 & 0 & -1\\
    0 & 0 & -1
    \end{pmatrix},
    \quad
    b\mapsto\begin{pmatrix}
      0 & 0 & 1\\
      1 & 0 & 0\\
      0 & 1 & 0
    \end{pmatrix},
  \]
  defines a representation $G\to\GL_3(\C)$. 
\end{example}

\begin{example}
    Let $Q_8=\{-1,1,i,-i,j,-j\}$ be the quaternion group. Recall that
    \[
    i^2=j^2=k^2=-1,\quad
    ijk=-1.
    \]
    The group $Q_8$ is generated by $\{i,j\}$ 
    and the map $\rho\colon Q_8\to\GL_2(\C)$, 
    \[
    i\mapsto\begin{pmatrix}
    i&0\\0&-i
    \end{pmatrix},
    \quad
    j\mapsto\begin{pmatrix}
    0&1\\-1&0
    \end{pmatrix},
    \]
    is a representation.
\end{example}

\begin{example}
  Let $G$ be a finite group that acts on a finite set $X$. 
  Let $V=\C X$ the complex vector space with basis $\{x:x\in
  X\}$. The map 
  \[
	\rho\colon G\to\GL(V),\quad
	\rho_g\left(\sum_{x\in X}\lambda_x x\right)
	=\sum_{x\in X}\lambda_x\rho_g(x)
	=\sum_{x\in X}\lambda_{g^{-1}\cdot x}x, 
  \]
  is a representation of degree $|X|$.
\end{example}

\begin{example}
    The sign $\sgn\colon\Sym_n\to\GL_1(\C)=\C^{\times}$ is a representation of $\Sym_n$.
\end{example}

An important fact is that there exists a bijective
correspondence 
between 
representations of a finite group $G$ 
and 
finite-dimensional modules over $\C[G]$. The correspondence
is given as follows. If $\rho\colon G\to\GL(V)$ is a representation, 
then $V$ is a $\C[G]$-module with
\[
\left(\sum_{g\in G}\lambda_gg\right)\cdot v=\sum_{g\in G}\lambda_g\rho_g(v).
\]
Conversely, if $V$ is a $\C[G]$-module, then
$\rho\colon G\to\GL(V)$, $\rho_g\colon V\to V$, $v\mapsto g\cdot v$, 
is a representation. 

\begin{exercise}
    Let $G$ be a finite group and 
    $\rho\colon G\to\GL(V)$ be a representation. Prove that 
    each $\rho_g$ is diagonalizable. 
\end{exercise}

The previous exercise uses properties of the minimal polynomial. We will 
see a different proof later. 

\begin{definition}
\index{Equivalent representations}
Let $G$ be a group and $\phi\colon G\to\GL(V)$ and $\psi\colon G\to\GL(W)$ be representations of $G$. 
We say that $\phi$ and $\psi$ are \emph{equivalent} if 
there exists a linear isomorphism $T\colon V\to W$ such that 
\[
	\psi_g T=T \phi_g
\]
for all $g\in G$. In this case, we write $\phi\simeq\psi$. 
\end{definition}

Note that $\phi\simeq\psi$ if and only if $V$
and $W$ are isomorphic as $\C[G]$-modules.

\begin{example}
  The representation 
  \begin{gather*}
  \phi\colon\Z/n\to\GL_2(\C),\quad
  \phi(m)=
  \begin{pmatrix}
    \cos(2\pi m/n) & -\sin(2\pi m/n)\\
    \sin(2\pi m/n) & \cos(2\pi m/n)
  \end{pmatrix},
  \shortintertext{is equivalent to the representation}
  \psi\colon\Z/n\to\GL_2(\C),
  \quad 
  \psi(m)=\begin{pmatrix}
    e^{2\pi im/n} & 0\\
    0 & e^{-2\pi im/n}
  \end{pmatrix}.
  \end{gather*}
  The equivalence is obtained with the matrix $T=\begin{pmatrix} i & -i\\
    1&1\end{pmatrix}$, as a direct calculation shows that
    $\phi_m T=T\psi_m$ for all $m$.
\end{example}

\begin{exercise}
    Let $\rho\colon G\to\GL(V)$ be a representation. Fix a basis 
    of $V$ and consider the corresponding matrix representation $\phi$ 
    of $\rho$. Prove that $\rho$ and $\phi$ are equivalent. 
\end{exercise}

\begin{definition}
    Let $\phi\colon G\to\GL(V)$ be a representation. A subspace 
    $W\subseteq V$ is said to be \emph{$G$-invariant} if
    $\phi_g(W)\subseteq W$ for all $g\in G$.  
\end{definition}

Let $\rho\colon G\to\GL(V)$ be a representation. 
If $W$ is a $G$-invariant subspace of $V$, 
then the restriction $\rho|_W\colon G\to\GL(W)$
is a representation. In particular, $W$ is a submodule (over $\C[G]$) 
of $V$. 

\begin{definition}
\index{Representation!irreducible}
\index{Module!simple}
    A representation $\rho\colon G\to\GL(V)$ is 
    said to be \emph{irreducible} if 
    $\{0\}$ and $V$ are the only 
    $G$-invariant subspaces of $V$. 
\end{definition}

Note that a representation $\rho\colon G\to\GL(V)$ is irreducible
if and only if $V$ is simple. 

\begin{example}
    Degree-one representations are irreducible. 
\end{example}

\begin{exercise}
\label{xca:degree-one}
    Let $G$ be a finite group. 
    Prove that there exists a bijective correspondence between 
    degree-one representations of $G$ and 
    degree-one representations of $G/[G,G]$. 
\end{exercise}

In the following example, we work over the real numbers. 

\begin{example}
Let $G=\langle g\rangle$ be the cyclic group of three elements and 
\[
\rho\colon G\to\GL(\R^3),\quad
\rho_g(x,y,z)=(y,z,x).
\]
%\mapsto\begin{pmatrix}
%  0&1&0\\
%  0&0&1\\
%  1&0&0
%\end{pmatrix}.
%\]
%Thus $g$ acts on $\R^{3}$ by left matrix multiplication,
%\[
%g\cdot (x,y,z)=
%\begin{pmatrix}
%  0&1&0\\
%  0&0&1\\
%  1&0&0
%\end{pmatrix}\begin{pmatrix}
%x\\
%y\\
%z
%\end{pmatrix}.
%\]
The set 
\[
N=\{(x,y,z)\in\R^{3}:x+y+z=0\}
\]
is a $G$-invariant subspace of $\R^3$. 

We claim that $N$ is irreducible. 
If $N$ contains a non-zero $G$-invariant subspace $S$, 
let $(x_0,y_0,z_0)\in S\setminus\{(0,0,0)\}$. Since $S$ is $G$-invariant, 
\[
(y_0,z_0,x_0)=g\cdot (x_0,y_0,z_0)\in S. 
\]
We claim that $\{(x_0,y_0,z_0),(y_0,z_0,x_0)\}$ is linearly independent. If there exists $\lambda\in\R$ 
such that $\lambda(x_0,y_0,z_0)=(y_0,z_0,x_0)$, then $x_0=\lambda^3 x_0$. Since $x_0=0$ implies 
$y_0=z_0=0$, it follows that $\lambda=1$. In particular, $x_0=y_0=z_0$, a contradiction, as $x_0+y_0+z_0=0$. 
Hence $\dim S=2$ and therefore $S=N$. 
\end{example}

What happens in the previous example if we consider complex numbers?

\begin{exercise}
  \label{xca:deg2}
  Let $\phi\colon G\to \GL(V)$, $g\mapsto\phi_g$, be a degree-two representation. Prove that
  $\phi$ is irreducible if and only if there is no common eigenvector for all the $\phi_g$.
\end{exercise}

\begin{example}
\label{exa:S3_deg2}
  Recall that $\Sym_3$ is generated by $(12)$ and $(23)$. The map 
  \[(12)\mapsto\begin{pmatrix}
    -1 & 1\\
    0 & 1
  \end{pmatrix},
  \quad
  (23)\mapsto\begin{pmatrix}
    1 & 0\\
    1 & -1
  \end{pmatrix},
  \]
  defines a representation $\phi$ of $\Sym_3$. 
  Exercise \ref{xca:deg2} shows that $\phi$ is  
  irreducible.
\end{example}