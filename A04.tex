\section{The theorems of Wedderburn and Artin--Wedderburn}

In this note, all algebras are unitary algebras over a fixed field $K$. 

A \textbf{left ideal} (resp. right ideal) 
of an algebra
$A$ is an additive subgroup $I$ of $A$ such that 
$ax\in I$ (resp $xa\in I$) 
for all $a\in A$ and $x\in I$. An \textbf{ideal} of $A$ 
is a subset that is both a left and a right ideal of $A$. 

\begin{exercise}
Prove that every left ideal $I$ of an algebra 
$A$ is a subspace of $A$ with $\lambda x=(\lambda 1_A)x$ 
for all $\lambda\in K$ and $x\in I$.
\end{exercise}

For left ideals $X$ and $Y$ of 
an algebra $A$, let $XY$ be the additive subgroup of $A$ 
generated by $\{xy:x\in X,y\in Y\}$, that is 
\[
XY=\left\{\sum_{i=1}^n:n\geq0,\,x_1,\dots,x_n\in X,\,y_1,\dots,y_n\in Y\right\},
\]
with the usual convention that the empty sum is equal to zero. 
    
\begin{exercise}
    Let $A$ be an algebra and $X$ and $Y$ be left ideals of $A$. 
    Prove the following statements:
    \begin{enumerate}
        \item $XY$ is a left ideal of $A$.
        \item $AX=X$ and $XR$ is a right ideal of $A$. 
        \item If $Z$ is a left ideal of $A$, then $(XY)Z=X(YZ)$. 
    \end{enumerate}
\end{exercise}

\begin{definition}
An algebra $A$ is said to be 
\textbf{semiprime} if $I^2\ne\{0\}$ for every non-zero ideal $I$ of~$A$.	
\end{definition}

\begin{definition}
A non-zero left ideal $I$ of $A$ is \textbf{minimal} if 
$J\subseteq I$ for some non-zero 
left ideal $J$ of $A$ implies $J=I$. 
\end{definition}

\begin{exercise}
\label{xca:minimal}
    Let $A$ be a finite-dimensional algebra and $I$ be a non-zero left
    ideal of $A$. Prove that $I$ contains a minimal left ideal of $A$. 
\end{exercise}

\begin{definition}
    An element $e$ of an algebra $A$ is said to be 
    \textbf{idempotent} if $e^2=e$. 
\end{definition}

Trivial examples of idempotents of an algebra $A$ 
are $0_A$ and $1_A$. The
matrix 
$\begin{pmatrix}1&0\\0&0\end{pmatrix}$ 
is a non-trivial idempotent of the algebra $M_2(\R)$ of real $2\times 2$ matrices. 

\begin{exercise}
    Let $e$ and $f$ be idempotents of an algebra $A$. 
    Prove that $eAe\subseteq fAf$ if and only if $ef=f=fe$.
\end{exercise}

For the proof of Wedderburn's theorem we shall need some lemmas. 

\begin{lemma}[Brauer]
\label{lem:Brauer}
Let $A$ be an algebra and $K$ be a non-zero 
minimal left ideal of $A$ such that $K^2\ne\{0\}$. 
Then $K=Ae$ for some idempotent $e\in A$ and
$eAe$ is a division algebra.
\end{lemma}

\begin{proof}
Since $K^2$ is a non-zero left ideal of $A$ and 
$K^2\subseteq K$, the minimality of $K$ implies that $K^2=K$. In particular, 
there exists $u\in K$ such that $\{0\}\ne Ku\subseteq K$. By the minimality of $K$, 
$Ku=K$. In particular, $u=eu$ for some $e\in K$. 
Let 
\[
L=\{x\in K:xu=0\}\subseteq K.
\] 
We claim that $xe-x\in L$ for all $x\in K$. In fact, 
if $x\in K$, then 
\[
(xe-x)u=x(eu)-xu=xu-xu=0.
\]
Note that $L$ is a left ideal of $A$. Moreover, $L\ne K$ (otherwise, 
if $L=K$, then $\{0\}\ne Ku=\{0\}$, a contradiction). 
Since $K$ is a minimal left ideal, 
$L=\{0\}$. Thus $xe=e$ for all $x\in K$. In particular, $e^2=e$. 

Since $Ae\subseteq AK\subseteq K$ and $Ae$ is a non-zero 
left ideal of $A$, 
the minimality of $K$ implies that $Ae=K$. 

It is an exercise to show that $eAe$ is an algebra. 

Let us prove
that every non-zero element $x$ of $eAe$ is invertible. Since  
\[
\{0\}\ne Ax\subseteq A(eAe)\subseteq (AeA)e\subseteq AK\subseteq K
\]
and $Ax$ is a left ideal of $A$, the minimality of $K$ implies that 
$Ax=K$. Thus $e=yx$ for some $y\in A$. Write $x=eae$ with $a\in A$. Then 
$ex=e(eae)=e^2ae=eae=x$ and 
$xe=x$. Now 
\[
(eye)x=(ey)(ex)=e(yx)=e^2=e.
\]
Since $0\ne eye\in eAe$, there exists $z\in eAe$ such that 
$z(eye)=e$. Thus 
\[
z=ze=z((eye)x)=(z(eye))x=ex=x
\]
and $x$ is invertible. 
\end{proof}

% Observemos que si $e$ y $g$ son idempotentes, entonces 
% \[
% eAe\subseteq gAg\Longleftrightarrow eg=g=ge.
% \]
% Si $eg=e=ge$, entonces $eAe=(ge)A(ge)\subseteq gAg$. Recíprocamente, si $eAe\subseteq gAg$, entonces
% $e=e^2\subseteq gAg$, digamos $e=gag$ para algún $a\in A$. Esto implica que 
% \[
% eg=(gag)g=gag^2=gag=e,\quad
% ge=g(gag)=g^2ag=gag=e.
% \]  

We now present Henderson's proof~\cite{MR184969}
of Wedderburn's theorem. 

\begin{theorem}[Wedderburn]
    Let $A$ be a finite-dimensional simple algebra. 
    Then 
    \[
    A\simeq M_n(D)
    \]
    for some $n\geq1$ and 
    some division algebra $D$. 
\end{theorem}

\begin{proof}
    Let $K$ be a minimal left ideal of $A$. Since $KA$ is a non-zero 
    ideal of $A$ and $A$ is simple, 
    $KA=A$. Moreover,  
	\[
	A=A^2=(KA)^2=KAKA\subseteq K^2A
	\]
	and therefore $K^2\ne\{0\}$. By Lemma~\ref{lem:Brauer}, 
    $K=Ae$ for some idempotent $e\in K$ and 
    $D=eAe$ is a division algebra. 

    We claim that the map 
    \[
    K\times D\to K,\quad (x,\delta)\mapsto x\delta, 
    \]
    turns $K$ into a right $D$-module. For example, since $K=Ae$, 
    if $x=ae\in K=Ae$, then $xe=(ae)e=ae^2=ae=x$. 
    
    We claim that the map 
    $\varphi_a\colon K\to K$, $x\mapsto ax$, is a $D$-module homomorphism. 
    \[
	\varphi_a(x\delta)=a(x\delta)=(ax)\delta=\varphi_a(x)\delta.
	\]  
    Since $\varphi_{ab}(x)=\varphi_a(\varphi_b(x))$ 
    for all 
    $a,b\in A$ and $x\in K$, the map 
    \[
    \varphi\colon A\to\End_D(K),\quad a\mapsto\varphi_a,
    \]
    is an algebra homomorphism. 

    We claim that $\varphi$ is injective. Let $a\in A$ be such that 
    $\varphi_a=0$. Then 
    \[
	0=\varphi_a(K)=aK=aAe.
	\]
    Since $A$ is simple and $AeA$ is a non-zero ideal of $A$, 
    we get $AeA=A$. Then 
    \[
    0=aAe=(aAe)(Ae)=(aA)(eAe)=aA
    \]
    and therefore $a=0$. 

    We claim that $\varphi$ is surjective. Since 
    $A=AeA$ by the simplicity of $A$,
    \[
	1=a_1eb_1+\cdots+a_neb_n
	\]
    for some $a_1,\dots,a_n,b_1,\dots,b_n\in A$. 
    We claim that for every $\alpha\in\End_D(K)$
    there exists $a\in A$ such that 
    $\alpha=\varphi_a$.  
	Let $a=\sum_{i=1}^n\alpha(a_ie)eb_i$ and $x\in A$. Since 
    \[
    x=a_1eb_1x+\cdots+a_neb_nx=\sum_{i=1}^n(a_ie)(eb_ix), 
    \]
    and $e(b_ix)e\in D$ for all $i\in\{1,\dots,n\}$, using 
    that $\alpha$ is a $D$-module homomorphism we get 
    \begin{align*}
        \alpha(xe)&=\alpha(a_1eb_1xe+\cdots+a_neb_nxe)\\
        &=\alpha(a_1e)eb_1xe+\cdots+\alpha(a_ne)eb_nxe\\
        &=\left(\alpha(a_1e)eb_1+\cdots+\alpha(a_ne)eb_n\right)xe\\
        &=\varphi_a(xe).
    \end{align*}
    Thus $A\simeq\End_D(K)$. 
    We now note that the set 
    \[
	\{\alpha\in\End_D(K):\dim\alpha(K)<\infty\}
	\]
    is a non-zero proper ideal of $\End_D(K)\simeq A$, a contradiction
    because $A$ is simple. We conclude that 
     $A\simeq\End_D(K)\simeq M_n(D)$ for some $n\geq1$. 
\end{proof}

\section{The Artin--Wedderburn theorem}

Before going into the Artin-Wedderburn theorem, 
we need some preliminaries. 

\begin{lemma}
    \label{lem:idempotents}
    Let $A$ be a finite-dimensional semiprime algebra. Every 
    non-zero left ideal contains a non-zero idempotent. 
\end{lemma}

\begin{proof}
Let $J$ be a non-zero left ideal of $A$. 
Since $A$ is finite-dimensional, there exists a non-zero 
minimal left ideal $I$ contained in $J$ (Exercise~\ref{xca:minimal})
Note that $IA$ is a non-zero ideal of $A$. 
Since $A$ 
is semiprime, $(IA)^2\ne\{0\}$. Then 
\[
(IA)^2=IAIA\subseteq I^2A
\]
and hence 
$I^2\ne\{0\}$, otherwise, $(IA)^2=\{0\}$. By Brauer's lemma, there exists an idempotent 
$e\in I$ such that $I=Ae$. 
\end{proof}

Recall that a proper (left) ideal $M$ of an algebra $A$ is said to be \textbf{maximal} 
if the only ideals $K$ such that $M\subseteq K\subseteq A$ are $K=M$ and $K=A$. Every
finite-dimensional algebra has maximal ideals. 

\begin{exercise}
\label{xca:maximal}
    Let $M$ be a maximal ideal of an algebra $A$. Prove
    that $A/M$ is a simple algebra. 
\end{exercise}

We now present Nicholson's proof of the  
Artin--Wedderburn theorem of~\cite{MR1244013}. We shall need a lemma.

\begin{lemma}
    \label{lem:ArtinWedderburn}
    Let $A$ be an algebra, $M$ be a maximal ideal of $A$ and $K$ be a non-zero ideal of $A$ 
    such that $K\cap M=\{0\}$. Then $K$ and $M$ are unitary algebras, $K$ 
    is simple and $A\simeq K\times M$. 
\end{lemma}

\begin{proof}
    Since $K+M=\{k+m:k\in K,\,m\in M\}$ is an ideal of $A$ containing $M$ and   
    $M$ is maximal, either $K+M=M$ or $K+M=A$. If $K+M=M$, then $K\subseteq K+M$ and
    hence $K=K\cap M=\{0\}$, a contradiction. Thus $K+M=A$. Let $e\in K$ and $f\in M$ be such 
    that $1=e+f$. If $k\in K$, then 
    $kf=fk=0$, as $K$ is an ideal and $K\cap M=\{0\}$. In particular, 
    \[
    k=k1=ke+kf=ke,\quad 
    k=1k=ek+fk=ek.
    \]
    for all $k\in K$. Thus $e$ is the unit of $K$. Similarly, $f$ is the unit of $M$. 

    Let $\varphi\colon K\times M\to A$, $(k,m)\mapsto k+m$. A direct calculation shows that
    $\varphi$ is an algebra isomorphism. Moreover, 
    the map $A\to K$, $k+m\mapsto k$, is a surjective algebra homomorphism 
    with kernel $M$. By the first isomorphism theorem, $A/M\simeq K$. Since $M$ is maximal, 
    $K$ is then simple. 
\end{proof}

\begin{theorem}[Artin--Wedderburn]
    Let $A$ be a semiprime finite-dimensional algebra. Then 
    \[
    A\simeq\prod_{i=1}^kM_{n_i}(D_i)
    \]
    for some $n_1,\dots,n_k\geq0$ and some 
    division algebras $D_1,\dots,D_k$. 
\end{theorem}

\begin{proof}
    We proceed by induction on $\dim A$. If $\dim A=1$, the result trivially holds, as 
    $A\simeq F$. Assume
    then that $\dim A>1$ and that the result holds for algebras
    of dimension $<\dim A$. Let $M$ be a maximal ideal of $A$ and 
    $K=\{a\in A:Ma=0\}$. Then $K$ is an ideal of $A$. 
    Moreover, $K\cap M$ is an ideal of $A$
    such that 
    \[
    (K\cap M)^2\subseteq MK=\{0\}.
    \]
    Since $A$ is semiprime, $K\cap M=\{0\}$. 
 
    Assume first that $M=\{0\}$. Then $A\simeq A/M$ is a simple 
    algebra (Exercise~\ref{xca:maximal}). By Wedderburn's theorem, 
    and $A\simeq M_n(D)$ for some division algebra $D$. 
  
    Assume now that $M\ne\{0\}$. By Lemma~\ref{lem:idempotents}, 
    $M$ contains non-zero idempotents. 
    Let $e\in M$ be a non-zero idempotent
    such that $Ae$ has maximal dimension. Since $M$ is maximal, $e\ne 1$. We claim that 
    $e-1\in K$, that is $M(1-e)=\{0\}$. If not, 
    since $M(1-e)$ is a left ideal of $A$,  
    there exists a non-zero idempotent $f\in M(1-e)$  
    by Lemma~\ref{lem:idempotents}. 
    In particular, 
    $f=m(1-e)$ for some $m\in M$ and hence 
    $fe=m(1-e)e=0$. Let $g=e+f-ef\in M$. Then
    $g$ is an idempotent:
    \begin{align*}
    g^2 & =(e+f-ef)(e+f-ef)\\
    &=e^2+ef-e^2f+fe+f^2-fef-efe-ef^2+efef\\
    &=e+f-ef\\
    &=g.
    \end{align*}
    Moreover, 
	\begin{align*}
	&eg=e(e+f-ef)=e^2+ef-e^2f=e+ef-ef=e,\\
	&ge=(e+f-ef)e=e^2+fe-efe=e,
	\end{align*}
    Thus $e=eg\in Rg$ and therefore $Re\subseteq Ag$. The maximality of  
    $\dim Ae$ implies that $Ae=Ag$. Let $a\in A$ be such that $g=ae$. 
    Then
    \[
    e=ge=(ae)e=g
    \]
    and thus $f=ef$. This implies that
    \[
    f=f^2=e(fe)f=0,
    \]
    a contradiction. Therefore $0\ne e-1\in K$ and $K$ is non-zero. 
    By Lemma~\ref{lem:ArtinWedderburn}, $K$ and $M$ are unitary 
    algebras, $K$ is simple and $A\simeq K\times M$. By Wedderburn's theorem,
    $A\simeq M_{n}(D)$ for some division algebra $D$. Since 
    $\dim M<\dim A$, the inductive hypothesis implies that 
    $M\simeq M_{n_1}(D_1)\times M_{n_k}(D_k)$ for
    division algebras $D_1,\dots,D_k$. 
\end{proof}
d