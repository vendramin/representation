\section*{Some other topics for final projects}

\pagestyle{plain}
\fancyhf{}
\fancyhead[LE,RO]{Representation theory of algebras}
\fancyhead[RE,LO]{Final projects}
\fancyfoot[CE,CO]{\leftmark}
\fancyfoot[LE,RO]{\thepage}

We collect here some topics for final presentations. Some topics
can also be used as bachelor or master theses. 

\subsection*{Kolchin's theorem}

If $V$ is a finite-dimensional complex 
vector space and $G$ is a subgroup of $\GL(V)$ such that every element $g$ of $G$ is unipotent (i.e., $g - 1$ is a nilpotent linear transformation), then there exists a basis of $V$ in which 
all the element of $G$ are represented by upper triangular matrices with ones on the diagonal. See my
notes for \href{https://github.com/vendramin/associative/}{Associative Algebra}) or \cite[Chapter 2]{MR1369573}.

\subsection*{Staircase groups}

This topic describes a situation similar to that of Kolchin's theorem (see the course \href{https://github.com/vendramin/associative/}{Associative Algebra}), but
more general. See \cite[Chapter 5]{MR1369573}.

\subsection*{Solvable and nilpotent groups}

The character table of a finite group
detects solvability and nilpotency of groups, see
\cite[Chapter 6]{MR1369573}.

% \subsection*{Kegel--Wielandt theorem}

% Prove Kegel--Wielandt theorem \ref{thm:KegelWielandt}. 
% For the proof see \cite[Theorem 2.13]{MR1211633}. 

% \subsection*{The Drinfeld double of a finite group}

% See \cite[Chapter IX]{MR1321145} and 
% \cite[Chapter 8]{MR3752618}.

% \subsection*{Ito's theorem}

% Ito's theorem is stated in Theorem~\ref{thm:Ito}. It  generalize Frobenius' theorem
% (Theorem \ref{thm:Frobenius_chi(1)})  
% and Schur's theorem (Theorem \ref{thm:Schur_chi(1)}). 
% The theorem states that if $\chi$ is an irreducible character
% of a finite group $G$, then $\chi(1)$ divides 
% $(G:A)$ for every normal abelian subgroup $A$ of $G$. 
% See \cite[\S8.1]{MR0450380}. 

\subsection*{Characters of $\GL_2(q)$ and $\SL_2(q)$}

One possible topic is the character table of $\GL_2(q)$, see
\cite[\S5.2]{MR2867444}. Alternatively, one can 
present the character table of the group $\SL_2(p)$  
following Humphreys's paper \cite{MR364478}. 
The character theory of $\SL_2(q)$ appears in 
\cite[\S5.2]{MR2867444}, see 
\cite[Chapter 20]{MR1650707} for details. 

\subsection*{Representations of the symmetric group}

See for example \cite[\S10]{MR2867444} and 
\cite{MR1153249}. 

\subsection*{Random walks on finite groups}

The goal is to construct the character table or 
the irreducible representations of the symmetric group. 
The topic has connections with combinatorics and applications 
to voting and card shuffling. 
See \cite[4]{MR1153249} and \cite[\S11]{MR2867444}.

% \subsection*{Fourier analysis on finite groups}

% See \cite[\S5]{MR2867444} for a very elementary approach and some
% basic applications. Other applications 
% appear in \cite{MR1695775}.

% \subsection*{Mackey's irreducibility criterion}

% It is not at all clear that 
% induction of an irreducible character will produce an irreducible character. In fact, 
% inducing the trivial character of the trivial subgroup to the whole group produces the 
% regular representation, which in general is not irreducible. Mackey found a criterion 
% that describes when an induced character is irreducible. See \cite[\S8.3]{MR2867444}. 

\subsection*{McKay's conjecture}

Prove McKay's conjecture \ref{conjecture:McKay} for all sporadic simple groups. 
This was first proved by Wilson in \cite{MR1643110}. 
Note that
for some ``small" sporadic simple groups this can be done
with the script presented in \S\ref{McKay}. However, 
for several sporadic simple groups a different approach is needed. One needs
to know the structure of normalizers. 
% http://www.math.rwth-aachen.de/~Thomas.Breuer/ctblocks/doc/overview.html

% \subsection*{Ore's conjecture}

% Prove Ore's conjecture \ref{conjecture:Ore} for alternating simple groups,
% see for example \cite{MR40298}. It is also interesting to prove the conjecture
% for other "small" simple groups such as $\PSL(3,2)$.  


\subsection*{Hirsh's theorem}

In \cite{MR36755} Hirsch found a generalization of Burnside's Theorem \ref{thm:Burnside_mod16}.  
If $G$ is a finite group and $d$ is the greatest common divisor of all 
the numbers $p^2-1$, where the $p$'s are prime divisors of $|G|$ and $r$ the number of conjugate sets in $G$. Then 
\[
|G|\equiv\begin{cases} 
    r\bmod 2d &\text{if $|G|$ odd,}\\
    r\bmod 3 & \text{if $|G|$ even and $\gcd(|G|,3)=1$.}
    \end{cases}
\]
The proof is elementary and does not use character theory. Is it possible
to prove Hirsch's theorem using characters?

% \subsection*{Clifford's theorem}

% Clifford’s theorem provides a description of the restriction of  irreducible representations to normal subgroups. It is a tool
% that tries to construct representations of the group
% from representations of a normal subgroup. See for example
% \cite[Chapter~7]{MR3970262}. 

\subsection*{Irreducible characters of groups of order $pq$}

Let $G$ be a non-abelian 
group of order $pq$, where 
$p$ and $q$ are prime numbers with $p>q$. Then 
$q\mid p-1$ and $G$ is a Frobenius group (see
Exercise~\ref{xca:Frobenius_pq}). 
The character table of Frobenius groups of order $pq$ can be found in~\cite[Chapter 25]{MR1864147}.

\subsection*{Irreducible characters of the simple group of order 168}

The smallest non-abelian simple group is $\Alt_5$, of
order $60$. The next smallest
is a certain group of order $168$. The character
table of this group can be found in~\cite[Chapter 27]{MR1864147}. 

\subsection*{Irreducible characters of semidirect products}

What can be said about irreducible characters
of semidirect products? The case of 
semidirect products by abelian groups is treated 
in \cite[Section 8.2]{MR0450380}.

% \subsection*{Hurwitz's theorem}

% It states that 
% if there is an identity of the form 
% 	\begin{equation*}
% 		(x_1^2+\cdots+x_n^2)(y_1^2+\cdots+y_n^2)=z_1^2+\cdots+z_n^2,
% 	\end{equation*}
% 	where the $x_j$'s and the $y_j$'s are real (or complex) numbers and
% 	each $z_k$ is a bilinear function in the $x_j$'s and the $y_j$'s, then 
% 	$n\in\{1,2,4,8\}$. There are several 
%     proofs of this result, and one of them uses representation theory! 

% Hurwitz's theorem has a nice application to elementary 
% linear algebra: When can we find  
% If $V$ is a real vector space with an inner product
% such that 
% see
% \cite{MR1534187} for more information. 

% \begin{theorem}
% 	Let $V$ be a real vector space (with an inner product) 
% 	such that $\dim
% 	V=n\geq3$. If there exists a bilinear function 
% 	$V\times V\to\R$, $(v,w)\mapsto v\times
% 	w$, such that $v\times w$ is orthogonal both 
% 	to $v$ and $w$ and 
% 	\[
% 		\|v\times w\|^2=\|v\|^2\|w\|^2-\langle v,w\rangle^2,
% 	\]
% 	where $\|v\|^2=\langle v,v\rangle$, then $n\in\{3,7\}$. 
% \end{theorem}

% \subsection*{Poincar\'e--Birkhoff--Witt theorem}

% There are several proofs of the
% Poincar\'e--Birkhoff--Witt theorem \ref{thm:PBW}, see for example 
% \cite[\S17.4]{MR499562} or \cite[Theorem 2.17]{MR938524}. 
% Bergman's proof based on the diamond lemma 
% appears in \cite{MR506890}. 

% \subsection*{Weyl's theorem}

% Weyl's theorem states that every finite-dimensional module over
% a semisimple Lie algebra is completely irreducible. 
% See \cite[Theorem 17.4]{MR2218355} for a proof. 

% \subsection*{Irreducible representations of $U_q(\sl(2,\C))$}

% Let $q\in\C\setminus\{0,1,-1\}$. 
% Let $U_q(\sl(2))$ be the (complex) algebra generated by 
% variables $E$, $F$, $K$ and $K^{-1}$ with relations
% \begin{align*}
%     &KK^{-1}=K^{-1}K=1,
%     &&
%     KEK^{-1}=q^2E,\\
%     &
%     KFK^{-1}=q^{-2}F,
%     &&
%     [E,F]=\frac{1}{(q-q^{-1})}(K-K^{-1}).
% \end{align*}

% This algebra is a \emph{deformation} of the enveloping algebra
% of $\sl(2,\C)$. The goal is to study the 
% representation theory of $U_q(\sl(2))$. This splits into
% two cases, depending on whether $q$ is a root of one or not. 
% Finite-dimensional simple $U_q(\sl(2))$-modules are studied 
% in \cite[VI]{MR1321145}. In particular, if 
% $q$ is not a root of one, finite-dimensional simple $U_q(\sl(2))$-modules
% are classified in \cite[Theorem VI.3.5]{MR1321145}. 

% \subsection*{Semisimple modules of $U_q(\sl(2,\C))$}

% Prove that if $q$ is not a root of one, any finite-dimensional
% $U_q(\sl(2))$-module is semisimple. 
% See \cite[Theorem VII.2.2]{MR1321145}. 