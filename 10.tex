\section{Lecture: Week 10}

\subsection{The character table of \texorpdfstring{$\Alt_5$}{A5}}

\index{Character table!of $\Alt_5$}
Let $G=\Alt_5$. 
The group $G$ is a non-abelian simple group of order 60. It has five conjugacy classes, namely

\bigskip 
\begin{center}
    \begin{tabular}{c|ccccc}
         Representative & $\id$  & $(12)(34)$ & $(123)$  & $(12345)$ & $(12354)$\\
         \hline 
         Size & $1$ & $15$ & $20$ & $12$ & $12$ \\
    \end{tabular}
\end{center}
\bigskip 

One can easily get the conjugacy classes of 
$\Alt_5$ with Magma:
\begin{lstlisting}
> A5 := Alt(5);
> ConjugacyClasses(A5);
Conjugacy Classes of group A5
-----------------------------
[1]     Order 1       Length 1
        Id(A5)

[2]     Order 2       Length 15
        (1, 2)(3, 4)

[3]     Order 3       Length 20
        (1, 2, 3)

[4]     Order 5       Length 12
        (1, 2, 3, 4, 5)

[5]     Order 5       Length 12
        (1, 3, 4, 5, 2)    
\end{lstlisting}

Let us see how to obtain all conjugacy classes
of $\Alt_5$ without computers. Let $\sigma\in\Alt_5$ and $C$ be its
conjugacy class in $\Sym_5$. Thus $|C|=(\Sym_5:C_{\Sym_5}(\sigma))$. There are two cases to consider

Assume first that $C_{\Sym_5}(\sigma)\not\subseteq\Alt_5$. Since $\Alt_5$ is a maximal subgroup of $\Sym_5$, it follows that 
$\Alt_5C_{\Sym_5}(\sigma)=\Sym_5$. Using the isomorphism theorems, 
\[
\Sym_5/\Alt_5=\Alt_5C_{\Sym_5(\sigma)}/\Alt_5
\simeq C_{\Sym_5}(\sigma)/(C_{\Sym_5}(\sigma)\cap\Alt_5)
=C_{\Sym_5}(\sigma)/C_{\Alt_5}(\sigma).
\]
Hence 
\[
(\Alt_5:C_{\Alt_5}(\sigma))=\frac{(\Sym_5:C_{\Alt_5}(\sigma))}{(\Sym_5:\Alt_5)}
=\frac{(\Sym_5:C_{\Alt_5}(\sigma))}{(C_{\Sym_5}(\sigma):C_{\Alt_5}(\sigma))}
=(\Sym_5:C_{\Sym_5}(\sigma))=|C|.
\]
Therefore $C$ is the class of $\sigma$ in $\Alt_5$. 

Assume now that $C_{\Sym_5}(\sigma)\subseteq\Alt_5$. Then 
$C_{\Alt_5}(\sigma)=C_{\Sym_5}(\sigma)\cap\Alt_5=C_{\Sym_5}(\sigma)$
and therefore 
\[
(\Alt_5:C_{\Alt_5}(\sigma))=(\Alt_5:C_{\Sym_5}(\sigma))
=\frac12(\Sym_5:C_{\Sym_5}(\sigma))=\frac12|C|.
\]
Thus $C$ splits into two conjugacy classes of $\Alt_5$ of equal size. 

The identity permutation is central. The even permutations 
$(12)(34)$ and $(123)$ both commutes with some odd permutation in $\Sym_5$ (e.g. 
$[(12)(34),(34)]=[(123),(45)]=\id$). Thus these classes do not split
in $\Alt_5$. There are twenty-four 5-cycles in $\Sym_5$. Since $24$ does not
divide $|\Alt_5|=60$, it follows that the class of 5-cycles
splits in $\Alt_5$. As representatives of these classes
we can take $(12345)$ and $(12354)$. 

Since $\Alt_5$ has five conjugacy classes, $|\Irr(G)|=5$. We already know one irreducible character of $G$, namely the trivial character 
$\Tchar_G$. 

Let $H=\Alt_4$. We compute $\Ind_H^G\Tchar_H$, where $\Tchar_H$ is the trivial character of $H$. 
By Corollary~\ref{cor:reciprocity}, 
\[
\left(\Ind_H^G\Tchar_H\right)(\id) = 5.
\]
And a direct calculation shows
\begin{align*}
    &\left(\Ind_H^G\Tchar_H\right)((12)(34)) = 1,\\
    &\left(\Ind_H^G\Tchar_H\right)((123)) = 2,\\
    &\left(\Ind_H^G\Tchar_H\right)((12345)) = 0\\ 
    &\left(\Ind_H^G\Tchar_H\right)((12354)) = 0.
\end{align*}

By Frobenius' reciprocity,
\begin{align*}
    \langle\Ind_H^G\Tchar_H,\Tchar_G\rangle = \langle\Tchar_H,\Res_H^G\Tchar_G\rangle
    = \langle\Tchar_H,\Tchar_H\rangle=
    1.
\end{align*}

Let $\chi_2=\Ind_H^G\Tchar_H-\Tchar_G$. Since 
\[
\langle\Ind_H^G\Tchar_H-\Tchar_G,\Ind_H^G\Tchar_H-\Tchar_G\rangle=1, 
\]
it follows that $\chi_2\in\Irr(G)$. 

\begin{exercise}
\label{xca:A5_chi2}
    Use Proposition~\ref{pro:2transitive} to derive (once again) the values of $\chi_2$.
\end{exercise}

So far we have the 
following table: 

\bigskip 
\begin{center}
        \begin{tabular}{|c|ccccc|}
        \hline  
         & $\id$ & $(12)(34)$ & $(123)$ & $(12345)$ & $(12354)$\\
        \hline 
        $\Tchar_G$ & $1$ & $1$ & $1$ & $1$ & $1$\\
        $\chi_2$ & $4$ & $0$ & $1$ & $-1$ & $-1$\\
        $\chi_3$ & $n_3$ & $\cdot$ & $\cdot$ & $\cdot$& $\cdot$\\
        $\chi_4$ & $n_4$ & $\cdot$ & $\cdot$ & $\cdot$& $\cdot$\\
        $\chi_5$ & $n_5$ & $\cdot$ & $\cdot$ & $\cdot$& $\cdot$\\
        \hline 
    \end{tabular}
\end{center}
\bigskip 

As $G$ is simple non-abelian, 
$|G/[G,G]|=1$. It follows that
$\Tchar_G$ is the only linear character of $G$. Moreover, 
$\chi_j(1)\geq3$ by Theorem~\ref{thm:simple}. Since 
\[
60=1+16+n_3^2+n_4^2+n_5^2
\]
and each $n_j$ divides $|G|=60$ 
(see Theorem \ref{thm:Frobenius_chi(1)}), it follows that 
$n_j\in\{3,4,5,6\}$. If some $n_j=6$, say without
loss of generality $n_3=6$, then 
\[
7=43-36=n_2^2+n_3^2, 
\]
a contradiction. Thus $n_j\in\{3,4,5\}$ for 
all $j\in\{3,4,5\}$. Without loss of generality, 
we may assume that $n_3=n_4=3$ and $n_5=5$. 

\bigskip 
\begin{center}
        \begin{tabular}{|c|ccccc|}
        \hline  
         & $\id$ & $(12)(34)$ & $(123)$ & $(12345)$ & $(12354)$\\
        \hline 
        $\Tchar_G$ & $1$ & $1$ & $1$ & $1$ & $1$\\
        $\chi_2$ & $4$ & $0$ & $1$ & $-1$ & $-1$\\
        $\chi_3$ & $3$ & $\cdot$ & $\cdot$ & $\cdot$& $\cdot$\\
        $\chi_4$ & $3$ & $\cdot$ & $\cdot$ & $\cdot$& $\cdot$\\
        $\chi_5$ & $5$ & $\cdot$ & $\cdot$ & $\cdot$& $\cdot$\\
        \hline 
    \end{tabular}
\end{center}
\bigskip 

The group $\Alt_5$ acts on the set $Y$ of subsets 
of $\{1,2,\dots,5\}$ of two elements, namely
\[
g\cdot \{a,b\}=\{g\cdot a,g\cdot b\}.
\]
Note that $|Y|=\binom{5}{2}=10$. Moreover, 
this action is transitive. Let us compute 
the character $\psi$ of the corresponding 
$\C\Alt_5$-module and the difference 
$\psi-\Tchar_G$ (We know $\psi$ counts
fixed points.)

\bigskip 
\begin{center}
        \begin{tabular}{|c|ccccc|}
        \hline  
         & $\id$ & $(12)(34)$ & $(123)$ & $(12345)$ & $(12354)$\\
        \hline 
        $\psi$ & $10$ & $2$ & $1$ & $0$ & $0$\\
        $\psi-\Tchar_G$ & $9$ & $1$ & $0$ & $-1$ & $-1$\\
        \hline 
    \end{tabular}
\end{center}
\bigskip 

The identity, of course, fixes all the ten elements
of $Y$. The permutation 
$(12)(34)$ fixed two two-elements subsets, namely
$\{1,2\}$ and $\{3,4\}$. The permutation 
$(123)$ fixes only one two-elements subset, namely
$\{4,5\}$. Finally, $(12345)$ and 
$(12354)$ fix no two-element subsets. 

Now we compute 
\[
\langle \psi-\Tchar_G,\psi-\Tchar_G\rangle=2
\]
and hence $\psi-\Tchar_G$ is the sum of two irreducible
characters (see Exercise~\ref{xca:n_irreducible}). Since
\[
\langle \psi-\Tchar_G,\chi_2\rangle=1,
\]
it follows that $\psi-\Tchar_G-\chi_2\in\Irr(G)$. Let 
$\chi_5=\psi-\Tchar_G-\chi_2$. Then

\bigskip 
\begin{center}
        \begin{tabular}{|c|ccccc|}
        \hline  
         & $\id$ & $(12)(34)$ & $(123)$ & $(12345)$ & $(12354)$\\
        \hline 
        $\Tchar_G$ & $1$ & $1$ & $1$ & $1$ & $1$\\
        $\chi_2$ & $4$ & $0$ & $1$ & $-1$ & $-1$\\
        $\chi_3$ & $3$ & $\cdot$ & $\cdot$ & $\cdot$& $\cdot$\\
        $\chi_4$ & $3$ & $\cdot$ & $\cdot$ & $\cdot$& $\cdot$\\
        $\chi_5$ & $5$ & $1$ & $-1$ & $0$& $0$\\
        \hline 
    \end{tabular}
\end{center}
\bigskip 

Let $K=\langle(12345)\rangle$ and 
$\eta\in\Irr(K)$ be such that $\eta((12345))=\zeta$, where
$\zeta=\exp(2\pi i/5)$ is a primitive $5$-th root of one. We can then compute 
$\Ind_K^G\eta$. 

\bigskip 
\begin{center}
        \begin{tabular}{|c|ccccc|}
        \hline  
         & $\id$ & $(12)(34)$ & $(123)$ & $(12345)$ & $(12354)$\\
         \hline 
         $\Ind_K^G\eta$ & $12$ & $0$ & $0$ & $\zeta^2+\zeta^3$ & $\zeta+\zeta^4$\\
         \hline 
\end{tabular}
\end{center}
\bigskip 

Since 
$\langle\Ind_K^G\eta,\chi_2\rangle=1=\langle\Ind_H^G\eta,\chi_5\rangle$,
it follows that 
\bigskip 
\begin{center}
        \begin{tabular}{|c|ccccc|}
        \hline  
         & $\id$ & $(12)(34)$ & $(123)$ & $(12345)$ & $(12354)$\\
         \hline 
         $\Ind_K^G\eta-\chi_2-\chi_5$ & $3$ & $-1$ & $0$ & $-\zeta-\zeta^4$ & $-\zeta^2-\zeta^3$\\
         \hline 
\end{tabular}
\end{center}
\bigskip 
Let $\chi_3=\Ind_K^G\eta-\chi_2-\chi_5$. Then $\chi_3\in\Irr(G)$, because it is
not the sum of three copies of the trivial character. Thus this is how our character table looks like: 
\bigskip
\begin{center}
        \begin{tabular}{|c|ccccc|}
        \hline  
         & $\id$ & $(12)(34)$ & $(123)$ & $(12345)$ & $(12354)$\\
        \hline 
        $\Tchar_G$ & $1$ & $1$ & $1$ & $1$ & $1$\\
        $\chi_2$ & $4$ & $0$ & $1$ & $-1$ & $-1$\\
        $\chi_3$ & $3$ & $-1$ & $0$ & $-\zeta-\zeta^4$ & $-\zeta^2-\zeta^3$\\
        $\chi_4$ & $3$ & $\cdot$ & $\cdot$ & $\cdot$& $\cdot$\\
        $\chi_5$ & $5$ & $1$ & $-1$ & $0$& $0$\\
        \hline 
    \end{tabular}
\end{center}
\bigskip 

\begin{exercise}
    Use the orthogonality relations
    to compute the missing row of the character table
    of $\Alt_5$. 
\end{exercise}

The previous exercise finishes the calculation
of the character table of $\Alt_5$; see Table~\ref{tab:A5}. 


\begin{table}[h]
\caption{The character table of $\Alt_5$.}
\label{tab:A5}
        \begin{tabular}{|c|ccccc|}
        \hline  
        & $1$ & $15$ & $20$ & $12$ & $12$ \\
         & $\id$ & $(12)(34)$ & $(123)$ & $(12345)$ & $(12354)$\\
        \hline 
        $\chi_1$ & $1$ & $1$ & $1$ & $1$ & $1$\\
        $\chi_2$ & $4$ & $0$ & $1$ & $-1$ & $-1$\\
        $\chi_3$ & $3$ & $-1$ & $0$ & $-\zeta-\zeta^4$ & $-\zeta^2-\zeta^3$\\
        $\chi_4$ & $3$ &  $-1$ & $0$ & $-\zeta^2-\zeta^3$ & $-\zeta-\zeta^4$ \\
        $\chi_5$ & $5$ & $1$ & $-1$ & $0$& $0$\\
        \hline 
    \end{tabular}
\end{table}

One last observation: 
Since $\zeta=\exp(2\pi i/5)$, it follows
that 
\[
-\zeta-\zeta^4=\frac{1-\sqrt{5}}{2},
\quad 
-\zeta^2-\zeta^3=\frac{1+\sqrt{5}}{2}.
\]

% Let us see what Magma says:

% \begin{lstlisting}
% > CharacterTable(Alt(5));


% Character Table
% ---------------


% ---------------------------
% Class |   1  2  3    4    5
% Size  |   1 15 20   12   12
% Order |   1  2  3    5    5
% ---------------------------
% p  =  2   1  1  3    5    4
% p  =  3   1  2  1    5    4
% p  =  5   1  2  3    1    1
% ---------------------------
% X.1   +   1  1  1    1    1
% X.2   +   3 -1  0   Z1 Z1#2
% X.3   +   3 -1  0 Z1#2   Z1
% X.4   +   4  0  1   -1   -1
% X.5   +   5  1 -1    0    0


% Explanation of Character Value Symbols
% --------------------------------------

% # denotes algebraic conjugation, that is,
% #k indicates replacing the root of unity w by w^k

% Z1     = (CyclotomicField(5: Sparse := true)) ! [ RationalField() | 0, 0, -1, -1 ]    
% \end{lstlisting}

\subsection{Mackey's theorem}

\begin{definition}
    Two representations of a finite group 
    are said to be \emph{disjoint} if they have no common 
    irreducible constituent. 
\end{definition}

\begin{exercise}
\label{xca:disjoint}
    Prove that 
    two representations are disjoint if and only if their characters are orthogonal. 
\end{exercise}

Let $H$ and $K$ be subgroups of a group $G$. 
The group $H\times K$ acts on $G$ via $(h,k)\cdot g=hgk^{-1}$. 
The orbit of $g$ under this action is 
the \emph{double coset} 
\[
HgK=\{(h,k)\cdot g:h\in H,k\in K\}
=\{hgk^{-1}:h\in H,k\in K\}
\]
with representative $g$. 

\begin{theorem}[Mackey]
\label{thm:Mackey}
\index{Mackey!theorem}
    Let $G$ be a finite group and $H$ and $K$ be 
    subgroups of $G$. Let $S$ be a complete set 
    of representatives of double $(H,K)$-cosets. If $\alpha\in\cf(K)$, then 
    \[
    \Res_H^G\Ind_K^G\alpha=\sum_{s\in S}\Ind_{H\cap sKs^{-1}}^H\Res_{H\cap sKs^{-1}}^{sKs^{-1}}(s\cdot f).
    \]
\end{theorem}

\begin{proof}
    For $s\in S$, let $X(s)$ be a left transversal 
    for $H\cap sKs^{-1}$ on $H$. Then 
    \[
    H=\bigcup_{x\in X(s)}x(H\cap sKs^{-1}),
    \]
    where the union is disjoint. 

    \begin{claim}
        $HsK=\bigcup_{x\in X(s)}xsK$, where the union is disjoint.
    \end{claim} 
    
    Let $z\in HsK$. Then $z=hsk$ for some $h\in H$ and $k\in K$. Since $h\in x(H\cap sKs^{-1})$ 
    for some $x\in X(s)$, 
    \[
    z=hsk\in x(H\cap sKs^{-1})sK\subseteq xsK. 
    \]
    Conversely, let $z\in xsK$ for some $x\in X(s)\subseteq H$. Then $z\in xsK\subseteq HsK$. To see that the union is disjoint, 
    suppose that $xsK=x_1sK$ for some $x,x_1\in X(s)$. Then 
    $x_1^{-1}x\in sKs^{-1}\cap H$. Thus $x(sKs^{-1}\cap H)=x_1(sKs^{-1}\cap H)$ and hence $x=x_1$, because $X(s)$ 
    is a left transversal for $sKs^{-1}\cap H$ in $H$. 

    \bigskip 
    Let $T(s)=\{xs:x\in X(s)\}$ and 
    \[
    T=\bigcup_{s\in S}T(s)
    \]
    To see that the union is disjoint, we proceed as follows. 
    Let $xs=x_1s_1$ for some $s,s_1\in S$, $x\in X(s)$ and $x_1\in X(s_1)$. 
    Since $x^{-1}x_1\in H$ and 
    $HsK=Hx^{-1}x_1s_1K=Hs_1K$, $s=s_1$ and hence $x=x_1$. 

    Then
    \[
    G=\bigcup_{s\in S}HsK
    =\bigcup_{s\in S}\bigcup_{x\in X(s)}xsK
    =\bigcup_{s\in S}\bigcup_{t\in T(s)}tK
    =\bigcup_{t\in T}tK.
    \]
    Since the unions are disjoint, 
    it follows that $T$ is a left transversal of $K$ in $G$. 

    For $h\in H$, 
    \begin{align*}
        (\Ind_K^G\alpha)(h) &= \sum_{t\in T}\alpha^0(t^{-1}ht)\\
        &=\sum_{s\in S}\sum_{t\in T(s)}\alpha^0(t^{-1}ht)\\
        &=\sum_{s\in S}\sum_{x\in X(s)}\alpha^0(s^{-1}x^{-1}hxs)\\
        &=\sum_{s\in S}\sum_{\substack{x\in X(s)\\x^{-1}hx\in sKs^{-1}}}(s\cdot \alpha)(x^{-1}hx)\\
        &=\sum_{s\in S}\sum_{\substack{x\in X(s)\\x^{-1}hx\in H\cap sKs^{-1}}}\Res_{H\cap sKs^{-1}}^{sKs^{-1}}(s\cdot \alpha)(\underbrace{x^{-1}hx)}_{\in H\cap sKs^{-1}}\\
        &=\sum_{s\in S}\Ind_{H\cap sKs^{-1}}^H\Res_{H\cap sKs^{-1}}^{sKs^{-1}}(s\cdot \alpha)(h).\qedhere 
    \end{align*}
\end{proof}

\begin{theorem}[Mackey's irreducibility criterion]
\label{thm:Mackey_irreducibility}
\index{Mackey!irreducibility criterion}
Let $H$ be a subgroup of a finite group $G$ and $\chi\in\Char(H)$.
Then $\Ind_H^G\chi\in\Irr(G)$ if and only if $\chi\in\Irr(H)$ and $\Res_{H\cap sHs^{-1}}^H\chi$ and 
$\Res_{H\cap sHs^{-1}}^{sHs^{-1}}(s\cdot\chi)$ are disjoint for all $s\not\in H$. 
\end{theorem}
\begin{proof}
    Let $S$ be a complete set of representatives of $(H,H)$-double cosets. Without loss of generality, 
    we may assume that $1\in S$. Note that if $s=1$, then 
    $H\cap sHs^{-1}=H$ and $s\cdot\chi=\chi$. By Mackey's theorem, 
    \begin{align*}
        \Res_H^G\Ind_H^G\chi 
        &=\sum_{s\in S}\Ind_{H\cap sHs^{-1}}^H\Res_{H\cap sHs^{-1}}^{sHs^{-1}}(s\cdot\chi)\\
        &=\chi+\sum_{1\ne s\in S}\Ind_{H\cap sHs^{-1}}^H\Res_{H\cap sHs^{-1}}^{sHs^{-1}}(s\cdot\chi).
    \end{align*}
    By Frobenius' reciprocity, 
    \begin{align*}
        \langle\Ind_H^G\chi,\Ind_H^G\chi\rangle
        &=\langle\Res_H^G\Ind_H^G\chi,\chi\rangle\\
        &=\underbrace{\langle\chi,\chi\rangle}_{\geq1}+\sum_{1\ne s\in S}\underbrace{\langle\Ind_{H\cap sHs^{-1}}^H\Res_{H\cap sHs^{-1}}^{sHs^{-1}}(s\cdot\chi),\chi\rangle}_{\geq0}.
    \end{align*}

    If $\chi\in\Irr(H)$ and $\Res_{H\cap sHs^{-1}}^H\chi$ and 
    $\Res_{H\cap sHs^{-1}}^{sHs^{-1}}(s\cdot\chi)$ are disjoint for all $s\not\in H$, then 
    $\langle\Ind_H^G\chi,\Ind_H^G\chi\rangle=1$ and hence $\Ind_H^G\chi\in\Irr(G)$. 

    Conversely, if $\Ind_H^G\chi\in\Irr(G)$, then $\langle\Ind_H^G\chi,\Ind_H^G\chi\rangle=1$. Thus 
    $\langle\chi,\chi\rangle=1$ and 
    \[
    \langle\Res_{H\cap sHs^{-1}}^{sHs^{-1}}(s\cdot\chi),\Res_{H\cap sHs^{-1}}^H\chi\rangle=
    \langle\Ind_{H\cap sHs^{-1}}^H\Res_{H\cap sHs^{-1}}^{sHs^{-1}}(s\cdot\chi),\chi\rangle=0
    \]
    for all $s\in S\setminus\{1\}$. As every element $s\notin H$ 
    could serve as a representative of an $(H,H)$-double coset, the claim follows.
\end{proof}

Theorem~\ref{thm:Mackey_irreducibility} takes a particularly elegant form when the subgroup is normal.

\begin{exercise}
\label{xca:Mackey}
    Let $H$ be a normal subgroup of a finite group $G$ and $\chi\in\Char(H)$.
    Then $\Ind_H^G\chi\in\Irr(G)$ if and only if $\chi\in\Irr(H)$ and $\chi\ne s\cdot\chi$ 
    for all $s\not\in H$. 
\end{exercise}

\begin{example}
\label{exa:p(p-1)}
    For a prime number $p\geq3$, let 
    \[
        G=\left\{\begin{pmatrix}a&b\\0&1\end{pmatrix}:0\ne a\in\Z/p,\,b\in\Z/p\right\}\text{ and }
        H=\left\{\begin{pmatrix}1&b\\0&1\end{pmatrix}:b\in\Z/p\right\}.
    \]
    Then $|G|=p(p-1)$, $H$ is a normal subgroup of $G$ and $|G/H|=p-1$. Let 
    \[
    \chi\colon H\to\C^{\times},\quad\begin{pmatrix}1&b\\0&1\end{pmatrix}\mapsto \exp(2\pi ib/p).
    \]
    Then $\chi$ is a group homomorphism. For each $a\in\Z/p\setminus\{0,1\}$, let $s(a)=\begin{pmatrix}a&0\\0&1\end{pmatrix}$. Then 
    \[
    (s(a)\cdot\chi)\begin{pmatrix}1&b\\0&1\end{pmatrix}=\chi\begin{pmatrix}1&a^{-1}b\\0&1\end{pmatrix}=\exp(2\pi ia^{-1}b/p)
    \ne \exp(2\pi ib/p).
    \]
    Hence $s(a)\cdot\chi\ne\chi$ for all $a\in\Z/p\setminus\{0,1\}$. By Exercise~\ref{xca:Mackey}, 
    $\Ind_H^G\chi\in\Irr(G)$ and 
    \[
    \deg\Ind_H^G\chi=(\Ind_H^G\chi)(1)=(G:H)\chi(1)=p-1.
    \]
    
    Since 
    $|G|-(p-1)^2=p-1$, 
    we still need additional irreducible characters to fully determine $\Irr(G)$. 
    The group $G/H$ is cyclic of order $p-1$, so it has $p-1$ irreducible characters, all of degree one. 
    These irreducible characters lift to irreducible characters of $G$ (see Theorem~\ref{thm:correspondence}). 
\end{example}

\begin{bonus}
\label{xca:p(p-1)}
    Find the character table of the group of Example~\ref{exa:p(p-1)}. 
\end{bonus}